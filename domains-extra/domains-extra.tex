\chapter{Свойства на областите на Скот}

\section{Област на Скот от непрекъснати изображения}

Следващата теорема е важна, защото с нейна помощ се доказват много свойства на непрекъснатите изображения.

\begin{thm}
  \label{th:double-chain}
  \marginpar{\cite[стр. 127]{winskel}}
  \marginpar{\cite[стр. 178]{models-of-computation}}
  Нека $\A = (A,\sqsubseteq,\bot)$ да бъде област на Скот и нека множеството 
  \[E = \{a_{m,n} \mid m,n \in \Nat\}\]
  от елементи на $A$ притежава свойството, че 
  \[n \leq n^\prime\ \&\ m \leq m^\prime\ \Rightarrow\ a_{n,m} \sqsubseteq a_{n^\prime,m^\prime}.\]
  Тогава множеството $E$ има точна горна граница, която означаваме с $\bigsqcup E$, и са изпълнени равенствата
  \[\bigsqcup E = \bigsqcup_m(\bigsqcup_n a_{n,m}) = \bigsqcup_n(\bigsqcup_{m} a_{n,m}) = \bigsqcup_n a_{n,n}.\]
\end{thm}
\begin{proof}
  Първо ще въведем някои означения.
  \begin{itemize}
  \item 
    Да фиксираме произволно $m$. Тогава можем да подредим елементите на множеството $\{a_{n,m} \mid n \in \Nat\}$ във възходящ ред: 
    \[a_{0,m} \sqsubseteq a_{1,m} \sqsubseteq a_{2,m} \sqsubseteq \cdots\]
    \marginpar{По дефиниция, всяка монотонно растяща редица в област на Скот притежава точна горна граница.}
    Следователно тя има точна горна граница $b_m \dff \bigsqcup \{a_{n,m} \mid n \in \Nat\}$.
  \item
    Аналогично, при фиксирано $n$, можем да подредим елементите на множеството $\{a_{n,m} \mid m \in \Nat\}$ в монотонно растяща редица:
    \[a_{n,0} \sqsubseteq a_{n,1} \sqsubseteq a_{n,2} \sqsubseteq \ldots,\]
    която притежава точна горна граница $c_n \dff \bigsqcup \{a_{n,m} \mid m \in \Nat\}$.
  \end{itemize}
  Това означава, че трябва да докажем следното:
  \[\bigsqcup E = \bigsqcup_mb_m = \bigsqcup_n c_n = \bigsqcup_n a_{n,n}.\]
  \begin{enumerate}[1)]
  \item 
    Първо да съобразим, че множеството $\{b_m \mid m \in \Nat\}$ образува верига в $\A$ и следователно притежава точна горна граница $\bigsqcup_m b_m$.
    Нека да разгледаме произволни $m \leq m^\prime$.
    Тогава \[(\forall n)[a_{n,m} \sqsubseteq a_{n,m^\prime} \sqsubseteq \bigsqcup_k a_{k,m^\prime} = b_{m^\prime}].\]
    Следователно $b_{m^\prime}$ е горна граница на веригата $(a_{n,m})^{\infty}_{n=0}$ и понеже $b_m$ е точна горна граница на $(a_{n,m})^{\infty}_{n=0}$, то получаваме, че \[b_m \sqsubseteq b_{m^\prime}.\]
    Това означава, че $\chain{b}{m}$ е верига в $\A$ и тя притежава точна горна граница $\bigsqcup_m b_m$.  
  \item
    С подобни разсъждения можем да докажем, че множеството $\{c_n \mid n \in \Nat\}$ образува верига в $\A$, която притежава точна горна граница $\bigsqcup_n c_n$.
  \item
    Сега ще докажем, че \[\bigsqcup_m b_m = \bigsqcup_n c_n.\]
    Имаме, че 
    \[(\forall m)(\forall n)[a_{n,m} \sqsubseteq \bigsqcup_na_{n,m} = b_m \sqsubseteq \bigsqcup_m b_m],\]
    което е еквивалентно на 
    \[(\forall n)(\forall m)[a_{n,m} \sqsubseteq b_m \sqsubseteq \bigsqcup_m b_m].\]
    Да фиксираме произволно $n$.
    Тогава $\bigsqcup_m b_m$ е горна граница на веригата $(a_{n,m})^\infty_{m=0}$.
    Следователно, $c_n = \bigsqcup_m a_{n,m} \sqsubseteq \bigsqcup_m b_m$.
    Така получаваме, че $\bigsqcup_m b_m$ е горна граница и на веригата $\chain{c}{n}$
    и тогава \[\bigsqcup_n c_n \sqsubseteq \bigsqcup_m b_m.\]
    С аналогични разсъждения можем да докажем също, че 
    \[\bigsqcup_m b_m \sqsubseteq \bigsqcup_n c_n.\]
    Така доказахме, че \[\bigsqcup_m b_m = \bigsqcup_n c_n.\]
  \item
    Сега ще докажем, че \[\bigsqcup E = \bigsqcup_m b_m.\]
    За целта ще проверим следното:
    \begin{enumerate}[a)]
    \item 
      $\bigsqcup_m b_m$ е горна граница на $E$.
    \item
      Ако $d$ е друга горна граница на $E$, то $\bigsqcup_m b_m \sqsubseteq d$.
    \end{enumerate}
    \begin{itemize}
    \item 
      $\bigsqcup_m b_m$ е горна граница на $E$, защото
      \[(\forall m)(\forall n)[a_{n,m} \sqsubseteq b_m \sqsubseteq \bigsqcup_m b_m].\]
    \item
      Нека $d$ е друга горна граница на $E$, т.е.
      \[(\forall m)(\forall n)[a_{n,m} \sqsubseteq d].\]
      Да фиксираме произволно $m$.
      Тогава $d$ е горна граница на веригата $(a_{n,m})^\infty_{n=0}$.
      Това означава, че $b_m = \bigsqcup_n a_{n,m} \sqsubseteq d$.
      Получаваме, че $d$ е горна граница на веригата $\chain{b}{m}$,
      откъдето следва, че $\bigsqcup_m b_m \sqsubseteq d$.
    \end{itemize}
    Заключаваме, че $\bigsqcup_m b_m$ е точната горна граница на $E$.
    Обобщавайки всичко от по-горе, следва, че:
    \[\bigsqcup E = \bigsqcup_m b_m = \bigsqcup_n c_n.\]
  \item
    Остана да видим, че 
    \[\bigsqcup E = \bigsqcup_n a_{n,n}.\]
    \begin{itemize}
    \item 
      Да разгледаме произволен елемент $a_{n,m} \in E$.
      Нека $k = \max\{n,m\}$.
      Ясно е, че $a_{n,m} \sqsubseteq a_{k,k} \sqsubseteq \bigsqcup_na_{n,n}$.
      Следователно, $\bigsqcup_n a_{n,n}$ е горна граница на $E$, откъдето получаваме
      \[\bigsqcup E \sqsubseteq \bigsqcup_n a_{n,n}.\]
    \item
      Нека $d$ е горна граница на $E$.
      Тогава $(\forall n)(\forall m)[a_{n,m} \sqsubseteq d]$
      и в частност, $(\forall n)[a_{n,n} \sqsubseteq d]$.
      Сега можем да заключим, че $\bigsqcup_n a_{n,n} \sqsubseteq d$.    
    \end{itemize}
    Така доказахме, че $\bigsqcup_n a_{n,n}$ е точна горна граница на $E$.
    \end{enumerate}
  С това доказателството на теоремата е завършено.
\end{proof}

\begin{framed}
  \begin{lemma}
    Нека $\A$ и $\B$ са области на Скот.
    Нека $\chain{f}{k}$ е верига от елементи на $\Cont{\A}{\B}$.
    Да дефинираме изображението $h$ на $\A$ в $\B$ по следния начин
    \[h(a) \dff \bigsqcup\{f_k(a) \mid k \in \Nat\}.\]
    Изображението $h$ е {\em непрекъснато} и е {\em точна горна граница} на веригата $\chain{f}{k}$,
    т.е. $h = \bigsqcup_k f_k$.
  \end{lemma}
\end{framed}
\marginpar{Ако $b_k = f_k(a)$, то $h(a)$ е точната горна граница на веригата $\chain{b}{k}$ в $\B$}
\begin{proof}
  \ifhints
  Доказателството, че $h$ е точна горна граница на веригата $\chain{f}{k}$ е лесно.
  \begin{itemize}
  \item 
    Да разгледаме произволен елемент $a \in A$.
    Лесно се вижда, че понеже $\chain{f}{k}$ е верига, то $(f_k(a))^\infty_{k=0}$ също е верига.
    Това е така, защото всяко непрекъснато изображение е също така и монотонно.

    \marginpar{$\bigsqcup_n f_n(a)$ е съкратен запис за $\bigsqcup\{f_n(a) \mid n \in \Nat\}$.}
    Получаваме, че за всяко $k$, $f_k(a) \sqsubseteq^\B \bigsqcup_n f_n(a) \dff h(a)$.
    Понеже това е вярно за произволно $a \in A$, $(\forall k)[f_k \sqsubseteq h]$,
    което означава, че $h$ е горна граница на веригата.
  \item
    Да разгледаме произволно изображение $g$, което е горна граница на веригата $\chain{f}{k}$.
    За произволен елемент $a \in A$, 
    \[(\forall k)[f_k(a) \sqsubseteq^\B g(a)].\]
    Това означава, че $g(a)$ е горна граница на веригата $(f_k(a))^\infty_{k=0}$.
    Понеже $h(a) = \bigsqcup_k \{f_k(a)\}$ е точната горна граница на веригата $(f_k(a))^\infty_{k=0}$,
    то $h(a) \sqsubseteq^\B g(a)$.
    Оттук следва, че $h \sqsupseteq g$.
  \end{itemize}
  \fi
  По-сложната част на доказателството е проверката, че $h$ е непрекъснато изображение.
  Да вземем една монотонно растяща редица $\chain{a}{k}$ от елементи на $A$.
  \marginpar{За момента дори не е ясно дали $\{h(a_k) \mid k \in \Nat\}$ е верига в $\B$}
  Ще докажем, че \[h(\bigsqcup_k a_k) = \bigsqcup_k \{h(a_k)\}.\]
  Нека $e_{n,m} \dff f_n(a_m)$.
  Понеже всяко $f_n$ е непрекъснато и следователно монотонно изображение, то имаме
  \[n \leq n^\prime\ \&\ m \leq m^\prime\ \Rightarrow\ e_{n,m} \sqsubseteq^{\B} e_{n^\prime,m^\prime}.\]
  Следователно,
  \begin{align*}
    h(\bigsqcup_m a_m) & = \bigsqcup_n(f_n(\bigsqcup_m a_m)) & \comment{\text{от деф. на }h}\\
                       & = \bigsqcup_n(\bigsqcup_m f_n(a_m)) & \comment{\text{ защото } f_n \text{ е непр.}}\\
                       & = \bigsqcup_n(\bigsqcup_m e_{n,m}) = \bigsqcup_m(\bigsqcup_n e_{n,m}) & \comment{\text{от \Th{double-chain}}}\\
                       & = \bigsqcup_m(\bigsqcup_n f_n(a_m)) & \comment{\text{от деф. на }e_{n.m}}\\
                       & = \bigsqcup_m \{h(a_m)\}. & \comment{\text{от деф. на }h}
  \end{align*}
\end{proof}

Да напомним, че релацията $\sqsubseteq$ между две изображения е дефинирана като
\[f \sqsubseteq g \dfff (\forall a\in A)[f(a) \sqsubseteq^\B g(a)].\]
\begin{framed}
  \begin{cor}
    $(\Cont{\A}{\B}, \sqsubseteq, \bm{\bot} )$ е област на Скот.
  \end{cor}
\end{framed}



% \begin{remark}
%   Интересен въпрос е при какви условия областта на Скот $\Cont{\A_1}{\A_2}$ е алгебрична.
%   Ние по-късно ще разгледаме този въпрос в един частен, но достатъчно общ, случай, 
%   а именно когато $\A_1$ и $\A_2$ са областта на Скот от лениви списъци.
% \end{remark}

% \begin{problem}
%   Нека $\A = (A, \sqsubseteq^\A, \bot^\A)$ е област на Скот, а $X$ е произволно непразно множество.
%   Определяме $\A^X = (A^X,\sqsubseteq^X,\bot^X)$ като:
%   \begin{itemize}
%   \item 
%     $A^X \dff \{f:X \to A\mid f\mbox{ е тотална}\}$;
%   \item
%     $f \sqsubseteq^X g\ \iff\ (\forall x \in X)[f(x) \sqsubseteq^\A g(x)]$;
%   \item
%     $\bot^X \in A^X$, като $(\forall x \in X)[\bot^X(x) = \bot^\A]$.
%   \end{itemize}
%   Докажете, че $\A^X$ е област на Скот.
% \end{problem}
% \begin{proof}
%   \begin{enumerate}[1)]
%   \item 
%     Лесно се вижда, че $\sqsubseteq^X$ е частична наредба.
%   \item
%     Също така, $\bot^X$ е най-малкият елемент на $A^X$ относно $\sqsubseteq^X$.
%   \item
%     Нека $(f_n)^\infty_{n=0}$ е верига от елементи на $A^X$. 
%     Ще докажем, че функцията $h$, дефинирана като $h(x) = \bigsqcup_n f_n(x)$,
%     е точна горна граница на $(f_n)^\infty_{n=0}$.
%     Но преди това, първо да съобразим, че $h \in A^X$.
%     Това следва от факта, че $\A$ е област на Скот и за всяко $x\in X$,
%     $(f_n(x))^\infty_{n=0}$ е верига в $\A$ 
%     и следователно има точна горна граница, която е равна на $h(x)$.
    
%     Сега ще покажем, че $h$ е горна граница на $(f_n)^\infty_{n=0}$.
%     Имаме, че:
%     \begin{align*}
%       (\forall x\in X)(\forall n)[f_n(x) \sqsubseteq^\A \bigsqcup_n f_n(x) = h(x)] & \iff (\forall n)(\forall x\in X)[f_n(x) \sqsubseteq^\A h(x)] \\
%       & \iff (\forall n)[f_n \sqsubseteq^\A h].
%     \end{align*}
    
%     Остана да видим, че $h$ е най-малката измежду горните граници на редицата $(f_n)^\infty_{n=0}$.
%     Нека $g$ е произволна горна граница на $(f_n)^\infty_{n=0}$. Това означава, че:
%     \begin{align*}
%       (\forall n)[f_n \sqsubseteq^X g] & \iff (\forall n)(\forall x\in X)[f_n(x) \sqsubseteq^\A g(x)]\\
%       & \iff (\forall x\in X)(\forall n)[f_n(x) \sqsubseteq^\A g(x)]
%     \end{align*}
%     Да разгледаме произволно $x \in X$. Тогава:
%     \[(\forall n)[f_n(x) \sqsubseteq^\A \bigsqcup_n f_n(x) \sqsubseteq^\A g(x)].\]
%     Понеже $h(x) = \bigsqcup_n f_n(x)$, получаваме, че за всяко $x \in X$,
%     \[h(x) \sqsubseteq^\A g(x),\]
%     което означава, че
%     $h \sqsubseteq^X g$.
%   \end{enumerate}
% \end{proof}

Нека $\A_1,\dots,\A_n$ и $\A$ са области на Скот и да разгледаме $f: \A_1\times \dots \times \A_n \to \A$.
Казваме, че $f$ е {\bf непрекъснато изображение по $i$-тия аргумент}, ако 
за всяка верига $\chain{a}{k}$ в $\A_i$, то
\[f(b_1,\dots, b_{i-1}, \bigsqcup_k a_k, b_{i+1},\dots,b_n) = \bigsqcup_kf(b_1,\dots, b_{i-1}, a_k, b_{i+1},\dots,b_n).\]

\begin{prop}
  Нека $\A_1,\dots,\A_n$ и $\A$ са области на Скот.
  Едно изображение $f: \A_1\times \dots \times \A_n \to \A$ 
  е непрекъснато точно тогава, когато $f$ е непрекъснато по всеки от аргументите си.
\end{prop}
\Stefan{Интересно е, че това твърдение не е вярно за непрекъснати свойства. Добре е да се обясни някъде.}

\begin{proof}
  \marginpar{\writedown Обобщете това твърдение за $n > 2$.}
  За по-просто изложение, да разгледаме случая $n = 2$.

  $(\Rightarrow)$ Лесно се съобразява, че ако $f$ е непрекъснато изображение, то $f$ е непрекъснато по всеки от аргументите си.
  \Stefan{Все пак е добре тук да се напише нещо.}
  
  $(\Leftarrow)$ Нека сега $f$ е непрекъснато по всеки от аргументите си. Ще докажем, че $f$ е непрекъснато.
  Нека $\{\pair{a_n,b_n}\}^\infty_{n=0}$ е верига в $\A_1\times \A_2$.
  Понеже от \Prop{cartesian} знаем, че
  \[\bigsqcup_n\pair{a_n,b_n} = \pair{\bigsqcup_na_n,\bigsqcup_n b_n},\]
  ще докажем, че 
  \[\bigsqcup_n f(a_n,b_n) = f(\bigsqcup_n a_n,\bigsqcup_n b_n).\]
  Да положим $e_{n,m} = f(a_n,b_m)$.
  Понеже $f$ е непрекъснато по всеки от аргументите си, лесно се вижда, че $f$
  е монотонно изображение по всеки от аргументите си. Следователно, 
  \[n \leq n^\prime\ \&\ m \leq m^\prime\ \Rightarrow\ e_{n,m} \sqsubseteq e_{n^\prime,m^\prime}.\]  
  Получаваме, че
  \begin{align*}
    \bigsqcup_n f(a_n,b_n) & = \bigsqcup_n e_{n,n} & \comment{\text{от опр. на }e_{n,m}}\\
                           & = \bigsqcup_n(\bigsqcup_m e_{n,m}) & \comment{\text{от \Th{double-chain}}}\\
                           & = \bigsqcup_n(\bigsqcup_m f(a_n,b_m)) & \comment{\text{от опр. на }e_{n,m}}\\
                           & = \bigsqcup_nf(a_n,\bigsqcup_m b_m) & \comment{f \text{ е непр. по втория си аргумент}}\\
                           & = f(\bigsqcup_n a_n,\bigsqcup_m b_m) & \comment{f \text{ е непр. по първия си аргумент}}.
  \end{align*}
\end{proof}


%%% Local Variables:
%%% mode: latex
%%% TeX-master: "../sep"
%%% End:


 \section{Оператор за най-малка неподвижна точка}

\Stefan{Това да се остави само с упътване.}

\begin{thm}
  % \index{$Y_\A$}
  Нека $\A$ е област на Скот и нека $f \in \Cont{\A}{\A}$.
  \marginpar{Знаем от \Th{knaster-tarski}, че най-малката неподвижна точка на $f$ е елемента $\bigsqcup_n f^n(\bot^\A)$.}
  Тогава изображението $Y_\A : \Cont{\A}{\A} \to \A$, определено като
  \[Y_\A(f) = \lfp(f),\]
  е непрекъснато, т.е.
  $Y_\A \in \Cont{\Cont{\A}{\A}}{\A}$.
\end{thm}
\begin{proof}
  Нека да вземем една верига $(f_n)^\infty_{n=0}$ от непрекъснати изображения.
  Нашата цел е да докажем, че
  \[Y_\A(\bigsqcup_n f_n) = \bigsqcup_n Y_\A(f_n).\]
  Да означим с $h$ точната горна граница на $(f_n)^\infty_{n=0}$.
  Знаем, че $h(a) = \bigsqcup_n f_n(a)$.
  \begin{prop}
    За всяко $k \geq 1$, $h^k(a) = \bigsqcup_n f^k_n(a)$.
  \end{prop}
  \begin{proof}
    Ще докажем твърдението с индукция по $k$, като случая $k = 1$ следва от дефиницията на $h$.
    Нека приемем, че твърдението е вярно за произволно $k \geq 1$.
    Ще докажем, че твърдението е вярно за $k+1$.
    \begin{align*}
      h^{k+1}(a) & = h(h^k(a)) & \\
      & = h(\bigsqcup_n f^k_n(a))& \comment{\text{ от инд. предположение}}\\
      & = \bigsqcup_n h(f^k_n(a))& \comment{h \text{ е непрекъснато изображение}}\\
      & = \bigsqcup_n (\bigsqcup_m f_m(f^k_n(a))). & 
    \end{align*}
    
    Да положим $b_n = f^k_n(a)$, за всяко $n$.
    Понеже $f_n \sqsubseteq f_{n^\prime}$, лесно се съобразява, че за $n \leq n^\prime$
    имаме $b_n \sqsubseteq^\A b_{n^\prime}$.

    Сега да положим $e_{m,n} = f_m(b_n)$.
    Отново, понеже $(b_n)^\infty_{n=0}$ и $(f_m)^\infty_{m=0}$ са вериги, имаме 
    \[m \leq m^\prime\ \&\ n\leq n^\prime\ \Rightarrow\ e_{m,n} \sqsubseteq^\A e_{m^\prime,n^\prime}.\]
    Това означава, че можем да приложим \Th{double-chain} за множеството $E = \{e_{m,n} \mid m,n \in \Nat\}$.
    Получаваме, че
    \begin{align*}
      h^{k+1}(a) & = \bigsqcup_n (\bigsqcup_m f_m(f^k_n(a))) & \comment{\text{ от горното равенство}}\\
      & = \bigsqcup_n (\bigsqcup_m e_{m,n}) & \comment{\text{ от определението на }e_{m,n}}\\
      & = \bigsqcup_n e_{n,n} & \comment{\text{ от \Th{double-chain}}}\\
      & = \bigsqcup_n f_n(f^k_n(a))  = \bigsqcup_n f^{k+1}_n(a) & 
    \end{align*}
    С това твърдението е доказано.
  \end{proof}
  Сега вече сме готови да докажем непрекъснатостта на $Y_\A$.
  Имаме, че:
  \begin{align*}
    Y_\A(\bigsqcup_n f_n) & = Y_\A(h) & \comment{\text{ от опр. на }h}\\
    & = \bigsqcup_m h^m(\bot^\A) & \comment{\text{ от опр. на }Y_\A }\\
    & = \bigsqcup_m (\bigsqcup_n f^m_n(\bot^\A)) & \comment{\text{ от горното твърдение}}.
  \end{align*}
  
  Да положим $e_{m,n} = f^m_n(\bot^\A)$.
  Отново лесно се съобразява, че 
  \[m \leq m^\prime\ \&\ n\leq n^\prime\ \Rightarrow\ e_{m,n} \sqsubseteq^\A e_{m^\prime,n^\prime}.\]
  Получаваме, че
  \begin{align*}
    Y_\A(\bigsqcup_n f_n) & = \bigsqcup_m (\bigsqcup_n f^m_n(\bot^\A)) & \comment{\text{ от горното равенство}}\\
    & = \bigsqcup_m (\bigsqcup_n e_{m,n}) & \comment{\text{ от опр. на }e_{m,n}}\\
    & = \bigsqcup_n(\bigsqcup_m e_{m,n}) & \comment{\text{ от \Th{double-chain}}}\\
    & = \bigsqcup_n (\bigsqcup_m f^m_n(\bot^\A)) = \bigsqcup_n Y_\A(f_n). & \comment{\text{ от опр. на }Y_\A}.
  \end{align*}
\end{proof}



%%% Local Variables:
%%% mode: latex
%%% TeX-master: "../sep"
%%% End:


\section{Изоморфни области на Скот}
\index{изоморфизъм}

Нека $\A_1 = (A_1,~\sqsubseteq_1,~\bot_1)$ и $\A_2 = (A_2,~\sqsubseteq_2~,~\bot_2)$ 
са области на Скот.
Ще казваме, че $\A_1$ е {\bf изоморфна} на $\A_2$, което ще означаваме като 
\[\A_1 \cong \A_2,\]
ако съществува {\em биективна} функция $F:A_1 \to A_2$ със свойството:
\marginpar{Ясно е, че $F(\bot_1) = \bot_2$}
\[(\forall a,b\in A_1)[\ a \sqsubseteq_1 b \iff F(a) \sqsubseteq_2 F(b)\ ].\]
В такъв случай ще казваме, че $F$ задава изоморфизъм между $\A_1$ и $\A_2$.

Когато искаме да означим, че $\A_1$ е изоморфна на $\A_2$ чрез $F$,
то понякога ще пишем $\A_1 \cong_F \A_2$.

\begin{prop}
  \label{pr:isomorphism-is-continuous}
  Ако $\A_1 \cong_F \A_2$ , то $F \in \Cont{\A_1}{\A_2}$.
\end{prop}
\begin{hint}
  Да разгледаме произволна верига $\chain{a}{i}$ от елементи на $\A_1$.
  Ще докажем, че 
  \[F(\bigsqcup_i a_i) = \bigsqcup_iF(a_i).\]
  
  \begin{itemize}
  \item 
    Първо, от дефиницията веднага следва, че $F$ е монотонно изображение,
    защото $a \sqsubseteq_1 b \implies F(a) \sqsubseteq_2 F(b)$.
    Това означава, че $(F(a_i))^\infty_{i=0}$ е монотонно растяща верига от елементи на $\A_2$.
    От \Prop{monotone-chain} получаваме, че 
    \[\bigsqcup_i F(a_i) \sqsubseteq_2 F(\bigsqcup_i a_i).\]
  \item
    За другата посока, нека $b \in \A_2$ е горна граница на веригата $(F(a_i))^\infty_{i=0}$, т.е. 
    \[(\forall i)[\ F(a_i) \sqsubseteq_2 b\ ].\]
    Ще докажем, че $F(\bigsqcup_i a_i) \sqsubseteq_2 b$.
    Понеже $F$ е {\em върху} $A_2$, то съществува елемент $a \in A_1$, такъв че $F(a) = b$.
    Тогава:
    \begin{align*}
      (\forall i)[\ F(a_i) \sqsubseteq_2 F(a)\ ] & \implies (\forall i)[\ a_i \sqsubseteq_1 a\ ] & \comment{F \text{ е изоморфизъм }}\\
                                                 & \implies \bigsqcup_i a_i \sqsubseteq_1 a & \comment{a\text{ е горна граница}}\\
                                                 & \implies F(\bigsqcup_i a_i) \sqsubseteq_1 F(a). & \comment{F\text{ е изоморфизъм }}
    \end{align*}
    Понеже $b = F(a)$, заключаваме, че
    \[F(\bigsqcup_i a_i) \sqsubseteq_2 b.\]
  \end{itemize}
\end{hint}

\begin{prop}
  \label{pr:isomorphic-pair}
  Нека $f \in \Mon{\A_1}{\A_2}$ и $g \in \Mon{\A_2}{\A_1}$,
  като 
  \begin{itemize}
  \item 
    $f \circ g = \texttt{id}_2$;
  \item
    $g \circ f = \texttt{id}_1$.
  \end{itemize}
  \marginpar{$\texttt{id}_i(a) \dff a$ за вс. $a \in \A_i$}
  Тогава са изпълнени свойствата:
  \begin{enumerate}[(1)]
  \item
    $\A_1 \cong_f \A_2$;
  \item
    $\A_2 \cong_g \A_1$;
  \end{enumerate}
\end{prop}
% \begin{hint}
%   За Свойство $(1)$ трябва да проверим, че $f$ отговаря на дефиницията за изоморфизъм.
%   \begin{itemize}
%   \item
%     Ще докажем, че $f$ е инективна като покажем, че за произволни $a, b\in A_1$,
%     ако $f(a) = f(b)$, то $a = b$.
%     Но това е лесно, защото
%     \[a = \texttt{id}_1(a) = g(f(a)) = g(f(b)) = \texttt{id}_1(b) = b.\]
%   \item
%     Нека сега $b \in \A_2$.
%     Знаем, че $f(g(b)) = \texttt{id}_2(b) = b$. Това означава, че $f$ е {\em сюрективна},
%     защото за всеки елемент $b \in A_2$ съществува елемент $a \in A_1$, а именно $a = g(b)$,
%     за който $f(a) = b$.
%   \item
%     Понеже $f$ е монотонно изображение, то директно имаме, че
%     \[a \sqsubseteq_1 b \implies f(a) \sqsubseteq_2 f(b).\]
%   \item
%     Нека $f(a) \sqsubseteq_2 f(b)$.
%     Сега пък понеже $g$ е монотонно изображение, 
%     \[a = \texttt{id}_1(a) = g(f(a)) \sqsubseteq_1 g(f(b)) = \texttt{id}_1(b) = b.\]
%     Така показахме, че
%     \[f(a) \sqsubseteq_2 f(b)\ \implies\ a \sqsubseteq_1 b.\]
%   \end{itemize}
%   Доказахме Свойство $(1)$, т.е. $\A_1 \cong_f \A_2$.
%   Разсъжденията за Свойство $(2)$ са аналогични.
% \end{hint}


\begin{prop}
  \label{pr:isomorphic-higher-order}
  Нека $\A_1 \cong_F \A_2$. Тогава:
  \begin{enumerate}[(1)]
  \item 
    $\Cont{\A_1}{\A_1} \cong_G \Cont{\A_2}{\A_2}$, където 
    \[G(f) \dff F \circ f \circ F^{-1};\]
    Графично това може да се изобрази така:

    \shorthandoff{"}%
    \begin{center}
    \begin{tikzcd}[sep=large]
      \A_1 \arrow[r, "f"] & \A_1 \arrow[d, "F"]\\
      \A_2 \arrow[u, "F^{-1}"]\arrow[r, dashed, "G(f)"] & \A_2 
    \end{tikzcd}
    \end{center}
    \shorthandon{"}%
  \item
    ако $f \in \Cont{\A_1}{\A_1}$, то 
    \[F(\lfp(f)) = \lfp(G(f)).\]
  \end{enumerate}
\end{prop}
\begin{hint}
  Ще докажем $(1)$ като използвме \Prop{isomorphic-pair}.

  \begin{itemize}
  \item 
    Ще докажем, че $G$ е монотонно изображение.
    Нека $f,h \in \Cont{\A_1}{\A_1}$ и $f \sqsubseteq h$, т.е.
    \[(\forall a \in \A_1)[\ f(a) \sqsubseteq_1 h(a)\ ].\]
    Ще докажем, че $G(f) \sqsubseteq G(h)$, т.е.
    \[(\forall b \in \A_2)[\ G(f)(b) \sqsubseteq_1 G(h)(b)\ ].\]
    Да разгледаме произволен елемент $b \in \A_2$. 
    Понеже $F$ е биекция, то съществува елемент $a \in A_1$, такъв че $F(a) = b$,
    т.е. $F^{-1}(b) = a$. Тогава:
    \begin{align*}
      G(f)(b) & \dff F(f(F^{-1}(b)))\\
              & = F(f(a)) & \comment{F^{-1}(b) = a}\\
              & \sqsubseteq_2 F(h(a)) & \comment{f(a) \sqsubseteq h(a)\text{ и $F$ е изом.}}\\
              & = F(h(F^{-1}(b))) & \comment{F^{-1}(b) =a}\\
              & \dff G(h)(b).
    \end{align*}
  \item
    Нека $G(f) \sqsubseteq G(h)$. Ще докажем, че $f \sqsubseteq h$.
    За целта, нека $a \in A_1$.
    Понеже $F$ е сюрективна, то съществува $b \in A_2$, за който $F^{-1}(b) =a$.
    Понеже
    \[G(f) \dff F \circ f \circ F^{-1} \sqsubseteq F \circ h \circ F^{-1} = G(h),\]
    то получаваме, че
    \[F(f(F^{-1}(b))) \sqsubseteq_2 F(h(F^{-1}(b))).\]
    Оттук,
    \[F(f(a)) \sqsubseteq_2 F(h(a)) \implies f(a) \sqsubseteq_1 h(a),\]
    защото $F$ е изоморфизъм.
  \end{itemize}
  Сега преминаваме към доказателството на $(2)$.
  Да напомним, че за $f \in \Cont{\A_1}{\A_1}$, означаваме
  \begin{align*}
    f^0  & = \lambda x. \bot_1\\
    f^{n+1} & = f \circ f^n.
  \end{align*}
  Понеже $f$ е непрекъснато изображение е ясно, че $(f^n(\bot_1))^{\infty}_{n=0}$ е верига.
  Също така знаем, че
  \[\lfp(f) = \bigsqcup_n f^n(\bot_1).\]
  След аналогични разсъждения можем да съобразим, че
  \[\lfp(G(f)) = \bigsqcup_n G(f)^n(\bot_2).\]
  Първо ще докажем с индукция по $n$, че 
  \begin{equation}
    \label{eq:2}
    (\forall n)[\ (G(f))^n = G(f^n)\ ].
  \end{equation}
  \begin{itemize}
  \item 
    За $n = 0$ имаме, че за произволен елемент $b \in \A_2$,
    \begin{align*}
      (G(f))^{0}(b) & \dff \bot_2\\
                    & = F(\bot_1) & \comment{F \text{ е изом.}}\\
                    & = F(f^{0}(F^{-1}(b))) & \comment{f^{0}(F^{-1}(b)) \dff \bot_1}\\
                    & = (F \circ f^{0} \circ F^{-1})(b) \\
                    & \dff G(f^{0})(b).
    \end{align*} 
  \item
    Нека да приемем, че твърдението е вярно за $n$.
    Тогава за $n+1$ имаме, че:
    \begin{align*}
      (G(f))^{n+1} & \dff G(f) \circ (G(f))^n\\
                   & = G(f) \circ G(f^n) & \comment{\text{ от И.П.}}\\
                   & \dff (F \circ f \circ F^{-1}) \circ (F\circ f^n \circ F^{-1})\\
                   & = F \circ f \circ (F^{-1} \circ F)\circ f^n \circ F^{-1} \\
                   & = F \circ f \circ f^{n} \circ F^{-1} & \comment{F^{-1}\circ F = id}\\
                   & = F \circ f^{n+1} \circ F^{-1} & \comment{f\circ f^n = f^{n+1}}\\
                   & \dff G(f^{n+1}).
    \end{align*}
  \end{itemize}
  Тогава:
  \begin{align*}
    F(\lfp(f)) & = F(\bigsqcup_n f^n(\bot_1)) & \comment{\lfp(f) = \bigsqcup_n f^n(\bot_1)}\\
               & = \bigsqcup_n F(f^n(\bot_1))& \comment{F\text{ е непр.}}\\
               & = \bigsqcup_n F(f^n(F^{-1}(\bot_2))) & \comment{F^{-1}(\bot_2) = \bot_1}\\
               & = \bigsqcup_n (F \circ f^n \circ F^{-1})(\bot_2) \\
               & \dff \bigsqcup_n G(f^n)(\bot_2)\\
               & = \bigsqcup_n G(f)^n(\bot_2) & \comment{\text{от }(\ref{eq:2})}\\
               & = \lfp(G(f)).
  \end{align*}
\end{hint}

\begin{framed}
  \begin{prop}
    За произволни области на Скот $\A$, $\B$ и $\C$ е изпълнено, че
    \[\Cont{\A}{\Cont{\B}{\C}}\ \cong\ \Cont{\A\times\B}{\C}.\]
  \end{prop}  
\end{framed}
\begin{hint}
  % \begin{itemize}
  % \item 
    Докажете, че изображението
    \[\texttt{curry}:\Cont{\A\times \B}{\C} \to \Cont{\A}{\Cont{\B}{\C}},\]
    където
    \[\texttt{curry}(f)(a)(b) \dff f(a,b)\]
    задава изоморфизъм.
  % \item
  %   Докажете, че изображението
  %   \[\texttt{uncurry}:\Cont{\A}{\Cont{\B}{\C}} \to \Cont{\A\times \B}{\C},\]
  %   където
  %   \[\texttt{uncurry}(f)(a,b) \dff f(a)(b)\]
  %   е монотонно.
  % \item
  %   Лесно се съобразява, че
  %   \[\texttt{curry} \circ \texttt{uncurry} = \texttt{id}\]
  %   \[\texttt{uncurry} \circ \texttt{curry} = \texttt{id}.\]    
  % \item
  %   Приложете \Prop{isomorphic-pair}.
  % \end{itemize}
\end{hint}

\begin{remark}
  Когато на хаскел пишем типовата сигнатура на някоя функция като 
  \mint{haskell}|f :: a -> b -> c|
  в действителност се има предвид следното
  \mint{haskell}|f :: a -> (b -> c)|
  
  На практика тези две задачи ни казват, че няма значение дали използваме {\em curried}
  или {\em uncurried} версията на една функция. На хаскел е по-удобно да използваме {\em curried}
  версията, защото като фиксираме първия аргумент на една функция получаваме нова функция наготово.
  Например, 
  \begin{haskellcode}
ghci> let plus x y = x + y
ghci> :t plus
plus :: Num a => a -> a -> a
ghci> let plus1 = plus 1
ghci> :t plus1
plus1 :: Num a => a -> a
   \end{haskellcode}

  Всъщност, хаскел има функциите \texttt{curry} и \texttt{uncurry} вградени в стандартната библиотека:
  \begin{haskellcode}
ghci> :t curry
curry :: ((a, b) -> c) -> a -> b -> c
ghci> :t uncurry
uncurry :: (a -> b -> c) -> (a, b) -> c
  \end{haskellcode}
\end{remark}

\begin{problem}
  Докажете, че съществуват области на Скот $\A$, $\B$ и $\C$, за които
  \[\Cont{\Cont{\A}{\B}}{\C} \not\cong \Cont{\A}{\Cont{\B}{\C}}.\]  
\end{problem}

\subsection*{Точни продължения на частични функции}

За еднa частична функция $f:\Nat^n \to \Nat$, определяме изображението $f^\star \in \Strict{\Nat^n_\bot}{\Nat_\bot}$ по следния начин:
\begin{align*}
  f^\star(\ov{a}) \dff
  \begin{cases}
    f(\ov{a}), & \text{ако }\bot\not\in\{a_1,\dots,a_n\}\ \&\ f(\ov{a})\text{ е деф.}\\
    \bot, & \text{иначе}.
  \end{cases}
\end{align*}
Изображението $f^\star$ се нарича {\bf точно продължение} на $f$.

\begin{example}
  Да разгледаме частичната функция $f:\Nat^2 \to \Nat$, дефинирана като
  $f(x,y) = x+y$. Тогава точното продължение на $f$ е 
  \[f^\star(x,y) = 
  \begin{cases}
    x+y, & x,y\in\Nat\\
    \bot, & x = \bot \vee y = \bot.
  \end{cases}\]
\end{example}

\noindent Да дефинираме оператор $\Sigma^\star_n : \Partial{\Nat^n}{\Nat} \to \Strict{\Nat^n_\bot}{\Nat_\bot}$ като
\[\Sigma^\star_n(f) = f^\star.\] 
Ще наричаме $\Sigma^\star_n$ {\bf продължаващ} оператор, защото на всяка частична функция дава нейното точно продължение.


\begin{thm}
  За всяко естествено число $n$,
  \[\Partial{\Nat^n}{\Nat}\ \cong\ \Strict{\Nat^n_\bot}{\Nat_\bot}\]
  чрез изображението $\Sigma^\star_n$.
\end{thm}
\begin{hint}
  \begin{itemize}
  \item 
    Лесно се съобразява, че $\Sigma^\star_n$ е биективна функция на $\Partial{\Nat^n}{\Nat}$ върху $\Strict{\Nat^n_\bot}{\Nat_\bot}$.
  \item
    Лесно се проверява също, че $f \subseteq g \iff \Sigma^\star_n(f) \sqsubseteq \Sigma^\star_n(g)$.    
  \end{itemize}
\end{hint}

\begin{cor}
  $\Sigma^\star_n$ е непрекъснато изображение, т.е.
  \[\Sigma^\star \in \Cont{\Partial{\Nat^n}{\Nat}}{\Strict{\Nat^n_\bot}{\Nat_\bot}}.\]
\end{cor}

\begin{example}
  Да разгледаме следната рекурсивна програма $\vv{P}$:

  \begin{haskellcode}
f(x, y) = if x == y then 0
            else 1 + f(x, y + 1)
   \end{haskellcode}

  Ако искаме да намерим $\D_V\val{\vv{P}}(x,y)$,
  то формално погледнато трябва да намерим най-малката неподвижна точка на оператора
  \[\Delta \in \Cont{\Strict{\Nat^2_\bot}{\Nat_\bot}}{\Strict{\Nat^2_\bot}{\Nat_\bot}},\]
  където
  \[\Delta(g)(x,y) = 
  \begin{cases}
    0, & \text{ако } x,y\in\Nat\ \&\ x = y\\
    1 + g(x,y+1), & \text{ако } x,y,\in\Nat\ \&\ x \neq y\\
    \bot, & \text{ако }x = \bot\ \lor\ y = \bot.
  \end{cases}\]
  Лесно се съобразява, че
  \[\D_V\val{\vv{P}}(a,b) = \lfp(\Delta)(a,b) = 
  \begin{cases}
    a - b, & \text{ако }a,b\in\Nat\ \&\ a \geq b\\
    \bot, & \text{иначе}
  \end{cases}\]

В някои отношения е по-лесно да подходим по друг начин.
Да разглеждаме оператора
\[\Gamma \in \Cont{\Partial{\Nat^2}{\Nat}}{\Partial{\Nat^2}{\Nat}},\]
където
\[\Gamma(f)(x,y) \simeq
\begin{cases}
  0, & \text{ако }x = y\\
  1 + f(x,y+1), & \text{ако } x \neq y.
\end{cases}\]
Лесно се съобразява, че
\begin{align*}
  \lfp(\Gamma)(n,k) \simeq
  \begin{cases}
    n - k, & \text{ако }n \geq k\\
    \neg !, & \text{иначе}.
  \end{cases}
\end{align*}

Понеже $\Sigma^\star_2$ задава изоморфизъм между областите на Скот $\Partial{\Nat^2}{\Nat}$ и $\Strict{\Nat^2_\bot}{\Nat_\bot}$,
то от \Prop{isomorphic-higher-order} имаме, че
\[\Cont{\Partial{\Nat^2}{\Nat}}{\Partial{\Nat^2}{\Nat}} \cong_{\mathcal{G}} \Cont{\Strict{\Nat^2_\bot}{\Nat_\bot}}{\Strict{\Nat^2_\bot}{\Nat_\bot}},\]
където
\[\mathcal{G}(\Gamma) = \Sigma^\star_2 \circ \Gamma \circ (\Sigma^\star_2)^{-1}.\]
Графично това може да се представи така:
\shorthandoff{"}%
\begin{center}
  \begin{tikzcd}[sep=large]
    \Partial{\Nat^n}{\Nat} \arrow[r, "\Gamma"] & \Partial{\Nat^n}{\Nat} \arrow[d, "\Sigma^\star_2"]\\
    \Strict{\Nat^n_\bot}{\Nat_\bot} \arrow[u, "(\Sigma^\star_2)^{-1}"]\arrow[r, dashed, "\mathcal{G}(\Gamma)"] & \Strict{\Nat^n_\bot}{\Nat_\bot}
  \end{tikzcd}
\end{center}
\shorthandon{"}%
Сега директно получаваме, че
\begin{align*}
  \D_V\val{\vv{P}} & \dff \lfp(\Delta)\\
                   & = \Sigma^\star_2(\lfp(\Gamma)),
\end{align*}
защото $\Delta = \mathcal{G}(\Gamma)$ и от \Prop{isomorphic-higher-order} знаем, че
\[\lfp(\mathcal{G}(\Gamma)) = \Sigma^\star_2(\lfp(\Gamma)).\]
\end{example}

%%% Local Variables:
%%% mode: latex
%%% TeX-master: "../sep"
%%% End:


\section{Задачи}  

Тук с $\A$, $\B$ и $\C$ ще означаваме области на Скот.

\marginpar{Много от задачите са от \cite[стр. 31]{abramsky94}}

\begin{problem}
  Да разгледаме операторите \[\Gamma,\Delta \in \Cont{\Cont{\A}{\A}}{\Cont{\A}{\A}}.\]
  Знаем, че операторът $\Gamma \circ \Delta$ е непрекъснат, където
  \[(\Gamma\circ\Delta)(f) \dff \Gamma(\Delta(f)).\]
  Вярно ли е, че
  \[\lfp(\Gamma \circ \Delta) \sqsubseteq \lfp(\Gamma) \circ \lfp(\Delta)?\]
  Обосновете се!
\end{problem}
\ifhints
\begin{hint}
  Нека $\A = \Nat_\bot$.
  Нека например да разгледаме
  \begin{align*}
    & \Delta(f)(x) \dff f(x+1)\\
    & \Gamma(f)(x) \dff
      \begin{cases}
        0, & x \neq \bot\\
        \bot, & x = \bot.
      \end{cases}
  \end{align*}
  Да положим $f_\Gamma \dff \lfp(\Gamma)$ и $f_\Delta \dff \lfp(\Delta)$.
  Ясно е, че 
  \begin{align*}
    & f_\Delta(x) = \bot\\
    & f_\Gamma(x) =
    \begin{cases}
      0, & x \neq \bot\\
      \bot, & x = \bot.
    \end{cases}  
  \end{align*}
  Тогава за произволно $x \in \Nat_\bot$,
  \[(f_\Gamma\circ f_\Delta)(x) = f_\Gamma(f_\Delta(x)) = f_\Gamma(\bot)  = \bot.\]
  От друга страна, понеже $(\Gamma \circ \Delta)(f) = \Gamma(\Delta(f))$, то 
  \begin{align*}
    & (\Gamma \circ \Delta)(f)(x) = \Gamma(\Delta(f))(x) = 
      \begin{cases}
        0, & x \neq \bot\\
        \bot, & x = \bot.
      \end{cases}
  \end{align*}
  Лесно се съобразява, че 
  \[\lfp(\Gamma \circ \Delta)(x) =
  \begin{cases}
    0, & x \neq \bot\\
    \bot, & x = \bot.
  \end{cases}\]
  Заключаваме, че 
  \[\lfp(\Gamma \circ \Delta) \sqsupset \lfp(\Gamma) \circ \lfp(\Delta).\]
\end{hint}
\fi

\begin{problem}
  Да разгледаме операторите \[\Gamma,\Delta \in \Cont{\Cont{\A}{\A}}{\Cont{\A}{\A}}.\]
  Знаем, че операторът $\Gamma \circ \Delta$ е непрекъснат, където
  \[(\Gamma\circ\Delta)(f) \dff \Gamma(\Delta(f)).\]
  Вярно ли е, че
  \[\lfp(\Gamma \circ \Delta) \sqsupseteq \lfp(\Gamma) \circ \lfp(\Delta)?\]
  Обосновете се!
\end{problem}
\ifhints
\begin{hint}
  Нека $\A = \Nat_\bot$.
  Нека например да разгледаме
  \begin{align*}
    & \Delta(f)(x) \dff 0\\
    & \Gamma(f)(x) \dff
      \begin{cases}
        0, & x = 0\\
        \bot, & \text{ иначе}.
      \end{cases}
  \end{align*}
  Да положим $f_\Gamma \dff \lfp(\Gamma)$ и $f_\Delta \dff \lfp(\Delta)$.
  Ясно е, че 
  \begin{align*}
    & f_\Delta(x) = 0\\
    & f_\Gamma(x) =
    \begin{cases}
      0, & x = 0\\
      \bot, & \text{ иначе}.
    \end{cases}  
  \end{align*}
  Тогава за произволно $x \in \Nat_\bot$,
  \[(f_\Gamma\circ f_\Delta)(x) = f_\Gamma(f_\Delta(x)) = f_\Gamma(0)  = 0.\]
  От друга страна, понеже $(\Gamma \circ \Delta)(f) = \Gamma(\Delta(f))$, то 
  \begin{align*}
    & (\Gamma \circ \Delta)(f)(x) = \Gamma(\Delta(f))(x) = 
      \begin{cases}
        0, & x = 0\\
        \bot, & \text{ иначе}.
      \end{cases}
  \end{align*}
  Лесно се съобразява, че 
  \[\lfp(\Gamma \circ \Delta)(x) =
  \begin{cases}
    0, & x = 0\\
    \bot, & \text{ иначе}.
  \end{cases}\]
  Заключаваме, че 
  \[\lfp(\Gamma \circ \Delta) \sqsubset \lfp(\Gamma) \circ \lfp(\Delta).\]
\end{hint}
\fi

\begin{problem}
  Нека $f_0 \sqsubseteq f_1 \sqsubseteq f_2 \sqsubseteq \cdots$
  е верига от елементи на $\Cont{\A}{\A}$.
  Да положим $h = \bigsqcup_n f_n$.
  Вярно ли е, че 
  \[h \circ h = \bigsqcup_n (f\circ f)?\]
  Обосновете се!
\end{problem}

\begin{problem}
  Да разгледаме едно изображение $f: \A \times \B \to \C$.
  За произволно $a \in \A$, дефинираме изображението $g_a: \B \to \C$, където
  \[g_a(b) \dff f(a,b).\]
  Аналогично, за произволно $b \in \B$, дефинираме изображението $h_b: \A \to \C$, където
  \[h_b(a) \dff f(a,b).\]
  Докажете, че следните твърдения са еквивалентни:
  \begin{enumerate}[1)]
  \item 
    $f$ е непрекъснато изображение;
  \item
    $g_a$ и $h_b$ са непрекъснати изображения, за всяко $a \in \A$ и $b \in \B$.
  \end{enumerate}
\end{problem}

\begin{problem}
  Да разгледаме $f \in \Cont{\A \times \B}{\C}$.
  За произволно $a \in \A$, дефинираме изображението $g_a: \B \to \C$, където
  \[g_a(b) \dff f(a,b).\]
  Вече знаем, че $g_a \in \Cont{\B}{\C}$, за всяко $a \in \A$.
  Да разгледаме изображението $h:\A \to \Cont{\B}{\C}$, където
  \[h(a) \dff g_a.\]
  Докажете, че $h$ е непрекъснато изображение.
\end{problem}

\begin{problem}
  \marginpar{\cite[стр. 129]{nikolova-soskova}}
  Да разгледаме $f \in \Cont{\A \times \B}{\B}$.
  За произволно $a \in \A$, дефинираме изображението $g_a: \B \to \B$, където
  \[g_a(b) \dff f(a,b).\]
  Вече знаем, че $g_a \in \Cont{\B}{\B}$, за всяко $a \in \A$,
  следователно $\lfp(g_a)$ съществува.
  Да разгледаме изображението $h:\A \to \B$, където
  \[h(a) \dff \lfp(g_a).\]
  Докажете, че $h$ е непрекъснато изображение.
\end{problem}


\begin{problem}
  Да разгледаме $f \in \Cont{\A}{\Cont{\B}{\C}}$.
  За произволно $a \in \A$, дефинираме изображението $g_a \in \Cont{\B}{\C}$, където
  \[g_a(b) \dff f(a).\]
  Да разгледаме изображението $h:\A\times \B \to \C$, където
  \[h(a,b) \dff g_a(b).\]
  Докажете, че $h$ е непрекъснато изображение.
\end{problem}

% \begin{problem}
%   Нека са дадени областите на Скот $\D$ и $\E$ и изображението 
%   \[\texttt{eval}: \Cont{\D}{\E} \times \D \to \E,\]
%   където 
%   \[\texttt{eval}(f,d) \dff f(d).\]
%   Докажете, че $\texttt{eval}$ е непрекъснато изображение.
% \end{problem}
% \ifhints
% \begin{hint}
%   Понеже $\bigsqcup_n(f_n,d_n) = (\bigsqcup_m f_m,\bigsqcup_n d_n)$, 
%   ще докажем, че \[\texttt{eval}(\bigsqcup_m f_m, \bigsqcup_n d_n) = \bigsqcup_n \texttt{eval}(f_n,d_n).\]
%   Знаем, че
%   \begin{align*}
%     \texttt{eval}(\bigsqcup_m f_m, \bigsqcup_n d_n) & = (\bigsqcup_m f_m)(\bigsqcup_n d_n) & (\mbox{от опр. на }\texttt{eval})\\
%     & = \bigsqcup_m (f_m(\bigsqcup_n d_n)) & (\mbox{от опр. на }\bigsqcup_mf_m)\\
%     & = \bigsqcup_m (\bigsqcup_n (f_m(d_n)) & (\mbox{всяка }f_m\mbox{ е непр.} )\\
%   \end{align*}
%   Нека да положим $e_{m,n} = f_m(d_n)$.
%   Лесно се съобразява, че
%   \[m \leq m^\prime\ \&\ n \leq n^\prime\ \Rightarrow\ e_{m,n} \sqsubseteq^\E e_{m^\prime,n^\prime}.\]
%   Така получаваме, че 
%   \begin{align*}
%     \texttt{eval}(\bigsqcup_m f_m, \bigsqcup_n d_n) & = \bigsqcup_m (\bigsqcup_n (f_m(d_n)) & (\mbox{от по-горе})\\
%     & = \bigsqcup_{m,n} e_{m,n} = \bigsqcup_{n} e_{n,n} & (\mbox{от \Th{double-chain}})\\
%     & = \bigsqcup_{n} f_n(d_n) & (\text{ от опр. на }e_{m,n})\\
%     & = \bigsqcup_n \texttt{eval}(f_n,d_n).
%   \end{align*}
% \end{hint}
% \fi

\begin{problem}
  Нека са дадени областите на Скот $\D$ и $\E$ и изображението 
  \[\texttt{eval}: \Cont{\D}{\E} \times \D \to \E,\]
  където 
  \[\texttt{eval}(f,d) \dff f(d).\]
  Докажете, че $\texttt{eval}$ е непрекъснато изображение.
\end{problem}
\ifhints
\begin{hint}
  Понеже $\bigsqcup_n(f_n,d_n) = (\bigsqcup_m f_m,\bigsqcup_n d_n)$, 
  ще докажем, че \[\texttt{eval}(\bigsqcup_m f_m, \bigsqcup_n d_n) = \bigsqcup_n \texttt{eval}(f_n,d_n).\]
  Знаем, че
  \begin{align*}
    \texttt{eval}(\bigsqcup_m f_m, \bigsqcup_n d_n) & = (\bigsqcup_m f_m)(\bigsqcup_n d_n) & \comment\text{от опр. на }\texttt{eval}\\
                                                    & = \bigsqcup_m (f_m(\bigsqcup_n d_n)) & \comment\text{от опр. на }\bigsqcup_mf_m\\
                                                    & = \bigsqcup_m (\bigsqcup_n (f_m(d_n)) & \comment\text{всяка }f_m\mbox{ е непр.}\\
  \end{align*}
  Нека да положим $e_{m,n} = f_m(d_n)$.
  Лесно се съобразява, че
  \[m \leq m^\prime\ \&\ n \leq n^\prime\ \Rightarrow\ e_{m,n} \sqsubseteq^\E e_{m^\prime,n^\prime}.\]
  Така получаваме, че 
  \begin{align*}
    \texttt{eval}(\bigsqcup_m f_m, \bigsqcup_n d_n) & = \bigsqcup_m (\bigsqcup_n (f_m(d_n)) & \comment\text{от по-горе}\\
    & = \bigsqcup_{m,n} e_{m,n} = \bigsqcup_{n} e_{n,n} & \comment\text{от \Th{double-chain}}\\
    & = \bigsqcup_{n} f_n(d_n) & \comment\text{ от опр. на }e_{m,n}\\
    & = \bigsqcup_n \texttt{eval}(f_n,d_n).
  \end{align*}
\end{hint}
\fi

%%% Local Variables:
%%% mode: latex
%%% TeX-master: "../sep"
%%% End:

\begin{problem}
  Нека изображението \[\texttt{comp}:(\Cont{\B}{\C} \times \Cont{\A}{\B}) \to \Cont{\A}{\C}\]
  е определено като 
  \[\texttt{comp}(g,f) \dff g\circ f.\]
  \marginpar{$(g \circ f)(a) = g(f(a))$}
  Докажете, че $\texttt{comp}$ е непрекъснато изображение.
\end{problem}
\ifhints
\begin{hint}
  \marginpar{\cite[стр. 124]{reynolds}}
  Трябва да докажем, че за всяка монотонно растяща редица $\{(g_n,f_n)\}_{n\in\Nat}$,
  \[\Gamma(\bigsqcup_n(g_n,f_n))(a) = \bigsqcup_n\Gamma(g_n,f_n)(a),\]
  за произволно $a \in A$.
  Да фиксираме едно $a\in A$ и да положим $g_n(f_k(a)) = e_{n,k}$.
  Лесно се вижда, че 
  \[n\leq n^\prime\ \&\ k \leq k^\prime\ \Rightarrow\ e_{n,k} \sqsubseteq e_{n^\prime,k^\prime}.\]
  Тогава:
  \begin{align*}
    \Gamma(\bigsqcup_n(g_n,f_n))(a) & = \Gamma(\bigsqcup_n g_n, \bigsqcup_k f_k)(a) & \\
                                    & = (\bigsqcup_n g_n)(\bigsqcup_k f_k(a)) & \comment\text{ по деф. на }\Gamma\\
                                    & = (\bigsqcup_n g_n)(\bigsqcup_k b_k) & \comment\text{ полагаме }b_k = f_k(a)\\
                                    & = \bigsqcup_k (\bigsqcup_n g_n)(b_k) & \comment\bigsqcup_n g_n\text{ е непр.}\\
                                    & = \bigsqcup_k(\bigsqcup_n g_n(b_k)) & \comment\text{ по деф. на }\bigsqcup_n g_n\\
                                    & = \bigsqcup_k (\bigsqcup_n g_n(f_k(a))) & \comment\text{ полагаме }e_{n,k} = g_n(f_k(a))\\
                                    & = \bigsqcup_k\bigsqcup_n e_{n,k} = \bigsqcup_n e_{n,n} & \comment\text{ от \Th{double-chain}}\\
                                    & = \bigsqcup_n g_n(f_n(a)) = \bigsqcup_n \Gamma(g_n, f_n)(a).
  \end{align*}
\end{hint}
\fi

\begin{remark}
  \marginpar{Когато на хаскел пишем $(.)$, означава, че операцията е инфиксна}
  В хаскел има операция композиция на функции.
  \begin{haskellcode}
ghci> :t (.)
(.) :: (b -> c) -> (a -> b) -> a -> c
  \end{haskellcode}
\end{remark}


%%% Local Variables:
%%% mode: latex
%%% TeX-master: "../sep"
%%% End:


% \begin{problem}
%   Нека изображението \[\texttt{comp}:(\Cont{\B}{\C} \times \Cont{\A}{\B}) \to \Cont{\A}{\C}\]
%   е определено като 
%   \[\texttt{comp}(g,f) \dff g\circ f.\]
%   \marginpar{$(g \circ f)(a) = g(f(a))$}
%   Докажете, че $\texttt{comp}$ е непрекъснато изображение.
% \end{problem}
% \ifhints
% \begin{hint}
%   \marginpar{\cite[стр. 124]{reynolds}}
%   Трябва да докажем, че за всяка монотонно растяща редица $\{(g_n,f_n)\}_{n\in\Nat}$,
%   \[\Gamma(\bigsqcup_n(g_n,f_n))(a) = \bigsqcup_n\Gamma(g_n,f_n)(a),\]
%   за произволно $a \in A$.
%   Да фиксираме едно $a\in A$ и да положим $g_n(f_k(a)) = e_{n,k}$.
%   Лесно се вижда, че 
%   \[n\leq n^\prime\ \&\ k \leq k^\prime\ \Rightarrow\ e_{n,k} \sqsubseteq e_{n^\prime,k^\prime}.\]
%   Тогава:
%   \begin{align*}
%     \Gamma(\bigsqcup_n(g_n,f_n))(a) & = \Gamma(\bigsqcup_n g_n, \bigsqcup_k f_k)(a) & \\
%     & = (\bigsqcup_n g_n)(\bigsqcup_k f_k(a)) & (\text{ по деф. на }\Gamma )\\
%     & = (\bigsqcup_n g_n)(\bigsqcup_k b_k) & (\text{ полагаме }b_k = f_k(a))\\
%     & = \bigsqcup_k (\bigsqcup_n g_n)(b_k) & (\bigsqcup_n g_n\text{ е непр.})\\
%     & = \bigsqcup_k(\bigsqcup_n g_n(b_k)) & (\text{ по деф. на }\bigsqcup_n g_n)\\
%     & = \bigsqcup_k (\bigsqcup_n g_n(f_k(a))) & (\text{ полагаме }e_{n,k} = g_n(f_k(a)))\\
%     & = \bigsqcup_k\bigsqcup_n e_{n,k} = \bigsqcup_n e_{n,n} & (\text{ от \Th{double-chain}})\\
%     & = \bigsqcup_n g_n(f_n(a)) = \bigsqcup_n \Gamma(g_n, f_n)(a).
%   \end{align*}
% \end{hint}
% \fi

% \begin{remark}
%   \marginpar{Когато пишем $(.)$ означава, че операцията е инфиксна}
%   В хаскел има операция композиция на функции.
%   \begin{haskellcode}
% ghci> :t (.)
% (.) :: (b -> c) -> (a -> b) -> a -> c
%   \end{haskellcode}
% \end{remark}

\begin{problem}
  \marginpar{\cite[стр. 131]{nikolova-soskova}}
  Нека $f \in \Cont{\A}{\B}$ и $g \in \Cont{\B}{\A}$.
  Докажете, че 
  \begin{itemize}
  \item 
    $\lfp(g \circ f) \sqsubseteq g(\lfp(f \circ g))$;
  \item
    $f(\lfp(g \circ f)) \sqsubseteq \lfp(f \circ g)$.
  \end{itemize}
  Оттук заключете, че 
  \[\lfp(g \circ f) = g(\lfp(f \circ g)) \text{ и }f(\lfp(g \circ f)) = \lfp(f \circ g).\]
\end{problem}


% \begin{problem}[Кантор-Шрьодер-Бернщайн]
%   \marginpar{\cite[стр. 639]{hanbook-cs}}
%   Нека имаме две инективни функции $f:A \to B$ и $g:B \to A$.
%   Тогава съществува биективна функция $h: A \to B$.  
% \end{problem}
% \begin{hint}
%   За множеството $B$, да дефинираме областта на Скот 
%   \[\D_B = (\Ps(B),\subseteq,\emptyset).\]
%   \begin{enumerate}[a)]
%   \item 
%     За дадените от условието инективни функции $f$ и $g$,
%     да разгледаме изображението $\Gamma:\D_B \to \D_B$ зададено като
%     \marginpar{Озн. $h(X) = \{h(x) \mid x\in X\}$, $h^{-1}(X) = \{z \mid h(z) \in X\}$}
%     \[\Gamma(X) = B\setminus f(A)\cup f(g(X)).\]
%     Докажете, че $F$ е непрекъснато изображение.
%   \item
%     \Stefan{Използвам, че $X_0$ е неподвижна точка, но не виждам къде използвам, че е най-малката.}
%     Нека $X_0 = \lfp(\Gamma)$. Тогава $X_0 = B\setminus f(A) \cup f(g(X_0))$.
%     Докажете, че 
%     \[B \setminus X_0 = f(A \setminus g(X_0)).\]
%   \item
%     Дефинираме функцията $h:A \to B$ по следния начин:
%     \begin{align*}
%       h(a) = 
%       \begin{cases}
%         g^{-1}(a), & a \in g(X_0)\\
%         f(a), & a \in A \setminus g(X_0).
%       \end{cases}
%     \end{align*}
%     Докажете, че $h$ е биекция.
%   \end{enumerate}
% \end{hint}

\begin{problem}% Gunter textbook
  Нека $f \in \Cont{\A}{\A}$.
  Да разгледаме множеството 
  \[B = \{a \in \A \mid f(a) = a\}.\]
  Вярно ли е, че 
  \[\B = (B, \sqsubseteq^\A, \lfp(f))\] е област на Скот?
  Обосновете се!
\end{problem}

\begin{problem}% Gunter textbook
  Нека $f \in \Cont{\A}{\A}$.
  Да разгледаме множеството 
  \[B = \{a \in \A \mid f(a) \sqsubseteq a\}.\]
  Вярно ли е, че 
  \[\B = (B, \sqsubseteq^\A, \lfp(f))\] е област на Скот?
  Обосновете се!
\end{problem}


\begin{problem} % Gunter textbook
  Да разгледаме множеството
  \[B = \{f \in \Mon{\A}{\A} \mid f\circ f = f\}.\]
  Вярно ли е, че 
  \[\B = (B,\ \sqsubseteq,\ \lambda x.\bot^\A)\] е област на Скот,
  където 
  \[f \sqsubseteq g \dfff (\forall a\in\A)[f(a) \sqsubseteq^\A g(a)] ?\]
  Обосновете се!
\end{problem}

% \begin{problem} % Gunter textbook
%   Да разгледаме множеството
%   \[B = \{f \in \Strict{\A}{\A} \mid f\circ f = f\}.\]
%   Вярно ли е, че 
%   \[\B = (B,\ \sqsubseteq,\ \lambda x.\bot^\A)\] е област на Скот,
%   където 
%   \[f \sqsubseteq g \dfff (\forall a\in\A)[f(a) \sqsubseteq^\A g(a)] ?\]
%   Обосновете се!
% \end{problem}

\begin{problem} % Gunter textbook
  Да разгледаме множеството
  \[B = \{f \in \Strict{\Nat_\bot}{\Nat_\bot} \mid f\circ f = f\}.\]
  Вярно ли е, че 
  \[\B = (B,\ \sqsubseteq,\ \lambda x.\bot)\] е област на Скот,
  където 
  \[f \sqsubseteq g \dfff (\forall a\in\Nat_\bot)[f(a) \sqsubseteq g(a)] ?\]
  Обосновете се!
\end{problem}

\begin{problem}
  % \marginpar{\cite[стр. 124]{reynolds}}
  Нека $f \in \Mon{\A}{\B}$ и $g \in \Mon{\B}{\A}$ имат свойствата:
  \begin{itemize}
  \item 
    $f\circ g = id_\B$;
  \item
    $g \circ f = id_\A$.
  \end{itemize}
  Докажете, че $f$ и $g$ са точни и непрекъснати.
\end{problem}


\begin{problem}
  % \marginpar{задачата е \href{http://www.cl.cam.ac.uk/teaching/exams/pastpapers/y2008p8q14.pdf}{оттук} и \href{http://www.cl.cam.ac.uk/teaching/exams/pastpapers/y1998p9q10.pdf}{оттук}}
  Да разгледаме областта на Скот 
  \[\O = (\{\bot,\top\},\sqsubseteq, \bot),\]
  където $\bot \sqsubseteq \top$.
  За произволна област на Скот $\A$ и елемент $a \in \A$, $a \neq \bot$, дефинираме изображенията:
  \begin{enumerate}[a)]
  \item
    $f_a:\A \to \O$, където
    \[f_a(x) \dff
    \begin{cases}
      \top, & a \sqsubseteq x\\
      \bot, & a \not\sqsubseteq x.
    \end{cases}\]
    Вярно ли е, че $f_a$ е точно непрекъснато изображение? Обосновете се!
  \item
    $\hat{f}_a:\A \to \O$, където
    \[\hat{f}_a(x) \dff
    \begin{cases}
      \bot, & a \sqsubseteq x\\
      \top, & a \not\sqsubseteq x.
    \end{cases}\]
    Вярно ли е, че $\hat{f}_a$ е точно непрекъснато изображение? Обосновете се!
  \item 
    $g_a:\A \to \O$, където
    \[g_a(x) \dff
    \begin{cases}
      \bot, & x \sqsubseteq a\\
      \top, & x \not\sqsubseteq a.
    \end{cases}\]
    Вярно ли е, че $g_a$ е точно непрекъснато изображение? Обосновете се!
  \item 
    $\hat{g}_a:\A \to \O$, където
    \[\hat{g}_a(x) \dff
    \begin{cases}
      \top, & x \sqsubseteq a\\
      \bot, & x \not\sqsubseteq a.
    \end{cases}\]
    Вярно ли е, че $\hat{g}_a$ е точно непрекъснато изображение? Обосновете се!
  \item
    Докажете, че 
    \[f \in \Cont{\D}{\A} \iff (\forall a \in \A)[g_a \circ f \in \Cont{\D}{\O}].\]
  \end{enumerate}
\end{problem}

\begin{problem}
  Да разгледаме изображението
  \[\Gamma: \Cont{\A}{\B} \times \Cont{\A}{\C} \to \Cont{\A}{\B\times\C},\]
  където $\Gamma(f,g)(a) \dff \pair{f(a),g(b)}$.
  \begin{itemize}
  \item
    Докажете, че $\Gamma$ е добре дефинирано изображение, т.е. за всеки непрекъснати $f$ и $g$,
    $\Gamma(f,g)$ е непрекъснато изображение.
  \item 
    Докажете, че $\Gamma$ е непрекъснато изображение.
  \end{itemize}
\end{problem}

\begin{problem}
  \marginpar{задачата е \href{http://www.cl.cam.ac.uk/teaching/exams/pastpapers/y2005p9q15.pdf}{оттук}}
  Докажете, че изображението
  \[\texttt{uncurry}:\Cont{\A}{\Cont{\B}{\C}} \to \Cont{\A\times \B}{\C},\]
  дефинирано като
  \[\texttt{uncurry}(f)(a,b) \dff f(a)(b),\]
  е непрекъснато.
\end{problem}

% \begin{problem}
%   Докажете, че изображението
%   \[\texttt{curry}:\Cont{\A\times \B}{\C} \to \Cont{\A}{\Cont{\B}{\C}},\]
%   дефинирано като
%   \[\texttt{curry}(f)(a)(b) \dff f(a,b),\]
%   е непрекъснато.
% \end{problem}

\begin{problem}
  Докажете, че изображението
  \[\texttt{curry}:\Cont{\A\times \B}{\C} \to \Cont{\A}{\Cont{\B}{\C}},\]
  дефинирано като
  \[\texttt{curry}(f)(a)(b) \dff f(a,b),\]
  е непрекъснато.
\end{problem}

%%% Local Variables:
%%% mode: latex
%%% TeX-master: "../sep"
%%% End:


\newpage
\subsection{Регулярни езици}

Да фиксираме азбуката $\Sigma = \{a_1,\dots,a_k\}$.
Да дефинираме полиномите над $\Sigma$ като
\[\tau ::= \emptyset\ |\ \varepsilon\ |\ a_i \cdot X_j\ |\ \tau_1 + \tau_2.\]
където $i = 1, \dots,k$, а $X$ е променлива.
За всеки полином $\tau[X_1,\dots,X_n]$ дефинираме оператора 
\[\val{\tau}: \mathcal{P}(\Sigma^\star)^n \to \mathcal{P}(\Sigma^\star)\]
 по следния начин:
\begin{itemize}
\item
    $\val{\emptyset}(L_1,\dots,L_n) = \emptyset$.
\item 
  $\val{\varepsilon}(L_1,\dots,L_n) = \varepsilon$.
\item 
  $\val{a_i \cdot X_j}(L_1,\dots,L_n) = \{a_i\} \cdot L_j$.
\item
  $\val{\tau_1 + \tau_2}(L_1,\dots,L_n) = \val{\tau_1}(L_1,\dots,L_n) \cup \val{\tau_2}(L_1,\dots,L_n)$.
\end{itemize}

\begin{problem}
  Докажете, че за всеки полином $\tau$ имаме, че $\val{\tau}$ е непрекъснато изображение в областта на Скот
  $\mathcal{S} = ( \mathcal{P}(\Sigma^\star),\subseteq, \emptyset)$.
\end{problem}


\begin{example}
  Да разгледаме системата 
  \marginpar{$\tau_1[X_1,X_2] \equiv b \cdot X_1 + a \cdot X_2$}
  \marginpar{$\tau_2[X_1,X_2] \equiv \varepsilon$}
  \begin{align*}
    & X_1 = b \cdot X_1 + a\cdot X_2\\
    & X_2 = \varepsilon.
  \end{align*}

  % Понеже $\val{\tau}$ е непрекъснат оператор, то той има най-малка неподвижна точка.
  Дефинираме непрекъснатия оператор 
  \[\Gamma:\mathcal{P}(\Sigma^\star)^2 \to \mathcal{P}(\Sigma^\star)^2,\]
  където:
  \[\Gamma(L_1,L_2) = (\val{\tau_1}(L_1,L_2), \val{\tau_2}(L_1,L_2)).\]

  От Теоремата на Клини ние знам как можем да намерим най-малката неподвижна точка на $\Gamma$,
  която ще бъде и най-малкото решение на горната система.

  \begin{itemize}
  \item 
    $(L_0,M_0) \df (\emptyset,\emptyset)$;
  \item
    $(L_1,M_1) \df \Gamma(L_0,M_0) = (\val{\tau_1}(L_0,M_0), \val{\tau_2}(L_0,M_0)) = (\emptyset, \varepsilon)$;
  \item
    $(L_2,M_2) \df \Gamma(L_1,M_1) = (\val{\tau_1}(L_1,M_1), \val{\tau_2}(L_1,M_1)) = (\{a\},\varepsilon)$;
  \item
    $(L_3,M_3) \df \Gamma(L_2,M_2) = (\val{\tau_1}(L_2,M_2), \val{\tau_2}(L_2,M_2)) = (\{ba,a\},\varepsilon)$;
  \item
    $(L_4,M_4) \df \Gamma(L_3,M_3) =(\val{\tau_1}(L_3,M_3), \val{\tau_2}(L_3,M_3)) = (\{bba, ba,a\},\varepsilon)$;
  \item
    $(L_5,M_5) \df \Gamma(L_4,M_4) = ( \val{\tau_1}(L_4,M_4), \val{\tau_2}(L_4,M_4)) = (\{bba, bba, ba,a\},\varepsilon)$.
  \end{itemize}
  Лесно се съобразява, че $L_n = \{ b^ka \mid k < n\}$.
  Тогава
  \[\lfp( \Gamma ) = (\bigcup_n L_n, \{\varepsilon\}) = (b^\star a, \{\varepsilon\} ).\]
\end{example}


\begin{problem}
  Докажете, че най-малкото решение на системата 
  \begin{align*}
    & X_1 = a \cdot X_1 + b \cdot X_2 + \varepsilon\\
    & X_2 = b \cdot X_2 + \varepsilon
  \end{align*}
  е двойката $(a^\star b^\star, b^\star)$.
\end{problem}

\begin{problem}
  Да разгледаме системата от оператори
  \begin{align*}
    & \val{\tau_1}(L_1,\dots,L_n) = L_1\\
    & \ \ \vdots\\
    & \val{\tau_n}(L_1,\dots,L_n) = L_n.
  \end{align*}
  Знаем, че тя притежава най-малко решение $(\hat{L}_1,\dots,\hat{L}_n)$.
  Докажете, че всеки от езиците $\hat{L}_i$ е регулярен.

  Докажете, че всеки регулярен език е елемент от най-малкото решение 
  на някоя система от оператори от горния вид.
\end{problem}


% \begin{problem}
%   Докажете, че всеки регулярен език е елемент на най-малкото решение на някоя система от $n$
%   полинома с $n$ променливи за някое $n$.
% \end{problem}

% \begin{problem}
%   Докажете, че всяко най-малко решение на система от $n$ полинома с $n$ променливи представлява $n$-орка от 
%   регулярни езици.
% \end{problem}

% \begin{problem}
%   Опишете алгоритъм, по който може от система от $n$ полинома с $n$ променливи да се построи 
%   краен автомат с $n$ състояния.
% \end{problem}

% \begin{problem}
%   Опишете алгоритъм, по който може от краен автомат с $n$ състояния може да се построи 
% \end{problem}



\subsection{Безконтекстни езици}

Да фиксираме азбуката $\Sigma = \{a_1,\dots,a_n\}$.
Да дефинираме термове от тип 1 като
\[\tau ::= X_i\ |\ a_j\ |\ \varepsilon\ |\ \emptyset\ |\ \tau_1 \cdot \tau_2\ |\ (\tau_1 + \tau_2),\]
където $j = 1, \dots,n$, а $X_i$ са изброимо безкрайна редица от променливи.
За всеки терм $\tau[X_1,\dots,X_n]$ дефинираме оператора 
\[\val{\tau}: (\mathcal{P}(\Sigma^\star))^n \to \mathcal{P}(\Sigma^\star)\]
 по следния начин:
\begin{itemize}
\item 
  $\val{X_i}(L_1,\dots,L_n) = L_i$.
\item 
  $\val{a_j}(L_1,\dots,L_n) = \{a_j\}$.
\item 
  $\val{\varepsilon}(L_1,\dots,L_n) = \varepsilon$.
\item 
  $\val{\emptyset}(L_1,\dots,L_n) = \emptyset$.
\item 
  $\val{\tau_1 \cdot \tau_2}(L_1,\dots,L_n) = \val{\tau_1}(L_1,\dots,L_n) \cdot \val{\tau_2}(L_1,\dots,L_n)$.
\item
  $\val{\tau_1 + \tau_2}(L_1,\dots,L_n) = \val{\tau_1}(L_1,\dots,L_n) \cup \val{\tau_2}(L_1,\dots,L_n)$.
\end{itemize}

\begin{problem}
  Докажете, че за всеки терм $\tau$, $\val{\tau}$ е непрекъснато изображение в областта на Скот
  $\mathcal{S} = ( \mathcal{P}(\Sigma^\star),\subseteq, \emptyset)$.
\end{problem}

\begin{problem}
  Докажете, че $\{a^nb^n \mid n\in \Nat\} = \lfp(\val{\tau})$, където 
  \[\tau[X] \equiv \varepsilon + a \cdot X \cdot b.\]
  С други думи, $\{a^nb^n \mid n \in \Nat\}$ е най-малкото решение на уравнението
  \[X = a \cdot X \cdot b + \varepsilon.\]
\end{problem}

Нека сега да разгледаме термовете $\tau_1[X_1,\dots,X_n], \dots, \tau_n[X_1,\dots,X_n]$.

\begin{problem}
  Да разгледаме системата от оператори
  \begin{align*}
    & \val{\tau_1}(L_1,\dots,L_n) = L_1\\
    & \ \ \vdots\\
    & \val{\tau_n}(L_1,\dots,L_n) = L_n.
  \end{align*}
  Знаем, че тя притежава най-малко решение $(\hat{L}_1,\dots,\hat{L}_n)$.
  Докажете, че всеки от езиците $\hat{L}_i$ е безконтекстен.

  Докажете, че всеки безконтекстен език е елемент от най-малкото решение 
  на някоя система от оператори от горния вид.
\end{problem}

\begin{problem}
  \marginpar{Това е аналог на нормалната форма на Чомски}
  Да дефинираме термове от тип 2 като
  \[\tau ::= a_j\ |\ \varepsilon\ |\ \emptyset\ |\ X_i \cdot X_k\ |\ (\tau_1 + \tau_2),\]
  където $j = 1, \dots,n$, а $X_i$ са изброимо безкрайна редица от променливи.
  Докажете горното твърдение, като замените термовете от тип 1 с тези от тип 2.
\end{problem}

\begin{example}
  Да разгледаме системата
  \begin{align*}
    & X_1 = X_3 \cdot X_2 + \varepsilon\\
    & X_2 = X_1 \cdot X_4\\
    & X_3 = a\\
    & X_4 = b.
  \end{align*}


  % \begin{align*}
  %   & \val{\varepsilon + X_3 \cdot X_2}(L_1, L_2, L_3, L_4) = L_1\\
  %   & \val{X_1 \cdot X_4}(L_1, L_2, L_3, L_4) = L_2\\
  %   & \val{a}(L_1, L_2, L_3, L_4) = L_3\\
  %   & \val{b}(L_1, L_2, L_3, L_4) = L_4\\
  % \end{align*}
  Нека $(\hat{L}_1, \hat{L}_2, \hat{L}_3, \hat{L}_4)$ е най-малкото решение на системата.
  Докажете, че $\hat{L}_1 = \{a^nb^n\mid n \in \Nat\}$ $\hat{L}_2 = \{a^nb^{n+1}\mid n \in \Nat\}$,
  $\hat{L}_3 = \{a\}$ и $\hat{L}_4 = \{b\}$.
\end{example}



%%% Local Variables:
%%% mode: latex
%%% TeX-master: "../sep"
%%% End:




% \subsection{Правило на Скот}

% \begin{itemize}
% \item 
%   Нека приемем, че имаме една програма, която работи върху списъци.
%   Ако искаме да докажем, че дадено свойство е вярно за тази програма, когато аргументите са {\em крайни} списъци,
%   то можем да подходим с обикновена индукция по структурата на списъците.
% \item
%   Трябва да сме по-внимателни, ако искаме да докажем, че дадено свойство е вярно за програмата, когато
%   допускаме като аргументи прозволни списъци, т.е. позволяваме частични и безкрайни списъци.
%   Понеже един безкраен списък $l = \bigsqcup_n l_n$, където $l_n $
% \item
%   Да разгледаме едно {\em непрекъснато свойство} $P \subseteq \PartL \cup \InfL$.
%   Тогава имаме правилото:
%   \begin{prooftree}
%     \AxiomC{$P(\bot)$}
%     \AxiomC{$(\forall a\in\Nat)(\forall l \in \PartL)[P(l) \implies P(\pair{a,l})]$}
%     \BinaryInfC{$(\forall l \in \PartL \cup \InfL)[P(l)]$}
%   \end{prooftree}
  
%   Да видим защо това правило е изпълнено.
%   Да разгледаме един произволен безкраен списък $l$.
%   Знаем, че $l = \bigsqcup_n (l\upharpoonright n)$.
%   Знаем, че $P(l \upharpoonright 0)$ и с индукция по естествените числа имаме, че за всяко $n$,
%   $P(l \upharpoonright n)$.
%   Тогава, понеже $P$ е непрекъснато свойство, следва 
%   \[P(\bigsqcup_n (l \upharpoonright n)).\]
% \end{itemize}

% \begin{problem}
%   % Нека е дадена програмата на
  
%   % \begin{haskellcode}
%   %   rev :: [a] -> [a]
%   %   rev x = f(x, []) where 
%   %     f([], y) = y
%   %     f(x:xs, y) = f(xs, x:y)
%   % \end{haskellcode}

%   Да разгледаме оператора:

%   \[
%   \Gamma(f)(l_1,l_2) = 
%   \begin{cases}
%     \bot, & l_1 = \bot\\
%     l_2, & l_1 = \nil\\
%     f(l'_1, \pair{a,l_2}), & l_1 = \pair{a,l'_1}.
%   \end{cases}
%   \]

%   Докажете, че $\Gamma$ е непрекъснат оператор.
%   Да положим 
%   \[\texttt{rev}(l) = \lfp(\Gamma)(l,\nil).\]
%   Докажете, че:
%   \begin{enumerate}[a)]
%   \item 
%     $(\forall l \in L)[\texttt{rev}(l) \neq \bot \iff l \in \FinL]$.
%   \item
%     $(\forall l \in \FinL)[\texttt{rev}(\texttt{rev}(l)) = l]$.
%   \item
%     $(\forall l \in \FinL)[\texttt{rev}(l) = l^R]$.
%   \item
%     $(\forall l \in \PartL \cup \InfL)[rev(l) = \bot]$.
%   \end{enumerate}
% \end{problem}
% \begin{hint}
%   % Ще използваме следното правило:
%   % \begin{prooftree}
%   %   \AxiomC{$P(\nil)$}
%   %   \AxiomC{$(\forall x \in \Sigma^\star)[x\neq\nil\ \&\ P(cdr(x)) \to P(x)]$}
%   %   \RightLabel{\scriptsize(1)}
%   %   \BinaryInfC{$(\forall x\in\Sigma^\star)[P(x)]$}
%   % \end{prooftree}
%   % % \item
%   %   Докажете валидността на правилото $(1)$. % е еквивалентно на структурна индукция върху фундираната наредба
%     % $(\Sigma^\star,\prec)$, където $x \prec y \iff (\exists z\in\Sigma^\star)[z\cdot x = y]$, т.е.
%     % $x$ е суфикс на $y$.
%   % Да разгледаме фундираната наредба $L = (\Sigma^\star, \prec)$, където
%   % $x \prec y \iff \abs{x} < \abs{y}$.
%   \begin{enumerate}[a)]
%   \item
%     Докажете, че $P$ е непрекъснато свойство, където
%     $P(f) \dfff (\forall l \in L)[f(l) \neq \bot \iff l \in \FinL]$.
    
%     % Да разгледаме свойството 
%     % \[P(x) \equiv (\forall y\in \Sigma^\star)[f(x,y)\text{ е дефинирана}].\]
%     % Докажете със структурна индукция по $L$, че $(\forall x\in\Sigma^\star)[P(x)]$.
%   \item
%     Да разгледаме свойството 
%     \[P(x) \equiv (\forall y\in \FinL)[\texttt{rev}(f(x,y)) = f(y,x)].\]
%     Докажете със структурна индукция по $L$, че $(\forall x\in\Sigma^\star)[P(x)]$.
%     Тогава в частния случай $y = \nil$, 
%     \[\texttt{rev}(\texttt{rev}(x)) = \texttt{rev}(f(x,\nil)) = f(\nil,x) = x.\]
%   \item
%     Разгледайте свойството
%     \[P(x) \equiv (\forall y\in L)[f(x,y) = x^R \cdot y].\]
%   \end{enumerate}
% \end{hint}

% \begin{problem}
%   Нека е дадена програмата на езика хаскел:

%   \begin{haskellcode}
% conc :: [Int] -> [Int] -> [Int]
% conc [] y = y
% conc (x:xs) y = x:conc xs y
%   \end{haskellcode}

% Да видим няколко примера:

% \begin{haskellcode}
% ghci> conc([1..10],undefined)  -- $ == \pair{1,2,3,4,5,6,7,8,9,10,\bot}$
% [1,2,3,4,5,6,7,8,9,10*** Exception: Prelude.undefined
% ghci> conc(undefined, [1..10]) -- $== \bot$
% *** Exception: Prelude.undefined
% \end{haskellcode}

%   На нея съответсва оператора

%   \[\Gamma(f)(x,y) = 
%   \begin{cases}
%     \bot, & x = \bot\\
%     y, & x = \nil\\
%     \pair{a, f(x',y)}, & x = \pair{a,x'}.
%   \end{cases}\]
  
%   Да положим $\texttt{conc}(x,y) \dff \lfp(\Gamma)(x,y)$.
  
%   \noindent Проверете дали са изпълнени свойствата:
%   \begin{enumerate}[a)]
%   \item
%     $(\forall x\in \FinL)(\forall y \in L)[\texttt{conc}(x, y) = x \cdot y]$;
%   \item
%     $(\forall x\in \PartL \cup \InfL)(\forall y \in L)[\texttt{conc}(x, y) = x]$;
%   \item 
%     $(\forall x,y,z\in L)[\texttt{conc}(\texttt{conc}(x, y), z) = \texttt{conc}(x, \texttt{conc}(y, z))]$;
%   \item
%     $(\forall x,y \in L)[\texttt{rev}(\texttt{conc}(x, y)) = \texttt{conc}(\texttt{rev}(y), \texttt{rev}(x))]$;
%   \end{enumerate}
% \end{problem}
% \begin{hint}
%   \begin{enumerate}[a)]
%   \item
%     Да разгледаме фундираната наредба $(\FinL, \prec)$, където
%     $x \prec y \iff \abs{x} < \abs{y}$.
%     Разгледайте свойството
%     \[P(x) \equiv (\forall y\in L)[\texttt{conc}(x, y) = x \cdot y].\]
%   \item
%     Разгледайте свойствотото:
%     \[P(x) \equiv (\forall y \in L)[\texttt{conc}(x, y) = x].\]
%     Съобразете, че това е непрекъснато свойство.
%     Докажете, че $P(\bot)$ и че $(\forall a \in \Nat)(\forall l \in \PartL \cup \InfL)[P(l) \implies P(\pair{a,l})]$.
%     Оттук, по правилото на Скот следва, че $P(l)$ за всяко $l \in \PartL \cup \InfL$.
%   \end{enumerate}
% \end{hint}

% \begin{problem}
%   На следната програмата на хаскел:
%   \begin{haskellcode}
% merge :: [Int] -> [Int] -> [Int]
% merge []     y       = y
% merge x      []      = x
% merge (x:xs) (y:ys)  = if x <= y then x:merge xs (y:ys)
%                          else y:merge (x:xs) ys
%   \end{haskellcode}
% отговаря операторът
% \[
% \Gamma(f)(x,y) =
% \begin{cases}
%   \bot, & x = \bot\ \vee\ y = \bot\\
%   y, & x = \nil \\
%   x, & y = \nil \\
%   \pair{a,f(x',y)}, & x = \pair{a,x'}\ \&\ y = \pair{b,y'}\ \&\ a \leq b\\
%   \pair{b,f(x,y')}, & x = \pair{a,x'}\ \&\ y = \pair{b,y'}\ \&\ a > b
% \end{cases}
% \]

% \begin{enumerate}[a)]
% \item 
%   Вярно ли е, че
%   \[(\forall x,y,z \in L)[\texttt{merge}(x,\texttt{merge}(y,z)) = \texttt{merge}(\texttt{merge}(x,y),z)]?\]
% \end{enumerate}
% \end{problem}


% \begin{problem}
%   Да разгледаме следната програма:
  
%   \begin{haskellcode}
% map :: (Int -> Int) -> [Int] -> [Int]
% map f []     = []
% map f (x:xs) = (f x):map f xs
%   \end{haskellcode}

%   Да разгледаме оператора
%   \[\Gamma(g)(f,l) = 
%   \begin{cases}
%     \bot, & l = \bot\\
%     \nil, & l = \nil\\
%     \pair{f(a),g(f,l')}, & l = \pair{a,l'}.
%   \end{cases}\]
%   Лесно се вижда, че този оператор е непрекъснат.
%   Нека $\texttt{map} \dff \lfp(\Gamma)$.

%   \begin{enumerate}[a)]
%   \item 
%     $\texttt{map}(f,\texttt{map}(g, l)) = \texttt{map}(f \circ g, l)$.
%   \end{enumerate}
% \end{problem}

% \begin{problem}
%   Да разгледаме следната програма:
  
%   \begin{haskellcode}
% foldr :: (Int -> Int -> Int) -> Int -> [Int] -> [Int]
% foldr f e []     = e
% foldr f e (x:xs) = f x (foldr f e xs)
%   \end{haskellcode}

%   Да разгледаме оператора
%   \[\Gamma(g)(f,z,l) = 
%   \begin{cases}
%     \bot, & l = \bot\\
%     z, & l = \nil\\
%     f(x, g(f,z,l')), & l = \pair{a,l'}.
%   \end{cases}\]
%   Лесно се вижда, че този оператор е непрекъснат.
%   Нека $\texttt{foldr} \dff \lfp(\Gamma)$.

%   \begin{enumerate}[a)]
%   \item 
%     Нека имаме свойствата 
%     \begin{itemize}
%     \item 
%       $f(\bot) = \bot$
%     \item
%       $f(g(x,y)) = h(x, f(y))$      
%     \end{itemize}
%     Тогава
%     $f(\texttt{foldr}(g, a, l)) = \texttt{foldr}(h, f(a), l)$
%   \end{enumerate}
% \end{problem}


% \begin{problem}
%   Да разгледаме следната програма:
  
%   \begin{haskellcode}
% foldl :: (Int -> Int -> Int) -> Int -> [Int] -> [Int]
% foldl f e []     = e
% foldl f e (x:xs) = foldl (f e x) xs
%   \end{haskellcode}

%   Да разгледаме оператора
%   \[\Gamma(g)(f,e,l) = 
%   \begin{cases}
%     \bot, & l = \bot\\
%     \nil, & l = \nil\\
%     g(f, f(e,a), l'), & l = \pair{a,l'}.
%   \end{cases}\]
%   Лесно се вижда, че този оператор е непрекъснат.
%   Нека $\texttt{foldl} \dff \lfp(\Gamma)$.
% \end{problem}

% \begin{problem}
%   Вярно ли е, че следните две програмите $\texttt{evens}$ и $\texttt{evens'}$ дефинират една и съща функция ?
%   \begin{haskellcode}
%     evens []       = []
%     evens [x]      = []
%     evens (_:x:xs) = x:evens xs
    
%     evens' xs = f xs [] where
%     f [] ys       = ys
%     f [x] ys      = ys
%     f (_:x:xs) ys = f xs (x:ys)  
%   \end{haskellcode}
% \end{problem}


% \subsection{Точни и неточни изображения}

% Една функция $f(x,y)$ е точна, ако $f(x,\bot) = \bot$ и $f(\bot,y) = \bot$,
% за произволни $x$ и $y$.

% \begin{haskellcode}
% -- Да видим, че стандартната операцията дизюнкция в хаскел е почти неточна
% ghci> :t (||)
% (||) :: Bool -> Bool -> Bool
% ghci> True || undefined
% True
% ghci> False || undefined
% *** Exception: Prelude.undefined
% ghci> undefined || True
% *** Exception: Prelude.undefined
% ghci> foldr (||) False [False, True, undefined, True]
% True
% ghci> foldr (||) False [False, undefined, True]
% *** Exception: Prelude.undefined
% ghci> :set +s
% ghci> foldr (||) False (map (>0) [-999999..])
% True
% (0.18 secs, 188,851,600 bytes)
% \end{haskellcode}

% Да разгледаме следната програма:
% \begin{haskellcode}
% mult x y = if x == 0 then 0
%           else mult (x-1) y + y
% \end{haskellcode}

% На тази програма съответства операторът:
% \begin{align*}
%   \Gamma(f)(x,y) = 
%   \begin{cases}
%     0, & x = 0\ \&\ y \in \Nat_\bot\\
%     f(x-1,y)+y, & x > 0\ \&\ y \in \Nat_\bot\\
%     \bot, & x = \bot\ \&\ y \in \Nat_\bot.
%   \end{cases}
% \end{align*}

% \marginpar{Тук използваме, че $\bot + x = x + \bot = \bot$}

% Тогава най-малката неподвижна точка на $\Gamma$, или еквивалентно, 
% семантиката на програмата $\texttt{mult}$ е функцията $f_\Gamma$, където
% \[f_\Gamma(x,y) =
% \begin{cases}
%   0, & x = 0\ \&\ y = \bot\\
%   x \cdot y, & x \in \Nat\ \& \ y \in \Nat\\
%   \bot, & \text{иначе}
% \end{cases}\]
% \marginpar{Това не е напълно коректно, защото на хаскел няма тип на естествените числа, а само на целите}
% Това означава, че $f_\Gamma$ е точна по първия си аргумент и неточна по втория си аргумент.

% \begin{haskellcode}
% ghci> mult 0 undefined
% 0
% ghci> mult undefined 0
% *** Exception: Prelude.undefined
% ghci> :set +s
% ghci> foldr mult 1 [0..]
% 0
% (0.02 secs, 8,959,824 bytes)
% ghci> foldr (*) 1 [0..]
% Interrupted. -- забива
% ghci> foldr mult 1 [1000,999..] -- [1000,999..] == [1000,999,998,997,..]
% 0
% (0.65 secs, 230,882,664 bytes)
% \end{haskellcode}

% Нека сега да разгледаме точен вариант на дизюнкцията.

% \begin{haskellcode}
% or' :: Bool -> Bool -> Bool
% or' !x !y = if x then x
%             else if y then y
%                  else False
% \end{haskellcode}

% Да разгледаме сега пак примера отгоре.
% \begin{haskellcode}
% ghci> True `or'` undefined
% undefined
% ghci> False `or'` undefined
% *** Exception: Prelude.undefined
% ghci> undefined `or'` True
% *** Exception: Prelude.undefined
% ghci> foldr or' False [False, True, undefined, True]
% *** Exception: Prelude.undefined
% ghci> foldr or' False [False, undefined, True]
% *** Exception: Prelude.undefined
% ghci> foldr or' False (map (>0) [-9999..])
% C-c C-cInterrupted.
% \end{haskellcode}

% \begin{example}
%   Защо не можем да използваме операция $+$, за която $x + \bot = x$?
%   Да разгледаме изображението $g:\Nat^2_\bot \to \Nat_\bot$, където
%   \[g(x,y) = 
%   \begin{cases}
%     x, & x \in \Nat_\bot\ \&\ y = \bot\\
%     x + y, & x\in \Nat\ \&\ y \in \Nat\\
%     \bot, & x = \bot\ \&\ y \in \Nat.
%   \end{cases}\]
%   Лесно се вижда, че $g$ не е монотонно, откъдето следва, че $g$ не е непрекъснато.
%   Например, $\pair{1,\bot} \sqsubseteq \pair{1,1}$, 
%   но $g(1,\bot) = 1 \not\sqsubseteq 2 = g(1,1)$.
%   Това означава, че е безсмислено да разглеждаме тази версия събирането, защото
%   ние се интересуваме само от непрекъснати изображения.
% \end{example}


% \section*{Бележки}

% \begin{itemize}
% \item
%   Много от теоретичните задачи мога да се намерят в \cite[Глава 2]{domains-book}.
% \item
%   На \cite[стр. 64]{bird-haskell} са дефинирани крайни, частични и безкрайни списъци в хаскел.
% \item
%   \cite[Глава 9]{bird-haskell} е посветена на безкрайни списъци.
% \item
%   Срещаната в литературата дефиниция на алгебрична област на Скот е по-обща от тази,
%   която ние разглеждаме. Тук се ограничаваме само до точни горни граници на вериги, а по-общата дефиниция е да се 
%   разглеждат точни горни граници на насочени множества.
% \item 
%   Ние се интересуваме основно само от алгебрични области на Скот.
% \item
%   Ситуацията с ленивите списъци в хаскел е малко по-сложна, защото примерно можем да 
%   разглеждаме списъци от вида $\pair{\bot,\bot,\bot,\nil}$.

%   \begin{haskellcode}
% ghci> length [undefined,undefined,undefined]
% 3
%   \end{haskellcode}
% \end{itemize}

% \begin{haskellcode}
% data LazyList a = Nil | Cons a (LazyList a) deriving (Show)

% data EagerList a = Nil' | Cons' !a !(EagerList a) deriving (Show)

% -- Еквиваленти на функцията take от Prelude

% grab 0 _ = Nil
% grab n Nil = Nil
% grab n (Cons x y) = Cons x (grab (n-1) y)

% grab' 0 _ = Nil'
% grab' n Nil' = Nil'
% grab' n (Cons' x y) = Cons' x (grab' (n-1) y)

% -- Безкрайни списъци започващи от x

% inf x = Cons x (inf (x+1))

% inf' x = Cons' x (inf' (x+1))

% h n = grab n (inf 0)

% h' n = grab' n (inf' 0)

% -- Какъв резултата от h inf ?
% -- Какъв резултата от h' inf' ?

% len Nil = 0
% len (Cons _ tail) = 1 + (len tail)

% len' Nil' = 0
% len' (Cons' _ tail) = 1 + (len' tail)

% x = Cons undefined (Cons undefined (Cons undefined Nil))
% x' = Cons' undefined (Cons' undefined (Cons' undefined Nil'))

% -- Какъв е резултата от len x ?
% -- Какъв е резултата от len' x' ?

% \end{haskellcode}



%%% Local Variables:
%%% mode: latex
%%% TeX-master: "../sep-notes"
%%% End:
