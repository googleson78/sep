\subsection{Сравнение между двете семантики}


\begin{example}
  Да разгледаме програмата $\vv{P}$ на езика $\REC$:
  \begin{haskellcode}
f(x) = g(f(x)) where
  g(x) = 0  
  \end{haskellcode}

Да проверим, че
$\D_V\val{\vv{P}} \neq \D_N\val{\vv{P}}$.
Имаме, че
\begin{align*}
  \tau_1[\vv{x},\vv{f},\vv{g}] & \equiv \vv{g}(\vv{f}(\vv{x}))\\
  \tau_2[\vv{x},\vv{f},\vv{g}] & \equiv \vv{0}.
\end{align*}
За $i = 1,2$ дефинираме операторите 
\[\Delta_{\tau_i} \in \Cont{\Strict{\Nat_\bot}{\Nat_\bot} \times \Strict{\Nat_\bot}{\Nat_\bot}}{\Strict{\Nat_\bot}{\Nat_\bot}},\]
  които имат следните дефиниции:  
  \begin{align*}
    \Delta_{\tau_1}(\varphi,\psi)(x) & \dff
    \begin{cases}
      \psi(\varphi(x)), & \text{ако }x \neq \bot\\
      \bot, & \text{ако }x = \bot\\
    \end{cases}\\
    \Delta_{\tau_2}(\varphi,\psi)(x) & \dff
    \begin{cases}
      0, & \text{ако }x \neq \bot,\\
      \bot, & \text{ако }x = \bot.
    \end{cases}
  \end{align*}
  Нека $\Delta \dff \Delta_{\tau_1} \times \Delta_{\tau_2}$. За да намерим най-малкото
  решение на системата, трябва да намерим най-малката неподвижна точка на $\Delta$.
  Ще получим това най-малко решение като точна горна граница на редица от двойки функции $(\varphi_i, \psi_i)$.
  Имаме, че $\varphi_0(x) \dff \lambda x.\bot$ и $\psi_0(x) \dff \lambda x.\bot$.
  \begin{align*}
    \varphi_1(x) & \dff \Delta_{\tau_1}(\varphi_0,\psi_0)(x)\\
                 & =
                   \begin{cases}
                     \psi_0(\varphi_0(x)), & \text{ако } x \neq \bot\\
                     \bot, & \text{ако }x = \bot
                   \end{cases}\\
                 & = \bot\\
    \psi_1(x) & \dff \Delta_{\tau_2}(\varphi_0,\psi_0)(x)\\
                 & = 
                   \begin{cases}
                     0, & \text{ако }x \neq \bot\\
                     \bot, & \text{ако }x = \bot\\
                   \end{cases}
  \end{align*}
  Това означава, че $\Delta(\varphi_0,\psi_0) = (\varphi_1,\psi_1)$.
  При следващата итерация получаваме следното:
  \begin{align*}
    \varphi_2(x) & \dff \Delta_{\tau_1}(\varphi_1,\psi_1)(x)\\
                 & =
                   \begin{cases}
                     \psi_1(\varphi_1(x)) & \text{ако } x \neq \bot\\
                     \bot, & \text{ако } x = \bot\\
                   \end{cases}\\
                 & = 
                   \begin{cases}
                     \psi_1(\bot) & \text{ако } x \neq \bot\\
                     \bot, & \text{ако } x = \bot\\
                   \end{cases}\\
                 & = \bot\\
    \psi_2(x) & \dff \Delta_{\tau_2}(\varphi_1,\psi_1)(x)\\
                 & = \begin{cases}
                   0, & \text{ако }x \neq \bot\\
                   \bot, & \text{ако }x = \bot\\
                 \end{cases}\\
                 & = \psi_1(x).
  \end{align*}
  Получихме, че $\Delta(\varphi_1,\psi_1) = (\varphi_1,\psi_1)$ и
  следователно $\lfp(\Delta) = (\varphi_1,\psi_1)$.
  Тогава за всяко $a \in \Nat_\bot$,
  \[\D_V\val{\vv{P}}(a) = \varphi_1(a) = \bot.\]

  Сега да намерим семантика по име на горната програма.
За $i = 1,2$ дефинираме операторите 
  \[\Gamma_{\tau_i} \in \Cont{\Cont{\Nat_\bot}{\Nat_\bot} \times \Cont{\Nat_\bot}{\Nat_\bot}}{\Cont{\Nat_\bot}{\Nat_\bot}},\]
  които имат следните дефиниции
  \begin{align*}
    \Gamma_{\tau_1}(\varphi,\psi)(x) & \dff \psi(\varphi(x))\\
    \Gamma_{\tau_2}(\varphi,\psi)(x) & \dff 0.
  \end{align*}
  Търсим най-малката неподвижна точна на оператора 
  \[\Gamma(\varphi,\psi) \dff (\Gamma_{\tau_0}(\varphi, \psi), \Gamma_{\tau_1}(\varphi,\psi)).\]
  \marginpar{т.е. това са функциите, които винаги връщат $\bot$}
  Имаме, че $\varphi_0(x) \dff \lambda x.\bot$ и $\psi_0(x) \dff \lambda x.\bot$.
  Сега, за произволно $x \in \Nat_\bot$, получаваме:
  \begin{align*}
    \varphi_1(x) & \dff \Gamma_{\tau_1}(\varphi_0,\psi_0)(x) = \psi_0(\varphi_0(x)) = \bot\\
    \psi_1(x) & \dff \Gamma_{\tau_2}(\varphi_0,\psi_0)(x) = 0.
  \end{align*}
  Това означава, че $\Gamma(\varphi_0,\psi_0) = (\varphi_1,\psi_1)$.
  Итерираме процеса още веднъж, и получаваме:
  \marginpar{Единствената разлика е, че тук не разглеждаме отделен случай дали $x = \bot$}
  \begin{align*}
    \varphi_2(x) & \dff \Gamma_{\tau_1}(\varphi_1,\psi_1)(x) = \psi_1(\varphi_1(x)) = 0\\
    \psi_2(x) & \dff \Gamma_{\tau_2}(\varphi_1,\psi_1)(x) = 0.
  \end{align*}
  Това означава, че $\Gamma(\varphi_1,\psi_1) = (\varphi_2,\psi_2)$.
  Отново итерираме процеса:
    \begin{align*}
      \varphi_3(x) & \dff \Gamma_{\tau_1}(\varphi_2,\psi_2)(x) = \psi_2(\varphi_2(x)) = 0 = \varphi_2(x)\\
      \psi_3(x) & \dff \Gamma_{\tau_2}(\varphi_2,\psi_2)(x) = 0 = \psi_2(x).
  \end{align*}
  Получихме, че $\Gamma(\varphi_2,\psi_2) = (\varphi_2,\psi_2)$ и
  следователно $\lfp(\Gamma) = (\varphi_2,\psi_2)$. 
  Тогава, за всяко $a \in \Nat_\bot$,
  \[\D_N\val{\vv{P}}(a) = \varphi_2(a) = 0.\]
  Това означава, че денотационната семантика по име на програмата $\vv{P}$ е константната функция, която винаги връща $0$.
  Обърнете внимание, че също $\D_N\val{\vv{P}}(\bot) = 0$.
\end{example}

Можем да проверим нашите изчисления за най-малките неподвижни точки на операторите $\Gamma$ и $\Delta$
от горния пример като преведем техните дефиниции на \texttt{хаскел}.

\begin{haskellcode}
ghci> let gamma1(f,g)(x) = g(f(x))
ghci> let gamma2(f,g)(x) = 0
ghci> let gamma(f,g) = (gamma1(f,g), gamma2(f,g))
ghci> let omega = \x -> undefined
ghci> let approxCBN = (omega, omega) : [gamma(f,g) | (f,g) <- approxCBN ]
ghci> let (f3, g3) = approxCBN !! 3
ghci> f3(undefined)
0
ghci> f3(55)
0
ghci> :set -XBangPatterns
ghci> let delta1(f,g)(!x) = g(f(x))
ghci> let delta2(f,g)(!x) = 0
ghci> let delta(f,g) = (delta1(f,g), delta2(f,g))
ghci> let approxCBV = (omega, omega) : [delta(f,g) | (f,g) <- approxCBV ]
ghci> let (f3', g3') = approxCBV !! 3
ghci> f3'(undefined)
undefined
ghci> f3'(0)
undefined
ghci> f3'(55)
undefined
\end{haskellcode}



  Горният пример показва, че съществуват програми $\vv{P}$, 
  за които двете денотационни семантики са различни, но все пак $\D_V\val{\vv{P}} \sqsubseteq \D_N\val{\vv{P}}$.
  Ще видим, че това свойство е вярно за всяка програма $\vv{P}$ на езика \REC.


  \begin{prop}
    \label{pr:delta-in-gamma}
    Да разгледаме произволен терм $\tau[\vv{x}_1,\dots,\vv{x}_n, \vv{f}_1,\dots,\vv{f}_k]$.
    Тогава всеки 
    \[\bar{\varphi} \in \Mapping{\Nat^{m_1}_\bot}{\Nat_\bot}\times\cdots\times\Mapping{\Nat^{m_k}_\bot}{\Nat_\bot}\]
    е изпълнено, че
    \[\Delta_\tau(\bar{\varphi}) \sqsubseteq \Gamma_\tau(\bar{\varphi}).\]
  \end{prop}
  \begin{hint}
    \marginpar{Обърнете внимание, че ако $\varphi$ е точна функция, то $\Gamma_\tau(\varphi)$ е непрекъсната, но може да не е точна.}
    Понеже за всяка $\varphi \in \Mapping{\Nat^n_\bot}{\Nat_\bot}$,
    \[\Sigma_n(\varphi) \sqsubseteq \varphi,\]
    по получаваме, че
    \[\Delta_\tau(\bar{\varphi}) \dff \Sigma_n(\Gamma_\tau(\bar{\varphi})) \sqsubseteq \Gamma_\tau(\bar{\varphi}).\]
\end{hint}

\begin{framed}
  \begin{thm}
    За всяка рекурсивна програма \vv{P} е изпълнено, че
    \[\D_V\val{\vv{P}} \sqsubseteq \D_N\val{\vv{P}}.\]
  \end{thm}
\end{framed}
\begin{proof}
  Първо с индукция по $k$ ще докажем, че
  \[(\forall k)[\ \bar{\delta}_k \sqsubseteq \bar{\gamma}_k\ ].\]
  За $k = 0$ е ясно, защото $\delta^i_0 = \Omega^{(m_i)} = \gamma^i_0$.
  \marginpar{Озн. $\ov{\delta}_k = (\delta^1_k,\dots,\delta^n_k)$}
  \begin{align*}
    \delta^i_{k+1} & \dff \Delta_{\tau_i}(\bar{\delta}_k)\\
                   & \sqsubseteq \Gamma_{\tau_i}(\bar{\delta}_k) & \comment{\text{\Prop{delta-in-gamma}}}\\
                   & \sqsubseteq \Gamma_{\tau_i}(\bar{\gamma}_k) & \comment{\text{$\Gamma_{\tau_i}$ е мон. и от {\bf И.П.} имаме, че } \ov{\delta}_k \sqsubseteq \ov{\gamma}_k}\\
                   & \dff \gamma^i_{k+1}.
  \end{align*}
  Оттук следва, че 
  \begin{equation}
    \label{eq:12}
    \bar{\delta} = \bigsqcup_i\bar{\delta}_i \sqsubseteq \bigsqcup_i\bar{\gamma}_i = \bar{\gamma}.
  \end{equation}
  Сега, за произволни елементи $\ov{a}$,
  \begin{align*}
    \D_V\val{\vv{h}}(\bar{a}) & \dff \delta_1(\ov{a})\\
                              & \sqsubseteq \gamma_1(\ov{a})  & \comment{\text{от (\ref{eq:12}) имаме, че }\bar{\delta}\sqsubseteq\bar{\gamma}}\\
                              & \dff \D_N\val{\vv{h}}(\ov{a}).
  \end{align*}
\end{proof}



%%% Local Variables:
%%% mode: latex
%%% TeX-master: "../sep-notes"
%%% End:
