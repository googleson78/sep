\subsection{Предаване на параметрите по стойност}
\index{денотационна семантика!по стойност}

Основната разлика между денотационната семантика по име и по стойност е, че при семантиката по стойност
изискваме да работим {\em само с точни функции}.

\begin{remark}
  \marginpar{Сигурни сме единствено, че термалните оператори запазват непрекъснатостта}
  Да обърнем внимание, че термалните оператори $\Gamma_\tau$ не запазват точните функции, т.е.
  възможно е $\varphi \in \Strict{\Nat^n_\bot}{\Nat_\bot}$, но $\Gamma_\tau(\varphi) \not \in \Strict{\Nat^m_\bot}{\Nat_\bot}$.
  Например, да разгледаме терма
  \[\tau[\vv{x},\vv{y},\vv{f}] \equiv \vv{x},\]
  т.е. изобщо не се обръщаме към функционалния аргумент.
  Тогава ако $a \neq \bot$, но $b = \bot$, то $\val{\tau}(\varphi)(a,b) \neq \bot$.

  Да се уверим, че хаскел ,,смята'' по подобен начин.
  \begin{haskellcode}
ghci> let f(x,y,g) = x
ghci> let h(x,y) = f(x,y,(\z -> z))
ghci> h(1,undefined)
1
  \end{haskellcode}
  Вижда се, че макар и да подаваме точна функция като аргумент на $\vv{f}$, то новополучената функция $\vv{h}$ не е точна.
\end{remark}

Поради тази причина, за да дефинираме семантика с предаване на параметрите по стойност, ще използваме най-малки неподвижни точки
на други оператори, {\em които винаги връщат точни функции}.

\begin{framed}
  \begin{theorem}
    За всеки терм $\tau[\varsx,\varsf]$, операторът 
    \[\Delta_\tau:\DomOpCBV\to \RanOpCBV,\]
    дефиниран като
    \[\Delta_\tau \dff \Sigma_\star \circ \Gamma_\tau,\]
    е непрекъснат.
  \end{theorem}
\end{framed}
\begin{proof}
  Първо да съобразим, че $\Delta_\tau$ е коректно дефиниран оператор, т.е. 
  за $\bar{\varphi} \in \DomOpCBV$ имаме, че $\Delta_\tau(\bar{\varphi}) \in \RanOpCBV$.
  Това е ясно, защото $\Sigma_\star$ превръща всяко изображение в точно.
  
  От \Prop{strict-operator} знаем, че $\Sigma_\star$ е непрекъснат оператор, а от \Prop{composition}
  знаем, че композиция на непрекъснати оператори е също непрекъснат оператор.
  Заключаваме, че $\Delta_\tau$ е непрекъснат оператор.    
\end{proof}

\marginpar{Тук отново можем да си мислим за $\vv{f}_1$ като за \texttt{main} функцията на програмата $\vv{P}$.}
Нека разгледаме рекурсивната програма $\vv{P}[\varsx,\varsf]$ на езика $\REC$, където:
\begin{align*}
  \vv{P} = & 
             \begin{cases}
               & \vv{f}_1(\vv{x}_1,\dots,\vv{x}_{m_1}) = \tau_1[\vv{x}_1,\dots,\vv{x}_{m_1},\vv{f}_1,\dots,\vv{f}_k]\\
               & \vv{f}_2(\vv{x}_1,\dots,\vv{x}_{m_2}) = \tau_2[\vv{x}_1,\dots,\vv{x}_{m_2},\vv{f}_1,\dots,\vv{f}_k]\\
               & \vdots\\
               & \vv{f}_k(\vv{x}_1,\dots,\vv{x}_{m_k}) = \tau_k[\vv{x}_1,\dots,\vv{x}_{m_k},\vv{f}_1,\dots,\vv{f}_k]
             \end{cases}
\end{align*}
Нека $\bar{\delta} \in \DomOpCBV$ е {\em най-малкото решение} на системата,
която съответства на програмата $\texttt{P}$:
\begin{align*}
  & \Delta_{\tau_1}(\psi_1,\dots,\psi_k) = \psi_1\\
  & \ \vdots \\
  & \Delta_{\tau_k}(\psi_1,\dots,\psi_k) = \psi_k.
\end{align*}
Понеже $\Delta_{\tau_i}$ са непрекъснати оператори, то \hyperref[th:knaster-tarski]{Теоремата на Клини} знаем, че такива точни изображения $\ov{\delta}$ съществуват.

\index{денотационна семантика!по стойност}
\begin{framed}
  За рекурсивната програма $\vv{P}[\varsx,\varsf]$, определяме {\bf денотационната семантика с предаване на параметрите по стойност} 
  като изображението $\D_V\val{\vv{P}} \in \Strict{\Nat^{m_1}_\bot}{\Nat_\bot}$, където:
  \[\D_V\val{\vv{P}}(a_1,\dots,a_{m_1}) \dff \delta_1(a_1,\dots,a_{m_1}).\]
\end{framed}

%%% Local Variables:
%%% mode: latex
%%% TeX-master: "../sep"
%%% End:
