% % \subsection{Термални оператори}

% Вече дефинирахме как на всеки терм $\tau[\vv{x}_1,\dots,\vv{x}_n,\vv{f}_1,\dots,\vv{f}_k]$
% съпоставяме изображението $\val{\tau}$ със сигнатура
% \[\val{\tau}:\Mapping{\Nat^{m_1}_\bot}{\Nat_\bot}\times\cdots\times\Mapping{\Nat^{m_k}_\bot}{\Nat_\bot} \to \Mapping{\Nat^n_\bot}{\Nat_\bot}\]

% Едно от основните свойства, които трябва да притежават тези изображения е да бъдат {\em непрекъснати} относно областите на Скот,
% за които са дефинирани.
% Причината за това е, че искаме да дефинираме семантиката на една програма като използваме най-малкото решение 
% на система от такива оператори, а според \hyperref[th:knaster-tarski]{Теоремата на Клини}, за да можем да направим това, трябва да работим
% с непрекъснати оператори. Следващият пример ни показва, че трябва да сме по-внимателни върху какви елементи разглеждаждаме тези изображения.

% \begin{example}
%   \label{ex:non-continuous}
%   Да разгледаме терма 
%   \[\tau[\vv{x},\vv{f},\vv{g}] \dfff \vv{f}(\vv{g}(\vv{x})),\]
%   и следните две изображения от тип $\Mapping{\Nat_\bot}{\Nat_\bot}$:
%   \marginpar{Обърнете внимание, че $\varphi$ не е монотонна функция. Тя дори не е точна. 
%     Функцията $\psi$ не е точна, но е монотонна. Знаем, че $\varphi$ не е непрекъсната}
%   \begin{align*}
%     & \varphi(x) \dff
%     \begin{cases}
%       42,   & \text{ако }x = \bot\\
%       \bot, & \text{ако }x \in \Nat
%     \end{cases}\\
%     & \psi(x) \dff 42 \text{, за всяко }x \in \Nat_\bot.
%   \end{align*}
%   Лесно се вижда, че $\pair{\varphi,\varphi} \sqsubseteq \pair{\varphi,\psi}$, но
%   \[\val{\tau}(\varphi,\varphi) \not\sqsubseteq \val{\tau}(\varphi,\psi),\]
%   защото за произволно $a \in \Nat$,
%   \begin{align*}
%     \val{\tau}(\varphi,\varphi)(a) & = \varphi(\varphi(a)) & \comment{\text{стойност на терм}}\\
%                                  & = 42  \\
%                                  & \not\sqsubseteq \bot & \comment{\text{плоска наредба}}\\
%                                  & = \varphi(\psi(a)) & \comment{\text{защото }\psi(a) \neq \bot}\\
%                                  & = \val{\tau}(\varphi,\psi)(a) & \comment{\text{от деф. на }\tau}\\
%   \end{align*}

%   \marginpar{Знаем, че всеки непрекъснат оператор е монотонен. Щом $\val{\tau}$ не е монотонен, то той със сигурност не е непрекъснат.}
%   Това означава, че за конкретния терм $\tau$, изображението $\val{\tau}$ не е монотонно, откъдето следва, че то не е непрекъснато.
% \end{example}

% % Горният пример ни казва, че за $\val{\tau}$, дефинирани върху произволни елементи от $\Mapping{\Nat^{m_i}_\bot}{\Nat_\bot}$
% % {\em не можем да гарантираме свойството непрекъснатост}. Ще видим, че ако се ограничим само до непрекъснатите изображения,
% % то тогава операторите ще бъдат непрекъснати.
% Всъщност на нас е необходимо да можем да подаваме като аргументи на $\val{\tau}$ само такива $\varphi$,
% които можем да дефинираме на езика \REC.
% Според семантиката на термовете, която дефинирахме по-горе, не е ясно дали можем да дефинираме терм на езика \REC, чиято семантика да бъде изображението $\varphi$. Естествено е да разгледаме терма 
% \[\tau[\vv{x}] \dfff \ifelse{\vv{x}\ \vv{==}\ \bot}{42}{\bot}.\]
% \marginpar{Това {\em не} е доказателство, че не можем да дефинираме функцията $\varphi$ на езика \REC. Тук ние твърдим само, че очевидният опит за дефиниция на $\varphi$ се проваля. По-нататък ще видим, че наистина не може да дефинираме $\varphi$ на нашия език, защото тя не е непрекъсната функция.}
%  Каква е семантиката на този терм? Следваме дефиницията на стойност на терм, за произволно $a \in \Nat_\bot$, и получаваме:
%  \begin{align*}
%    \val{\tau}(a) & =
%                    \begin{cases}
%                      42,   & \text{ако }\val{\vv{x}\ \vv{==}\ \bot}(a) \in \Nat^+\\
%                      \bot, & \text{ако }\val{\vv{x}\ \vv{==}\ \bot}(a) = 0\\
%                      \bot, & \text{ако }\val{\vv{x}\ \vv{==}\ \bot}(a) = \bot\\
%                    \end{cases}\\
%                  & = 
%                    \begin{cases}
%                      42,   & \text{ако }\underbrace{\val{\vv{x}}(a)}_{\in\Nat} = \underbrace{\val{\bot}(a)}_{\in \Nat}\\
%                      \bot, & \text{ако }\underbrace{\val{\vv{x}}(a)}_{\in\Nat} \neq \underbrace{\val{\bot}(a)}_{\in \Nat}\\
%                      \bot, & \text{ако }\val{\vv{x}}(a) = \bot\text{ или }\val{\bot}(a) = \bot\\
%                    \end{cases}\\
%                  & = \bot.
%  \end{align*}
 
% Да видим, какво ще стане ако преведем директно горния пример на \vv{хаскел}.
% \begin{haskellcode}
% ghci> let phi(x) = if x == undefined then 42 else undefined
% ghci> let psi(x) = 42
% ghci> phi(0)
% *** Exception: Prelude.undefined
% ghci> phi(undefined)
% *** Exception: Prelude.undefined
% ghci> psi(0)
% 42
% \end{haskellcode}

% Виждаме, че тук \texttt{хаскел} следва нашата семантика за езика {\bf REC}.
% Направените по-горе бележки ни насочват към следната дефиниция на {\bf термален оператор}.

% \begin{framed}
%   \begin{definition}
%     \index{термален оператор}
%     За всеки терм от вида $\tau[\vv{x}_1,\dots,\vv{x}_n, \vv{f}_1,\dots,\vv{f}_k]$
%     дефинираме оператора 
%     \[\Gamma_\tau: \DomOpCBN \to \RanOpCBN,\] % като
%     където
%     \[\Gamma_\tau(\bar{\varphi})(\bar{a}) \dff \val{\tau}(\ov{\varphi})(\ov{a}).\]
%   \end{definition}
% \end{framed}
% \marginpar{Обърнете внимание, че все още не е ясно защо $\Gamma_\tau(\ov{\varphi}) \in \RanOpCBN$}
% В следващия раздел ще се концентрираме върху доказателството на следния основен резултат:

% \begin{framed}
%   За всеки терм $\tau[\varsx, \varsf]$, операторът $\Gamma_\tau$ е непрекъснат.
% \end{framed}

\subsection{Непрекъснатост на термалните оператори}

% Първата ни задача ще бъде да проверим, че операторът $\Gamma_\tau$ е коректно дефиниран.
% Това означава, че трябва да се уверим, че винаги когато подадем като аргументи на $\Gamma_\tau$ непрекъснати изображения,
% то $\Gamma_\tau$ връща непрекъснато изображение.

\begin{remark}
  В този раздел всички доказателства се провеждат с индукция по построението на термовете.
\end{remark}

\begin{proposition}
  \label{pr:term-monotone}
  Да разгледаме един терм $\tau[\vv{x}_1,\dots,\vv{x}_n, \vv{f}_1,\dots,\vv{f}_k]$.
  \marginpar{Озн. $\ov{a}_i = (a^1_i,\dots,a^n_i)$}
  Нека $\ov{a} \sqsubseteq \ov{b}$.
  Тогава 
  \[\val{\tau}(\ov{\varphi})(\ov{a}) \sqsubseteq \val{\tau}(\ov{\varphi})(\ov{b}),\]
  където
  $\varphi_i \in \Cont{\Nat^{m_i}_\bot}{\Nat_\bot}$, за $i = 1,\dots,k$.
\end{proposition}
\begin{hint}
  Индукция по построението на терма $\tau$.
  \begin{itemize}
  \item
    Нека $\tau \equiv \vv{c}$. Този случай е ясен.
  \item
    Нека $\tau \equiv \vv{x}_i$. Тогава
    \begin{align*}
      \val{\tau}(\ov{\varphi})(\ov{a}) & = \val{\vv{x}_i}(\ov{\varphi})(\ov{a})\\
                                       & = a_i & \comment{\text{стойност на терм}}\\
                                       & \sqsubseteq b_i & \comment{\ov{a} \sqsubseteq \ov{b}}\\
                                       &= \val{\vv{x}_i}(\ov{\varphi})(\ov{b}) & \comment{\text{стойност на терм}}\\
                                       & = \val{\tau}(\ov{\varphi})(\ov{b}).
    \end{align*}
  \item
    Нека $\tau \equiv \tau_1 + \tau_2$. 
    Тук ще използваме, че от {\bf И.П.}, за $j = 1,2$, имаме
    \begin{equation}
      \label{eq:10}
      \val{\tau_j}(\ov{\varphi})(\ov{a}) \sqsubseteq \val{\tau_j}(\ov{\varphi})(\ov{b}).
    \end{equation}
    Освен това, понеже изображението $\texttt{plus}$ е непрекъснато, то то е и монотонно.
    \begin{align*}
      \val{\tau}(\ov{\varphi})(\ov{a}) & = \val{\tau_1 + \tau_2}(\ov{\varphi})(\ov{a})\\
                                       & \dff \texttt{plus}(\val{\tau_1}(\ov{\varphi})(\ov{a}), \val{\tau_2}(\ov{\varphi})(\ov{a}))\\
                                       & \sqsubseteq \texttt{plus}(\val{\tau_1}(\ov{\varphi})(\ov{b}), \val{\tau_2}(\ov{\varphi})(\ov{b})) & \comment{\text{от (\ref{eq:10}) и мон.}}\\
      & \dff \val{\tau_1 + \tau_2}(\ov{\varphi})(\ov{b}).
    \end{align*}
  \item
    \marginpar{\writedown За домашно!}
    Нека $\tau \equiv \tau_1\ \vv{==}\ \tau_2$.
  \item
    Нека $\tau \equiv \ifelse{\tau_0}{\tau_1}{\tau_2}$.  
  \item
    Нека $\tau \equiv \vv{f}_i(\tau_1,\dots,\tau_{m_i})$. 
    Тук ще използваме, че от {\bf И.П.}, за $j = 1,\dots,m_i$, имаме
    \begin{equation}
      \label{eq:6}
      \val{\tau_j}(\ov{\varphi})(\ov{a}) \sqsubseteq \val{\tau_j}(\ov{\varphi})(\ov{b}).
    \end{equation}
    Тогава
    \begin{align*}
      \val{\tau}(\ov{\varphi})(\ov{a}) & = \val{\vv{f}_i(\tau_1,\dots,\tau_{m_i})}(\ov{\varphi})(\ov{a})\\
                                      & \dff \varphi_i(\val{\tau_1}(\ov{\varphi})(\ov{a}), \dots, \val{\tau_{m_i}}(\ov{\varphi})(\ov{a}))\\
                                      & \sqsubseteq \varphi_i(\val{\tau_1}(\ov{\varphi})(\ov{b}), \dots, \val{\tau_{m_i}}(\ov{\varphi})(\ov{b})) & \comment{\text{от (\ref{eq:6}) и $\varphi_i$ е мон.}}\\
                                      & \dff \val{\tau}(\ov{\varphi})(\ov{b}).
    \end{align*}
  \end{itemize}
\end{hint}

\begin{corollary}
  \label{cr:tau-preserves-continuous}
  Да разгледаме един терм $\tau[\vv{x}_1,\dots,\vv{x}_n, \vv{f}_1,\dots,\vv{f}_k]$.
  Тогава за произволна верига $\chain{\bar{a}}{i}$ в $\Nat^{n}_\bot$,
  \[\val{\tau}(\ov{\varphi})(\bigsqcup_i\ov{a}_i) = \bigsqcup_i\val{\tau}(\ov{\varphi})(\ov{a}_i),\]
  където
  $\varphi_i \in \Cont{\Nat^{n_i}_\bot}{\Nat_\bot}$, за $i = 1,\dots,k$.
\end{corollary}
\begin{proof}
  Да разгледаме произволен терм $\tau$ и непрекъснати изображения $\ov{\varphi}$.
  От \Prop{term-monotone} следва, че $\val{\tau}(\ov{\varphi}) \in \Mon{\Nat^n_\bot}{\Nat_\bot}$.
  Понеже всяка верига в $\Nat^n_\bot$ се стабилизира, то директно от \Prop{stab-continuous}
  следва, че $\val{\tau}(\ov{\varphi}) \in \Cont{\Nat^n_\bot}{\Nat_\bot}$,
  което означава, че за произволна верига $\chain{\ov{a}}{i}$,
  \[\val{\tau}(\ov{\varphi})(\bigsqcup_i\ov{a}_i) = \bigsqcup_i\val{\tau}(\ov{\varphi})(\ov{a}_i).\]
\end{proof}

\begin{corollary}\label{cr:gamma-preserves-continuous}
  Да разгледаме произволен терм $\tau[\vv{x}_1,\dots,\vv{x}_n, \vv{f}_1,\dots,\vv{f}_k]$.
  Тогава за произволни $\varphi_i \in \Cont{\Nat^{m_i}}{\Nat_\bot}$, за $i = 1,\dots k$,
  имаме, че
  \[\val{\tau}(\bar\varphi) \in \RanOpCBN.\]
\end{corollary}

Щом термалните оператори $\Gamma_\tau$ са добре дефинирани, ще 
проверим, че са непрекъснати. Първата стъпка ще бъде проверката, че те са монотонни.

\begin{lemma}\label{lem:rec:functional:term:continuous}
  Нека $\mu[\vv{f}_1,\dots,\vv{f}_k]$ е {\em функционален} терм.
  \marginpar{Да напомним, че $m_i \dff \#\vv{f}_i$}
  Да разгледаме произволна верига $\chain{\ov{\varphi}}{r}$
  от елементи на
  \[\DomOpCBN.\]
  Тогава $\val{\mu}$ е непрекъснато изображение, т.е.
  \[\val{\mu}(\bigsqcup_r\ov{\varphi}_r) = \bigsqcup_r \val{\mu}(\ov{\varphi}_r).\]
\end{lemma}
\begin{proof}
  Индукция по построението на терма $\mu$.
  \begin{itemize}
  \item
    Нека $\mu \equiv \vv{c}$. Този случай е очевиден.
  \item
    Нека $\mu \equiv \mu_1 + \mu_2$. Ще използваме, че $\texttt{plus}$ е непрекъснато изображение.
    \begin{align*}
      \val{\mu}(\bigsqcup_r \ov{\varphi}_r) & \dff \texttt{plus}(\val{\mu_1}(\bigsqcup_r\ov{\varphi}_r), \val{\mu_2}(\bigsqcup_r\ov{\varphi}_r))\\
      & = \texttt{plus}(\bigsqcup_r\val{\mu_1}(\ov{\varphi}_r), \bigsqcup_r \val{\mu_2}(\ov{\varphi}_r)) & \comment{\text{от И.П.}}\\
      & = \bigsqcup_r \texttt{plus}(\val{\mu_1}(\ov{\varphi}_r), \val{\mu_2}(\ov{\varphi}_r)) & \comment{\texttt{plus}\text{ е непр.}}\\
      & \dff \bigsqcup_r \val{\mu}(\ov{\varphi}_r).
    \end{align*}
  \item
    Нека $\mu \equiv \mu_1\ \vv{==}\ \mu_2$.
    Използвайте, че $\texttt{eq}$ е непрекъснато изображение.
  \item
    Нека $\mu \equiv \ifelse{\mu_0}{\mu_1}{\mu_2}$.
    Използвайте \Problem{rec:if:continuous}.
  \item
    Нека $\mu \equiv \vv{f}_i(\mu_1,\dots,\mu_{m_i})$. 
    От И.П. знаем, че $\val{\mu_j}$ са непрекъснати изображения за $j = 1,\dots,m_i$ и следователно са монотонни. Тогава
    за произволни индекси $n \leq n'$ и $r \leq r'$ имаме, че
    \marginpar{$\ov{\varphi}_n \dff (\varphi^1_n,\dots,\varphi^{k}_n)$}
    \begin{align*}
      \varphi^i_{n}(\val{\mu_1}(\ov{\varphi}_r),\dots,\val{\mu_{m_i}}(\ov{\varphi}_r)) & \sqsubseteq \varphi^i_{n}(\val{\mu_1}(\ov{\varphi}_{r'}),\dots,\val{\mu_{m_i}}(\ov{\varphi}_{r'}))\\
                                                                                       & \sqsubseteq \varphi^i_{n'}(\val{\mu_1}(\ov{\varphi}_{r'}),\dots,\val{\mu_{m_i}}(\ov{\varphi}_{r'})).
    \end{align*}
    Това означава, че ако положим
    \[e_{n,r} \dff \varphi^i_{n}(\val{\mu_1}(\ov{\varphi}_r),\dots,\val{\mu_{m_i}}(\ov{\varphi}_r)),\]
    то
    \[n \leq n'\ \&\ r \leq r' \implies e_{n,r} \sqsubseteq e_{n',r'}.\]
    Сега можем да приложим \Th{double-chain}, според която
    \begin{equation}
      \label{eq:8}
      \bigsqcup_n(\bigsqcup_r e_{n,r}) = \bigsqcup_n e_{n,n}.
    \end{equation}

    Тогава 
    \begin{align*}
      \val{\mu}(\bigsqcup_n\ov{\varphi}_n) & = \val{\vv{f}_i(\mu_1,\dots,\mu_{m_i})}(\bigsqcup_n\ov{\varphi}_n)\\
                                           & \dff (\bigsqcup_n\varphi^i_n)(\val{\mu_1}(\bigsqcup_r\ov{\varphi}_r),\dots,\val{\mu_{m_i}}(\bigsqcup_r\ov{\varphi}_r))\\
                                           & = \bigsqcup_n \{\varphi^i_n(\val{\mu_1}(\bigsqcup_r\ov{\varphi}_r),\dots,\val{\mu_{m_i}}(\bigsqcup_r\ov{\varphi}_r))\} & \comment{\text{от \Th{monotone-is-domain}}}\\
                                           & =  \bigsqcup_n \{\varphi^i_n(\bigsqcup_r \val{\mu_1}(\ov{\varphi}_r),\dots,\bigsqcup_r \val{\mu_{m_i}}(\ov{\varphi}_r))\} & \comment{\text{от И.П. за }\mu_j}\\
                                           & =  \bigsqcup_n \{\bigsqcup_r \underbrace{\varphi^i_n(\val{\mu_1}(\ov{\varphi}_r),\dots,\val{\mu_{m_i}}(\ov{\varphi}_r))}_{e_{n,r}}\} & \comment{\varphi^i_n \text{ е непр.}}\\
                                           & =  \bigsqcup_n \underbrace{\varphi^i_n(\val{\mu_1}(\ov{\varphi}_n),\dots,\val{\mu_{m_i}}(\ov{\varphi}_n))}_{e_{n,n}} & \comment{\text{от (\ref{eq:8})}}\\
                                           & \dff \bigsqcup_n \val{\vv{f}_i(\mu_1,\dots,\mu_{m_i})}(\ov{\varphi}_n)\\
                                           & = \bigsqcup_n \val{\mu}(\ov{\varphi}_n).
    \end{align*}
  \end{itemize}
\end{proof}

\begin{corollary}
  \label{cr:rec:term:continuous}
  Нека $\tau[\varsx,\varsf]$ е терм.
  Да разгледаме произволни елементи $\ov{a}$ на $\Nat^n_\bot$ и 
  произволна верига $\chain{\ov{\varphi}}{r}$
  от елементи на
  \[\DomOpCBN.\]
  Тогава 
  \[\val{\tau}(\bigsqcup_r\ov{\varphi}_r)(\ov{a}) = \bigsqcup_r \val{\tau}(\ov{\varphi}_r)(\ov{a}).\]  
\end{corollary}
\begin{proof}
  \begin{align*}
    \val{\tau}(\bigsqcup_r\ov{\varphi}_r)(\ov{a}) & = \val{\tau[\varsx/\ov{\vv{a}}]}(\bigsqcup_r \ov{\varphi}_r) & \comment{\text{от \hyperref[lem:rec:substitution]{Лема за замяната}}}\\
                                                  & = \bigsqcup_r \val{\tau[\varsx/\ov{\vv{a}}]}(\ov{\varphi}_r) & \comment{\text{от \Lem{rec:functional:term:continuous}}}\\
                                                  & = \bigsqcup_r \val{\tau}(\ov{\varphi}_r)(\ov{a}) &  \comment{\text{от \hyperref[lem:rec:substitution]{Лема за замяната}}}\\
  \end{align*}
\end{proof}

Вече имаме всичко необходимо за да се убедим, че термалните оператори са непрекъснати.

\begin{framed}
\begin{theorem}
  \label{th:gamma-is-continuous}
  За всеки терм $\tau[\vv{x}_1,\dots,\vv{x}_n, \vv{f}_1,\dots,\vv{f}_k]$, операторът 
  \[\Gamma_\tau: \DomOpCBN \to \RanOpCBN,\]
  дефиниран като
  \[\Gamma_\tau(\bar{\varphi})(\bar{a}) = \val{\tau}(\ov{\varphi})(\bar{a}),\]
  е непрекъснат.
\end{theorem}
\end{framed}
\begin{proof}
  От \Cor{gamma-preserves-continuous} имаме, че $\Gamma_\tau$ е коректно дефиниран.
  Сега директно се позоваваме на \Cor{rec:term:continuous}.
\end{proof}

%%% Local Variables:
%%% mode: latex
%%% TeX-master: "../sep"
%%% End:
