\subsection{Предаване на параметрите по име}
\index{операционна семантика!по име}

Правилата за извод с предаване на параметрите по име, които означаваме като $\mu \to^P_N c$,
са същите като тези с предаване на параметрите по стойност като 
единствената разлика е, че вместо правилата $(4)_\Nat$ и $(4)_\bot$ имаме правилото $(4)$.

\begin{description}
\item
  За всяко $a \in \Nat_\bot$,
  \begin{figure}[h!]
    \begin{prooftree}
      \AxiomC{}
      \RightLabel{\scriptsize{(1)}}
      \UnaryInfC{$\vv{a}\to^P_N a$}
    \end{prooftree}
  \end{figure}
\item
  \begin{figure}[h!]
    \begin{prooftree}
      \AxiomC{$\mu_1\to^P_N a_1$}
      \AxiomC{$\mu_2\to^P_N a_2$}
      \AxiomC{$a = \texttt{plus}(a_1, a_2)$}
      \RightLabel{\scriptsize{$(2_+)$}}
      \TrinaryInfC{$\mu_1 + \mu_2 \to^P_N a$}
    \end{prooftree}
  \end{figure}
\item
  \begin{figure}[h!]
    \begin{prooftree}
      \AxiomC{$\mu_1\to^P_N a_1$}
      \AxiomC{$\mu_2\to^P_N a_2$}
      \AxiomC{$a = \texttt{eq}(a_1, a_2)$}
      \RightLabel{\scriptsize{$(2_{\vv{==}})$}}
      \TrinaryInfC{$\mu_1\ \vv{==}\ \mu_2 \to^P_N a$}
    \end{prooftree}
  \end{figure}
\item
  \begin{figure}[h!]
    \begin{prooftree}
      \AxiomC{$\mu_0\to^P_N a_0$}
      \AxiomC{$\mu_1 \to^P_N a_1$}
      \AxiomC{$a_0 \in \Nat^+$}
      \RightLabel{\scriptsize{$(3_\true)$}}
      \TrinaryInfC{$\ifelse{\mu_0}{\mu_1}{\mu_2} \to^P_N a_1$}
    \end{prooftree}
  \end{figure}  
\item
  \begin{figure}[h!]
    \begin{prooftree}
      \AxiomC{$\mu_0\to^P_N 0$}
      \AxiomC{$\mu_2 \to^P_N a_2$}
      \RightLabel{\scriptsize{$(3_\false)$}}
      \BinaryInfC{$\ifelse{\mu_0}{\mu_1}{\mu_2} \to^P_N a_2$}
    \end{prooftree}
  \end{figure}
\item
  \begin{figure}[h!]
    \begin{prooftree}
      \AxiomC{$\mu_0\to^P_N \bot$}
      \RightLabel{\scriptsize{$(3_\bot)$}}
      \UnaryInfC{$\ifelse{\mu_0}{\mu_1}{\mu_2} \to^P_N \bot$}
    \end{prooftree}
  \end{figure}
\item
  \begin{figure}[h!]
    \begin{prooftree}
      \AxiomC{$\tau_i[\vv{x}_1/\mu_1,\dots,\vv{x}_{m_i}/\mu_{m_i}] \to^P_N a$}
      \RightLabel{\scriptsize{(4)}}
      \UnaryInfC{$\vv{f}_i(\mu_1,\dots,\mu_{m_i}) \to^P_N a$}
    \end{prooftree}
  \end{figure}
\end{description}


\Stefan{Също както няма нужда от $(4_\bot)$, няма нужда и от $(3_\bot)$, но може би е добре да се остави, за да е по-явно какво става в този случай.}

За фиксирана декларация $P$, нека за всеки {\em функционален} терм $\mu$ да дефинираме
\[\evaln{\mu}\dff
\begin{cases}
  b, & \text{ ако }\mu \to^P_N b\\
  \bot, & \text{ ако }\mu \text{ няма извод до константа}.
\end{cases}\]

\begin{framed}
  Операционната семантика по име на рекурсивната програма $\vv{P}[\varsx,\varsf]$ представлява
  изображението 
  \[\O_N\val{\vv{P}} \in \Cont{\Nat^{m_1}_\bot}{\Nat_\bot},\] където
  \[\O_N\val{\vv{P}}(a_1,\dots,a_{m_1}) \dff \evaln{\vv{f}_1(\vv{a}_1,\dots,\vv{a}_{m_1})},\]
  за произволни $a_1,\dots,a_{m_1} \in \Nat_\bot$.
\end{framed}

\begin{remark}
  Всъщност ние все още няма как да знаем, че за всяка програма $\vv{P}$,
  $\O_N\val{\vv{P}}$ е непрекъснато изображение.
  Този факт може да се докаже директно, но вместо това, ние ще видим, че
  $\O_N\val{\vv{P}} = \D_N\val{\vv{P}}$ и оттам ще получим непрекъснатостта на $\O_N\val{\vv{P}}$,
  защото от дефиницията на $\D_N\val{\vv{P}}$ е ясно, че то непрекъснато изображение.
\end{remark}


\begin{example}
  Нека за програмата \vv{P}:
  \begin{haskellcode}
    f(x,y) = if x == y then 0 else 1 + f(x,y+1)
  \end{haskellcode}
  да разгледаме няколко извода с правилата на операционната семантика по име.
\end{example}


%%% Local Variables:
%%% mode: latex
%%% TeX-master: "../sep"
%%% End:
