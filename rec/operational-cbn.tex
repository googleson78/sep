\subsection{Предаване на параметрите по име}\index{операционна семантика!по име}

% Правилата за извод с предаване на параметрите по име, които означаваме като $\mu \Downarrow^P c$,
% са същите като тези с предаване на параметрите по стойност като 
% единствената разлика е, че вместо правилата $(4)_\Nat$ и $(4)_\bot$ имаме правилото $(4)$.

\marginpar{Това се нарича cost dynamics в \cite[стр. 58]{practical-foundations}.}
% В някои доказателства ще се наложи да правим индукция по дължината на извода $\to_P$.
% Затова дефинираме релацията $\mu \to^\ell_P \vv{a}$, която казва, че функционалният терм $\mu$
% се свежда до константата $\vv{a}$ след $\ell$ на брой прилагания на правилата на операционната семантика.
% Формално дефиницията е следната:


Дефинираме релация $\Downarrow^\ell_P$, която ни казва как един функционален терм $\mu$
се свежда до константата $\vv{a}$ след $\ell$ на брой стъпки следвайки следните правила:
\begin{description}
\item
  % За всяко $a \in \Nat$,
  \begin{figure}[h!]
    \begin{prooftree}
      \AxiomC{}
      \RightLabel{\scriptsize{(const)}}
      \UnaryInfC{$\vv{a}\Downarrow^0_P \vv{a}$}
    \end{prooftree}
  \end{figure}
\item
  \begin{figure}[h!]
    \begin{prooftree}
      \AxiomC{$\mu_1\Downarrow^{\ell_1}_P \vv{a}_1$}
      \AxiomC{$\mu_2\Downarrow^{\ell_2}_P \vv{a}_2$}
      \AxiomC{$a = \texttt{plus}(a_1, a_2)$}
      \RightLabel{\scriptsize{(plus)}}
      \TrinaryInfC{$\mu_1 + \mu_2 \Downarrow^{\ell_1+\ell_2+1}_P \vv{a}$}
    \end{prooftree}
  \end{figure}
\item
  \begin{figure}[h!]
    \begin{prooftree}
      \AxiomC{$\mu_1\Downarrow^{\ell_1}_P \vv{a}_1$}
      \AxiomC{$\mu_2\Downarrow^{\ell_2}_P \vv{a}_2$}
      \AxiomC{$a = \texttt{eq}(a_1, a_2)$}
      \RightLabel{\scriptsize{(eq)}}
      \TrinaryInfC{$\mu_1\ \vv{==}\ \mu_2 \Downarrow^{\ell_1+\ell_2+1}_P \vv{a}$}
    \end{prooftree}
  \end{figure}
\item
  \begin{figure}[h!]
    \begin{prooftree}
      \AxiomC{$\mu_0\Downarrow^{\ell_0}_P \vv{a}_0$}
      \AxiomC{$\mu_1 \Downarrow^{\ell_1}_P \vv{a}_1$}
      \AxiomC{$a_0 > 0$}
      \RightLabel{\scriptsize{(if$^+$)}}
      \TrinaryInfC{$\ifelse{\mu_0}{\mu_1}{\mu_2} \Downarrow^{\ell_0+\ell_1+1}_P \vv{a}_1$}
    \end{prooftree}
  \end{figure}  
\item
  \begin{figure}[h!]
    \begin{prooftree}
      \AxiomC{$\mu_0\Downarrow^{\ell_0}_P \vv{0}$}
      \AxiomC{$\mu_2 \Downarrow^{\ell_2}_P \vv{a}_2$}
      \RightLabel{\scriptsize{(if$_0$)}}
      \BinaryInfC{$\ifelse{\mu_0}{\mu_1}{\mu_2} \Downarrow^{\ell_0+\ell_2+1}_P \vv{a}_2$}
    \end{prooftree}
  \end{figure}
\item
  \begin{figure}[h!]
    \begin{prooftree}
      \AxiomC{$\tau_i[\vv{x}_1/\mu_1,\dots,\vv{x}_{m_i}/\mu_{m_i}] \Downarrow^{\ell}_P \vv{a}$}
      \RightLabel{\scriptsize{(cbn)}}
      \UnaryInfC{$\vv{f}_i(\mu_1,\dots,\mu_{m_i}) \Downarrow^{\ell+1}_P \vv{a}$}
    \end{prooftree}
  \end{figure}
\end{description}

Ще пишем $\mu \Downarrow_{\vv{P}} \vv{a}$, ако съществува $\ell$, за което $\mu \Downarrow^\ell_{\vv{P}} \vv{a}$.
Също така, понякога ще пишем $\mu \Downarrow^{<\ell}_P \vv{a}$, когато искаме да кажем, че
$\mu$ се свежда до $\vv{a}$ след прилагане на по-малко от $\ell$ на брой правила от операционната семантика.

\begin{lemma}
  Докажете, че за всеки затворен терм $\mu$,
  ако $\mu \Downarrow_P \vv{a}$ и $\mu \Downarrow_P \vv{a}'$, то $\vv{a} \equiv \vv{a}'$.
\end{lemma}

За фиксирана декларация $P$, нека за всеки {\em функционален} терм $\mu$ да дефинираме
\[\eval{\mu}\dff
  \begin{cases}
    b, & \text{ ако }\mu \Downarrow_P \vv{b}\\
    \bot, & \text{ ако }\mu \text{ няма извод до константа}.
\end{cases}\]

\begin{framed}
  Операционната семантика по име на рекурсивната програма $\vv{P}[\varsx,\varsf]$ представлява
  изображението 
  \[\O\val{\vv{P}} \in \Cont{\Nat^{m_1}_\bot}{\Nat_\bot},\] където
  \[\O\val{\vv{P}}(a_1,\dots,a_{m_1}) \dff
    \begin{cases}
      \eval{\vv{f}_1(\vv{a}_1,\dots,\vv{a}_{m_1})}, & \text{ако }\bot \not\in\{a_1,\dots,a_{m_1}\}\\
      \bot, & \text{ако }\bot \in \{a_1,\dots,a_{m_1}\}
    \end{cases}\]
  за произволни $a_1,\dots,a_{m_1} \in \Nat_\bot$.
\end{framed}

\begin{remark}
  Всъщност ние все още няма как да знаем, че за всяка програма $\vv{P}$,
  $\O\val{\vv{P}}$ е непрекъснато изображение.
  Този факт може да се докаже директно, но вместо това, ние ще видим, че
  $\O\val{\vv{P}} = \D\val{\vv{P}}$ и оттам ще получим непрекъснатостта на $\O\val{\vv{P}}$,
  защото от дефиницията на $\D\val{\vv{P}}$ е ясно, че то е непрекъснато изображение.
\end{remark}


% \begin{description}
% \item
%   % За всяко $a \in \Nat$,
%   \begin{figure}[h!]
%     \begin{prooftree}
%       \AxiomC{}
%       \RightLabel{\scriptsize{(const)}}
%       \UnaryInfC{$\vv{a} \Downarrow^0_P \vv{a}$}
%     \end{prooftree}
%   \end{figure}
% \item
%   \begin{figure}[h!]
%     \begin{prooftree}
%       \AxiomC{$\mu_1\Downarrow^{\ell_1}_P \vv{a}_1$}
%       \AxiomC{$\mu_2\Downarrow^{\ell_2}_P \vv{a}_2$}
%       \AxiomC{$a = a_1 + a_2$}
%       \RightLabel{\scriptsize{(plus)}}
%       \TrinaryInfC{$\mu_1 + \mu_2 \Downarrow^{\ell_1+\ell_2+1}_P \vv{a}$}
%     \end{prooftree}
%   \end{figure}
% \item
%   \begin{figure}[h!]
%     \begin{prooftree}
%       \AxiomC{$\mu_1\Downarrow^{\ell_1}_P \vv{a}_1$}
%       \AxiomC{$\mu_2\Downarrow^{\ell_2}_P \vv{a}_2$}
%       \AxiomC{$a = \texttt{eq}(a_1, a_2)$}
%       \RightLabel{\scriptsize{$(eq)$}}
%       \TrinaryInfC{$\mu_1\ \vv{==}\ \mu_2 \Downarrow^{\ell_1+\ell_2+1}_P \vv{a}$}
%     \end{prooftree}
%   \end{figure}
% \item
%   \begin{figure}[h!]
%     \begin{prooftree}
%       \AxiomC{$\mu_1\Downarrow^{\ell_1}_P \vv{a}_1$}
%       \AxiomC{$\mu_2 \Downarrow^{\ell_2}_P \vv{a}_2$}
%       \AxiomC{$\vv{a}_1 \not\equiv \vv{0}$}
%       \RightLabel{\scriptsize{(if$_\true$)}}
%       \TrinaryInfC{$\ifelse{\mu_1}{\mu_2}{\mu_3} \Downarrow^{\ell_1+\ell_2+1}_P \vv{a}_2$}
%     \end{prooftree}
%   \end{figure}  
% \item
%   \begin{figure}[h!]
%     \begin{prooftree}
%       \AxiomC{$\mu_1\Downarrow^{\ell_1}_P \vv{0}$}
%       \AxiomC{$\mu_3 \Downarrow^{\ell_3}_P \vv{a}_3$}
%       \RightLabel{\scriptsize{(if$_\false$)}}
%       \BinaryInfC{$\ifelse{\mu_1}{\mu_2}{\mu_3} \Downarrow^{\ell_1+\ell_3+1}_P \vv{a}_3$}
%     \end{prooftree}
%   \end{figure}
% \item
%   \begin{figure}[h!]
%     \begin{prooftree}
%       \AxiomC{$\tau_i[\vv{x}_1/\mu_1,\dots,\vv{x}_{m_i}/\mu_{m_i}] \Downarrow^{\ell}_P \vv{a}$}
%       \RightLabel{\scriptsize{(cbn)}}
%       \UnaryInfC{$\vv{f}_i(\mu_1,\dots,\mu_{m_i}) \Downarrow^{\ell+1}_P \vv{a}$}
%     \end{prooftree}
%   \end{figure}
% \end{description}

% \newpage

\begin{example}
  Нека за програмата \vv{P}:
  \begin{haskellcode}
    f(x,y) = if x == y then 0 else 1 + f(x,y+1)
  \end{haskellcode}
  да разгледаме няколко извода с правилата на операционната семантика по име.
\end{example}


%%% Local Variables:
%%% mode: latex
%%% TeX-master: "../sep"
%%% End:
