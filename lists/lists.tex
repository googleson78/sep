\chapter{Лениви спъсъци}
\marginpar{До известна степен следваме \cite[Глава 6]{bird-haskell}}

\Stefan{Дали не е по-добре вместо $[x,y,z]$ е съкратен запис за $x:y:z:[]$ ?}

\section{Основни понятия}
\begin{itemize}
\item
  Нека с $\nil$ да означаваме празния спъсък, т.е. единствения краен списък с дължина $0$.
\item
  {\bf Краен спъсък} с елементи естествените числа $a_0,\dots,a_{k-1}$
  ще записваме като $\pair{a_0,\dots,a_{k-1},\nil}$, т.е. това е елемент придандлежащ на множеството  
  \[\underbrace{\Nat \times \Nat \cdots \times \Nat}_{k} \times \{\nil\}.\]
  Дефинираме множеството от всички крайни списъци като
  \[\FinL \dff \bigcup_{k\geq 0} \Nat^k \times \{\nil\}.\]
\end{itemize}

Нека $P(l)$ е свойство на крайните списъци.
\begin{prooftree}
  \AxiomC{$P(\nil)$}
  \AxiomC{$(\forall a\in\Nat)(\forall l\in\FinL)[P(l) \implies P(\pair{a,l})]$}
  \BinaryInfC{$(\forall l\in \FinL)[P(l)]$}
\end{prooftree}

% \subsection{Частични списъци}

\marginpar{Това на практика са недовършени списъци}
{\bf Частичен списък} с елементи естествените числа $a_0,\dots,a_{k-1}$
ще записваме като $\pair{a_0,a_1,a_2,\dots,a_{k-1},\bot}$, т.е. това е елемент придандлежащ на множеството  
\[\underbrace{\Nat \times \Nat \cdots \times \Nat}_{k} \times \{\bot\}.\]
Дефинираме множеството от всички частични списъци като
\[\PartL \dff \bigcup_{k\geq 0} \Nat^k \times \{\bot\}.\]
Тогава $\bot$ може да се интерпретира като единствения частичен списък с дължина $0$.
Да видим какво ще стане ако изпълним следния ред:

\begin{haskellcode}
ghci> filter (<4) [1..]
[1,2,3                      -- == $(1:2:3:\bot)$
\end{haskellcode}
  
Програмата работи безкрайно дълго време след като вече е отпечатала първите три числа,
защото хаскел не знае, че след $3$ в безкрайния списък няма други числа по-малки от $4$.
Това означава, че резултатът от изпълнението на тази програма е частичния списък $\pair{1,2,3,\bot}$,

Нека $P(l)$ е свойство на частичните списъци.
\begin{prooftree}
  \AxiomC{$P(\bot)$}
  \AxiomC{$(\forall a\in\Nat)(\forall l\in\PartL)[P(l) \implies P( \pair{a,l} ) ]$}
  \BinaryInfC{$(\forall l\in \PartL)[P(l)]$}
\end{prooftree}


% \subsection*{Безкрайни списъци}

{\bf Безкраен списък} с елементи естествените числа $a_0,a_1,\dots$
ще записваме като $\pair{a_0,a_1,\dots}$, т.е. това е елемент придандлежащ на множеството  
\[\InfL \dff \Nat^\Nat = \{f:\Nat \to \Nat \mid f \text{ е тотална}\}.\]
Както вече знаем, на хаскел е много лесно да работим с безкрайни списъци.
\begin{haskellcode}
ghci> take 10 [1,3..]                 -- $a_n = 1 + (3-1)*n$
[1,3,5,7,9,11,13,15,17,19]
ghci> take 10 [x*x | x <- [0..]]      -- $a_n = n^2$
[0,1,4,9,16,25,36,49,64,81]
\end{haskellcode}

\marginpar{chain-complete property \cite[стр. 218]{bird-haskell}}
Нека $P(l)$ е {\em непрекъснато свойство} на частичните и безкрайни списъци, ако 
за всяка верига от частични списъци $(l_n)^\infty_{n=0}$ имаме, че $(\forall n)[P(l_n)]$,
то тогава имаме, че $P(\bigsqcup_n l_n)$.

\begin{prooftree}
  \AxiomC{$P(\bot)$}
  \AxiomC{$(\forall a\in\Nat)(\forall l\in\PartL)[P(l) \implies P(\pair{a,l})]$}
  \BinaryInfC{$(\forall l\in \PartL)[P(l)]$}
\end{prooftree}



% \begin{itemize}
% \item 
%   Нека с $\nil$ да означаваме празния спъсък, т.е. единствения краен списък с дължина $0$.
% \item
%   {\bf Краен спъсък} с елементи естествените числа $a_0,\dots,a_{k-1}$
%   ще записваме като $\pair{a_0,\dots,a_{k-1},\nil}$, т.е. това е елемент придандлежащ на множеството  
%   \[\underbrace{\Nat \times \Nat \cdots \times \Nat}_{k} \times \{\nil\}.\]
%   Дефинираме множеството от всички крайни списъци като
%   \[\FinL \dff \bigcup_{k\geq 0} \Nat^k \times \{\nil\}.\]
% \item
%   \marginpar{Това на практика са недовършени списъци}
%   {\bf Частичен списък} с елементи естествените числа $a_0,\dots,a_{k-1}$
%   ще записваме като $\pair{a_0,\dots,a_{k-1},\bot}$, т.е. това е елемент придандлежащ на множеството  
%   \[\underbrace{\Nat \times \Nat \cdots \times \Nat}_{k} \times \{\bot\}.\]
%   Дефинираме множеството от всички частични списъци като
%   \[\PartL \dff \bigcup_{k\geq 0} \Nat^k \times \{\bot\}.\]
%   Тогава $\bot$ може да се интерпретира като единствения частичен списък с дължина $0$.
%   Да видим какво ще стане ако изпълним следния ред:
  
%   \begin{haskellcode}
% ghci> filter (<4) [1..]
% [1,2,3
%   \end{haskellcode}
  
% Програмата работи безкрайно дълго време след като вече е отпечатала първите три числа,
% защото хаскел не знае, че след $3$ в безкрайния списък няма други числа по-малки от $4$.
% Това означава, че резултатът от изпълнението на тази програма е частичния списък $\pair{1,2,3,\bot}$,
% \item
%   {\bf Безкраен списък} с елементи естествените числа $a_0,a_1,\dots$
%   ще записваме като $\pair{a_0,a_1,\dots}$, т.е. това е елемент придандлежащ на множеството  
%   \[\InfL \dff \Nat^\Nat = \{f:\Nat \to \Nat \mid f \text{ е тотална}\}.\]
%   Както вече знаем, на хаскел е много лесно да работим с безкрайни списъци.
%   \begin{haskellcode}
% ghci> take 10 [1,3..]
% [1,3,5,7,9,11,13,15,17,19]
% ghci> take 10 [x*x | x <- [1..]]
% [1,4,9,16,25,36,49,64,81,100]
%   \end{haskellcode}
% \end{itemize}



\section{Област на Скот от списъци}


Дефинираме областта на Скот $L = (L,\sqsubseteq, \bot)$, където
\[L = \FinL \cup \PartL \cup \InfL,\]
а частичната наредба $\sqsubseteq$ е дефинирана следвайки правилата:
\begin{itemize}
\item
  $(a_0 : a_1 : \dots: a_{n-1} : \bot) \sqsubseteq (a_0 : a_1 : \dots : a_{n-1} : b_0 : \dots : b_{k-1} : \bot)$;
\item
  $(a_0 : a_1 : \dots : a_{n-1} : \bot) \sqsubseteq (a_0 : \dots : a_{n-1} : b_0 : \dots : b_{k-1} : \nil)$;
\item
  $(a_0 : \dots : a_{n-1} : \bot) \sqsubseteq (a_0 : \dots : a_{n-1} : b_0 : b_1 : \cdots)$.
\end{itemize}

% \begin{haskellcode}
% data LazyList = Nil | Cons Int LazyList deriving (Show)
% \end{haskellcode}

Според правилата, когато $n = 0$, получаваме, че $\bot$ е най-малкият елемент на $L$.
Обърнете внимание, че
\[(\forall \alpha_1,\alpha_2 \in \FinL \cup \InfL)[\alpha_1 \sqsubseteq \alpha_2 \iff \alpha_1 = \alpha_2].\]
Това означава, че например,
\begin{itemize}
% \item 
%   $\pair{0,1,2,\bot} \sqsubseteq \pair{0,1,2,\nil}$;
\item
  $(0:1:\nil) \not\sqsubseteq (0:1:2:\nil)$;
\item
  $\bot \sqsubseteq (0:\bot) \sqsubseteq (0:1:\bot) \sqsubseteq \cdots \sqsubseteq [0,1,\dots]$.
\end{itemize}

\begin{framed}
  \begin{figure}[H]
    \label{fig:lazy-list}
    \centering
    \begin{tikzpicture}[shorten >=1pt,->]
      \tikzstyle{vertex}=[circle,minimum size=17pt,inner sep=0pt]
      
      \node[vertex] (bot) at (3,0) {$\bot$};
      \node[vertex] (0) at (2,1) {$\nil$};
      \node[vertex] (1) at (4,1) {$(0:\bot)$};

      \node[vertex] (11) at (3,2.5) {$(0:\nil)$};
      \node[vertex] (12) at (5,2.5) {$(0:1:\bot)$};

      \node[vertex] (122) at (3.5,4.3) {$(0:1:\nil)$};
      \node[vertex] (123) at (7,4) {$(0:1:2:\bot)$};
      
      \node[vertex] (1233) at (4.6,6) {$(0:1:2:\nil)$};
      \node[vertex] (dots) at (9,5.5) {};

      \draw (bot) -- node[below left]{$\scriptstyle{\sqsupseteq}$} (0);
      \draw (bot) -- node[below right]{$\scriptstyle{\sqsubseteq}$} (1);
      \draw (1) -- node[below left]{$\scriptstyle{\sqsupseteq}$} (11);
      \draw (1) -- node[below right]{$\scriptstyle{\sqsubseteq}$} (12);
      \draw (12) -- node[below left]{$\scriptstyle{\sqsupseteq}$} (122);
      \draw (12) -- node[below right]{$\scriptstyle{\sqsubseteq}$} (123);
      \draw (123) -- node[below left]{$\scriptstyle{\sqsupseteq}$} (1233);
      \draw[dashed] (123) -- node[below right]{$\scriptstyle{\sqsubseteq}$} (dots);
    \end{tikzpicture}    
    \caption{Графично представяне на $\sqsubseteq$ върху част от $L$}
  \end{figure}
\end{framed}

\subsection*{Канонични апроксимации}

\marginpar{Тук следваме \cite[стр. 218]{bird-haskell}}
\index{канонична апроксимация}
За всеки елемент $l \in L$, дефинираме неговата {\bf $n$-та канонична апроксимация} $l\upharpoonright{n}$, където:
\begin{align*}
  & l \upharpoonright{0} \dff \bot\\
  & \nil \upharpoonright{(n+1)} \dff \nil\\
  % & l \upharpoonright{(n+1)} \dff \texttt{cons}(\texttt{car}(l), l\upharpoonright{n}).
  & (a:l) \upharpoonright{(n+1)} \dff a:(l\upharpoonright{n}).
\end{align*}
Имаме свойството, че за всеки списък $l \in L$, 
\[n \leq n' \implies l\upharpoonright n \sqsubseteq l\upharpoonright n'.\]
Това означава, че $(l\upharpoonright n)^{\infty}_{n=0}$ е верига.
Лесно се съобразява, че 
\[l = \bigsqcup_n (l \upharpoonright n).\]
Друго важно свойство, което имаме е, че 
\[l_1 \sqsubseteq l_2 \implies l_1 \upharpoonright n \sqsubseteq l_2 \upharpoonright n.\]
\begin{itemize}
\item 
  Да обърнем внимание на апроксимациите на крайни списъци.
  Нека $l = \pair{a_0,a_1,\dots,a_{n-1},\nil}$. Тогава според дефиницията:
  \begin{align*}
    & l \upharpoonright {0} = \bot,\\
    & l \upharpoonright k = \pair{a_0,\dots,a_{k-1},\bot} \text{, за }k = 1,\dots, n,\\
    & l \upharpoonright k = l \text{, за }k\geq n+1.
  \end{align*}

\item
  Да разгледаме няколко примера.
  \begin{itemize}
  \item 
    Нека $l = (0:1:\nil) \in \FinL$. Тогава 
    \[\underbrace{\bot}_{l\upharpoonright 0} \sqsubseteq \underbrace{(0:\bot)}_{l\upharpoonright 1} \sqsubseteq \underbrace{(0:1:\bot)}_{l\upharpoonright 2} \sqsubseteq \underbrace{(0:1:\nil)}_{l\upharpoonright 3} = \underbrace{(0:1:\nil)}_{l\upharpoonright 4} = \cdots\]
  \item
    Нека $l = (0:1:\bot) \in \PartL$. Тогава
    \[\underbrace{\bot}_{l\upharpoonright 0} \sqsubseteq \underbrace{(0:\bot)}_{l\upharpoonright 1} \sqsubseteq \underbrace{(0:1:\bot)}_{l\upharpoonright 2} = \underbrace{(0:1:\bot)}_{l\upharpoonright 3} = \underbrace{(0:1:\bot)}_{l\upharpoonright 4} = \cdots\]
  \item
    Нека $l = (0:1:2:\cdots) \in \InfL$. Тогава
    \[\underbrace{\bot}_{l\upharpoonright 0} \sqsubseteq \underbrace{(0:\bot)}_{l\upharpoonright 1} \sqsubseteq \underbrace{(0:1:\bot)}_{l\upharpoonright 2} \sqsubseteq \underbrace{(0:1:2:\bot)}_{l\upharpoonright 3} \sqsubseteq \underbrace{(0:1:2:3:\bot)}_{l\upharpoonright 4} \sqsubseteq \cdots\]
  \end{itemize}
\end{itemize}


Можем да дефинираме апроксимацията на списъци по следния начин на хаскел:
\begin{haskellcode}
approx _      0         = undefined         -- $l \upharpoonright 0 = \bot$
approx []     n | n > 0 = []                -- $\nil \upharpoonright (n+1) = \nil$
approx (x:xs) n | n > 0 = x:approx xs (n-1) -- $(a:l) \upharpoonright (n+1) = (a: (l\upharpoonright n))$
\end{haskellcode}

\begin{problem}
  \label{prob:approx}
  Да дефинираме изображението 
  \[\texttt{approx}_n(l) \dff l \upharpoonright n.\]
  Докажете, че $\texttt{approx}_n \in \Cont{L}{L}$.
\end{problem}

\begin{haskellcode}
ghci> approx [1..10] 0    --  == $\bot$
*** Exception: Prelude.undefined
ghci> approx [1..10] 10   --  == (1:2:3:4:5:6:7:8:9:10:$\bot$)
[1,2,3,4,5,6,7,8,9,10*** Exception: Prelude.undefined  
ghci> approx [1..10] 11   --  == (1:2:3:4:5:6:7:8:9:10:$\nil$)
[1,2,3,4,5,6,7,8,9,10]
ghci> approx [1..10] 101  --  == (1:2:3:4:5:6:7:8:9:10:$\nil$)
[1,2,3,4,5,6,7,8,9,10]
\end{haskellcode}

\Stefan{Да се обасни каква е разликата между take и approx}

\begin{prop}
  Областта на Скот $L$ е алгебрична като крайните елементи $K(L) = \FinL$ и $\PartL$.
\end{prop}
% \begin{hint}
%   Банално.
% \end{hint}

\section{Алгебричност на непрекъснатите изображения}

Целта ни тук ще бъде да докажем, че $\Cont{L}{L}$ е алгебрична област на Скот.
За да направим това трябва да намерим {\em крайните елементи} на $\Cont{L}{L}$.
Понеже крайните елементи на $L$ са $\FinL$ и $\PartL$, то е логично да предположим,
че крайните елементи на $\Cont{L}{L}$ са дефинирани чрез крайно много елементи на $K(L)$.

\begin{problem}
  Докажете, че за всяко $f \in \Cont{L}{L}$ е изпълнено, че:
  \[f(l) = \bigsqcup_n \{f(l \upharpoonright n)\upharpoonright n\}.\]
\end{problem}
\begin{hint}
  Използвайте \Th{double-chain}.
\end{hint}

\begin{problem}
  За произволно изображение $f \in \Cont{L}{L}$ и число $n$, да ознчим с $f \upharpoonright n$ изображението, където
  \[(f\upharpoonright n)(l) \dff f(l\upharpoonright n)\upharpoonright n.\]
  Докажете, че $f\upharpoonright n \in \Cont{L}{L}$.
\end{problem}
\begin{hint}
  Използвайте \Problem{approx}.
\end{hint}

\Stefan{Това не е ясно!}
Можем да считаме изображенията $f\upharpoonright n$ като апроксимации на $f$.
Обаче те не са крайни апроксимации, защото за да дефинираме $f \upharpoonright n$
се нуждаем от безкрайна информация. Това е така, защото има безкрайно много списъци с дължина $\leq n$.
Оттук следва, че изображенията $f \upharpoonright n$ не са крайните елементи на $\Cont{L}{L}$.

\begin{problem}
  Нека $a \in K(\A)$ и $b \in K(\B)$.
  Да дефинираме изображението
  \[\theta(x) \dff
  \begin{cases}
    b, & \text{ ако } a \sqsubseteq x\\
    \bot, & \text{ иначе}.
  \end{cases}\]
  Докажете, че:
  \begin{enumerate}[1)]
  \item 
    $\theta \in \Cont{\A}{\B}$.
  \item
    ако $f \in \Cont{\A}{\B}$, то $\theta \sqsubseteq f \iff b \sqsubseteq f(a)$.
  \end{enumerate}
\end{problem}
\begin{hint}
  
\end{hint}

Сега трябва да видим кога и по какъв начин можем да правим крайни обединения на такива изображения.
Това не е толкова просто, защото искаме полученото изображение също да бъде непрекъснато.

Две редици от елементи на $K(L)$
\begin{align*}
  \bar{a} & = (a_0,\dots,a_{n-1}),\\
  \bar{b} & = (b_0,\dots,b_{n-1})
\end{align*}
се наричат {\bf съвместими}, ако е изпълнено свойството:
\begin{equation}
  \label{eq:mon}
  (\forall i,j < n)[a_i \sqsubseteq a_j \implies b_i \sqsubseteq b_j].
\end{equation}

\begin{example}
  Редиците от крайни елементи
  \begin{align*}
    \bar{a} & = (\ (0:\bot),\ (1:\bot),\ (0:1:\nil)\ ),\\
    \bar{b} & = (\ (1:2:\nil),\ \bot,\ (1:2:3:\nil)\ ),
  \end{align*}
  не са съвместими, защото
  \[(0:\bot) \sqsubseteq (0:1:\nil)\text{, но } (1:2:\nil) \not\sqsubseteq (1:2:3:\nil).\]
\end{example}

Възможно е някои от елементите на $\bar{a}$ да са несравними помежду си, но обърнете внимание, че за произволен елемент $x$, 
всички елементи на множеството 
\[A \dff \{a_i \mid i < n\ \&\ a_i \sqsubseteq x\}\]
са сравними помежду си.
Това означава, че 
\[A = \{a_{i_0} \sqsubseteq a_{i_1} \sqsubseteq \cdots \sqsubseteq a_{i_k}\},\] за някое $k$,
и следователно, $A$ притежава точна горна граница.
Естествено, множеството $A$ може и да е празно. Тогава точната горна граница на $A$ ще бъде $\bot$,
защото по дефиниция $\bigsqcup\emptyset = \bot$.

Да разгледаме изображението 
\begin{align*}
  \theta_{\bar{a},\bar{b}}(x) & \dff \bigsqcup_{i<n}\{b_i \mid a_i \sqsubseteq x\}\\
  & = \begin{cases}
    b_0, & \text{ ако } a_0 = \bigsqcup_{i<n}\{a_i \mid a_i \sqsubseteq x\}\\
    \vdots & \\
    b_{n-1}, & \text{ ако } a_{n-1} = \bigsqcup_{i<n}\{a_i \mid a_i \sqsubseteq x\}\\
    \bot, & \text{ иначе}.
  \end{cases}
\end{align*}

\Stefan{На картинка нещата не изглеждат толкова сложни.}

\begin{prop}
  Ако $\bar{a}$ и $\bar{b}$ са съвместими редици от крайни елементи на $L$, то имаме свойствата:
  \begin{enumerate}[1)]
  \item 
    $\theta_{\bar{a},\bar{b}} \in \Cont{L}{L}$;
  \item
    За всяко $f \in \Cont{L}{L}$, 
    \[\theta_{\bar{a},\bar{b}} \sqsubseteq f\ \iff\ (\forall i < n)[b_i \sqsubseteq f(a_i)];\]
  \item
    $\theta_{\bar{a},\bar{b}}$ е компактен елемент в $\Cont{L}{L}$.
  \end{enumerate}
\end{prop}
\begin{proof}
  \begin{enumerate}[1)]
  \item 
    Лесно. Използва се, че $\bar{a}$ са крайни елементи.
  \item
    Използва се само монотонността на $f$.
  \item
    Тук съществено се използва, че $\bar{b}$ са крайни елементи.

    Нека $(f_r)^\infty_{r=0}$ е верига в $\Cont{L}{L}$ и 
    $\theta_{\bar{a},\bar{b}} \sqsubseteq \bigsqcup_r f_r$.
    От $2)$, това означава, че за $i < n$,
    \[b_i \sqsubseteq (\bigsqcup_r f_r)(a_i) = \bigsqcup_r \{f_r(a_i)\}.\]
    Понеже $b_i$ е краен елемент, то съществува индекс $r_i$, за който
    \[b_i \sqsubseteq f_{r_i}(a_i).\]
    Накрая взимаме $r = \max\{r_1,\dots,r_n\}$.
    Получаваме, че за $i < n$,
    \[b_i \sqsubseteq f_{r}(a_i).\]
    Тогава отново от 2) следва, че $\theta_{\bar{a},\bar{b}} \sqsubseteq f_r$.
  \end{enumerate}
\end{proof}

\begin{framed}
\begin{prop}
  Областта на Скот $\Cont{L}{L}$ е алгебрична.
\end{prop}
\end{framed}
\begin{proof}
  Да разгледаме произволен елемент $f \in \Cont{L}{L}$.
  Ще покажем, че съществува верига от крайни елементи $(\theta_n)^\infty_{n=0}$, за която $f = \bigsqcup_n \theta_n$.
  Нека подредим елементите на $K(L)$ в една редица $a_0,a_1,\dots$,
  което можем да направим, защото те са изброимо много.
  Да означим 
  \begin{align*}
    \bar{a}_n & = (a_0,a_1,\dots,a_{n})\\
    \bar{b}_n & = (f(a_0)\upharpoonright n, \dots, f(a_{n})\upharpoonright n).
  \end{align*}
  Нека да положим
  \[\theta_n \dff \theta_{\bar{a}_n,\bar{b}_n}.\]

  Лесно се проверява, че $(\theta_n)^\infty_{n=0}$ е верига, защото $f$ е монотонно изображение.
  
  Трябва да проверим, че $f = \bigsqcup_n \theta_n$. 
  От дефиницията на $\theta_n$ е ясно, че $\theta_n \sqsubseteq f$ за всяко $n$.
  Следователно, $\bigsqcup_n \theta_n \sqsubseteq f$.
  За другата посока, нека разгледаме произволен елемент $l \in L$.
  Ще докажем, че $f(l) \sqsubseteq (\bigsqcup_n \theta_n)(l)$.
  Да разгледаме верига от елементи на $K(L)$, $(a_{i_n})^\infty_{n=0}$, за която $i_0 < i_1 < \cdots$
  и $\bigsqcup_n a_{i_n} = l$. Знаем, че такава съществува, защото $L$ е алгебрична област на Скот.
  \Stefan{Трябва някъде да сложа твърдение, че ако $\bigsqcup_n a_n = b$, то за произволна подредица
  $\bigsqcup_n a_{i_n} = b$.}
  Тогава 
  \begin{align*}
    f(l) & = \bigsqcup_n \{f(a_{i_n}) \upharpoonright n\}\\
         & \sqsubseteq \bigsqcup_n \{f(a_{i_n}) \upharpoonright i_n \} & (i_n \geq n)\\
         & = \bigsqcup_n \{\theta_{i_n}(a_{i_n})\} & (\text{от деф. на }\theta_{i_n})\\
         & = \bigsqcup_n\{ \bigsqcup_m \{\theta_{i_n}(a_{i_m})\}\} & (\text{\Th{double-chain}})\\
         & = \bigsqcup_n \{\theta_{i_n}(\bigsqcup_m a_{i_m})\} & (\theta_{i_n} \in \Cont{L}{L})\\
         & = \bigsqcup_n \{\theta_{i_n}(l)\} & (l = \bigsqcup_m a_{i_m})\\
         & = (\bigsqcup_n \theta_{i_n})(l) \\
         & \sqsubseteq (\bigsqcup_n \theta_n)(l).
  \end{align*}
\end{proof}


\section{Задачи}

\Stefan{Дали да дефинираме изображение cons, което да докажа, че е непрекъснато?}

\begin{problem}
  Да се даде пример за $f \in \Mon{L}{L}$, но $f \not\in \Cont{L}{L}$.
\end{problem}

\begin{problem}
  Да се даде пример за $f \in \Strict{L}{L}$, но $f \not\in \Mon{L}{L}$.
\end{problem}

\begin{problem}
  Да разгледаме изображението $f \in \Cont{L}{L}$, където
  \[f(l) \dff (0:l).\]
  Намерете $\lfp(f)$.
\end{problem}
\begin{solution}
  Прилагаме \Th{knaster-tarski}.
  \begin{itemize}
  \item 
    $l_0 = \bot$;
  \item
    $l_1 = f(\bot) = (0:\bot)$;
  \item
    $l_2 = f(l) = (0:0:\bot)$;
  \end{itemize}
  Така получаваме, че
  \[l_n = \pair{\underbrace{0,0,\dots,0}_{n},\bot} \in \PartL.\]
  Тогава 
  \[\lfp(f) = \bigsqcup_n l_n = \pair{0,0,\dots}.\]



\begin{haskellcode}
ghci> let f(l) = (0:l)
ghci> let approx = undefined : [f(x) | x <- approx]
ghci> approx !! 10
[0,0,0,0,0,0,0,0,0,0*** Exception: Prelude.undefined  
\end{haskellcode}

\end{solution}

\begin{problem}
  Да разгледаме изображението 
  \[\Gamma \in \Cont{\Cont{\Nat_\bot\times L}{L}}{\Cont{\Nat_\bot\times L}{L}},\] където
  \[\Gamma(f)(n,l) =
  \begin{cases}
    \bot, & n = \bot\\
    \nil, & n = 0\\
    \nil, & n > 0\ \&\ l = \nil\\
    \bot, & n > 0\ \&\ l = \bot\\
    a : f(l'), & n > 0\ \&\ l = (a:l')\\
  \end{cases}\]
  Намерете $\lfp(\Gamma)$.
\end{problem}


\begin{haskellcode}
ghci> let l = (0:l)
ghci> take 10 l
[0,0,0,0,0,0,0,0,0,0]
ghci> take 10 [1,2,3]
[1,2,3]
ghci> take 0 undefined
[]
ghci> take 2 [1,2,undefined]
[1,2]
ghci> take 3 [1,2,undefined]
[1,2,*** Exception: Prelude.undefined
ghci> take undefined []
*** Exception: Prelude.undefined
\end{haskellcode}


\Stefan{Да се разгледа функцията drop n l}

% \begin{problem}
%   Да разгледаме оператора $\Gamma:[\Nat\to L] \to [\Nat \to L]$, където
%   \[\Gamma(f)(a,b) = \pair{a, f(b,a+b)}.\]
%   Намерете безкрайния списък $\lfp(\Gamma)(1,1)$.
% \end{problem}

\begin{problem}
  Да разгледаме оператора $\Gamma:[L\times L \to L] \to [L\times L \to L]$, където
  \[\Gamma(f)(l_1,l_2) = 
  \begin{cases}
    \bot, & \text{ ако }l_1 = \bot \\
    l_2, & \text{ ако }l_1 = \nil\\
    a:f(l'_1,l_2) & \text{ ако } l_1 = a:l'_1.
  \end{cases}\]
  Да означим $\texttt{conc} \dff \lfp(\Gamma)$.
  Докажете, че
  \[(\forall l_1 \in \FinL)(\forall l_2,l_3 \in L)[\texttt{conc}(l_1, \texttt{conc}(l_2,l_3)) = \texttt{conc}(\texttt{conc}(l_1, l_2), l_3))]\]
\end{problem}

%%% Local Variables:
%%% mode: latex
%%% TeX-master: "../sep-notes"
%%% End:


%%% Local Variables:
%%% mode: latex
%%% TeX-master: "../sep"
%%% End:
