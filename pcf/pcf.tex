\chapter{Езикът PCF}\label{ch:pcf}
\marginpar{Plotkin’s language PCF is often called the \emph{E. coli} of programming languages, the  subject  of  countless  studies  of  language  concept.}

\marginpar{Тази глава се оповава основно на \cite[Глава 19]{practical-foundations} и \cite{cambridge-den-sem}.}

\newcommand{\nat}{\vv{nat}}
\newcommand{\type}[2]{\vv{#1}~:~\vv{#2}}
\newcommand{\lamb}[3]{\lambda~\type{#1}{#2}~.~{#3}}


\section{Синтаксис}

\newcommand{\fix}{\texttt{fix}}
\newcommand{\fv}{\texttt{fv}}

\begin{itemize}
\item
  Типове
  \[\vv{a} ::= \vv{nat}\ |\ \vv{a} \to \vv{a}.\]
\item
  Изрази
  \[\tau ::= \vv{n}\ |\ \vv{x}\ |\ \tau + \tau\ |\ \tau\ \vv{==}\ \tau\ |\ \ifelse{\tau}{\tau}{\tau}\ |\ \tau\tau\ |\ \lambda \vv{x:a}.\tau\ |\ \fix(\tau).\]
  Ще казваме, че един израз $\tau$ е {\bf затворен}, ако $\fv(\tau) = \emptyset$.
  В противен случай, ще казваме, че изразът е {\bf отворен}.
\item
  \marginpar{или в канонична форма?}
  Ще казваме, че един израз е {\bf стойност}, ако той е съставен по следния начин:
  \[\vv{v} ::= \vv{n}\ |\ \lambda\type{x}{a}\ .\ \vv{v}.\]
\item
  Контексти
  \[\Gamma ::= \emptyset\ |\ \Gamma,\type{x}{a}.\]
\end{itemize}

Да обърнем внимание, че ние няма да правим разлика между два израза, които са $\alpha$-еквивалентни, т.е.
ще считаме, че $\lambda x:a. (x+1)$ е същият израз като $\lambda y:a. (y+1)$.

Понеже тук вече имам свободни и свързани променливи, трябва да дефинираме точно какво означава това.
\[\fv:\mathcal{E} \to \mathcal{V}\]
Ще дефинираме функцията $\texttt{fv}$ със структурна индукция по построението на термовете.

\begin{itemize}
\item
  $\fv(\vv{n}) = \emptyset$;
\item
  $\fv(\vv{x}) = \{\vv{x}\}$;
\item
  $\fv(\tau_1 + \tau_2) = \fv(\tau_1\ \vv{==}\ \tau_2) = \fv(\tau_1\tau_2) = \fv(\tau_1) \cup \fv(\tau_2)$;
\item
  $\fv(\ifelse{\tau_1}{\tau_2}{\tau_3}) = \fv(\tau_1) \cup \fv(\tau_2) \cup \fv(\tau_3)$;
\item
  $\fv(\lambda \type{x}{a}\ .\ \tau) = \fv(\tau) \setminus \{x\}$.
\item
  $\fv(\fix(\tau)) = \fv(\tau)$;
\end{itemize}



%%% Local Variables:
%%% mode: latex
%%% TeX-master: "../sep"
%%% End:


\section{Типове}

Всяка \emph{крайна} функция от вида $\Gamma : \mathcal{V} \to \mathcal{T}$
ще наричаме {\bf типов контекст}.
Ние искаме да работим само с коректно типизирани изрази.
Ако в изразите имаме свободни променливи, то дали един израз е коректно типизиран ще зависи от типовия контекст.

\begin{prooftree}
  \AxiomC{}
  \RightLabel{\scriptsize{(const)}}
  \UnaryInfC{$\Gamma \vdash \vv{n} : \vv{nat}$}
\end{prooftree}

\begin{prooftree}
  \AxiomC{$\vv{x} \in \texttt{dom}(\Gamma)$}
  \AxiomC{$\Gamma(\vv{x}) = \vv{a}$}
  \RightLabel{\scriptsize{(var)}}
  \BinaryInfC{$\Gamma \vdash \vv{x} : \vv{a}$}
\end{prooftree}

\begin{prooftree}
  \AxiomC{$\Gamma \vdash \tau_1:\vv{nat}$}
  \AxiomC{$\Gamma \vdash \tau_2:\vv{nat}$}
  \RightLabel{\scriptsize{(plus)}}
  \BinaryInfC{$\Gamma \vdash \tau_1 + \tau_2 : \vv{nat}$}
\end{prooftree}

\begin{prooftree}
  \AxiomC{$\Gamma \vdash \tau_1:\vv{nat}$}
  \AxiomC{$\Gamma \vdash \tau_2:\vv{nat}$}
  \RightLabel{\scriptsize{(eq)}}
  \BinaryInfC{$\Gamma \vdash \tau_1\ \vv{==}\ \tau_2 : \vv{nat}$}
\end{prooftree}

\begin{prooftree}
  \AxiomC{$\Gamma \vdash \tau_1:\vv{nat}$}
  \AxiomC{$\Gamma \vdash \tau_2:\vv{a}$}
  \AxiomC{$\Gamma \vdash \tau_3:\vv{a}$}
  \RightLabel{\scriptsize{(if)}}
  \TrinaryInfC{$\Gamma \vdash \ifelse{\tau_1}{\tau_2}{\tau_3} : \vv{a}$}
\end{prooftree}

\begin{prooftree}
  \AxiomC{$\Gamma \vdash \tau_1:\vv{a}\to\vv{b}$}
  \AxiomC{$\Gamma \vdash \tau_2:\vv{a}$}
  \RightLabel{\scriptsize{(app)}}
  \BinaryInfC{$\Gamma \vdash \tau_1\tau_2 : \vv{b}$}
\end{prooftree}

\begin{prooftree}
  \AxiomC{$\Gamma \vdash \tau:\vv{a}\to\vv{a}$}
  \RightLabel{\scriptsize{(fix)}}
  \UnaryInfC{$\Gamma \vdash \fix(\tau) : \vv{a}$}
\end{prooftree}

\begin{prooftree}
  \AxiomC{$\vv{x} \not\in\vv{dom}(\Gamma)$}
  \AxiomC{$\Gamma, \type{x}{a} \vdash \tau:\vv{b}$}
  \RightLabel{\scriptsize{(lambda)}}
  \BinaryInfC{$\Gamma \vdash \lambda \type{x}{a}\ .\ \tau : \vv{a} \to \vv{b}$}
\end{prooftree}

Ако имаме затворен израз $\tau$, то ще пишем $\tau : \vv{a}$ вместо $\emptyset \vdash \tau : \vv{a}$.

\begin{proposition}
  Ако $\Gamma \vdash \tau : \vv{a}$, то $\fv(\tau) \subseteq \vv{dom}(\Gamma)$.
  Ако $\Gamma \vdash \tau : \vv{a}$ и $\Gamma \vdash \tau : \vv{b}$, то $\vv{a} \equiv \vv{b}$.
\end{proposition}
\begin{proof}
  Доказателството протича с индукция по построението на термовете.
  
\end{proof}

\begin{corollary}
  Всеки затворен терм има най-много един тип.
\end{corollary}


\begin{proposition}
  Ако $\Gamma \vdash \tau : \vv{a}$ и $\Gamma, \type{x}{a} \vdash \rho : \vv{b}$,
  то $\Gamma \vdash \rho[\vv{x}/\tau] : \vv{b}$.
\end{proposition}



%%% Local Variables:
%%% mode: latex
%%% TeX-master: "../sep"
%%% End:


\section{Операционна семантика}

Дефинираме релацията $\Downarrow_{\vv{a}}$ върху затворените изрази и стойностите.

\begin{prooftree}
  \AxiomC{$\type{v}{a}$}
  \RightLabel{\scriptsize{(val)}}
  \UnaryInfC{$\vv{v} \Downarrow^0_{\vv{a}} \vv{v}$}
\end{prooftree}

\begin{prooftree}
  \AxiomC{$\tau_1 \Downarrow^{\ell_1}_{\vv{nat}} \vv{n}_1$}
  \AxiomC{$\tau_2 \Downarrow^{\ell_2}_{\vv{nat}} \vv{n}_2$}
  \AxiomC{$n = \texttt{eq}(n_1,n_2)$}
  \RightLabel{\scriptsize{(eq)}}
  \TrinaryInfC{$\tau_1\ \vv{==}\ \tau_2 \Downarrow^{\ell_1+\ell_2+1}_{\vv{nat}} \vv{n}$}
\end{prooftree}

\begin{prooftree}
  \AxiomC{$\tau_1 \Downarrow_{\vv{nat}} \vv{n}_1$}
  \AxiomC{$\tau_2 \Downarrow_{\vv{nat}} \vv{n}_2$}
  \AxiomC{$n = n_1 + n_2$}
  \RightLabel{\scriptsize{(plus)}}
  \TrinaryInfC{$\tau_1\ \vv{+}\ \tau_2 \Downarrow_{\vv{nat}} \vv{n}$}
\end{prooftree}


\begin{prooftree}
  \AxiomC{$\tau_1 \Downarrow_{\vv{nat}} \vv{0}$}
  \AxiomC{$\tau_3 \Downarrow_{\vv{a}} \vv{v}$}
  \RightLabel{\scriptsize{(if$_0$)}}
  \BinaryInfC{$\ifelse{\tau_1}{\tau_2}{\tau_3} \Downarrow_{\vv{a}} \vv{v}$}
\end{prooftree}

\begin{prooftree}
  \AxiomC{$\tau_1 \Downarrow_{\vv{nat}} \vv{n}$}
  \AxiomC{$\tau_2 \Downarrow_{\vv{a}} \vv{v}$}
  \AxiomC{$\vv{n} \not\equiv \vv{0}$}
  \RightLabel{\scriptsize{(if$^+$)}}
  \TrinaryInfC{$\ifelse{\tau_1}{\tau_2}{\tau_3} \Downarrow_{\vv{a}} \vv{v}$}
\end{prooftree}

\begin{prooftree}
  \AxiomC{$\tau_1 \Downarrow_{\vv{a}\to\vv{b}} \lamb{x}{a}{\tau'_1}:\vv{b}$}
  \AxiomC{$\tau'_1[x/\tau_2] \Downarrow_{\vv{b}} \vv{v}$}
  \RightLabel{\scriptsize{(cbn)}}
  \BinaryInfC{$\tau_1 \tau_2 \Downarrow_{\vv{b}} \vv{v} $}
\end{prooftree}

\begin{prooftree}
  \AxiomC{$\tau\ \fix(\tau) \Downarrow_{\vv{a}} \vv{v}$}
  \RightLabel{\scriptsize{(fix)}}
  \UnaryInfC{$\fix(\tau) \Downarrow_{\vv{a}} \vv{v} $}
\end{prooftree}

\begin{lemma}
  За произволен затворен терм $\tau$ и стойности $\vv{v}$ и $\vv{u}$,
  ако $\tau \Downarrow_{\vv{a}} \vv{v}$ и $\tau \Downarrow_{\vv{a}} \vv{u}$, то $\vv{v} \equiv \vv{u}$.
\end{lemma}

\begin{example}
  \[\Omega_{\vv{a}} \equiv \fix(\lamb{x}{a}{\vv{x}}).\]
  Лесно се вижда, че:
  \begin{itemize}
  \item
    $\emptyset \vdash \Omega_{\vv{a}} : \vv{a}$.
  \item
    За всяка стойност $\vv{v}$,
    $\Omega_{\vv{a}} \not\Downarrow_{\vv{a}} \vv{v}$.
  \end{itemize}
\end{example}

\begin{example}
  Нека $\vv{a} = \vv{nat} \to (\vv{nat} \to \vv{nat})$ и 
  \[\tau \equiv \fix(\lamb{f}{a}{\lamb{x}{nat}{\lamb{y}{nat}{\ifelse{\vv{y == 0}}{0}{\vv{x + (f x (y-1))} }}}}).\]
  Лесно се вижда, че
  \[\tau:\vv{nat} \to (\vv{nat} \to \vv{nat}).\]
  Също така за всяко $n$ и $k$, ако $m = n + k$, то
  \[\tau\ \vv{n}\ \vv{k} \Downarrow_{\vv{nat}} \vv{m}.\]
  Нека сега
  \[\rho \equiv \lamb{g}{a}{\fix(\lambda \vv{f} : \vv{nat}\to\vv{nat}\ .\ \lamb{x}{nat}{\ifelse{\vv{x == 0}}{\vv{1}}{\vv{g x f(x-1)}}}}).\]
  Лесно се вижда, че
  \[ \rho : \vv{a} \to (\vv{nat} \to \vv{nat}).\]
  Също така, за всяко $n$, ако $k = n!$, то
  \[ \rho\ \tau\ \vv{n} \Downarrow_{\vv{nat}} \vv{k}.\]
\end{example}

\begin{haskellcode}
> tm = \f x y -> if y == 0 then 0 else x + (f x (y-1))
> fix f = f (fix f)
> times = fix tm
> times 2 3
6
>times 10 11
110
> fct = \g -> fix(\f x -> if x == 0 then 1 else g x (f (x-1)))
> fact = fct times
> fact 5
120
\end{haskellcode}

\begin{problem}
  \marginpar{\cite[стр. 104]{types-programming-languages}}
  Докажете или опровергайте дали е възможно да съществува контекст $\Gamma$ и тип $\vv{a}$, такива че
  \[\Gamma \vdash \type{xx}{a}.\]
\end{problem}


%%% Local Variables:
%%% mode: latex
%%% TeX-master: "../sep"
%%% End:

\newpage
\section{Денотационна семантика с предаване на параметрите по име}

Семантиката на всеки тип ще бъде област на Скот както следва:
\begin{align*}
  & \val{\vv{nat}} \dff \Nat_\bot\\
  & \val{\vv{a} \to \vv{b}} \dff \Cont{\val{\vv{a}}}{\val{\vv{b}}}.
\end{align*}
\marginpar{Да напомним, че \[\emptyset_\bot = (\{\bot\},\sqsubseteq,\bot).\]}
\marginpar{От Раздел~\ref{subsect:domains:product} знаем, че $\val{\Gamma}$ е област на Скот.}
За един типов контекст $\Gamma$, дефинираме $\val{\Gamma}$ по следния начин:
\begin{itemize}
\item
  Ако $\Gamma = \emptyset$, то $\val{\Gamma} = \emptyset_\bot$;
\item
  Ако $\Gamma = \Gamma', \vv{x:a}$, то $\val{\Gamma} = \val{\Gamma'} \times \val{\vv{a}}$.
\end{itemize}
Например, ако $\Gamma = \vv{x}_1 : \vv{a}_1,\ \vv{x}_2 : \vv{a}_2,\ \vv{x}_3 : \vv{a}_3$, то
\[\val{\Gamma} = (\val{\vv{a}_1} \times \val{\vv{a}_2})\times \val{\vv{a}_3}.\]

Сега трябва да дефинираме семантика на термовете.
За всеки терм, за който $\fv(\tau) \subseteq \texttt{dom}(\Gamma)$ и
за произволни $\overline{u} \in \val{\Gamma}$, дефинираме неговата стойност $\val{\tau}_\Gamma(\overline{u})$ по следния начин:
\begin{itemize}
\item
  Нека $\tau \equiv \vv{n}$. Тогава
  \[\val{\vv{n}}_\Gamma(\overline{u}) \dff n.\]
\item
  Нека $\tau \equiv \vv{x}_i$. Тогава
  \[\val{\vv{x}_i}_\Gamma(\overline{u}) \dff u_i.\]
\item
  \marginpar{За $\texttt{plus}$ вижте Раздел~\ref{subsect:rec:term-value}.}
  Нека $\tau \equiv \tau_1 + \tau_2$. Тогава
  \[\val{\tau_1 + \tau_2}_\Gamma(\overline{u}) \dff \texttt{plus}(\val{\tau_1}_\Gamma(\overline{u}), \val{\tau_2}_\Gamma(\overline{u})).\]
\item
  \marginpar{За $\texttt{eq}$ вижте Раздел~\ref{subsect:rec:term-value}.}
  Нека $\tau \equiv \tau_1\ \vv{==}\ \tau_2$. Тогава
  \[\val{\tau_1\ \vv{==}\ \tau_2}_\Gamma(\overline{u}) \dff \texttt{eq}(\val{\tau_1}_\Gamma(\overline{u}), \val{\tau_2}_\Gamma(\overline{u})).\]
\item
  \marginpar{За $\texttt{if}$ вижте \Def{if}.}
  Нека $\tau \equiv \ifelse{\tau_1}{\tau_2}{\tau_3}$. Тогава
  \[\val{\ifelse{\tau_1}{\tau_2}{\tau_3}}_\Gamma(\overline{u}) \dff \texttt{if}(\val{\tau_1}_\Gamma(\overline{u}),
  \val{\tau_2}_\Gamma(\overline{u}), \val{\tau_3}_\Gamma(\overline{u})).\]
\item
  \marginpar{За $\texttt{eval}$ вижте \Def{eval}.}
  Нека $\tau \equiv \tau_1 \tau_2$. Тогава
  \[\val{\tau_1 \tau_2}_\Gamma(\overline{u}) \dff \texttt{eval}(\val{\tau_1}_\Gamma(\overline{u}), \val{\tau_2}_\Gamma(\overline{u})).\]
\item
  \marginpar{За $\lfp$ вижте Раздел~\ref{sect:lfp}.}
  Нека $\tau \equiv \fix(\tau')$. Тогава 
  \[\val{\fix(\tau')}_\Gamma(\overline{u}) \dff \lfp(\val{\tau'}_\Gamma(\overline{u})).\]
\item
  \marginpar{За $\curry$ вижте \Def{curry}.}
  Нека $\tau \equiv \lamb{y}{b}{\tau'}$, като $\vv{y} \not \in \texttt{dom}(\Gamma)$.
  Нека $\Gamma' \df \Gamma, \type{y}{b}$. Тогава
  \[\val{\lamb{y}{b}{\tau'}}_\Gamma(\overline{u}) \dff \curry(\val{\tau'}_{\Gamma'})(\overline{u}).\]
\end{itemize}

\begin{remark}
  За $\Gamma = \emptyset$, ще пишем $\val{\tau}$ вместо $\val{\tau}_\emptyset$.
\end{remark}

Не е ясно дали винаги горните дефиниции имат смисъл.
Сега ще докажем, че винаги, когато един терм е добре типизиран, то горната дефиниция има смисъл.

\begin{framed}
  \begin{lemma}
    Ако $\Gamma \vdash \tau : \vv{a}$, то $\val{\tau}_\Gamma \in \Cont{\val{\Gamma}}{\val{\vv{a}}}$.
  \end{lemma}  
\end{framed}
\begin{proof}
  Доказателството протича с индукция по построението на термовете
  като съществено използваме \Prop{composition} според което, ако $f \in \Cont{\A}{\B}$ и $g \in \Cont{\B}{\C}$, то
  $g \circ f \in \Cont{\A}{\C}$.
  \marginpar{Изображението $f \times g$ е дефинирано в \Prop{cartesian-continuous}.}
  \begin{itemize}
  \item
    Нека $\tau \equiv \vv{n}$. Щом $\Gamma \vdash \tau : \vv{a}$, то
    по правилата за типизиране следва, че $\vv{a} = \vv{nat}$.
    Сега лесно се съобразява, че изображението $\val{\vv{n}}_\Gamma \in \Cont{\val{\Gamma}}{\val{\vv{nat}}}$, където
    $\val{\vv{n}}_\Gamma(\overline{u}) = n$.
    Това е така, защото за всяка верига $\chain{\overline{u}}{i}$ от елементи на $\val{\Gamma}$,
    \[\val{\vv{n}}_\Gamma(\bigsqcup_i\overline{u}_i) = n = \bigsqcup_i\{ \val{\vv{n}}(\overline{u}_i)\}.\]
  \item
    Нека $\tau \equiv \vv{x}_i$. Щом $\Gamma \vdash \tau : \vv{a}$, то
    по правилата за типизиране следва, че $\vv{a} = \vv{a}_i$.
    Сега лесно се съобразява, че изображението $\val{\vv{n}}_\Gamma \in \Cont{\val{\Gamma}}{\val{\vv{a}_i}}$, където
    $\val{\vv{n}}_\Gamma(\overline{u}) = u_i$.
    \marginpar{$\overline{u}_k = (u_{1,k},\dots,u_{n,k})$.}
    Това е така, защото за всяка верига $\chain{\overline{u}}{n}$ от елементи на $\val{\Gamma}$,
    \[\val{\vv{x}_i}_\Gamma(\bigsqcup_n\overline{u}_n) = \bigsqcup_n u_{i,n} = \bigsqcup_i\{ \val{\vv{x}_i}(\overline{u}_n)\}.\]
  \item
    Нека $\tau \equiv \tau_1 + \tau_2$. Щом $\Gamma \vdash \tau : \vv{a}$, то
    по правилата за типизиране следва, че $\vv{a} = \vv{nat}$, а също и $\Gamma \vdash \tau_1 : \vv{nat}$ и $\Gamma \vdash \tau_2
    : \vv{nat}$.
    От И.П. имаме, че
    \begin{align*}
      & \val{\tau_1}_\Gamma \in \Cont{\val{\Gamma}}{\val{\vv{nat}}};\\
        & \val{\tau_2}_\Gamma \in \Cont{\val{\Gamma}}{\val{\vv{nat}}}.
    \end{align*}
    Това означава, че $(\val{\tau_1} \times \val{\tau_2}) \in \Cont{\val{\Gamma}}{\val{\vv{nat}} \times \val{\vv{nat}}}$.
    Тогава имаме следното равенство
    \marginpar{Използваме, че композиция на непрекъснати изображения е непрекъснато изображение.}
    \[\val{\tau_1 + \tau_2}_\Gamma = \texttt{plus} \circ (\val{\tau_1} \times \val{\tau_2}) \in \Cont{\val{\Gamma}}{\val{\vv{a}}},\]
    защото за произволни $\overline{u} \in \val{\Gamma}$,
    \begin{align*}
      (\texttt{plus} \circ (\val{\tau_1} \times \val{\tau_2}))(\overline{u}) & = \texttt{plus}((\val{\tau_1} \times \val{\tau_2})(\overline{u}))\\ 
                                                                             & = \texttt{plus}(\val{\tau_1}_\Gamma(\overline{u}), \val{\tau_2}_\Gamma(\overline{u}))\\
                                                                             & \dff \val{\tau}_\Gamma(\overline{u}).
    \end{align*}
  \item
    Нека $\tau \equiv \tau_1\ \vv{==}\ \tau_2$. Съобразете сами, че 
    \[\val{\tau_1\ \vv{==}\ \tau_2}_\Gamma = \texttt{eq} \circ (\val{\tau_1}_\Gamma \times \val{\tau_2}_\Gamma) \in \Cont{\val{\Gamma}}{\val{\vv{a}}}.\]
  \item
    Нека $\tau \equiv \ifelse{\tau_1}{\tau_2}{\tau_3}$. Съобразете сами, че 
    \[\val{\ifelse{\tau_1}{\tau_2}{\tau_3}}_\Gamma = \texttt{if} \circ (\val{\tau_1}_\Gamma \times \val{\tau_2}_\Gamma \times \val{\tau_3}_\Gamma)  \in \Cont{\val{\Gamma}}{\val{\vv{a}}}.\]
  \item
    Нека $\tau \equiv \tau_1 \tau_2$.
    Щом $\Gamma \vdash \tau_1 \tau_2 : \vv{a}$, то от правилата за типизиране следва, че
    \begin{align*}
      & \Gamma \vdash \tau_1 : \vv{b} \to \vv{a}\\
      & \Gamma \vdash \tau_2 : \vv{b}.
    \end{align*}
    От И.П. за $\tau_1$ и $\tau_2$ знаем, че
    \begin{align*}
      & \val{\tau_1}_\Gamma \in \Cont{\val{\Gamma}}{\Cont{\val{\vv{b}}}{\val{\vv{a}}}} \\
      & \val{\tau_2}_\Gamma \in \Cont{\val{\Gamma}}{\val{\vv{b}}}
    \end{align*}
    Оттук получаваме, че за произволни $\overline{u} \in \val{\Gamma}$,
    \begin{align*}
      & \val{\tau_1}_\Gamma(\overline{u}) \in \Cont{\val{\vv{b}}}{\val{\vv{a}}} \\
      & \val{\tau_2}_\Gamma(\overline{u}) \in \val{\vv{b}}.
    \end{align*}
    Тогава 
    \[\val{\tau_1 \tau_2}_\Gamma = \texttt{eval} \circ (\val{\tau_1}_\Gamma \times \val{\tau_2}_\Gamma) \in \Cont{\val{\Gamma}}{\val{\vv{a}}},\]
    защото за произволни $\overline{u} \in \val{\Gamma}$,
    \begin{align*}
      (\texttt{eval} \circ \val{\tau_1}_\Gamma \times \val{\tau_2}_\Gamma)(\overline{u}) & = \texttt{eval}((\val{\tau_1}_\Gamma \times \val{\tau_2}_\Gamma)(\overline{u}))\\
                                                                                         & = \texttt{eval}(\val{\tau_1}_\Gamma(\overline{u}), \val{\tau_2}_\Gamma(\overline{u}))\\
                                                                                         & \dff \val{\tau_1\tau_2}_\Gamma(\overline{u}).
    \end{align*}
    
  \item
    Нека сега $\tau \equiv \fix(\tau')$.
    Понеже $\Gamma \vdash \fix(\tau') : \vv{a}$, то от правилата за типизиране имаме, че
    $\Gamma \vdash \tau' : \vv{a} \to \vv{a}$.
    От И.П. знаем, че
    \[\val{\tau'}_\Gamma \in \Cont{\val{\Gamma}}{\Cont{\val{\vv{a}}}{\val{\vv{a}}}}.\]
    Това означава, че за произволни $\overline{u} \in \val{\Gamma}$,
    \[\val{\tau'}_\Gamma(\overline{u}) \in \Cont{\val{\vv{a}}}{\val{\vv{a}}}.\]
    Следователно
    $\val{\tau'}_\Gamma(\overline{u})$ е изображение, което според \Th{knaster-tarski}
    притежава най-малка неподвижна точка.
    \marginpar{Непрекъснатото изображението $Y$ е дефинирано \Th{Y}.}
    Тогава
    \[\val{\texttt{fix}(\tau')}_\Gamma = Y \circ \val{\tau'}_\Gamma \in \Cont{\val{\Gamma}}{\val{\vv{a}}},\]
    защото за произволни $\overline{u} \in \val{\Gamma}$,
    \begin{align*}
      (Y \circ \val{\tau'}_\Gamma)(\overline{u}) & = Y(\val{\tau'}_\Gamma(\overline{u}))\\
                                                 & = \lfp(\val{\tau'}_\Gamma(\overline{u}))\\
                                                 & \dff \val{\fix(\tau')}_\Gamma(\ov{u}).
    \end{align*}
  \item
    Нека $\tau \equiv \lamb{y}{b}{\tau'}$, като $\vv{y} \not \in \texttt{dom}(\Gamma)$.
    Щом $\Gamma \vdash \lamb{y}{b}{\tau'} : \vv{a}$, то от правилата за типизиране следва, че $\vv{a} = \vv{b} \to \vv{c}$
    и 
    \[\Gamma, \type{y}{b} \vdash \tau' : \vv{c}.\]
    
    Нека $\Gamma' = \Gamma, \vv{y}:\vv{b}$. Тогава $\val{\Gamma'} = \val{\Gamma} \times \val{\vv{b}}$, а от И.П. имаме, че
    \[\val{\tau'}_{\Gamma'} \in \Cont{\val{\Gamma} \times \val{\vv{b}}}{\val{\vv{c}}}.\]
    Тогава от \Prop{curry} следва, че
    \[\val{\lamb{y}{b}{\tau}}_\Gamma \dff \curry(\val{\tau'}_{\Gamma'}) \in \Cont{\val{\Gamma}}{\Cont{\val{\vv{b}}}{\val{\vv{c}}}}.\]
  \end{itemize}
\end{proof}

\begin{remark}
  В случая $\Gamma = \emptyset$, формално погледнато,
  $\val{\tau}_\emptyset \in \Cont{\emptyset_\bot}{\A}$, за някоя област на Скот $\A$.
  Но ние знаем, че $\Cont{\emptyset_\bot}{\A} \cong \A$.
  Следователно, можем да считаме, че $\val{\tau} \in \A$.
  В противен случай, трябва винаги да пишем $\val{\tau}(\bot)(a)$ вместо $\val{\tau}(a)$.
\end{remark}


\begin{proposition}
  \marginpar{Ясно е, че това твърдение се обобщава за произволна пермутация на индекстите $1,\dots,n$.}
  Нека имаме следните типови контексти:
  \begin{align*}
    &\Gamma = \vv{x}_1:\vv{a}_1, \dots, \vv{x}_i:\vv{a}_i, \dots, \vv{x}_j:\vv{a}_j, \dots, \vv{x}_n:\vv{a}_n;\\
    &\Delta = \vv{x}_1:\vv{a}_1, \dots, \vv{x}_j:\vv{a}_j, \dots, \vv{x}_i:\vv{a}_i, \dots, \vv{x}_n:\vv{a}_n,
  \end{align*}
  т.е. $\Delta$ се получава от $\Gamma$ като разменим местата на $i$-тата и $j$-тата двойка.
  Тогава за всеки терм $\tau$, такъв че $\Gamma \vdash \tau : \vv{a}$, е изпълено, че за всеки $(u_1,\dots,u_n) \in \val{\Gamma}$,
  \[\val{\tau}_\Gamma(u_1,\dots,u_i,\dots,u_j,\dots,u_n) = \val{\tau}_\Delta(u_1,\dots,u_j,\dots,u_i,\dots,u_n).\]
\end{proposition}
\begin{hint}
  Индукция по построението на терма $\tau$.
\end{hint}



%%% Local Variables:
%%% mode: latex
%%% TeX-master: "../sep"
%%% End:


%%% Local Variables:
%%% mode: latex
%%% TeX-master: "../sep"
%%% End:

\newpage
\section{Коректност}
Понеже вече имаме дефинирани операционна и денотационна семантика на термовете,
следващата стъпка е да разгледаме каква е връзката между тях.
В този раздел ще докажем едната (по-лесната) посока.

\begin{framed}
  \begin{theorem}[Теорема за коректност]\label{th:pcf:soundness}
    За всеки затворен терм $\tau : \vv{b}$ и стойност $\vv{v}$, е изпълнено:
    \[\tau \Downarrow_{\vv{b}} \vv{v}\ \implies\ \val{\tau} = \val{\vv{v}} \in \val{\vv{b}}.\]
  \end{theorem}  
\end{framed}
\begin{proof}
  Индукция по дължината $\ell$ на извода $\Downarrow^\ell_{\vv{b}}$ за всеки тип $\vv{b}$.
  Нека $\ell = 0$. Имаме два случая.
  \begin{itemize}
  \item
    Нека $\tau \equiv \vv{n}$.
    Ясно е, че $\vv{b} = \vv{nat}$ и от правилата на операционната семантика имаме, че:
    \begin{prooftree}
      \AxiomC{}
      \RightLabel{\scriptsize{(val)}}
      \UnaryInfC{$\tau \Downarrow^0_{\vv{nat}} \vv{n}$}
    \end{prooftree}
    От дефиницията на семантика на терм, директно получаваме, че
    $\val{\tau} = n = \val{\vv{n}}$.    
  \item
    Нека $\tau \equiv \lamb{x}{c}{\tau'}$. Тогава $\vv{b} = \vv{a}\to\vv{c}$ и от правилата на операционната семантика имаме, че:
    \begin{prooftree}
      \AxiomC{}
      \RightLabel{\scriptsize{(val)}}
      \UnaryInfC{$\tau \Downarrow^0_{\vv{a}\to\vv{c}} \tau$}
    \end{prooftree}
    Ясно е, че $\val{\tau} = \val{\tau} \in \val{b}$.
  \end{itemize}
  Така доказахме, че
  \[\tau \Downarrow^{0}_{\vv{b}} \vv{v}\ \implies\ \val{\tau} = \val{\vv{v}} \in \val{\vv{b}}.\]
  Нека сега $\ell > 0$ и да приемем, че имаме следното индукционно предположение:
  \[\tau \Downarrow^{<\ell}_{\vv{b}} \vv{v}\ \implies\ \val{\tau} = \val{\vv{v}} \in \val{\vv{b}}.\]
  Ще докажем, че
  \[\tau \Downarrow^{\ell}_{\vv{b}} \vv{v}\ \implies\ \val{\tau} = \val{\vv{v}} \in \val{\vv{b}}.\]
  \begin{itemize}
  \item
    Нека $\tau \equiv \tau_1 + \tau_2$. Тогава от правилата на операционната семантика имаме, че:
    \begin{prooftree}
      \AxiomC{$\tau_1 \Downarrow_{\vv{nat}} \vv{n}_1$}
      \AxiomC{$\tau_2 \Downarrow_{\vv{nat}} \vv{n}_2$}
      \RightLabel{\scriptsize{(plus)}}
      \BinaryInfC{$\tau_1 + \tau_2 \Downarrow_{\vv{nat}} \vv{n},$}
    \end{prooftree}
    където $n = n_1 + n_2$.
    От И.П. получаваме, че
    \begin{align*}
      & \val{\tau_1} = \val{\vv{n}_1} = n_1\\
      & \val{\tau_2} = \val{\vv{n}_2} = n_2.
    \end{align*}
    Тогава
    \begin{align*}
      \val{\tau_1 + \tau_2} & = \texttt{plus}(\val{\tau_1}, \val{\tau_2}) & \comment\text{от деф.}\\
                            & = n_1 + n_2 & \comment\text{от И.П.}\\
                            & = n.
    \end{align*}
  \item
    Случаят $\tau \equiv \tau_1\ \vv{==}\ \tau_2$ е аналогичен.
  \item
    Нека $\tau \equiv \ifelse{\tau_1}{\tau_2}{\tau_3}$. Тогава от правилата на операционната семантика имаме, че:
    \begin{prooftree}
      \AxiomC{$\tau_1 \Downarrow_{\vv{nat}} \vv{n}_1$}
      \AxiomC{$\tau_2 \Downarrow_{\vv{a}} \vv{v}_2$}
      \RightLabel{\scriptsize{(if$^+$)}}
      \BinaryInfC{$\ifelse{\tau_1}{\tau_2}{\tau_3} \Downarrow_{\vv{a}} \vv{v}_2,$}
    \end{prooftree}
    където $n_1 > 0$.
    Тогава от И.П. получаваме, че:
    \begin{align*}
      & \val{\tau_1} = n_1\\
      & \val{\tau_2} = \val{\vv{v}_2}.
    \end{align*}
    Тогава
    \begin{align*}
      \val{\ifelse{\tau_1}{\tau_2}{\tau_3}} & = \texttt{if}(\val{\tau_1}, \val{\tau_2}, \val{\tau_3})\\
                                            & = \texttt{if}(n_1,\val{\tau_2}, \val{\tau_3}) & \comment\text{от И.П.}\\
                                            & = \val{\tau_2} & \comment\text{от деф. на }\texttt{if}\\
                                            & = \val{\vv{v}_2}. & \comment\text{от И.П.}
    \end{align*}
    
    Случаят, когато $n_1 = 0$ е аналогичен.
  \item
    Нека $\tau \equiv \tau_1 \tau_2$. Тогава от правилата на операционната семантика имаме, че:
    \begin{prooftree}
      \AxiomC{$\tau_1 \Downarrow_{\vv{a}\to\vv{b}} \lamb{x}{a}{\tau'_1}$}
      \AxiomC{$\tau'_1[x/\tau_2] \Downarrow_{\vv{b}} \vv{v}$}
      \RightLabel{\scriptsize{(cbn)}}
      \BinaryInfC{$\tau_1 \tau_2 \Downarrow_{\vv{b}} \vv{v} $}
    \end{prooftree}
    Тогава от И.П. получаваме, че:    
    \begin{align*}
      & \val{\tau_1} = \val{\lamb{x}{a}{\tau'_1}} \in \Cont{\val{\vv{a}}}{\val{\vv{b}}}\\
      & \val{\tau'_1\subst{x}{\tau_2}} = \val{\vv{v}} \in \val{\vv{b}}.
    \end{align*}
    Тогава
    \begin{align*}
      \val{\tau_1\tau_2} & = \texttt{eval}(\val{\tau_1},\val{\tau_2}) & \comment\text{от деф.}\\ 
                         & = \val{\tau_1}(\val{\tau_2}) & \comment \val{\tau_1} \in \Cont{\val{\vv{a}}}{\val{\vv{b}}}\\
                         & = \val{\lamb{x}{a}{\tau'_1}}(\val{\tau_2}) & \comment\text{от И.П.}\\
                         & = \val{\tau'_1}_{\Gamma}(\val{\tau_2}) & \comment \Gamma \dff \type{x}{a}\\
                         & = \val{\tau'_1\subst{x}{\tau_2}} & \comment\text{от \hyperref[lem:pcf:substitution]{Лема за замяната}}\\
                         & = \val{\vv{v}} & \comment\text{от И.П.}
    \end{align*}
  \item
    Нека $\tau \equiv \fix(\tau')$. Тогава от правилата на операционната семантика имаме, че:
    \begin{prooftree}
      \AxiomC{$\tau'\ \fix(\tau') \Downarrow_{\vv{a}} \vv{v}$}
      \RightLabel{\scriptsize{(fix)}}
      \UnaryInfC{$\fix(\tau') \Downarrow_{\vv{a}} \vv{v} $}
    \end{prooftree}
    Тогава от И.П. имаме, че:
    \[\val{\tau'\ \fix(\tau')} = \val{\vv{v}}.\]
    Тогава
    \begin{align*}
      \val{\tau} & = \lfp(\val{\tau'})\\
                 & = \val{\tau'}(\lfp(\val{\tau'}))\\
                 & = \val{\tau'}(\val{\fix(\tau')})\\
                 & = \val{\tau'\fix(\tau')}\\
                 & = \val{\vv{v}}. & \comment\text{от И.П.}
    \end{align*}
  \end{itemize}
\end{proof}



%%% Local Variables:
%%% mode: latex
%%% TeX-master: "../sep"
%%% End:

\newpage
\section{Адекватност}
\marginpar{Adequacy ???}
Нашата цел в този раздел е да докажем следната теорема.
\begin{framed}
  \begin{theorem}[Теорема за адекватност]
    За всеки затворен терм $\tau : \vv{nat}$, 
    \[\val{\tau} = n \neq \bot \implies \tau \Downarrow_{\vv{nat}} \vv{n}.\]
  \end{theorem}
\end{framed}


Сега дефинираме за всеки тип $\vv{a}$ релацията 
$\triangleleft_{\vv{a}} \subseteq \val{\vv{a}} \times PCF_{\vv{a}}$
с индукция по построението на типовете.

\begin{itemize}
\item
  Нека $\vv{a} = \vv{nat}$. Тогава дефинираме
  \[d \triangleleft_{\vv{nat}} \tau \iff ( d\neq\bot \implies \tau \Downarrow_{\vv{nat}} \vv{d}).\]
\item
  Нека $\vv{a} = \vv{b} \to \vv{c}$. Тогава дефинираме
  \[f \triangleleft_{\vv{b}\to\vv{c}} \tau \iff (\forall e\in \val{\vv{b}})(\forall \mu \in PCF_{\vv{b}})[\ e \triangleleft_{\vv{b}} \mu \implies f(e) \triangleleft_{\vv{c}} \tau(\mu)\ ].\]
\item
  Нека $\Gamma = \vv{x}_1:\vv{a}_1, \dots, \vv{x}_n:\vv{a}_n$. Тогава дефинираме 
  \[(u_1,\dots,u_n) \triangleleft_\Gamma (\tau_1,\dots,\tau_n) \iff u_1 \triangleleft_{\vv{a}_1} \tau_1\ \&\ \cdots\ \&\ u_n \triangleleft_{\vv{a}_n} \tau_n.\]
\end{itemize}


\begin{lemma}\label{lem:pcf:relation}
  \marginpar{\cite[стр. 197]{gunter}}
  Нека $\tau : \vv{a}$. Тогава:
  \begin{enumerate}[1)]
  \item
    $\bot^{\val{\vv{a}}} \triangleleft_{\vv{a}} \tau$;
  \item
    $D = \{d \in \val{\vv{a}} \mid d \triangleleft_{\vv{a}} \tau\}$ е непрекъснато свойство в областта на Скот $\val{\vv{a}}$;
  \item
    Ако $u \sqsubseteq d$, $d \triangleleft_{\vv{a}} \tau$ и $(\forall \vv{v})[\tau \Downarrow_{\vv{a}} \vv{v} \implies \rho
    \Downarrow_{\vv{a}} \vv{v}]$, то $u \triangleleft_{\vv{a}} \rho$.
  \end{enumerate}
\end{lemma}
\begin{proof}
  Индукция по построението на типовете $\vv{a}$.
  
  Нека $\vv{a} = \vv{b} \to \vv{c}$.
  \begin{enumerate}[1)]
  \item
    Тук имаме, че $\bot^{\val{\vv{a}}} \in \Cont{\val{\vv{b}}}{\val{\vv{c}}}$ е изображение,
    за което $\bot^{\val{\vv{a}}}(e) =  \bot^{\val{\vv{c}}}$ за всеки елемент $e \in \val{\vv{b}}$.
    Нека $e \triangleleft_{\vv{b}} \mu$, където $\mu : \vv{b}$.
    От правилата за типизиране е ясно, че $\tau(\mu) : \vv{c}$.
    Сега от И.П. е ясно, че $\bot^{\val{\vv{a}}}(e) = \bot^{\val{\vv{c}}} \triangleleft_{\vv{c}} \tau(\mu)$.
  \item
    Нека $\chain{f}{i}$ е верига от елементи на $\Cont{\val{\vv{b}}}{\val{\vv{c}}}$,
    за които е изпълнено, че $f_i \triangleleft_{\vv{a}} \tau$. Трябва да докажем, че $\bigsqcup_i f_i \triangleleft_{\vv{a}} \tau$,
    т.е. за произволни $e \in \val{\vv{b}}$ и произволни $\mu : \vv{b}$, за които $e \triangleleft_{\vv{b}} \mu$, то
    $(\bigsqcup_if)(e) \triangleleft_{\vv{c}} \tau(\mu)$.
    Но ние знаем, че $(\bigsqcup_if)(e) = \bigsqcup_i\{f_i(e)\}$.
    Щом $f_i \triangleleft_{\vv{b}\to\vv{c}} \tau$, то за разглежданите $e$ и $\mu$ имаме, че $f_i(e) \triangleleft_{\vv{c}} \tau(\mu)$.
    Ние знаем, че ${(f_i(e))}^\infty_{i=0}$ е верига и от И.П. следва, че $\bigsqcup_i\{f_i(e)\} \triangleleft_{\vv{c}} \tau(\mu)$.
  \item
    Нека $g \sqsubseteq f$, $f \triangleleft_{\vv{b}\to\vv{c}} \tau$ и $\tau \Downarrow_{\vv{b}\to\vv{c}} \vv{v} \implies \rho
    \Downarrow_{\vv{b}\to\vv{c}} \vv{v}$. Ще докажем, че $g \triangleleft_{\vv{b} \to \vv{c}} \rho$.
    За целта, нека $e \in \val{\vv{b}}$, $\mu : \vv{b}$ и $e \triangleleft_{\vv{b}} \mu$.
    Ще докажем, че $g(e) \triangleleft_{\vv{c}} \rho(\mu)$.
    За момента знаем само, че $f(e) \triangleleft_{\vv{c}} \tau(\mu)$.
    Щом $g \sqsubseteq f$, то $g(e) \sqsubseteq f(e)$ в областта на Скот $\val{\vv{c}}$.
    Понеже имаме следното правило в операционната семантика:
    \begin{prooftree}
      \AxiomC{$\tau \Downarrow_{\vv{b}\to\vv{c}} \lamb{x}{b}{\tau'}$}
      \AxiomC{$\tau'\subst{x}{\mu} \Downarrow_{\vv{c}} \vv{v}'$}
      \RightLabel{\scriptsize{(app)}}
      \BinaryInfC{$\tau(\mu) \Downarrow_{\vv{c}} \vv{v}'$}
    \end{prooftree}
    то получаваме, че
    \begin{prooftree}
      \AxiomC{$\rho \Downarrow_{\vv{b}\to\vv{c}} \lamb{x}{b}{\tau'}$}
      \AxiomC{$\tau'\subst{x}{\mu} \Downarrow_{\vv{c}} \vv{v}'$}
      \RightLabel{\scriptsize{(app)}}
      \BinaryInfC{$\rho(\mu) \Downarrow_{\vv{c}} \vv{v}'$}
    \end{prooftree}
    Оттук следва, че
    \[(\forall \vv{v}')[\tau(\mu) \Downarrow_{\vv{c}} \vv{v}' \implies \rho(\mu) \Downarrow_{\vv{c}} \vv{v}'].\]
    Сега от И.П. директно следва, че $g(e) \triangleleft_{\vv{c}} \rho(\mu)$.
  \end{enumerate}
\end{proof}


\begin{framed}
  \begin{theorem}[Фундаментално свойство на $\triangleleft_{\vv{a}}$]\label{th:pcf:fundamental}
    Нека $\Gamma = \vv{x}_1:\vv{a}_1,\dots,\vv{x}_n:\vv{a}_n$. Тогава
    \begin{prooftree}
      \AxiomC{$\Gamma \vdash \tau : \vv{a}$}
      \AxiomC{$(u_1,\dots,u_n) \triangleleft_\Gamma (\mu_1,\dots,\mu_n)$}
      \BinaryInfC{$\val{\tau}_\Gamma(\ov{u}) \triangleleft_{\vv{a}} \tau[\ov{\vv{x}}/\ov{\mu}]$}
    \end{prooftree}
  \end{theorem}  
\end{framed}
\begin{proof}
  Индукция по построението на термовете.
  \begin{itemize}
  \item
    Нека $\tau = \tau_1\tau_2$. От правилата за типизиране имаме, че
    \begin{prooftree}
      \AxiomC{$\Gamma \vdash \tau_1 : \vv{b} \to \vv{a}$}
      \AxiomC{$\Gamma \vdash \tau_2 : \vv{b}$}
      \BinaryInfC{$\Gamma \vdash \tau_1\tau_2 : \vv{a}$}
    \end{prooftree}
    Да напомним, че
    \[\val{\tau_1\tau_2}_\Gamma(\ov{u}) \dff \texttt{eval}(\val{\tau_1}_\Gamma(\ov{u}), \val{\tau_2}_\Gamma(\ov{u})).\]
    От И.П. имаме следното:
    \begin{align*}
      & \val{\tau_1}_\Gamma(\ov{u}) \triangleleft_{\vv{b}\to\vv{a}} \tau_1[\ov{x}/\ov{\mu}];\\
      & \val{\tau_2}_\Gamma(\ov{u}) \triangleleft_{\vv{b}} \tau_2[\ov{x}/\ov{\mu}].
    \end{align*}
    Тогава директно следва, че
    \[\texttt{eval}(\val{\tau_1}_\Gamma(\ov{u}), \val{\tau_2}_\Gamma(\ov{u})) \triangleleft_{\vv{a}} \tau_1[\ov{x}/\ov{\mu}](\tau_2[\ov{x}/\ov{\mu}])\]
  \item
    Нека $\tau = \lamb{y}{b}{\tau'}$. Тогава от правилата за типизиране следва, че $\vv{a} = \vv{b} \to \vv{c}$ и
    $\Gamma' \vdash \tau' : \vv{c}$, където $\Gamma' = \Gamma, \type{y}{b}$.
    Да напомним, че
    \[\val{\tau}_\Gamma(\ov{u}) \dff \texttt{curry}(\val{\tau'}_{\Gamma'})(\ov{u}) \in \Cont{\val{\vv{b}}}{\val{\vv{c}}}.\]
    Да положим $f \dff \val{\tau}_\Gamma(\ov{u})$.
    Трябва да докажем, че $f \triangleleft_{\vv{b} \to \vv{c}} \tau[\ov{x}/\ov{\mu}]$.
    Това означава, че за произволни $e \in \val{\vv{b}}$ и $\rho : \vv{b}$, за които $e \triangleleft_{\vv{b}} \rho$,
    трябва да докажем, че $f(e) \triangleleft_{\vv{c}} \tau[\ov{x}/\ov{\mu}](\rho)$.
    Имаме, че
    \begin{prooftree}
      \AxiomC{$\Gamma' \vdash \tau' : \vv{c}$}
      \AxiomC{$(u_1,\dots,u_n,e) \triangleleft_{\Gamma'} (\mu_1,\dots,\mu_n,\rho)$}
      \RightLabel{\scriptsize{(И.П.)}}
      \BinaryInfC{$\val{\tau'}(\ov{u},e) \triangleleft_{\vv{c}} \tau'[\ov{x}/\ov{\mu}][y/\rho]$}
      \UnaryInfC{$f(e) \triangleleft_{\vv{c}} \tau'[\ov{x}/\ov{\mu}][y/\rho]$}
    \end{prooftree}
    % Нека сега за улеснение да положим $\tau'' \dff \tau'[\ov{x}/\ov{\mu}]$.
    От правилата на операционната семантика имаме следното:
    \begin{prooftree}
      \AxiomC{$\rho : \vv{b}$}
      \AxiomC{$\tau'[\ov{x}/\ov{\mu}][y/\rho] \Downarrow_{\vv{c}} \vv{v}$}
      \BinaryInfC{$(\lamb{y}{b}{\tau'[\ov{x}/\ov{\mu}]})(\rho) \Downarrow_{\vv{c}} \vv{v}$}
      \UnaryInfC{$\tau[\ov{x}/\ov{\mu}](\rho) \Downarrow_{\vv{c}} \vv{v}$}
    \end{prooftree}
    От 3) на \Lem{pcf:relation} веднага заключаваме, че $f(e) \triangleleft_{\vv{c}} \tau[\ov{x}/\ov{\mu}](\rho)$.
  \item
    Нека $\tau \equiv \fix(\tau')$. Тогава от правилата за типизиране имаме, че $\tau' : \vv{a} \to \vv{a}$.
    \marginpar{По-лесно става като се позовем на правилото на Скот ?}
    От И.П. имаме, че
    \[\val{\tau'}(\ov{u}) \triangleleft_{\vv{a}\to\vv{a}} \tau'[\ov{x}/\ov{\mu}].\]
    Нека за улеснение да положим $f \dff \val{\tau'}(\ov{u})$.
    Да напомним, че
    \[\val{\fix(\tau')}_\Gamma(\ov{u}) = \lfp(f) = \bigsqcup_i f^{(i)}(\bot^{\val{\vv{a}}}).\]
    Сега ще докажем, че за всяко $i$,
    \[f^{(i)}(\bot^{\val{\vv{a}}}) \triangleleft_{\vv{a}} \fix(\tau'[\ov{x}/\ov{\mu}]).\]
    Понеже $f \triangleleft_{\vv{a}\to\vv{a}} \tau'[\ov{x}/\ov{\mu}]$, то
    за произволно $e \triangleleft_{\vv{a}} \rho$ е изпълнено, че
    $f(e) \triangleleft_{\vv{a}} \tau'[\ov{x}/\ov{\mu}](\rho)$.
    Нека $e = f^{(i)}(\bot^{\val{\vv{a}}})$ и $\rho = \fix(\tau'[\ov{x}/\ov{\mu}])$.
    Тогава \[f^{(i+1)}(\bot^{\val{\vv{a}}}) \triangleleft_{\vv{a}} \tau'[\ov{x}/\ov{\mu}](\fix(\tau'[\ov{x}/\ov{\mu}])).\]
    От правилата на операционната семантика имаме, че:
    \begin{prooftree}
      \AxiomC{$\tau'[\ov{x}/\ov{\mu}](\fix(\tau'[\ov{x}/\ov{\mu}])) \Downarrow_{\vv{a}} \vv{v}$}
      \UnaryInfC{$\fix(\tau'[\ov{x}/\ov{\mu}]) \Downarrow_{\vv{a}} \vv{v}$}
    \end{prooftree}
    Тогава от 3) на \Lem{pcf:relation} следва, че
    \[f^{(i+1)}(\bot^{\val{\vv{a}}}) \triangleleft_{\vv{a}} \fix(\tau'[\ov{x}/\ov{\mu}]).\]
    
    Сега от 2) на \Lem{pcf:relation} следва, че
    \[\val{\fix(\tau')}_\Gamma(\ov{u}) \triangleleft_{\vv{a}} \fix(\tau'[\ov{x}/\ov{\mu}])\]

    % Получаваме, че
    % \[\texttt{eval}(\val{\tau'}_\Gamma(\ov{u}), \val{\fix(\tau')}_\Gamma(\ov{u})) \triangleleft_{\vv{a}}
    %   \tau'[\ov{x}/\ov{\mu}](\fix(\tau'[\ov{x}/\ov{\mu}])).\]
    % От правилата на операционната семантика имаме, че:
    % \begin{prooftree}
    %   \AxiomC{$\tau' \fix(\tau') \Downarrow_{\vv{a}} \vv{v}$}
    %   \UnaryInfC{$\fix(\tau') \Downarrow_{\vv{a}} \vv{v}$}
    % \end{prooftree}
  \end{itemize}
\end{proof}

\begin{framed}
  \begin{corollary}\label{cr:pcf:fundamental}
    Ако $\tau : \vv{a}$, то $\val{\tau} \triangleleft_{\vv{a}} \tau$.
  \end{corollary}
\end{framed}

\begin{framed}
  \begin{theorem}[Теорема за адекватност]\label{th:pcf:adequacy}
    За всеки затворен терм $\tau : \vv{nat}$, 
    \[\val{\tau} = n \neq \bot \implies \tau \Downarrow_{\vv{nat}} \vv{n}.\]
  \end{theorem}
\end{framed}
\begin{proof}
  Да разгледаме произволен затворен терм $\tau : \vv{nat}$.
  Нека $\val{\tau} = n \neq \bot$.
  От \Cor{pcf:fundamental} имаме, че $\val{\tau} \triangleleft_{\vv{nat}} \tau$.
  Тогава от дефиницията на $\triangleleft_{\vv{nat}}$ получаваме, че $\tau \Downarrow_{\vv{nat}} \vv{n}$.
\end{proof}



%%% Local Variables:
%%% mode: latex
%%% TeX-master: "../sep"
%%% End:

\newpage
\section{Контексти}


\[\C ::= -\ |\ \vv{n}\ |\ \vv{x}\ |\ \C + \C\ |\ \C\ \vv{==}\ \C\ |\ \ifelse{\C}{\C}{\C}\ |\ \C\C\ |\ \lamb{x}{a}{\C}\ |\ \fix(\C).\]
Контекстите са като изразите, но имаме специален символ $-$.
За произволен израз $\tau$, с $\C[\tau]$ означаваме израза, който се получава като
заменим всички срещания на $-$ с $\tau$. Заместването, което правим е директно, т.е.
ако $\C = \lamb{x}{a}{-}$, то $\C[x] = \lamb{x}{a}{\vv{x}}$.

\begin{proposition}
  \marginpar{$\equiv$ е $\alpha$-еквивалентност.} 
  Ако $\tau \equiv \tau'$, то $\C[\tau] \equiv \C[\tau']$.
\end{proposition}



\begin{definition}
  $\Gamma \vdash \tau_1 \leq_{ctx} \tau_2 : \vv{a}$, ако
  \begin{enumerate}[1)]
  \item
    $\Gamma \vdash \tau_1 : \vv{a}$ и $\Gamma \vdash \tau_2 : \vv{a}$
  \item
    За всички контексти $C[-]$, за които $\emptyset \vdash C[\tau_1] : \vv{nat}$ и $\emptyset \vdash C[\tau_2] : \vv{nat}$, то
    \[C[\tau_1] \Downarrow_{\vv{nat}} \vv{n} \implies C[\tau_2] \Downarrow_{\vv{nat}} \vv{n}.\]
  \end{enumerate}  
\end{definition}

\marginpar{Някои наричат тази релация observational equivalence. Тук наричаме релацията contextual equivalence.}
Ще пишем $\Gamma \vdash \tau_1 \cong_{ctx} \tau_2 : \vv{a}$, ако
$\Gamma \vdash \tau_1 \leq_{ctx} \tau_2 : \vv{a}$ и $\Gamma \vdash \tau_2 \leq_{ctx} \tau_1 : \vv{a}$.
Ако $\Gamma = \emptyset$, то ще пишем $\tau_1 \cong_{ctx} \tau_2 : \vv{a}$ вместо $\emptyset \vdash \tau_1 \cong_{ctx} \tau_2 : \vv{a}$.

\begin{proposition}\label{pr:pcf:compositionality}
  \marginpar{Това твърдение ни казва, че нашата денотационна семантика е композиционална.}
  Нека $\tau_1$ и $\tau_2$ са термове, за които $\Gamma \vdash \tau_1 : \vv{a}$ и $\Gamma \vdash \tau_2 : \vv{a}$,
  като също така $\val{\tau_1}_\Gamma = \val{\tau_2}_\Gamma$.
  и нека $\C[-]$ е контекст, за който $\Gamma' \vdash \C[\tau_1] : \vv{b}$ и $\Gamma' \vdash \C[\tau_2] : \vv{b}$.
  Тогава
  \[\val{\C[\tau_1]}_{\Gamma'} = \val{\C[\tau_2]}_{\Gamma'}.\]
\end{proposition}
\begin{proof}
  Преобразуваме $\C[-]$ в терм $\mu$ като заместваме всяко срещане на $-$ с новата променлива $z$
  и нека $\Gamma,\ \type{z}{a} \vdash \mu : \vv{b}$.
  Сега използваме \Lem{pcf:substitution}, защото $\C[\tau_i] \equiv \mu\subst{z}{\tau_i}$, защото
  \begin{align*}
    \val{\mu\subst{z}{\tau_1}}_{\Gamma'}(\ov{u}) & = \val{\mu}_{\Gamma''}(\ov{u},\val{\tau_1}_\Gamma)\\
                                                 & = \val{\mu}_{\Gamma''}(\ov{u},\val{\tau_2}_\Gamma)\\
                                                 & = \val{\mu\subst{z}{\tau_2}}_{\Gamma'}(\ov{u}).
  \end{align*}
\end{proof}


\begin{proposition}
  За произволни затворени термове $\tau_1$ и $\tau_2$ от тип $\vv{a}$,
  \[\tau_1 \leq_{ctx} \tau_2 : \vv{a} \iff \val{\tau_1} \triangleleft_{\vv{a}} \tau_2.\]
\end{proposition}
\begin{proof}
  
\end{proof}


\begin{proposition}\label{pr:pcf:extensionality}
  \begin{enumerate}[1)]
  \item
    $\tau_1 \leq_{ctx} \tau_2 : \vv{nat} \iff (\forall \vv{v})[\tau_1 \Downarrow_{\vv{nat}} \vv{v} \implies \tau_2    
    \Downarrow_{\vv{nat}} \vv{v}]$;
  \item
    $\tau_1 \leq_{ctx} \tau_2 : \vv{a}\to\vv{b} \iff (\forall \rho:\vv{a})[\tau_1\rho \leq_{ctx} \tau_2 \rho : \vv{b}]$.
  \end{enumerate}
\end{proposition}


\begin{framed}
  \begin{theorem}
    За произволни $\tau_1, \tau_2 \in \vv{PCF}_{\vv{a}}$,
    \[\val{\tau_1} \sqsubseteq \val{\tau_2} \implies \tau_1 \leq_{ctx} \tau_2 : \vv{a}.\]
  \end{theorem}  
\end{framed}
\begin{proof}
  \marginpar{\cite[стр. 179]{gunter}}
  Достатъчно е да докажем, че
  \[\val{\tau_1} = \val{\tau_2} \implies \tau_1 \leq_{ctx} \tau_2 : \vv{a}.\]
  Нека $\C[-]$ е контекст, за който $\C[\tau_1] : \vv{nat}$ и $\C[\tau_2] : \vv{nat}$.
  Нека $\val{\tau_1} = \val{\tau_2}$. Тогава
  \begin{align*}
    \C[\tau_1] \Downarrow_{\vv{nat}} \vv{n} & \implies \val{\C[\tau_1]} = \val{\vv{n}} & \comment\text{\Th{pcf:soundness}}\\
                                            & \implies \val{\C[\tau_2]} = \val{\vv{n}} & \comment\text{\Prop{pcf:compositionality}}\\
                                            & \implies \C[\tau_2] \Downarrow_{\vv{nat}} \vv{n}. & \comment\text{\Th{pcf:adequacy}}
  \end{align*}
\end{proof}


%%% Local Variables:
%%% mode: latex
%%% TeX-master: "../sep"
%%% End:

\newpage
\section{Пълна абстракция}\label{pcf:sect:full-abstraction}
\marginpar{Full abstraction на англ.}
\begin{definition}
  \marginpar{\cite[стр. 179]{gunter}}
  Денотационната семантика $\val{.}$ се нарича {\bf напълно абстрактна}, ако
  контекстната (операционната) и денотационната наредба съвпадат, т.е.
  за произволни термове $\tau_1,\tau_2$ от тип $\vv{a}$ е изпълнено, че
  \[\val{\tau_1} \sqsubseteq \val{\tau_2} \iff \tau_1 \leq_{ctx} \tau_2 : \vv{a}.\]
\end{definition}

\begin{framed}
  \begin{theorem}[Гордън Плоткин 1977]
    Денотационната семантика $\val{.}$ за езика PCF {\bf не е} напълно абстрактна.
  \end{theorem}
\end{framed}
Да напомним, че от \Th{pcf:context:connection} винаги имаме следното:
\[ \val{\tau_1} = \val{\tau_2} \implies \tau_1 \cong_{ctx} \tau_2 : \vv{a}.\]
Сега ще се захванем с търсенето на термове $\tau_1$ и $\tau_2$, за които
$\val{\tau_1} \neq \val{\tau_2}$ и $\tau_1 \cong \tau_2 : \vv{a}$.


\begin{problem}
  Да дефинираме функцията $sor:\Nat_\bot \to (\Nat_\bot \to \Nat_\bot)$ по следния начин:
  \marginpar{$sor$ идва от sequential or.}
  
  \begin{tabular}{|c|c|c|c|}
    \hline
    $sor$ & $\bot$ & $0$ & $y>0$ \\
    \hline
    $\bot$ & $\bot$ & $\bot$ & $\bot$\\
    \hline
    $0$ & $\bot$ & $0$ & $1$\\
    \hline
    $x>0$ & $1$ & $1$ & $1$\\
    \hline
  \end{tabular}
  
  Докажете, че $sor$ е определима в PCF.
\end{problem}
\begin{hint}
  Разгледайте затворения терм
  \[\tau \dff \lamb{x}{nat}{\lamb{y}{nat}{\ifelse{\vv{x}}{\vv{1}}{\ifelse{\vv{y}}{\vv{1}}{\vv{0}}}}}\]
  Докажете, че $\val{\tau} = sor$.
\end{hint}


\begin{problem}

Да дефинираме изображението $por:\Nat_\bot\to(\Nat_\bot \to \Nat_\bot)$ по следния начин:

\begin{tabular}{|c|c|c|c|}
  \hline
  $por$ & $\bot$ & $0$ & $y>0$\\
  \hline
  $\bot$ & $\bot$ & $\bot$ & $1$\\
  \hline
  $0$ & $\bot$ & $0$ & $1$\\
  \hline
  $x>0$ & $1$ & $1$ & $1$\\
  \hline
\end{tabular}
\marginpar{$por$ идва от parallel or.}

  Докажете, че $por$ е непрекъснато изображение.
\end{problem}
\begin{hint}
  Достатъчно е да се съобрази, че $por$ е монотонно изображение.
\end{hint}

\begin{framed}
  \begin{lemma}[Гордън Плоткин 1977]
    Изображението $por$ не е определимо в PCF, т.е. не съществува затворен терм $\rho$,
    за който $\val{\rho} = por$.
  \end{lemma}
\end{framed}

\begin{example}
Да видим, че операторът ,,или'' в хаскел не е паралелен.
\begin{haskellcode}
ghci> True || undefined
True
ghci> undefined || True
*** Exception: Prelude.undefined
\end{haskellcode}
\end{example}

\begin{problem}\label{prob:pcf:full-abstraction:por}
  Да разгледаме $f \in \Cont{\Nat_\bot}{\Cont{\Nat_\bot}{\Nat_\bot}}$, за което имаме ограниченията:

  \begin{tabular}{|c|c|c|c|}
    \hline
    $f$ & $\bot$ & $0$ & $y>0$\\
    \hline
    $\bot$ & $?$ & $?$ & $1$\\
    \hline
    $0$ & $?$ & $0$ & $?$\\
    \hline
    $x>0$ & $1$ & $?$ & $?$\\
    \hline
  \end{tabular}

  Докажете, че $f = por$.
  
\end{problem}
\begin{hint}
  Използвайте монотонността на $f$.
\end{hint}

\begin{problem}\label{prob:pcf:full-abstraction:not-definable}
  Да разгледаме изображението $f \in \Cont{\Nat_\bot}{\Cont{\Nat_\bot}{\Nat_\bot}}$, за което
  \[f(0)(0) = 0\text{ и } f(1)(\bot) = f(\bot)(1) = 1.\]
  Докажете, че $f$ не е определима в PCF.
\end{problem}
\begin{hint}
  Да допуснем, че $f$ е определима в PCF.
  Тогава $f = \val{\tau}$, за някой затворен терм $\tau : \nat \to \nat \to \nat$.

  За произволна променлива $\vv{z}$, да положим
  \[\rho_{\vv{z}} \dff \ifelse{\vv{z == 0}}{\vv{0}}{\vv{1}}.\]
  Нека също положим
  \begin{align*}
    \tau' & \dff \tau\rho_{\vv{x}};\\
    \tau'' & \dff \tau'\rho_{\vv{y}}.
  \end{align*}
  Нека също така $\Gamma \dff \type{x}{nat}$ и $\Delta = \type{y}{nat}$.
  Ясно, че $\val{\tau'}_\Gamma \in \Cont{\Nat_\bot}{\Cont{\Nat_\bot}{\Nat_\bot}}$ и
  \begin{align*} 
    \val{\tau'}_\Gamma(u)  & = \val{\tau\rho_{\vv{x}}}_\Gamma(u)\\
                       & = \texttt{eval}(\val{\tau}, \val{\rho_{\vv{x}}}_\Gamma(u))\\
                       & = \val{\tau}(\val{\rho_{\vv{x}}}_\Gamma(u))
  \end{align*}
  Нека $f' = \val{\tau'}_\Gamma$. Получаваме следното за $f'$.
  \[f'(u) = f(\val{\rho_{\vv{x}}}_\Gamma(u)) =
    \begin{cases}
      f(u), & \text{ако } u = \bot \text{ или } u = 0\\
      f(1), & \text{ако } u > 0.
    \end{cases}\]
  Аналогично, ясно е, че $\val{\tau''}_{\Gamma,\Delta} \in \Cont{\Nat_\bot\times\Nat_\bot}{\Nat_\bot}$ и
  \begin{align*} 
    \val{\tau''}_{\Gamma,\Delta}(u,v)  & = \val{\tau'\rho_{\vv{y}}}_{\Gamma,\Delta}(u,v)\\
                                  & = \texttt{eval}(\val{\tau'}_\Gamma(u), \val{\rho_{\vv{y}}}_\Delta(v))\\
                                  & = \val{\tau'}_\Gamma(u)(\val{\rho_{\vv{y}}}_\Delta(v))\\
                                  & = \val{\tau}(\val{\rho_{\vv{x}}}_\Gamma(u))(\val{\rho_{\vv{y}}}_\Delta(v)).
  \end{align*}
  Нека сега $f'' = \val{\tau''}_{\Gamma,\Delta}$. Тогава
  \begin{align*}
    f''(u,v) & = f'(u)(\val{\rho_{\vv{y}}}_\Delta(v))\\
             & = \begin{cases}
               f'(u)(v), & \text{ако } v = \bot\text{ или } v = 0\\
               f'(u)(1), & \text{ако } v > 0
             \end{cases}
  \end{align*}
  Така получаваме следната характеризация на $f''$:

  \begin{tabular}{|c|c|c|c|}
    \hline
    $f''$ & $\bot$ & $0$ & $y>0$ \\
    \hline
    $\bot$ & $?$ & $?$ & $1$ \\
    \hline
    $0$ & $?$ & $0$ & $?$ \\
    \hline
    $x>0$ & $1$ & $?$ & $?$\\
    \hline
  \end{tabular}

  Нека сега $\rho \dff \lamb{x}{nat}{\lamb{y}{nat}{\tau''}}$.
  Тогава за $g = \val{\rho}$ имаме, че
  \[g(x)(y) = f''(x,y).\]
  От \Problem{pcf:full-abstraction:por} получаваме, че $g = por$.
  Достигаме до противоречие, защото $por$ не е определимо изображение.
\end{hint}

В следващите твърдения ще използваме типовете
\begin{align*}
  & \vv{a} \dff \nat \to (\nat \to \nat)\\
  & \vv{b} \dff (\nat \to (\nat \to \nat))\to\nat.
\end{align*}
За $n = 0,1$, нека дефинираме затворените термове

\begin{lstlisting}
  $\tau_n \equiv \lambda \vv{f:a}$.if (f 1 $\Omega_\nat$) == 1 then
              if (f $\Omega_\nat$ 1) == 1 then
                if (f 0 0) == 0 then n
                  else $\Omega_\nat$
                else $\Omega_\nat$
              else $\Omega_\nat$
\end{lstlisting}

Лесно се съобразява, че $\tau_0$ и $\tau_1$ са добре типизирани термове от тип $\vv{b}$.

\begin{problem}
  Докажете, че 
  \[\val{\tau_0} \neq \val{\tau_1}.\]
\end{problem}
\begin{hint}
  Докажете, че за $n = 0,1$ е изпълнено, че
  \[\val{\tau_n}(por) = n.\]  
\end{hint}

\begin{proposition}
  $\tau_1 \cong_{ctx} \tau_2 : \vv{b}$.
\end{proposition}
\begin{proof}
  Понеже $\vv{b} = \vv{a} \to \nat$, от \Prop{pcf:context:extensionality} следва, че е достатъчно да докажем, че
  за всеки затворен терм $\rho:\vv{a}$ е изпълнено, че
  \[\tau_1\rho \Downarrow_{\nat} \vv{n} \iff \tau_2\rho \Downarrow_{\nat} \vv{n}.\]
  Да видим какво означава $\tau_i \rho \Downarrow_{\nat}$ за $i = 0,1$.
  Това означава, че трябва да са изпълнени и трите свойства:
  \begin{itemize}
  \item
    $\rho\ \vv{1}\ \Omega_{\nat} \Downarrow_{\nat} \vv{1}$;% , за някое $\vv{k} \not\equiv \vv{0}$;
  \item
    $\rho\ \Omega_{\nat}\ \vv{1} \Downarrow_{\nat} \vv{1}$;% , за някое $\vv{m} \not\equiv \vv{0}$;
  \item
    $\rho\ \vv{0}\ \vv{0} \Downarrow_{\nat} \vv{0}$.
  \end{itemize}
  Понеже $\val{\Omega_{\nat}} = \bot$, от \hyperref[th:pcf:soundness]{теоремата за коректност} получаваме, че трябва да са изпълнени следните три свойства:
  \begin{itemize}
  \item
    $\val{\rho}(1)(\bot) = 1$;
  \item
    $\val{\rho}(\bot)(1) = 1$;
  \item
    $\val{\rho}(0)(0) = 0$.
  \end{itemize}
  Но тогава $\val{\rho} = por$, което е противоречие с \Problem{pcf:full-abstraction:not-definable}.
\end{proof}

Доказателството на следващата теорема излиза извън обхата на този курс.
\index{Плоткин}
\begin{framed}
  \begin{theorem}[Плоткин 1977]
    Денотационната семантика $\val{.}$ за езика PCF+\texttt{por} е напълно абстрактна.
  \end{theorem}
\end{framed}
\marginpar{\cite[стр. 188]{gunter}}

% \index{Плоткин}
% \index{Милнър}
% \begin{theorem}[Милнър,Плоткин]
%   A continuous, order-extensional model of PCF is fully abstract if and only if for every type $\sigma$, $\val{\sigma}$ is a domain whose finite elements are definable.
% \end{theorem}

%%% Local Variables:
%%% mode: latex
%%% TeX-master: "../sep"
%%% End:



%%% Local Variables:
%%% mode: latex
%%% TeX-master: "../sep"
%%% End:
