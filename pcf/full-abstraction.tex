\section{Пълна абстракция}\label{pcf:sect:full-abstraction}
\marginpar{Full abstraction на англ.}
\begin{definition}
  \marginpar{\cite[стр. 179]{gunter}}
  Денотационната семантика $\val{.}$ се нарича {\bf напълно абстрактна}, ако
  контекстната (операционната) и денотационната наредба съвпадат, т.е.
  за произволни термове $\tau_1,\tau_2$ от тип $\vv{a}$ е изпълнено, че
  \[\val{\tau_1} \sqsubseteq \val{\tau_2} \iff \tau_1 \leq_{ctx} \tau_2 : \vv{a}.\]
\end{definition}

\begin{framed}
  \begin{theorem}[Гордън Плоткин 1977]
    Денотационната семантика $\val{.}$ за езика PCF {\bf не е} напълно абстрактна.
  \end{theorem}
\end{framed}
Да напомним, че от \Th{pcf:context:connection} винаги имаме следното:
\[ \val{\tau_1} = \val{\tau_2} \implies \tau_1 \cong_{ctx} \tau_2 : \vv{a}.\]
Сега ще се захванем с търсенето на термове $\tau_1$ и $\tau_2$, за които
$\val{\tau_1} \neq \val{\tau_2}$ и $\tau_1 \cong \tau_2 : \vv{a}$.


\begin{problem}
  Да дефинираме функцията $sor:\Nat_\bot \to (\Nat_\bot \to \Nat_\bot)$ по следния начин:
  \marginpar{$sor$ идва от sequential or.}
  
  \begin{tabular}{|c|c|c|c|}
    \hline
    $sor$ & $\bot$ & $0$ & $y>0$ \\
    \hline
    $\bot$ & $\bot$ & $\bot$ & $\bot$\\
    \hline
    $0$ & $\bot$ & $0$ & $1$\\
    \hline
    $x>0$ & $1$ & $1$ & $1$\\
    \hline
  \end{tabular}
  
  Докажете, че $sor$ е определима в PCF.
\end{problem}
\begin{hint}
  Разгледайте затворения терм
  \[\tau \dff \lamb{x}{nat}{\lamb{y}{nat}{\ifelse{\vv{x}}{\vv{1}}{\ifelse{\vv{y}}{\vv{1}}{\vv{0}}}}}\]
  Докажете, че $\val{\tau} = sor$.
\end{hint}


\begin{problem}

Да дефинираме изображението $por:\Nat_\bot\to(\Nat_\bot \to \Nat_\bot)$ по следния начин:

\begin{tabular}{|c|c|c|c|}
  \hline
  $por$ & $\bot$ & $0$ & $y>0$\\
  \hline
  $\bot$ & $\bot$ & $\bot$ & $1$\\
  \hline
  $0$ & $\bot$ & $0$ & $1$\\
  \hline
  $x>0$ & $1$ & $1$ & $1$\\
  \hline
\end{tabular}
\marginpar{$por$ идва от parallel or.}

  Докажете, че $por$ е непрекъснато изображение.
\end{problem}
\begin{hint}
  Достатъчно е да се съобрази, че $por$ е монотонно изображение.
\end{hint}

\begin{framed}
  \begin{lemma}[Гордън Плоткин 1977]
    Изображението $por$ не е определимо в PCF, т.е. не съществува затворен терм $\rho$,
    за който $\val{\rho} = por$.
  \end{lemma}
\end{framed}

\begin{example}
Да видим, че операторът ,,или'' в хаскел не е паралелен.
\begin{haskellcode}
ghci> True || undefined
True
ghci> undefined || True
*** Exception: Prelude.undefined
\end{haskellcode}
\end{example}

\begin{problem}\label{prob:pcf:full-abstraction:por}
  Да разгледаме $f \in \Cont{\Nat_\bot}{\Cont{\Nat_\bot}{\Nat_\bot}}$, за което имаме ограниченията:

  \begin{tabular}{|c|c|c|c|}
    \hline
    $f$ & $\bot$ & $0$ & $y>0$\\
    \hline
    $\bot$ & $?$ & $?$ & $1$\\
    \hline
    $0$ & $?$ & $0$ & $?$\\
    \hline
    $x>0$ & $1$ & $?$ & $?$\\
    \hline
  \end{tabular}

  Докажете, че $f = por$.
  
\end{problem}
\begin{hint}
  Използвайте монотонността на $f$.
\end{hint}

\begin{problem}\label{prob:pcf:full-abstraction:not-definable}
  Да разгледаме изображението $f \in \Cont{\Nat_\bot}{\Cont{\Nat_\bot}{\Nat_\bot}}$, за което
  \[f(0)(0) = 0\text{ и } f(1)(\bot) = f(\bot)(1) = 1.\]
  Докажете, че $f$ не е определима в PCF.
\end{problem}
\begin{hint}
  Да допуснем, че $f$ е определима в PCF.
  Тогава $f = \val{\tau}$, за някой затворен терм $\tau : \nat \to \nat \to \nat$.

  За произволна променлива $\vv{z}$, да положим
  \[\rho_{\vv{z}} \dff \ifelse{\vv{z == 0}}{\vv{0}}{\vv{1}}.\]
  Нека също положим
  \begin{align*}
    \tau' & \dff \tau\rho_{\vv{x}};\\
    \tau'' & \dff \tau'\rho_{\vv{y}}.
  \end{align*}
  Нека също така $\Gamma \dff \type{x}{nat}$ и $\Delta = \type{y}{nat}$.
  Ясно, че $\val{\tau'}_\Gamma \in \Cont{\Nat_\bot}{\Cont{\Nat_\bot}{\Nat_\bot}}$ и
  \begin{align*} 
    \val{\tau'}_\Gamma(u)  & = \val{\tau\rho_{\vv{x}}}_\Gamma(u)\\
                       & = \texttt{eval}(\val{\tau}, \val{\rho_{\vv{x}}}_\Gamma(u))\\
                       & = \val{\tau}(\val{\rho_{\vv{x}}}_\Gamma(u))
  \end{align*}
  Нека $f' = \val{\tau'}_\Gamma$. Получаваме следното за $f'$.
  \[f'(u) = f(\val{\rho_{\vv{x}}}_\Gamma(u)) =
    \begin{cases}
      f(u), & \text{ако } u = \bot \text{ или } u = 0\\
      f(1), & \text{ако } u > 0.
    \end{cases}\]
  Аналогично, ясно е, че $\val{\tau''}_{\Gamma,\Delta} \in \Cont{\Nat_\bot\times\Nat_\bot}{\Nat_\bot}$ и
  \begin{align*} 
    \val{\tau''}_{\Gamma,\Delta}(u,v)  & = \val{\tau'\rho_{\vv{y}}}_{\Gamma,\Delta}(u,v)\\
                                  & = \texttt{eval}(\val{\tau'}_\Gamma(u), \val{\rho_{\vv{y}}}_\Delta(v))\\
                                  & = \val{\tau'}_\Gamma(u)(\val{\rho_{\vv{y}}}_\Delta(v))\\
                                  & = \val{\tau}(\val{\rho_{\vv{x}}}_\Gamma(u))(\val{\rho_{\vv{y}}}_\Delta(v)).
  \end{align*}
  Нека сега $f'' = \val{\tau''}_{\Gamma,\Delta}$. Тогава
  \begin{align*}
    f''(u,v) & = f'(u)(\val{\rho_{\vv{y}}}_\Delta(v))\\
             & = \begin{cases}
               f'(u)(v), & \text{ако } v = \bot\text{ или } v = 0\\
               f'(u)(1), & \text{ако } v > 0
             \end{cases}
  \end{align*}
  Така получаваме следната характеризация на $f''$:

  \begin{tabular}{|c|c|c|c|}
    \hline
    $f''$ & $\bot$ & $0$ & $y>0$ \\
    \hline
    $\bot$ & $?$ & $?$ & $1$ \\
    \hline
    $0$ & $?$ & $0$ & $?$ \\
    \hline
    $x>0$ & $1$ & $?$ & $?$\\
    \hline
  \end{tabular}

  Нека сега $\rho \dff \lamb{x}{nat}{\lamb{y}{nat}{\tau''}}$.
  Тогава за $g = \val{\rho}$ имаме, че
  \[g(x)(y) = f''(x,y).\]
  От \Problem{pcf:full-abstraction:por} получаваме, че $g = por$.
  Достигаме до противоречие, защото $por$ не е определимо изображение.
\end{hint}

В следващите твърдения ще използваме типовете
\begin{align*}
  & \vv{a} \dff \nat \to (\nat \to \nat)\\
  & \vv{b} \dff (\nat \to (\nat \to \nat))\to\nat.
\end{align*}
За $n = 0,1$, нека дефинираме затворените термове

\begin{lstlisting}
  $\tau_n \equiv \lambda \vv{f:a}$.if (f 1 $\Omega_\nat$) == 1 then
              if (f $\Omega_\nat$ 1) == 1 then
                if (f 0 0) == 0 then n
                  else $\Omega_\nat$
                else $\Omega_\nat$
              else $\Omega_\nat$
\end{lstlisting}

Лесно се съобразява, че $\tau_0$ и $\tau_1$ са добре типизирани термове от тип $\vv{b}$.

\begin{problem}
  Докажете, че 
  \[\val{\tau_0} \neq \val{\tau_1}.\]
\end{problem}
\begin{hint}
  Докажете, че за $n = 0,1$ е изпълнено, че
  \[\val{\tau_n}(por) = n.\]  
\end{hint}

\begin{proposition}
  $\tau_1 \cong_{ctx} \tau_2 : \vv{b}$.
\end{proposition}
\begin{proof}
  Понеже $\vv{b} = \vv{a} \to \nat$, от \Prop{pcf:context:extensionality} следва, че е достатъчно да докажем, че
  за всеки затворен терм $\rho:\vv{a}$ е изпълнено, че
  \[\tau_1\rho \Downarrow_{\nat} \vv{n} \iff \tau_2\rho \Downarrow_{\nat} \vv{n}.\]
  Да видим какво означава $\tau_i \rho \Downarrow_{\nat}$ за $i = 0,1$.
  Това означава, че трябва да са изпълнени и трите свойства:
  \begin{itemize}
  \item
    $\rho\ \vv{1}\ \Omega_{\nat} \Downarrow_{\nat} \vv{1}$;% , за някое $\vv{k} \not\equiv \vv{0}$;
  \item
    $\rho\ \Omega_{\nat}\ \vv{1} \Downarrow_{\nat} \vv{1}$;% , за някое $\vv{m} \not\equiv \vv{0}$;
  \item
    $\rho\ \vv{0}\ \vv{0} \Downarrow_{\nat} \vv{0}$.
  \end{itemize}
  Понеже $\val{\Omega_{\nat}} = \bot$, от \hyperref[th:pcf:soundness]{теоремата за коректност} получаваме, че трябва да са изпълнени следните три свойства:
  \begin{itemize}
  \item
    $\val{\rho}(1)(\bot) = 1$;
  \item
    $\val{\rho}(\bot)(1) = 1$;
  \item
    $\val{\rho}(0)(0) = 0$.
  \end{itemize}
  Но тогава $\val{\rho} = por$, което е противоречие с \Problem{pcf:full-abstraction:not-definable}.
\end{proof}

Доказателството на следващата теорема излиза извън обхата на този курс.
\index{Плоткин}
\begin{framed}
  \begin{theorem}[Плоткин 1977]
    Денотационната семантика $\val{.}$ за езика PCF+\texttt{por} е напълно абстрактна.
  \end{theorem}
\end{framed}
\marginpar{\cite[стр. 188]{gunter}}

% \index{Плоткин}
% \index{Милнър}
% \begin{theorem}[Милнър,Плоткин]
%   A continuous, order-extensional model of PCF is fully abstract if and only if for every type $\sigma$, $\val{\sigma}$ is a domain whose finite elements are definable.
% \end{theorem}

%%% Local Variables:
%%% mode: latex
%%% TeX-master: "../sep"
%%% End:
