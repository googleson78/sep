\section{Пълна абстракция}

\begin{definition}
  \marginpar{\cite[стр. 179]{gunter}}
  Денотационната семантика $\val{.}$ се нарича {\bf напълно абстрактна}, ако
  операционата и денотационната наредба съвпадат, т.е.
  за произволни термове $\tau_1,\tau_2 : \vv{a}$,
  \[\val{\tau_1} \sqsubseteq \val{\tau_2} \iff \tau_1 \leq_{ctx} \tau_2.\]
\end{definition}

\begin{framed}
  \begin{theorem}
    Денотационната семантика $\val{.}$ за езика PCF {\bf не е} напълно абстрактна.
  \end{theorem}
\end{framed}

\begin{lemma}
  Съществуват термове $\tau_1$ и $\tau_2$,
  за които $\tau_1 \cong_{ctx} \tau_2$, но $\val{\tau_1} \neq \val{\tau_2}$.
\end{lemma}



\begin{proposition}
  Не съществува затворен PCF терм $P$, за който
  \[\val{P} = \texttt{por}.\]
\end{proposition}

\begin{verbatim}
  if f 1 O then 
     if f O 1 then
        if f 0 0 then O else i
     else O
  else O
\end{verbatim}

За $i = 0,1$, нека
\[T_i \dff \ifelse{f\ 1\ \Omega}{\ifelse{f\ \Omega\ 1}{\ifelse{f\ 0\ 0}{\Omega}{i}}{\Omega}}{\Omega}\]

\begin{proposition}
  За $i = 0,1$ е изпълнено, че
  \[\val{T_i}(\texttt{por}) = i.\]
\end{proposition}

\begin{proposition}
  Нека $\vv{a} \dff (\vv{nat}\to(\vv{nat}\to\vv{nat}))\to\vv{nat}$. Тогава
  \[T_1 \cong_{ctx} T_2 : \vv{a}.\]
\end{proposition}


\begin{framed}
  \begin{theorem}
    Денотационната семантика $\val{.}$ за езика PCF+\texttt{por} е напълно абстрактна.
  \end{theorem}
\end{framed}
\marginpar{\cite[стр. 188]{gunter}}

%%% Local Variables:
%%% mode: latex
%%% TeX-master: "../sep"
%%% End:
