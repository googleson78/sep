\section{Операционна семантика}

Дефинираме релацията $\Downarrow_{\vv{a}}$ върху затворените изрази и стойностите.

\begin{prooftree}
  \AxiomC{$\type{v}{a}$}
  \RightLabel{\scriptsize{(val)}}
  \UnaryInfC{$\vv{v} \Downarrow^0_{\vv{a}} \vv{v}$}
\end{prooftree}

\begin{prooftree}
  \AxiomC{$\tau_1 \Downarrow^{\ell_1}_{\vv{nat}} \vv{n}_1$}
  \AxiomC{$\tau_2 \Downarrow^{\ell_2}_{\vv{nat}} \vv{n}_2$}
  \AxiomC{$n = \texttt{eq}(n_1,n_2)$}
  \RightLabel{\scriptsize{(eq)}}
  \TrinaryInfC{$\tau_1\ \vv{==}\ \tau_2 \Downarrow^{\ell_1+\ell_2+1}_{\vv{nat}} \vv{n}$}
\end{prooftree}

\begin{prooftree}
  \AxiomC{$\tau_1 \Downarrow_{\vv{nat}} \vv{n}_1$}
  \AxiomC{$\tau_2 \Downarrow_{\vv{nat}} \vv{n}_2$}
  \AxiomC{$n = n_1 + n_2$}
  \RightLabel{\scriptsize{(plus)}}
  \TrinaryInfC{$\tau_1\ \vv{+}\ \tau_2 \Downarrow_{\vv{nat}} \vv{n}$}
\end{prooftree}


\begin{prooftree}
  \AxiomC{$\tau_1 \Downarrow_{\vv{nat}} \vv{0}$}
  \AxiomC{$\tau_3 \Downarrow_{\vv{a}} \vv{v}$}
  \RightLabel{\scriptsize{(if$_0$)}}
  \BinaryInfC{$\ifelse{\tau_1}{\tau_2}{\tau_3} \Downarrow_{\vv{a}} \vv{v}$}
\end{prooftree}

\begin{prooftree}
  \AxiomC{$\tau_1 \Downarrow_{\vv{nat}} \vv{n}$}
  \AxiomC{$\tau_2 \Downarrow_{\vv{a}} \vv{v}$}
  \AxiomC{$\vv{n} \not\equiv \vv{0}$}
  \RightLabel{\scriptsize{(if$^+$)}}
  \TrinaryInfC{$\ifelse{\tau_1}{\tau_2}{\tau_3} \Downarrow_{\vv{a}} \vv{v}$}
\end{prooftree}

\begin{prooftree}
  \AxiomC{$\tau_1 \Downarrow_{\vv{a}\to\vv{b}} \lamb{x}{a}{\tau'_1}:\vv{b}$}
  \AxiomC{$\tau'_1[x/\tau_2] \Downarrow_{\vv{b}} \vv{v}$}
  \RightLabel{\scriptsize{(cbn)}}
  \BinaryInfC{$\tau_1 \tau_2 \Downarrow_{\vv{b}} \vv{v} $}
\end{prooftree}

\begin{prooftree}
  \AxiomC{$\tau\ \fix(\tau) \Downarrow_{\vv{a}} \vv{v}$}
  \RightLabel{\scriptsize{(fix)}}
  \UnaryInfC{$\fix(\tau) \Downarrow_{\vv{a}} \vv{v} $}
\end{prooftree}

\begin{lemma}
  За произволен затворен терм $\tau$ и стойности $\vv{v}$ и $\vv{u}$,
  ако $\tau \Downarrow_{\vv{a}} \vv{v}$ и $\tau \Downarrow_{\vv{a}} \vv{u}$, то $\vv{v} \equiv \vv{u}$.
\end{lemma}

\begin{example}
  \[\Omega_{\vv{a}} \equiv \fix(\lamb{x}{a}{\vv{x}}).\]
  Лесно се вижда, че:
  \begin{itemize}
  \item
    $\emptyset \vdash \Omega_{\vv{a}} : \vv{a}$.
  \item
    За всяка стойност $\vv{v}$,
    $\Omega_{\vv{a}} \not\Downarrow_{\vv{a}} \vv{v}$.
  \end{itemize}
\end{example}

\begin{example}
  Нека $\vv{a} = \vv{nat} \to (\vv{nat} \to \vv{nat})$ и 
  \[\tau \equiv \fix(\lamb{f}{a}{\lamb{x}{nat}{\lamb{y}{nat}{\ifelse{\vv{y == 0}}{0}{\vv{x + (f x (y-1))} }}}}).\]
  Лесно се вижда, че
  \[\tau:\vv{nat} \to (\vv{nat} \to \vv{nat}).\]
  Също така за всяко $n$ и $k$, ако $m = n + k$, то
  \[\tau\ \vv{n}\ \vv{k} \Downarrow_{\vv{nat}} \vv{m}.\]
  Нека сега
  \[\rho \equiv \lamb{g}{a}{\fix(\lambda \vv{f} : \vv{nat}\to\vv{nat}\ .\ \lamb{x}{nat}{\ifelse{\vv{x == 0}}{\vv{1}}{\vv{g x f(x-1)}}}}).\]
  Лесно се вижда, че
  \[ \rho : \vv{a} \to (\vv{nat} \to \vv{nat}).\]
  Също така, за всяко $n$, ако $k = n!$, то
  \[ \rho\ \tau\ \vv{n} \Downarrow_{\vv{nat}} \vv{k}.\]
\end{example}

\begin{haskellcode}
> tm = \f x y -> if y == 0 then 0 else x + (f x (y-1))
> fix f = f (fix f)
> times = fix tm
> times 2 3
6
>times 10 11
110
> fct = \g -> fix(\f x -> if x == 0 then 1 else g x (f (x-1)))
> fact = fct times
> fact 5
120
\end{haskellcode}

\begin{problem}
  \marginpar{\cite[стр. 104]{types-programming-languages}}
  Докажете или опровергайте дали е възможно да съществува контекст $\Gamma$ и тип $\vv{a}$, такива че
  \[\Gamma \vdash \type{xx}{a}.\]
\end{problem}


%%% Local Variables:
%%% mode: latex
%%% TeX-master: "../sep"
%%% End:
