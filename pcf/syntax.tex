\section{Синтаксис}

\newcommand{\fix}{\texttt{fix}}
\newcommand{\fv}{\texttt{fv}}

\begin{itemize}
\item
  Типове
  \[\vv{a} ::= \vv{nat}\ |\ \vv{a} \to \vv{a}.\]
\item
  Изрази
  \[\tau ::= \vv{n}\ |\ \vv{x}\ |\ \tau + \tau\ |\ \tau\ \vv{==}\ \tau\ |\ \ifelse{\tau}{\tau}{\tau}\ |\ \tau\tau\ |\ \lamb{x}{a}{\tau}\ |\ \fix(\tau).\]
  Ще казваме, че един израз $\tau$ е {\bf затворен}, ако $\fv(\tau) = \emptyset$.
  В противен случай, ще казваме, че изразът е {\bf отворен}.
\item
  \marginpar{или в канонична форма?}
  Ще казваме, че един израз е {\bf стойност}, ако той е съставен по следния начин:
  \[\vv{v} ::= \vv{n}\ |\ \lamb{x}{a}{\vv{v}}.\]
\item
  Контексти
  \[\Gamma ::= \emptyset\ |\ \Gamma,\type{x}{a}.\]
\end{itemize}

Да обърнем внимание, че ние няма да правим разлика между два израза, които са $\alpha$-еквивалентни, т.е.
ще считаме, че $\lamb{x}{a}{(x+1)}$ е същият израз като $\lamb{y}{a}{(y+1)}$.


Сега ще дефинираме функция $var:\mathcal{E} \to \mathcal{V}$, която ни дава всички променливи в един израз.
\begin{itemize}
\item
  $var(\vv{n}) = \emptyset$;
\item
  $var(\vv{x}) = \{\vv{x}\}$;
\item
  $var(\tau_1 + \tau_2) = var(\tau_1\ \vv{==}\ \tau_2) = var(\tau_1\tau_2) = var(\tau_1) \cup var(\tau_2)$;
\item
  $var(\ifelse{\tau_1}{\tau_2}{\tau_3}) = var(\tau_1) \cup var(\tau_2) \cup var(\tau_3)$;
\item
  $var(\lamb{x}{a}{\tau}) = var(\tau)$.
\item
  $var(\fix(\tau)) = var(\tau)$;
\end{itemize}


Понеже тук вече имам свободни и свързани променливи, трябва да дефинираме точно какво означава това.
\[\fv:\mathcal{E} \to \mathcal{V}\]
Ще дефинираме функцията $\texttt{fv}$ със структурна индукция по построението на термовете.

\begin{itemize}
\item
  $\fv(\vv{n}) = \emptyset$;
\item
  $\fv(\vv{x}) = \{\vv{x}\}$;
\item
  $\fv(\tau_1 + \tau_2) = \fv(\tau_1\ \vv{==}\ \tau_2) = \fv(\tau_1\tau_2) = \fv(\tau_1) \cup \fv(\tau_2)$;
\item
  $\fv(\ifelse{\tau_1}{\tau_2}{\tau_3}) = \fv(\tau_1) \cup \fv(\tau_2) \cup \fv(\tau_3)$;
\item
  $\fv(\lamb{x}{a}{\tau}) = \fv(\tau) \setminus \{x\}$.
\item
  $\fv(\fix(\tau)) = \fv(\tau)$;
\end{itemize}


\newcommand{\rename}[2]{\{\vv{#1}/\vv{#2}\}}
\newcommand{\subst}[2]{[\vv{#1}/{#2}]}

С $\tau\rename{x}{u}$ ще означаваме изразът получен от $\tau$, в който всяко \emph{свободно} срещане на променливата $\vv{x}$
е заменена с променливата $\vv{u}$. Можем да дадем формална дефиниция с индукция по построението на изразите:
\begin{itemize}
\item
  Ако $\tau \equiv \vv{n}$, то $\tau\rename{x}{u}$ e израза $\vv{n}$.
\item
  Ако $\tau \equiv \vv{x}$, то $\tau\rename{x}{u}$ е израза $\vv{u}$.
\item
  Ако $\tau \equiv \vv{y}$ и $\vv{y}$ не е променливата $\vv{x}$, то $\tau\rename{x}{u}$ е израза $\vv{y}$.
\item
  Ако $\tau \equiv \tau_1 + \tau_2$, то
  $\tau\rename{x}{y}$ е израза $\tau_1\rename{x}{u} + \tau_2\rename{x}{u}$.
\item
  Ако $\tau$ е израза $\tau_1\ \vv{==}\ \tau_2$, то $\tau\rename{x}{u}$ е израза
  \[\tau_1\rename{x}{u}\ \vv{==}\ \tau_2\rename{x}{u}.\]
\item
  Ако $\tau$ е израза $\ifelse{\tau_1}{\tau_2}{\tau_3}$, то $\tau\rename{x}{u}$ е израза
  \[\ifelse{\tau_1\rename{x}{u}}{\tau_2\rename{x}{u}}{\tau_3\rename{x}{u}}.\]
\item
  Ако $\tau$ е израза $\lamb{x}{a}{\tau'}$, то
  $\tau\rename{x}{u}$ е израза $\tau$;
\item
  Ако $\tau$ е израза $\lamb{y}{a}{\tau'}$ и $\vv{y}$ е променлива различна от $\vv{x}$, то
  $\tau\rename{x}{u}$ е израза $\lamb{y}{a}{\tau'\rename{x}{u}}$;
\end{itemize}

Сега ще дефинираме бинарна релация между изрази, която ще наричаме $\alpha$-еквивалентност.
Това е най-малката релация $\equiv$ между изрази, за която са изпълнени свойствата:
\begin{itemize}
\item
  $\vv{x} \equiv \vv{x}$;
\item
  $\vv{n} \equiv \vv{n}$;
\item
  ако $\tau_1 \equiv \rho_1$ и $\tau_2 \equiv \rho_2$, то имаме, че
  \begin{align*}
    & \tau_1 + \tau_2 \equiv \rho_1 + \rho_2,\\
    & \tau_1\ \vv{==}\ \tau_2 \equiv \rho_1\ \vv{==}\ \rho_2,\\
    & \tau_1 \tau_2 \equiv \rho_1 \rho_2;
  \end{align*}
\item
  ако $\tau_1 \equiv \rho_1$, $\tau_2 \equiv \rho_2$ и $\tau_3 \equiv \rho_3$, то
  \[\ifelse{\tau_1}{\tau_2}{\tau_3} \equiv \ifelse{\rho_1}{\rho_2}{\rho_3};\]
\item
  ако $\tau \equiv \rho$, то $\fix(\tau) \equiv \fix(\rho)$;
\item
  ако $\tau\rename{x}{z} \equiv \rho\rename{y}{z}$, където $z \not\in \fv(\tau) \cup \fv(\rho)$, то
  \[\lamb{x}{a}{\tau} \equiv \lamb{y}{a}{\rho}.\]
\end{itemize}

Сега трябва да обобщим операцията преименуване на една променлива с друга с
операцията замяна на една променлива с израз.
Трябва да внимаваме как правим това, защото можем да получим
израз с различен смисъл. Например, нека $\tau_1$ е изразът $\lamb{x}{a}{xz}$,
а $\tau_2$ е изразът $\lamb{y}{a}{yz}$. Ясно е, че $\tau_1 \equiv \tau_2$, т.е. двата израза описват един и същ терм.
От друга страна обаче, нека $\rho$ е изразът $\lamb{x}{a}{yx}$. Тогава
$\tau_1[z/\rho]$ е изразът
\[\lamb{x}{a}{x(\lamb{x}{a}{yx})},\]
а $\tau_2[z/\rho]$ е изразът
\[\lamb{y}{a}{y(\lamb{x}{a}{yx})}.\]
Лесно се съобразява, че
\[\tau_1[z/\rho] \not\equiv \tau_2[z/\rho].\]
Това означава, че ако не внимаваме, то след извършена замяна можем да получим различни термове в
зависимост от това какъв представител на съответния терм сме избрали.
Проблемът възниква от това, че $$


С $\tau\subst{x}{\rho}$ ще означаваме термът получен от $\tau$, в който всяко \emph{свободно} срещане на променливата $\vv{x}$
е заменена с терма $\rho$. Можем да дадем формална дефиниция с индукция по построението на изразите:
\begin{itemize}
\item
  Ако $\tau \equiv \vv{n}$, то $\tau\subst{x}{\rho}$ e израза $\vv{n}$.
\item
  Ако $\tau \equiv \vv{x}$, то $\tau\subst{x}{\rho}$ е израза $\rho$.
\item
  Ако $\tau \equiv \vv{y}$ и $\vv{y}$ не е променливата $\vv{x}$, то $\tau\subst{x}{\rho}$ е израза $\vv{y}$.
\item
  Ако $\tau \equiv \tau_1 + \tau_2$, то
  $\tau\subst{x}{y}$ е израза $\tau_1\subst{x}{\rho} + \tau_2\subst{x}{\rho}$.
\item
  Ако $\tau$ е израза $\tau_1\ \vv{==}\ \tau_2$, то $\tau\subst{x}{\rho}$ е израза
  \[\tau_1\subst{x}{\rho}\ \vv{==}\ \tau_2\subst{x}{\rho}.\]
\item
  Ако $\tau$ е израза $\ifelse{\tau_1}{\tau_2}{\tau_3}$, то $\tau\subst{x}{\rho}$ е израза
  \[\ifelse{\tau_1\subst{x}{\rho}}{\tau_2\subst{x}{\rho}}{\tau_3\subst{x}{\rho}}.\]
\item
  Ако $\tau$ е израза $\lamb{x}{a}{\tau'}$, то
  $\tau\subst{x}{\rho}$ е израза $\tau$;
\item
  Ако $\tau$ е израза $\lamb{y}{a}{\tau'}$ и $\vv{y}$ е променлива различна от $\vv{x}$, то
  $\tau\subst{x}{\rho}$ е израза
  \[\lamb{z}{a}{(\tau'\rename{y}{z}\subst{x}{\rho})},\]
  където $\vv{z} \not\in \fv(\tau') \cup \fv(\rho) \cup \{x\}$.
\end{itemize}



PCF терм е клас на еквивалентност от PCF изрази относно релацията $\alpha$-еквивалентност.

%%% Local Variables:
%%% mode: latex
%%% TeX-master: "../sep"
%%% End:
