\subsection{Непрекъснати изображения}\index{изображение!непрекъснато}

\marginpar{На англ. {\em continuous}}
Едно изображение $f:\A_1\to \A_2$ се нарича {\bf непрекъснато}, ако са изпълнени свойствата:
\begin{itemize}
\item
  $f$ е монотонно изображение;
\item
  при всеки избор на верига $\chain{a}{i}$ в $\A_1$, то имаме равенството
  \marginpar{Понеже $\A_1$ е област на Скот знаем, че $\bigsqcup_i a_i \in \A_1$}
  \marginpar{Понеже $f$ е монотонно, то ${(f(a_i))}^{\infty}_{i=0}$ е верига.}
  \marginpar{$\bigsqcup_i f(a_i) \dff \bigsqcup\{f(a_i) \mid i \in \Nat\}$}
  \[f(\bigsqcup_i a_i) = \bigsqcup \{f(a_i) \mid i \in \Nat\}.\]  
\end{itemize}
Да означим
\[\Cont{\A_1}{\A_2} \dff (\{f: \A_1 \to \A_2 \mid f\text{ е непр. изобр.}\},\ \sqsubseteq,\ \bm{\bot}).\]

% \begin{framed}
%   \begin{prop}
%     \label{pr:continuous-is-monotone}
%     За произволни области на Скот $\A_1$ и $\A_2$, всяко непрекъснато изображение $f:\A_1 \to \A_2$ е монотонно, т.е.
%     \[\Cont{\A_1}{\A_2}\ \subseteq\ \Mon{\A_1}{\A_2}.\]
%   \end{prop}
% \end{framed}
% \begin{proof}
%   Нека $f \in \Cont{\A_1}{\A_2}$.
%   Да вземем два произволни елемента в $\A_1$, за които $a \sqsubseteq_1 b$.
%   Ще докажем, че $f(a) \sqsubseteq_2 f(b)$.
%   Да разгледаме веригата $\chain{a}{i}$ в $\A_1$, където:
%   \[\underbrace{a_0}_{a} \sqsubseteq \underbrace{a_1}_{b} = \underbrace{a_2}_{b} = \underbrace{a_3}_{b} = \cdots\]
%   Ясно е, че 
%   \[\bigsqcup\{a_i \mid i \in \Nat\} = \bigsqcup\{a,b\} = b.\]
%   Тогава от непрекъснатостта на $f$ имаме, че
%   \begin{align*}
%     f(a) & = f(a_0) & \comment{\text{защото }a_0 \dff a}\\
%     & \sqsubseteq_2 \bigsqcup\{f(a_i) \mid i \in \Nat\} & \comment{\text{защото }f(a_0) \in \{f(a_i) \mid i\in\Nat\}}\\
%     & = f(\bigsqcup_i a_i) & \comment{\text{защото $f$ е непр.}}\\
%     & = f(\bigsqcup\{a,b,b,b,\dots\}) & \comment{\text{от избора на веригата }\chain{a}{i}}\\
%     & = f(b) & \comment{\text{защото }a \sqsubseteq_1 b}.
%   \end{align*}
%   Така получихме, че за произволни $a,b\in\A_1$, 
%   \[a \sqsubseteq_1 b \implies f(a) \sqsubseteq_2 f(b).\]
% \end{proof}

Ясно е, че всяко непрекъснато изображение е монотонно.
Ествено е да си зададем въпроса дали имаме и обратното включване.
Оказва се, че в общия случай не е вярно, че всяко монотонно изображение е непрекъснато.

\begin{proposition}
  Съществува област на Скот $\A$, за която
  \[\Cont{\A}{\A} \subsetneqq \Mon{\A}{\A}.\]
\end{proposition}
\begin{hint}
  Нека $A = \{a_n \mid n \in \Nat\} \cup \{a_\omega, b_0\}$.
  Да разгледаме областта на Скот $\A = (A, \sqsubseteq, a_0)$, където 
  наредбата между елементите е следната:
  \[a_0 \sqsubseteq a_1 \sqsubseteq \cdots \sqsubseteq a_n \sqsubseteq \cdots \sqsubseteq a_\omega \sqsubseteq b_0. \]
  Нека $f(a_n) = a_{n+1}$, $f(a_{\omega}) = b_0$ и $f(b_0) = b_0$.
  Очевидно е, че $f$ е монотонно изображение.
  Лесно се вижда, че $f$ не е непрекъснато изображение, 
  защото
  \[f(\bigsqcup_n a_n) = f(a_\omega) = b_0,\]
  но 
  \[\bigsqcup_n f(a_n) = \bigsqcup_n a_{n+1} = a_\omega.\]
\end{hint}

Сега да видим един важен за нас случай, при който имаме и обратното включване.

\begin{framed}
  \begin{proposition}\label{pr:stab-continuous}
    Ако всяка верига в $\A_1$ се {\em стабилизира}, то
    \[\Mon{\A_1}{\A_2} \subseteq \Cont{\A_1}{\A_2}.\]
  \end{proposition}
\end{framed}
\begin{hint}
  Да разгледаме една верига $\chain{a}{i}$ в $\A_1$ и $f \in \Mon{\A_1}{\A_2}$.
  Ще докажем, че \[f(\bigsqcup_i a_i) = \bigsqcup_i f(a_i).\]

  \begin{enumerate}[(1)]
  \item
    % \marginpar{Това включване е вярно за произволна област на Скот $\A_1$.}
    % Ясно е, че за всяко монотонно изображение $f$,
    % понеже $a_i \sqsubseteq \bigsqcup_i a_i$, то $f(a_i) \sqsubseteq f(\bigsqcup_i a_i)$.
    % Това означава, че $f(\bigsqcup_i a_i)$ е горна граница на веригата $(f(a_i))^\infty_{i=0}$ в $\A_2$
    % и следователно
    % \[\bigsqcup_i f(a_i) \sqsubseteq f(\bigsqcup_i a_i).\]
    От \Prop{monotone-chain} веднага получаваме, че
    \[\bigsqcup_i f(a_i) \sqsubseteq f(\bigsqcup_i a_i).\]
  \item
    За другата посока ще използваме свойството, че веригата $\chain{a}{i}$ се стабилизира.
    Нека $n_0$ е индекс, такъв че $(\forall k \geq n_0)[a_k = a_{n_0}]$.
    Това означава, че $\bigsqcup_i a_i = a_{n_0}$.
    Тогава
    \[f(\bigsqcup_i a_i) = f(a_{n_0}) \sqsubseteq \bigsqcup_i f(a_i).\]
  \end{enumerate}
  
  От $(1)$ и $(2)$ следва, че $f(\bigsqcup_i a_i) = \bigsqcup_i f(a_i)$.
\end{hint}

Понеже всяка верига в $\Nat^n_\bot$ се {\em стабилизира}, то
получаваме следното важно следствие.
\begin{framed}
\begin{cor}\label{cr:monotone-is-continuous}
  $\Mon{\Nat^n_\bot}{\Nat_\bot} = \Cont{\Nat^n_\bot}{\Nat_\bot}$.
\end{cor}  
\end{framed}

% От доказателството на (1) за \Prop{stab-continuous} можем да извлечем следното свойство,
% което ще ни бъде полезно по-нататък.
% \begin{proposition}\label{pr:monotone-chain}
%   За всяко изображение $f \in \Mon{\A}{\B}$ и всяка верига $\chain{a}{i}$, е изпълнено, че
%   \[\bigsqcup_i f(a_i) \sqsubseteq f(\bigsqcup_i a_i).\]
% \end{proposition}

Понеже от \Cor{flat-monotone-is-domain} имаме, че монотонните изображения образуват област на Скот, 
то директно получаваме следната важна теорема.

\begin{framed}
\begin{theorem}
  \label{th:continuous-is-domain}
  $\Cont{\Nat^n_\bot}{\Nat_\bot}$ е област на Скот.
\end{theorem}
\end{framed}
% \begin{proof}
%   От \Cor{monotone-is-continuous} имаме, че 
%   \[\Mon{\Nat^n_\bot}{\Nat_\bot} = \Cont{\Nat^n_\bot}{\Nat_\bot}.\]
%   От \Cor{flat-monotone-is-domain} имаме, че 
%   \[(\Mon{\Nat^n_\bot}{\Nat_\bot}, \sqsubseteq, \Omega^{(n)})\]
%   е област на Скот. 
%   Оттук директно получаваме, че 
%   \[(\Cont{\Nat^n_\bot}{\Nat_\bot}, \sqsubseteq, \Omega^{(n)})\] е област на Скот.
% \end{proof}


% \begin{prop}
%   \label{pr:composition}
%   \index{изображения!композиция}
%   Ако $f \in \Cont{\A}{B}$ и $g \in \Cont{\B}{\C}$, то $g \circ f \in \Cont{\A}{\C}$,
%   където \[(g\circ f)(a) \dff g(f(a)).\]
% \end{prop}
% \begin{hint}
%   Нека $\chain{a}{i}$ е верига в $\A$.
%   Да обърнем внимание, че понеже $f \in \Cont{\A}{\B}$,
%   то $f$ е монотонно изображение и тогава $(f(a_i))^\infty_{i=0}$ е верига в $\B$.
%   Тогава:
%   \begin{align*}
%     (g \circ f)(\bigsqcup_i a_i) & = g(f(\bigsqcup_i a_i)) & \comment{\text{от деф.}}\\
%     & = g(\bigsqcup_i f(a_i)) & \comment{f \text{ е непр.}}\\
%     & = \bigsqcup_i g(f(a_i)) & \comment{g \text{ е непр.}}
%   \end{align*}
% \end{hint}

%%% Local Variables:
%%% mode: latex
%%% TeX-master: "../sep"
%%% End:
