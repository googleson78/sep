\section{Оператор за най-малка неподвижна точка}

\Stefan{Това да се остави само с упътване.}

\begin{theorem}
  % \index{$Y_\A$}
  Нека $\A$ е област на Скот и нека $f \in \Cont{\A}{\A}$.
  \marginpar{Знаем от \Th{knaster-tarski}, че най-малката неподвижна точка на $f$ е елемента $\bigsqcup_n f^n(\bot^\A)$.}
  Тогава изображението $Y_\A : \Cont{\A}{\A} \to \A$, определено като
  \[Y_\A(f) = \lfp(f),\]
  е непрекъснато, т.е.
  $Y_\A \in \Cont{\Cont{\A}{\A}}{\A}$.
\end{theorem}
\begin{proof}
  Нека да вземем една верига $(f_n)^\infty_{n=0}$ от непрекъснати изображения.
  Нашата цел е да докажем, че
  \[Y_\A(\bigsqcup_n f_n) = \bigsqcup_n Y_\A(f_n).\]
  Да означим с $h$ точната горна граница на $(f_n)^\infty_{n=0}$.
  Знаем, че $h(a) = \bigsqcup_n f_n(a)$.
  \begin{prop}
    За всяко $k \geq 1$, $h^k(a) = \bigsqcup_n f^k_n(a)$.
  \end{prop}
  \begin{proof}
    Ще докажем твърдението с индукция по $k$, като случая $k = 1$ следва от дефиницията на $h$.
    Нека приемем, че твърдението е вярно за произволно $k \geq 1$.
    Ще докажем, че твърдението е вярно за $k+1$.
    \begin{align*}
      h^{k+1}(a) & = h(h^k(a)) & \\
      & = h(\bigsqcup_n f^k_n(a))& \comment{\text{ от инд. предположение}}\\
      & = \bigsqcup_n h(f^k_n(a))& \comment{h \text{ е непрекъснато изображение}}\\
      & = \bigsqcup_n (\bigsqcup_m f_m(f^k_n(a))). & 
    \end{align*}
    
    Да положим $b_n = f^k_n(a)$, за всяко $n$.
    Понеже $f_n \sqsubseteq f_{n^\prime}$, лесно се съобразява, че за $n \leq n^\prime$
    имаме $b_n \sqsubseteq^\A b_{n^\prime}$.

    Сега да положим $e_{m,n} = f_m(b_n)$.
    Отново, понеже $(b_n)^\infty_{n=0}$ и $(f_m)^\infty_{m=0}$ са вериги, имаме 
    \[m \leq m^\prime\ \&\ n\leq n^\prime\ \Rightarrow\ e_{m,n} \sqsubseteq^\A e_{m^\prime,n^\prime}.\]
    Това означава, че можем да приложим \Th{double-chain} за множеството $E = \{e_{m,n} \mid m,n \in \Nat\}$.
    Получаваме, че
    \begin{align*}
      h^{k+1}(a) & = \bigsqcup_n (\bigsqcup_m f_m(f^k_n(a))) & \comment{\text{ от горното равенство}}\\
      & = \bigsqcup_n (\bigsqcup_m e_{m,n}) & \comment{\text{ от определението на }e_{m,n}}\\
      & = \bigsqcup_n e_{n,n} & \comment{\text{ от \Th{double-chain}}}\\
      & = \bigsqcup_n f_n(f^k_n(a))  = \bigsqcup_n f^{k+1}_n(a) & 
    \end{align*}
    С това твърдението е доказано.
  \end{proof}
  Сега вече сме готови да докажем непрекъснатостта на $Y_\A$.
  Имаме, че:
  \begin{align*}
    Y_\A(\bigsqcup_n f_n) & = Y_\A(h) & \comment{\text{ от опр. на }h}\\
    & = \bigsqcup_m h^m(\bot^\A) & \comment{\text{ от опр. на }Y_\A }\\
    & = \bigsqcup_m (\bigsqcup_n f^m_n(\bot^\A)) & \comment{\text{ от горното твърдение}}.
  \end{align*}
  
  Да положим $e_{m,n} = f^m_n(\bot^\A)$.
  Отново лесно се съобразява, че 
  \[m \leq m^\prime\ \&\ n\leq n^\prime\ \Rightarrow\ e_{m,n} \sqsubseteq^\A e_{m^\prime,n^\prime}.\]
  Получаваме, че
  \begin{align*}
    Y_\A(\bigsqcup_n f_n) & = \bigsqcup_m (\bigsqcup_n f^m_n(\bot^\A)) & \comment{\text{ от горното равенство}}\\
    & = \bigsqcup_m (\bigsqcup_n e_{m,n}) & \comment{\text{ от опр. на }e_{m,n}}\\
    & = \bigsqcup_n(\bigsqcup_m e_{m,n}) & \comment{\text{ от \Th{double-chain}}}\\
    & = \bigsqcup_n (\bigsqcup_m f^m_n(\bot^\A)) = \bigsqcup_n Y_\A(f_n). & \comment{\text{ от опр. на }Y_\A}.
  \end{align*}
\end{proof}



%%% Local Variables:
%%% mode: latex
%%% TeX-master: "../sep"
%%% End:
