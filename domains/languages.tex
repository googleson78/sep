\section{Допълнителен материал}
\subsection{Регулярни езици}\index{език!регулярен}

Да фиксираме азбуката $\Sigma = \{a_1,\dots,a_k\}$.
Да дефинираме полиномите над $\Sigma$ като
\[\tau ::= \emptyset\ |\ \varepsilon\ |\ a_i \cdot X_j\ |\ \tau_1 + \tau_2.\]
където $i = 1, \dots,k$, а $X$ е променлива.
За всеки полином $\tau[X_1,\dots,X_n]$ дефинираме оператора 
\marginpar{Тук за променливите използваме главни букви за да се сещаме, че те приемат стойности множества от думи.}
\[\val{\tau}: \mathcal{P}(\Sigma^\star)^n \to \mathcal{P}(\Sigma^\star)\]
 по следния начин:
\begin{itemize}
\item
    $\val{\emptyset}(L_1,\dots,L_n) = \emptyset$.
\item 
  $\val{\varepsilon}(L_1,\dots,L_n) = \{\varepsilon\}$.
\item 
  $\val{a_i \cdot X_j}(L_1,\dots,L_n) = \{a_i\} \cdot L_j$.
\item
  $\val{\tau_1 + \tau_2}(L_1,\dots,L_n) = \val{\tau_1}(L_1,\dots,L_n) \cup \val{\tau_2}(L_1,\dots,L_n)$.
\end{itemize}

\begin{problem}
  Докажете, че за всеки полином $\tau$ имаме, че $\val{\tau}$ е непрекъснато изображение в областта на Скот
  $\mathcal{S} = ( \mathcal{P}(\Sigma^\star),\subseteq, \emptyset)$.
\end{problem}


\begin{example}
  Да разгледаме системата 
  \marginpar{$\tau_1[X_1,X_2] \equiv b \cdot X_1 + a \cdot X_2$}
  \marginpar{$\tau_2[X_1,X_2] \equiv \varepsilon$}
  \begin{align*}
    & X_1 = b \cdot X_1 + a\cdot X_2\\
    & X_2 = \varepsilon.
  \end{align*}

  % Понеже $\val{\tau}$ е непрекъснат оператор, то той има най-малка неподвижна точка.
  Дефинираме непрекъснатия оператор 
  \[\Gamma:\mathcal{P}(\Sigma^\star)^2 \to \mathcal{P}(\Sigma^\star)^2,\]
  където:
  \[\Gamma(L_1,L_2) = (\val{\tau_1}(L_1,L_2), \val{\tau_2}(L_1,L_2)).\]

  От Теоремата на Клини ние знам как можем да намерим най-малката неподвижна точка на $\Gamma$,
  която ще бъде и най-малкото решение на горната система.

  \begin{itemize}
  \item 
    $(L_0,M_0) \df (\emptyset,\emptyset)$;
  \item
    $(L_1,M_1) \df \Gamma(L_0,M_0) = (\val{\tau_1}(L_0,M_0), \val{\tau_2}(L_0,M_0)) = (\emptyset, \varepsilon)$;
  \item
    $(L_2,M_2) \df \Gamma(L_1,M_1) = (\val{\tau_1}(L_1,M_1), \val{\tau_2}(L_1,M_1)) = (\{a\},\varepsilon)$;
  \item
    $(L_3,M_3) \df \Gamma(L_2,M_2) = (\val{\tau_1}(L_2,M_2), \val{\tau_2}(L_2,M_2)) = (\{ba,a\},\varepsilon)$;
  \item
    $(L_4,M_4) \df \Gamma(L_3,M_3) =(\val{\tau_1}(L_3,M_3), \val{\tau_2}(L_3,M_3)) = (\{bba, ba,a\},\varepsilon)$;
  \item
    $(L_5,M_5) \df \Gamma(L_4,M_4) = ( \val{\tau_1}(L_4,M_4), \val{\tau_2}(L_4,M_4)) = (\{bba, bba, ba,a\},\varepsilon)$.
  \end{itemize}
  Лесно се съобразява, че $L_n = \{ b^ka \mid k < n\}$.
  Тогава
  \[\lfp( \Gamma ) = (\bigcup_n L_n, \{\varepsilon\}) = (b^\star a, \{\varepsilon\} ).\]
\end{example}


\begin{problem}
  Докажете, че най-малкото решение на системата 
  \begin{align*}
    & X_1 = a \cdot X_1 + b \cdot X_2 + \varepsilon\\
    & X_2 = b \cdot X_2 + \varepsilon
  \end{align*}
  е двойката $(a^\star b^\star, b^\star)$.
\end{problem}

\begin{problem}
  Да разгледаме системата от оператори
  \begin{align*}
    & \val{\tau_1}(L_1,\dots,L_n) = L_1\\
    & \ \ \vdots\\
    & \val{\tau_n}(L_1,\dots,L_n) = L_n.
  \end{align*}
  Знаем, че тя притежава най-малко решение $(\hat{L}_1,\dots,\hat{L}_n)$.
  Докажете, че всеки от езиците $\hat{L}_i$ е регулярен.

  Докажете, че всеки регулярен език е елемент от най-малкото решение 
  на някоя система от оператори от горния вид.
\end{problem}

\subsection{Безконтекстни езици}\index{език!безконтекстен}

Да фиксираме азбуката $\Sigma = \{a_1,\dots,a_n\}$.
Да дефинираме термове от тип 1 като
\[\tau ::= X_i\ |\ a_j\ |\ \varepsilon\ |\ \emptyset\ |\ \tau_1 \cdot \tau_2\ |\ (\tau_1 + \tau_2),\]
където $j = 1, \dots,n$, а $X_i$ са изброимо безкрайна редица от променливи.
За всеки терм $\tau[X_1,\dots,X_n]$ дефинираме оператора 
\[\val{\tau}: (\mathcal{P}(\Sigma^\star))^n \to \mathcal{P}(\Sigma^\star)\]
 по следния начин:
\begin{itemize}
\item 
  $\val{X_i}(L_1,\dots,L_n) = L_i$.
\item 
  $\val{a_j}(L_1,\dots,L_n) = \{a_j\}$.
\item 
  $\val{\varepsilon}(L_1,\dots,L_n) = \{\varepsilon\}$.
\item 
  $\val{\emptyset}(L_1,\dots,L_n) = \emptyset$.
\item 
  $\val{\tau_1 \cdot \tau_2}(L_1,\dots,L_n) = \val{\tau_1}(L_1,\dots,L_n) \cdot \val{\tau_2}(L_1,\dots,L_n)$.
\item
  $\val{\tau_1 + \tau_2}(L_1,\dots,L_n) = \val{\tau_1}(L_1,\dots,L_n) \cup \val{\tau_2}(L_1,\dots,L_n)$.
\end{itemize}

\begin{problem}
  Докажете, че за всеки терм $\tau$, $\val{\tau}$ е непрекъснато изображение в областта на Скот
  $\mathcal{S} = ( \mathcal{P}(\Sigma^\star),\subseteq, \emptyset)$.
\end{problem}

\begin{problem}
  Докажете, че $\{a^nb^n \mid n\in \Nat\} = \lfp(\val{\tau})$, където 
  \[\tau[X] \equiv \varepsilon + a \cdot X \cdot b.\]
  С други думи, $\{a^nb^n \mid n \in \Nat\}$ е най-малкото решение на уравнението
  \[X = a \cdot X \cdot b + \varepsilon.\]
\end{problem}

Нека сега да разгледаме термовете $\tau_1[X_1,\dots,X_n], \dots, \tau_n[X_1,\dots,X_n]$.

\begin{problem}
  Да разгледаме системата от оператори
  \begin{align*}
    & \val{\tau_1}(L_1,\dots,L_n) = L_1\\
    & \ \ \vdots\\
    & \val{\tau_n}(L_1,\dots,L_n) = L_n.
  \end{align*}
  Знаем, че тя притежава най-малко решение $(\hat{L}_1,\dots,\hat{L}_n)$.
  Докажете, че всеки от езиците $\hat{L}_i$ е безконтекстен.

  Докажете, че всеки безконтекстен език е елемент от най-малкото решение 
  на някоя система от оператори от горния вид.
\end{problem}

\begin{problem}
  \marginpar{Това е аналог на нормалната форма на Чомски}
  Да дефинираме термове от тип 2 като
  \[\tau ::= a_j\ |\ \varepsilon\ |\ \emptyset\ |\ X_i \cdot X_k\ |\ (\tau_1 + \tau_2),\]
  където $j = 1, \dots,n$, а $X_i$ са изброимо безкрайна редица от променливи.
  Докажете горното твърдение, като замените термовете от тип 1 с тези от тип 2.
\end{problem}

\begin{example}
  Да разгледаме системата
  \begin{align*}
    & X_1 = X_3 \cdot X_2 + \varepsilon\\
    & X_2 = X_1 \cdot X_4\\
    & X_3 = a\\
    & X_4 = b.
  \end{align*}


  % \begin{align*}
  %   & \val{\varepsilon + X_3 \cdot X_2}(L_1, L_2, L_3, L_4) = L_1\\
  %   & \val{X_1 \cdot X_4}(L_1, L_2, L_3, L_4) = L_2\\
  %   & \val{a}(L_1, L_2, L_3, L_4) = L_3\\
  %   & \val{b}(L_1, L_2, L_3, L_4) = L_4\\
  % \end{align*}
  Нека $(\hat{L}_1, \hat{L}_2, \hat{L}_3, \hat{L}_4)$ е най-малкото решение на системата.
  Докажете, че $\hat{L}_1 = \{a^nb^n\mid n \in \Nat\}$ $\hat{L}_2 = \{a^nb^{n+1}\mid n \in \Nat\}$,
  $\hat{L}_3 = \{a\}$ и $\hat{L}_4 = \{b\}$.
\end{example}

%%% Local Variables:
%%% mode: latex
%%% TeX-master: "../sep"
%%% End:
