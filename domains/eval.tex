\begin{definition}\label{def:eval}\index{eval}
  Нека $\D$ и $\E$ са области на Скот. Дефинираме изображението 
  \[\texttt{eval}: \Cont{\D}{\E} \times \D \to \E,\]
  по следния начин:
  \[\texttt{eval}(f,d) \dff f(d).\]  
\end{definition}

\begin{problem}\label{prob:eval}
  \marginpar{\cite[стр. 186]{models-of-computation}}
  Докажете, че $\texttt{eval}$ е непрекъснато изображение, т.е.
  \[\texttt{eval} \in \Cont{\Cont{\D}{\E} \times \D}{\E}.\]
\end{problem}
\begin{proof}
  Според \Prop{continuous-arguments}, достатъчно е да докажем, че $\texttt{eval}$ е непрекъснато
  изображение по всеки от двата си аргумента поотделно.
  
  Първо, нека $\chain{f}{n}$ е верига от елементи на $\Cont{\D}{\E}$ и $d$ е произволен елемент на $\D$.
  Тогава
  \[\texttt{eval}(\bigsqcup_n f_n,d) = (\bigsqcup_n f_n)(d) = \bigsqcup_n \{f_n(d)\} = \bigsqcup_n \texttt{eval}(f_n,d),\]
  т.е. изображението $\texttt{eval}$ е непрекъснато по първия си аргумент.
  
  Нека сега $\chain{d}{n}$ е верига от елементи на $\D$.
  Тогава за произволен елемент $f$ на $\Cont{\D}{\E}$ получаваме, че
  \[\texttt{eval}(f,\bigsqcup_n d_n) = f(\bigsqcup_n d_n) = \bigsqcup_n \{f(d_n)\} = \bigsqcup_n \texttt{eval}(f,d_n).\]
\end{proof}

%%% Local Variables:
%%% mode: latex
%%% TeX-master: "../sep"
%%% End:
