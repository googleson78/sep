\section{Частична коректност}

\begin{problem}
  Дадена е следната програма на езика \REC:
  
  \begin{haskellcode}
h(x,y) = f(x,y,1) where
  f(x,y,z) = if x <= 1 then z 
               else f(x-1, y, z*g(x,y))
  g(x,y) = if y == 0 then 1 
             else x * g(x, y - 1)
  \end{haskellcode}
  
  Докажете, че $(\forall x,y\in\Nat)[\D_V\val{\vv{h}}(x,y) \neq \bot \implies \D_V\val{\vv{h}}(x,y) = (x!)^y]$.
\end{problem}
\begin{solution}
  \Stefan{В главата за REC означавах операторите с $\Delta$}
  \marginpar{Понеже разглеждаме денотационна семантика по стойност, то работим с {\bf точни} изображения}
  Да разгледаме непрекъснатите термални оператори 
  \begin{align*}
    & \Delta_1:\S_3\times\S_2 \to \S_3\\
    & \Delta_2:\S_3\times\S_2 \to \S_2,
  \end{align*}
  съответстващи на термовете дефиниращи \vv{f} и \vv{g}:
  \begin{align*}
    \Delta_1(f,g)(x,y,z) = &
    \begin{cases}
      z & \text{, ако } x \leq 1\\
      f(x-1,y,z\cdot g(x,y)) &  \text{, ако } x > 1\\
      \bot & \text{, ако }\bot \in \{x,y,z\}
    \end{cases}
    \\
    \Delta_2(f,g)(x,y) = &
    \begin{cases}
      1 & \text{, ако } y = 0\\
      x \cdot g(x,y-1) & \text{, ако }y > 0\\
      \bot & \text{, ако } \bot \in \{x,y\}.
    \end{cases}
  \end{align*}
  Тогава операторът $\Delta:\S_3\times\S_2 \to \S_3\times\S_2$, дефиниран като:
  \marginpar{Знаем, че $\delta_1, \delta_2$ е най-малкото решение на системата от уравнения:
    \begin{align*}
      & \Delta_1(f,g) = f\\
      & \Delta_2(f,g) = g
    \end{align*}}
  \[\Delta(f,g) = (\Delta_1(f,g),\Delta_2(f,g)),\] също е непрекъснат.
  Ако означим $(\delta_1,\delta_2) = \lfp(\Delta)$, то по дефиниция
  % \marginpar{Следвайки дефинициите, $\D_V\val{\vv{h}}(x,y) \simeq \delta_1(x,y,1)$}
  \[\D_V\val{\vv{h}}(x,y) = \delta_1(x,y,1).\]
  Сега дефинираме следните свойства:
  \begin{align*}
    & P_1(f,g) \dfff (\forall x,y,z\in\Nat)[f(x,y,z) \neq \bot \implies f(x,y,z) = z.(x!)^y],\\
    & P_2(f,g) \dfff (\forall x,y \in \Nat)[g(x,y) \neq \bot \implies g(x,y) = x^y].
  \end{align*}
  Понеже те са от тип частична коректност, те са и непрекъснати.
  Ще докажем с индукционно правило на Скот, приложено върху областта на Скот $\S_3 \times \S_2$ за непрекъснатото изображение $\Gamma$,
  че $P(\delta_1,\delta_2)$, където 
  \[P(f,g) \dfff P_1(f,g)\ \&\ P_2(f,g).\]
  $P$ също е непрекъснато свойство, защото е конюнкция на две непрекъснати свойства.

  Очевидно е, че $P(\Omega^{(3)},\Omega^{(2)})$.
  Сега да приемем, че $P(f,g)$ е изпълнено. Ще докажем $P(\Delta(f,g))$.
  \begin{enumerate}[1)]
  \item 
    Ще докажем, че е изпълнено $P_1(\Delta(f,g))$, т.е.
    \[(\forall x,y,z\in\Nat)[\Delta_1(f,g)(x,y,z) \neq \bot \implies \Delta_1(f,g)(x,y,z) = z.(x!)^y].\]
    Нека $\Delta_1(f,g)(x,y,z) \neq \bot$, за произволни $x,y,z \in \Nat$.
    \begin{itemize}
    \item 
      ако $x \leq 1$, то $\Delta_1(f,g)(x,y,z) = z = z.(x!)^y$, което следва от дефиницията на $\Delta_1$.
    \item
      ако $x > 1$, тогава имаме, че
      \[\Delta_1(f,g)(x, y, z) = f(x-1, y, z.g(x,y)) \neq \bot,\]
      откъдето следва, че $g(x,y) \neq \bot$, защото $f$ е точна функция.
      Тогава от $P_2(f,g)$ получаваме, че $g(x,y) = x^y$.
      Обединявайки всичко, получаваме:
      \begin{align*}
        \Delta_1(f,g)(x,y,z)&  = f(x-1,y,z.g(x,y)) & (\text{от деф.})\\
                            & = f(x-1,y,z.x^y) & (\text{от }P_2(f,g))\\
                            & = (z.x^y).((x-1)!)^y & (\text{от }P_1(f,g))\\
                            & = z.(x!)^y.
      \end{align*}
    \end{itemize}
  \item
    Сега ще докажем, че е изпълнено $P_2(\Delta(f,g))$, т.е.
    \[(\forall x,y\in\Nat)[\Delta_2(f,g)(x,y) \neq \bot \implies \Delta_2(f,g)(x,y) = x^y].\]
    Нека $\Delta_2(f,g)(x,y) \neq \bot$, за произволни $x,y \in \Nat$.
    \begin{itemize}
    \item 
      ако $y = 0$, то $\Delta_2(f,g)(x,0) = 1 = x^0$, което следва от дефиницията на $\Delta_2$.
    \item
      ако $y > 0$, то имаме $\Delta_2(f,g)(x,y) = x \cdot g(x, y-1) \neq \bot$.
      Оттук следва, че $g(x,y-1) \neq \bot$ и от $P_2(f,g)$ получаваме, че 
      $g(x,y-1) = x^{y-1}$.
      Обединявайки всичко, получаваме:
      \begin{align*}
        \Delta_2(f,g)(x,y) & = x.g(x,y-1) & (\text{от деф.})\\
        & = x.x^{y-1} & (\text{от }P_2(f,g))\\
        & = x^y.
      \end{align*}
    \end{itemize}
  \end{enumerate}
  Заключаваме, че $P(\Delta(f,g))$.
  Следователно, $P(\delta_1,\delta_2)$ и в частност, 
  \begin{align*}
    \D_V\val{\vv{h}}(x) & = \delta_1(x,y,1) & (\text{от деф.})\\
    & = 1.(x!)^y & (\text{от }P_1(\delta_1,\delta_2))\\
    & = (x!)^y.
  \end{align*}
\end{solution}

\begin{problem}
  Дадена е следната програма на езика $\REC$:
  
  \begin{haskellcode}
h(x) = g(x,0) where
  f(x,y) = if y == 0 then 2^x
             else if x == y then 3^y
               else 3*f(x - 1, y - 1) + 2*f(x- 1, y)
  g(x,y) = if x < y then 0 
             else g(x, y + 1) + f(x, y)
  \end{haskellcode}
  
  Докажете, че $(\forall x\in \Nat)[\D_V\val{\vv{h}}(x) \neq \bot \implies \D_V\val{\vv{h}}(x) = 5^x]$.
\end{problem}
\begin{hint}
  Използвайте следните свойства:
  \marginpar{$\binom{x}{y}$ - Нютонов бином}
  \begin{itemize}
  \item 
    $5^x = \sum^x_{i=0}3^i2^{x-i}\binom{x}{i}$;
  \item
    $\binom{x}{i} = \binom{x-1}{i-1} + \binom{x-1}{i}$;
  \item
    $3^i2^{x-i}\binom{x}{i} = 3.3^{i-1}2^{x-i}\binom{x-1}{i-1} + 2.3^{i}2^{x-1-i}\binom{x-1}{i}$.
  \end{itemize}
  Тогава приложете правилото на Скот за:
  \begin{align*}
    & P_1(f,g) \dfff (\forall x,y)[f(x,y) \neq \bot\ \&\ x\geq y \implies f(x,y) = 3^y2^{x-y}\binom{x}{y}],\\
    & P_2(f,g) \dfff (\forall x,y)[g(x,y) \neq \bot \implies g(x,y) = \sum^{x}_{i=y}3^{i}2^{x-i}\binom{x}{i}].
  \end{align*}
\end{hint}

Сега ще дадем пълно решение на тази задача.

\begin{solution}
  Да разгледаме следните две непрекъснати изображения, които съответстват на термовете дефиниращи 
  \vv{f} и \vv{g}:
  
  \begin{align*}
    \Delta_1(f,g)(x,y) \dff &
    \begin{cases}
      \bot, & \text{ако } \bot \in \{x,y\}\\
      2^x, & \text{ако } y = 0\\
      3^y, & \text{ако } x = y\\
      2\cdot  f(x-1,y-1)+3\cdot f(x-1,y), & \text{иначе}
    \end{cases}
    \\
    \Delta_2(f,g)(x,y) \dff &
    \begin{cases}
      \bot, & \text{ако } \bot \in \{x,y\}\\
      0, & \text{ако } y > x\\
      g(x,y+1) + f(x,y), & \text{иначе}.
    \end{cases}
  \end{align*}
  Тогава $\Delta:\S_2\times\S_2 \to \S_2\times\S_2$, дефинирано като:
  \[\Delta(f,g) \dff (\Delta_1(f,g),\Delta_2(f,g)),\] също е непрекъснато изображение.
  Ако означим с $(\delta_1,\delta_2) = \lfp(\Delta)$, то по дефиниция
  \[\D_V\val{\vv{h}}(x) = \delta_2(x,0).\]
  Сега дефинираме следните свойства:
  \begin{align*}
    P_1(f,g) & \iff (\forall x,y\in\Nat)[f(x,y) \neq \bot\ \&\ y\leq x \implies f(x,y) = 2^{y}3^{x-y}\binom{x}{y}],\\
    P_2(f,g) & \iff (\forall x,y\in\Nat)[g(x,y) \neq \bot \implies g(x,y) = \sum^{x}_{z=y}2^z3^{x-z}\binom{x}{z}].
  \end{align*}
  Понеже те са от тип частична коректност, те са и непрекъснати.
  Ще докажем с индукционно правило на Скот, приложено върху областта на Скот $\S_2 \times \S_2$ за непрекъснатото изображение $\Delta$,
  че $P(\delta_1,\delta_2)$, където 
  \[P(f,g) \iff P_1(f,g)\ \&\ P_2(f,g).\]
  $P$ също е непрекъснато свойство, защото е конюнкция на две непрекъснати свойства.

  Очевидно е, че $P(\Omega^{(2)},\Omega^{(2)})$.
  Сега да приемем, че $P(f,g)$ е изпълнено. Ще докажем $P(\Delta(f,g))$.
  \begin{enumerate}[1)]
  \item 
    Ще докажем, че $P_1(\Delta(f,g))$, т.е.
    \[(\forall x,y\in\Nat)[\Delta_1(f,g)(x,y) \neq \bot\ \&\ y\leq x \implies \Delta_1(f,g)(x,y) = 2^{y}3^{x-y}\binom{x}{y}].\]
    Нека $\Delta_1(f,g)(x,y) = u \neq \bot$ и $y \leq x$.
    Според дефиницията на $\Delta_1$, трябва да разгледаме три случая:
    \begin{itemize}
    \item 
      \marginpar{$\binom{x}{0} = 1$}
      ако $y = 0$, тогава 
      \[\Delta_1(f,g)(x,y) = 3^x = 2^{0}3^{x-0}\binom{x}{0}.\]
    \item
      \marginpar{$\binom{x}{x} = 1$}
      ако $x = y$, тогава
      \[\Delta_1(f,g)(x,y) = 2^y = 2^{y}3^{x-y}\binom{x}{y}.\]
    \item
      иначе, $0 < y < x$ и 
      \[\Delta_1(f,g)(x,y) = 2.\underbrace{f(x-1,y-1)}_{\neq \bot} + 3.\underbrace{f(x-1,y)}_{\neq \bot} = u \neq \bot.\]
      % Щом $u \neq \bot$, то $f(x-1,y-1) \neq \bot$ и $f(x-1,y) \neq \bot$.
      Понеже е изпълнено $P_1(f,g)$, то имаме, че:
      \begin{align*}
        & f(x-1,y-1) = 2^{y-1}3^{x-1-(y-1)}\binom{x-1}{y-1},\\
        & f(x-1,y) = 2^{y}3^{x-1-y}\binom{x-1}{y}.
      \end{align*}
      Обединявай всичко това, получаваме:
      \begin{align*}
        \Delta_1(f,g)(x,y) & = 2.f(x-1,y-1) + 3.f(x-1,y) & (\text{от деф.})\\
        & = 2.2^{y-1}3^{x-y}\binom{x-1}{y-1} + 3.2^{y}3^{x-1-y}\binom{x-1}{y} & (\text{от }P_1(f,g))\\
        & = 2^y3^{x-y}\binom{x-1}{y-1} + 2^y3^{x-y}\binom{x-1}{y}\\
        & = 2^y3^{x-y}[\binom{x-1}{y-1} + \binom{x-1}{y}]\\
        & = 2^y3^{x-y}\binom{x}{y}.
      \end{align*}
    \end{itemize}
    \item
      Ще докажем, че $P_2(\Delta(f,g))$, т.е.
      \[(\forall x,y\in\Nat)[\Delta_2(f,g)(x,y) \neq \bot \implies \Delta_2(f,g)(x,y) = \sum^{x}_{z=y}2^{z}3^{x-z}\binom{x}{z}].\]
      Нека $\Delta_2(f,g)(x,y) = u \neq \bot$. Според дефиницията на $\Delta_2$, трябва да разгледаме два случая:
      \begin{itemize}
      \item 
        ако $y > x$, то
        \[\Delta_2(f,g)(x,y) = 0 = \sum^{x}_{z=y}2^{z}3^{x-z}\binom{x}{z},\]
        защото сумата от елементите на празното множество е $0$.
      \item
        ако $y \leq x$, то
        \[\Delta_2(f,g)(x,y) = \underbrace{g(x,y+1)}_{\neq \bot} + \underbrace{f(x,y)}_{\neq \bot} = u \neq \bot.\]
        Понеже е изпълнено $P_1(f,g)$ и $P_2(f,g)$, то имаме, че:
        \begin{align*}
          \Delta_2(f,g)(x,y) & = g(x,y+1) + f(x,y) & (\text{от деф. на }\Delta_2)\\
          & = g(x,y+1) + 2^{y}3^{x-y}\binom{x}{y} & (\text{от }P_1(f,g))\\
          & = \sum^{x}_{z=y+1}2^{z}3^{x-z}\binom{x}{z} +  2^{y}3^{x-y}\binom{x}{y} & (\text{от }P_2(f,g))\\
          & = \sum^x_{z=y}2^{z}3^{x-z}\binom{x}{z}.
        \end{align*}
      \end{itemize}
    \end{enumerate}
    Накрая заключаваме, че $P(\Delta(f,g))$.
    Следователно, $P(\delta_1,\delta_2)$ и в частност, 
    \begin{align*}
      D_V\val{\vv{h}}(x) & = \delta_2(x,0) & (\text{от деф.})\\
      & = \sum^x_{z=0}2^z3^{x-z}\binom{x}{z} & (\text{от }P_2(\delta_1,\delta_2))\\
      & = (2+3)^x\\
      & = 5^x.
    \end{align*}
\end{solution}

\begin{problem}
  Дадена е следната програма на езика $\REC$:
  
  \begin{haskellcode}
h(x) = g(x, x) where
  f(x, y) = if y == 0 || x == y then 1
              else f(x - 1, y - 1) + f(x - 1, y)
  g(x, y) = if y == 0 then 1
              else g(x, y - 1) + f(x, y)^2
  \end{haskellcode}
  
  Докажете, че $(\forall x\in \Nat)[\D_V\val{\vv{h}}(x) \neq \bot \implies \D_V\val{\vv{h}}(x) = \binom{2x}{x}]$.
\end{problem}
\begin{hint}
  Използвайте, че:
  \begin{itemize}
  \item
    $\binom{x+y}{z} = \sum^z_{i=0} \binom{x}{i}\binom{y}{z-i}$;
  \item
    $\binom{x}{y} = \binom{x}{x-y}$;
  \item
    $\binom{2x}{x} = \sum^x_{i=0} \binom{x}{i}\binom{x}{x-i} = \sum^{x}_{i=0}\binom{x}{i}^2$.
  \end{itemize}
\end{hint}

\begin{problem}
  Да разгледаме програмата на езика $\REC$:
  
  \begin{haskellcode}
h(x) = g(x,x) where
  f(x,y) = if y == 0 || x == y then 1
             else f(x - 1,y - 1) + f(x - 1,y)
  g(x,y) = if y == 0 then 1
             else g(x,y-1) + f(x+y,y)
  \end{haskellcode}
  
  Докажете, че $(\forall x\in \Nat)[\D_V\val{\vv{h}}(x) \neq \bot \implies \D_V\val{\vv{h}}(x) = \binom{2x+1}{x}]$.
\end{problem}

\begin{problem}
  Дадена е следната програма на езика $\REC$:
  
  \begin{haskellcode}
h(x) = f(0, x, x + 1) where
  f(x, y, z) = if x > y then z 
                 else f(x + 1, y, g(0, x, z))
  g(x, y, z) = if x == y then z
                 else g(x + 1, y, 2 + z)
  \end{haskellcode}
  
  Да се докаже, че:
  $(\forall x\in\Nat)[\D_V\val{\vv{h}}(x) \neq \bot \implies \D_V\val{\mathbf{h}}(x) = (x+1)^2]$.
\end{problem}

\begin{problem}
  Дадена е следната програма на езика $\REC$:
  
  \begin{haskellcode}
h(y) = f(0, y) where
  f(x, y) = if x == y then x
              else  f(x + 1, y) + g(0, x)
  g(x, y) = if x == y then 1
              else g(x + 1, y) + 2
  \end{haskellcode}
  
  Да се докаже, че:
  $(\forall x\in\Nat)[\D_V\val{\vv{h}}(x) \neq \bot \implies \D_V\val{\vv{h}}(x) = x^2+x]$.
\end{problem}

\begin{problem}
  Да разгледаме програмата на езика $\REC$:
  
  \begin{haskellcode}
h(x) = f(x) where 
  f(x) = if x <= 1 then 4 
           else g((f(x-1))^2,(f(x-2))^4)
  g(x,y) = if y == 0 then 0 
             else g(x,y-1) + x
  \end{haskellcode}
  
  Докажете, че $(\forall a\in\Nat)[\D_V\val{\vv{h}}(a) \neq \bot \implies \log_2(\D_V\val{\vv{h}}(a)) \equiv 2 \bmod 10]$.
\end{problem}

\begin{problem}
  Да разгледаме следната програма на езика $\REC$:
  
  \begin{haskellcode}
rm(x,y) = f(x,y) where
  f(x, y) = if x == 0 then 0
              else if f(x - 1, y) + 1 == y then 0
                else f(x-1, y) + 1
  \end{haskellcode}
  
  Докажете частична коректност на $\D_V\val{\vv{rm}}$ относно $I$ и $O$, където
  \begin{align*}
    & I(x,y) \dfff x,y\in\Nat\\
    & O(x,y,z) \dfff z\text{ е остатъкът при делението на $x$ с $y$}.
  \end{align*}
\end{problem}


\begin{problem}
  Да разгледаме следната програма на езика $\REC$:
  
  \begin{haskellcode}
qt(x, y) = g(x, y, 0) where
  f(x, y) = if x == 0 then 0
              else if f(x - 1, y) + 1 == y then 0
                else f(x-1, y) + 1
  g(x, y, z) = if x == 0 then z
                 else if f(x-1, y) + 1 == y then g(x-1, y, z+1)
                   else g(x - 1, y, z)
  \end{haskellcode}
  
  Докажете частична коректност на $\D_V\val{\vv{qt}}$ относно $I$ и $O$, където
  \begin{align*}
    & I(x,y) \dfff x,y\in\Nat\\
    & O(x,y,z) \dfff z\text{ е цялата част при делението на $x$ с $y$}.
  \end{align*}
\end{problem}

\begin{problem}
  Лесно се вижда, че всяко крайно множество 
  \[D = \{y_0 < y_1 < \cdots < y_k\}\] можем
  еднозначно да кодираме като числото $x = \sum^k_{i=0} 2^{y_i}$. Нека с $D_x$ да означим крайното множество с код $x$.
  \marginpar{Имаме, че $x = \sum_{y \in D_x}2^y$}
  Да разгледаме следната програма на езика $\REC$:

  \begin{haskellcode}
mem(x, y) = f(g(x, 2^y, 0), 2) where
  f(x, y) = if x == 0 then 0
              else if f(x - 1, y) + 1 == y then 0
                else f(x-1, y) + 1
  g(x, y, z) = if x == 0 then z
                 else if f(x-1, y) + 1 == y then g(x-1, y, z+1)
                   else g(x - 1, y, z)
  \end{haskellcode}
  
  Докажете частична коректност на $\D_V\val{\vv{mem}}$ относно свойствата $I$ и $O$, където
  \begin{align*}
    & I(x,y) \dfff x,y\in\Nat\\
    & O(x,y,z) \dfff (z = 1 \iff y \in D_x).
  \end{align*}
\end{problem}

%%% Local Variables:
%%% mode: latex
%%% TeX-master: "../sep-notes"
%%% End:
