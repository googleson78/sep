\section{Правило на Скот}
\marginpar{\cite[стр. 166]{winskel}}
\index{правило на Скот}
\begin{itemize}
\item 
  Нека $\A$ е област на Скот и нека $f \in \Cont{\A}{\A}$.
\item
  Всяко $P \subseteq A$ ще наричаме свойство.
\item
  \marginpar{С $\texttt{lfp}(f)$ означаваме най-малката неподвижна точка на $f$ (от англ. least fixed point)}
  Тогава {\bf правилото на Скот} гласи следното:
  \begin{prooftree}
  \AxiomC{$P(\bot)$}
  \AxiomC{$(\forall a \in \A)[P(a) \implies P(f(a))]$}
  \BinaryInfC{$P(\texttt{lfp}(f))$}
\end{prooftree}
\end{itemize}

\begin{proof}
  
\end{proof}


\begin{problem}
  \marginpar{Сравнете с \Prop{prefix-point}}
  Нека $f \in \Cont{\A}{\A}$.
  Да означим множеството от преднеподвижни точки на $f$ като
  \[\texttt{Pref}(f) \dff \{a \in \A \mid f(a) \sqsubseteq a\}.\]
  Тогава 
  \[(\forall a \in \A)[a \in \texttt{Pref}(f) \implies \lfp(f) \sqsubseteq a].\]
\end{problem}
\begin{proof}
  Да фиксираме елемент $a \in \texttt{Pref}(f)$.
  Да разгледаме непрекъснатото свойство
  \marginpar{Сами се убедете, че $P$ е непрекъснато свойство!}
  \[P(b) \dfff b \sqsubseteq a.\]
  Ясно е, че $P(\bot)$.
  Нека $b\in \A$, за който $P(b)$. Ще докажем, че $P(f(b))$.
  \begin{align*}
    b \sqsubseteq a & \implies f(b) \sqsubseteq f(a) & \comment{f \text{ е мон.}}\\
    & \implies f(b) \sqsubseteq f(a) \sqsubseteq a & \comment{a \in \texttt{Pref}(f)}\\
    & \implies f(b) \sqsubseteq a & \comment{\sqsubseteq \text{ е транз. рел.}}.
  \end{align*}
  От правилото на Скот, заключаваме, че $P(\lfp(f))$, т.е.
  $\lfp(f) \sqsubseteq a$.
\end{proof}
  
\begin{example}
  Нека $f,g \in \Cont{\A}{\A}$ като имаме свойствата:
  \begin{itemize}
  \item
    $f(\bot) \sqsubseteq g(\bot)$;
  \item
    $f \circ g \sqsubseteq g \circ f$.
  \end{itemize}
  Докажете, че $\lfp(f) \sqsubseteq \lfp(g)$.
\end{example}
\begin{proof}
  Разгледайте непрекъснатото свойството 
  \[P(a) \dfff f(a) \sqsubseteq g(a).\]
  От условието имаме, че $P(\bot)$.
  Нека $P(a)$. Ще докажем, че $P(g(a))$.
  \begin{align*}
    P(a) & \iff f(a) \sqsubseteq g(a)\\
         & \implies g(f(a)) \sqsubseteq g(g(a)) & \comment{g \text{ е мон.}}\\
         & \implies f(g(a)) \sqsubseteq g(g(a)) & \comment{ f\circ g \sqsubseteq g\circ f}\\
         & \iff P(g(a)).
  \end{align*}
  Тогава по правилото на Скот ще заключим, че $P(\lfp(g))$, откъдето
  \[f(\lfp(g)) \sqsubseteq g(\lfp(g)) = \lfp(g).\]
  Това означава, че $\lfp(g)$ е преднеподвижна точка на $f$, т.е.
  \[\lfp(g) \in \texttt{Pref}(f).\]
  Понеже $\lfp(f)$ е най-малката преднеподвижна точка на $f$,
  то заключаваме, че $\lfp(f) \sqsubseteq \lfp(g)$.
\end{proof}

\begin{problem}
  Нека $h \in \Cont{\A}{\B}$, $f \in \Cont{\A}{\A}$, $g \in \Cont{\B}{\B}$,
  като имаме свойствата:
  \begin{itemize}
  \item 
    $h$ е точна, т.е. $h(\bot^\A) = \bot^\B$;
  \item
    $g\circ h = h \circ f$.
  \end{itemize}
  Докажете, че $\lfp(g) = h(\lfp(f))$.
\end{problem}
\ifhints
\begin{hint}
  \begin{itemize}
  \item 
    Разгледайте непрекъснатото свойство 
    \[P_1(a) \dfff h(a) \sqsubseteq g(h(a)).\]
    Докажете с правилото на Скот, че $P_1(\lfp(f))$.
    Тогава
    \begin{align*}
      h(\lfp(f)) \sqsubseteq g(h(\lfp(f)) & \iff h(f(\lfp(f))) \sqsubseteq g(h(\lfp(f)\\
                                          & \iff h(f(\lfp(f))) \sqsubseteq h(f(\lfp(f)))\\
                                          & \iff g(h(\lfp(f))) \sqsubseteq h(\lfp(f)).
    \end{align*}
    Това означава, че $h(\lfp(f))$ е преднеподвижна точка на $g$, т.е.
    \[h(\lfp(f)) \in \texttt{Pref}(g).\]
    Заключаваме, че $\lfp(g) \sqsubseteq h(\lfp(f))$.
  \item
    Другата посока е по-лесна. Разгледайте непрекъснатото свойство
    \[P_2(a) \dfff h(a) \sqsubseteq \lfp(g).\]
    Докажете, че $P_2(\lfp(f))$.
  \end{itemize}
\end{hint}
\fi

\begin{problem}
  Нека $f,g \in \Cont{\A}{\A}$ като имаме свойствата:
  \begin{itemize}
  \item
    $f(\bot) = g(\bot)$;
  \item
    $f \circ g = g \circ f$.
  \end{itemize}
  Докажете, че $\lfp(f \circ g) = \lfp(g \circ f)$.
\end{problem}
\ifhints
\begin{hint}
  Разгледайте непрекъснатото свойство
  \marginpar{Лесно се вижда, че $P$ е непрекъснато свойство, защото $f$ и $g$ са непрекъснати изображения.}
  \[P(a) \dfff g(f(a)) \sqsubseteq a.\]
  Ясно е, че $P(\bot)$.
  Нека $P(a)$. Ще докажем, че $P(f(g(a))$.
  \begin{align*}
    g(f(a)) \sqsubseteq a & \implies g(f(g(f(a)))) \sqsubseteq g(f(a))\\
    & \implies g(f(f(g(a)))) \sqsubseteq f(g(a))\\
    & \implies P(f(g(a))).
  \end{align*}
  От правилото на Скот заключваме, че $P(\lfp(f\circ g))$.
  Това означава, че 
  \[g(f(\lfp(f\circ g))) \sqsubseteq \lfp(f\circ g),\] т.е.
  $\lfp(f\circ g) \in \texttt{Pref}(g \circ f)$.
  Следователно,
  \[\lfp(g \circ f) \sqsubseteq \lfp(f\circ g).\]

  За другата посока разсъждаваме аналогично.
\end{hint}
\fi


\begin{problem}
  Нека $p \in \Cont{\A}{\Nat_\bot}$ и $h \in \Cont{\A}{\A}$, като $h$ е точна, т.е. $h(\bot) = \bot$.
  Да разгледаме 
  \[\Gamma \in \Cont{\Cont{\A\times\A}{\A}}{\Cont{\A\times\A}{\A}},\]
  където
  \begin{align*}
    \Gamma(f)(x,y) =
    \begin{cases}
      y, & p(x) = 0\\
      h(f(h(x),y)), & p(x) \in \Nat^+\\
      \bot, & p(x) = \bot.
    \end{cases}
  \end{align*}
  Докажете, че ако $f_\Gamma \dff \lfp(\Gamma)$, то
  \[(\forall a,b\in\A)[h(f_\Gamma(a,b)) = f_\Gamma(a,h(b))].\]
\end{problem}
\ifhints
\begin{hint}
  Разгледайте непрекъснатото свойство
  \[P(g) \dfff (\forall a,b\in\A)[h(g(a,b)) = g(a,h(b))].\]
\end{hint}
\fi

\begin{problem}
  Нека $p \in \Cont{\A}{\Nat_\bot}$ и $h \in \Cont{\A}{\A}$, като $p$ е точна, т.е. $p(\bot) = \bot$.
  Да разгледаме 
  \[\Gamma \in \Cont{\Cont{\A}{\A}}{\Cont{\A}{\A}},\] 
  където:
  \begin{align*}
    \Gamma(f)(x) =
    \begin{cases}
      x, & p(x) = 0\\
      f(f(h(x))), & p(x) \in \Nat^+\\
      \bot, & p(x) = \bot.
    \end{cases}
  \end{align*}
  Докажете, че ако $f_\Gamma \dff \lfp(\Gamma)$, то
  \[(\forall a\in\A)[f_\Gamma(f_\Gamma(a)) = f_\Gamma(a)].\]
\end{problem}
\ifhints
\begin{hint}
  Разгледайте непрекъснатото свойство
  \[P(g) \dfff (\forall a \in \A)[f_\Gamma(g(a)) = g(a)].\]
\end{hint}
\fi

\begin{problem}
  Нека $p \in \Cont{\A}{\Nat_\bot}$ и $h,k \in \Cont{\A}{\A}$, като $h$ е точна, т.е. $h(\bot) = \bot$.
  Да разгледаме $\Gamma_{1,2} \in \Cont{\Cont{\A\times\A}{\A}}{\Cont{\A\times\A}{\A}}$, където:
  \begin{align*}
    & \Gamma_1(f)(x,y) =
    \begin{cases}
      y, & p(x) = 0\\
      h(f(k(x),y)), & p(x) \in \Nat^+\\
      \bot, & p(x) = \bot;\\
    \end{cases}\\
   & \Gamma_2(f)(x,y) =
    \begin{cases}
      y, & p(x) = 0\\
      f(k(x),h(y)), & p(x) \in \Nat^+\\
      \bot, & p(x) = \bot;
    \end{cases}
  \end{align*}
  Докажете, че ако $f_1 \dff \lfp(\Gamma_1)$ и $f_2 = \lfp(\Gamma_2)$, то
  $f_1 = f_2$.
\end{problem}
\ifhints
\begin{hint}
  Разгледайте непрекъснатото изображение $\Delta$, където
  \[\Delta(f,g) = \pair{\Gamma_1(f),\Gamma_2(g)}.\]
  Разгледайте свойството:
  \[P(f,g) \dfff f = g\ \&\ (\forall a,b \in \A)[h(f(a,b))) = f(a,h(b))].\]
  Първо трябва да се съобрази, че това свойство е непрекъснато, което не е трудно.
  Ясно е, че $P(\Omega,\Omega)$.
  Докажете, че $P(f,g) \implies P(\Delta(f,g))$.
\end{hint}
\fi

%%% Local Variables:
%%% mode: latex
%%% TeX-master: "../sep-notes"
%%% End:
