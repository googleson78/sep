\section{Алгебрични области на Скот}

\marginpar{\cite{abramsky94}}

\begin{itemize}
\item 
  \index{компактен елемент}
  Нека $\A$ е област на Скот.
  Казваме, че елементът $c$ е {\bf компактен}, ако 
  за всяка верига $\chain{a}{i}$, за която $c \sqsubseteq \bigsqcup_i a_i$,
  съществува индекс $i_0$, за който $c \sqsubseteq a_{i_0}$.
  Компактните елементи на $\A$ ще означаваме с $K(\A)$.
\item
  \index{област на Скот!алгебрична}
  Ще казваме, че областта на Скот $\A$ е {\bf алгебрична}, ако за всеки елемент $a \in \A$,
  съществува верига от {\em компактни} елементи $\chain{c}{i}$ в $\A$, за която $a = \bigsqcup_i c_i$.
\end{itemize}

\begin{example}
  Нека да разгледаме $\A = \pair{A,\sqsubseteq, \bot}$,
  където 
  \[A = \{a_0,a_1,\dots,\} \cup \{a_\omega, b\},\]
  релацията $\sqsubseteq$ е представена на \Fig{noncompact-element}.
  Лесно се съобразява, че $\A$ е област на Скот.
  Всеки от елементите на $A$ е компактен, с изключение на $a_\omega$ и $b$.
  Също така е ясно, че $\A$ {\em не е} алгебрична област на Скот, защото 
  няма верига от крайни елементи, чиято точна горна граница да е елементът $b$.
  \begin{framed}
    \begin{figure}[H]
    \centering
    \begin{tikzpicture}[shorten >=1pt,->]
      \tikzstyle{vertex}=[circle,minimum size=17pt,inner sep=0pt]
      
      \node[vertex] (omega) at (0,5) {$a_\omega$};
      \node[vertex] (2) at (0,3) {$a_3$};
      \node[vertex] (1) at (0,2) {$a_2$};
      \node[vertex] (0) at (0,1) {$a_1$};
      \node[vertex] (bot) at (0,0) {$a_0$};
      
      \node[vertex] (a) at (3,3) {$b$};

      \draw (bot) -- (a);
      \draw (a)   -- (omega);
      \draw (bot) -- (0);
      \draw (0)   -- (1);
      \draw (1)   -- (2);
      \draw[dashed] (2) -- (omega);
    \end{tikzpicture}    
    \caption{Графично представяне на $\sqsubseteq$ върху $\A$}
    \label{fig:noncompact-element}
  \end{figure}
\end{framed}
\end{example}

\begin{example}
  Областта на Скот $\F_n$ е алгебрична.
  Компактните елементи са тези функции, за които $|Dom(f)| < \infty$, т.е.
  крайните функции.   
\end{example}


\begin{framed}
  \begin{thm}
    Нека $\A$ и $\B$ са области на Скот, където $\A$ е {\em алгебрична}.
    Тогава $f \in \Cont{\A}{\B}$ точно тогава, когато за произволен елемент $a \in A$,
    \[f(a) = \bigsqcup\{f(c) \mid c \sqsubseteq a\ \&\ c \in K(\A)\}.\]
  \end{thm}
\end{framed}
\begin{proof}
  \begin{enumerate}[(1)]
  \item
    Нека $f \in \Cont{\A}{\B}$ и да разгледаме произволен елемент $a \in A$.
    Нека $c$ е комапктен елемент, за който $c \sqsubseteq a$.
    \marginpar{Всяко непрекъснато изображение е монотонно}
    Тогава $f(c) \sqsubseteq f(a)$, защото $f$ е монотонно изображение.
    Това означава, че $f(a)$ е горна граница на множеството
    $\{f(c) \mid c \sqsubseteq a\ \&\ c\in K(\A)\}$.
    
    Нека сега $b$ е друга горна граница на $\{f(c) \mid c \sqsubseteq a\ \&\ c\in K(\A)\}$.
    Ще докажем, че $f(a) \sqsubseteq b$.

    Понеже $\A$ е алгебрична област на Скот, то $a = \bigsqcup_i c_i$, за някоя вергига $\chain{c}{i}$ от компактни елементи.
    Знаем, че $f(c_i) \sqsubseteq b$ за всеки компактен елемент $c_i \sqsubseteq a$.
    \marginpar{$\chain{f(c_i)}{i}$ образуват верига и следователно притежава точна горна граница}
    Тогава $\bigsqcup_i f(c_i) \sqsubseteq b$ и следователно
    $f(\bigsqcup_i c_i) \sqsubseteq b$, защото $f$ е непрекъснато изображение.
    Понеже $a = \bigsqcup_i c_i$, то получаваме, че $f(a) \sqsubseteq b$.

    От всичко това следва, че
    \[f(a) = \bigsqcup \{f(c) \mid c \sqsubseteq a\ \&\ c\in K(\A)\}.\]
  \item
    Сега да разгледаме обратната посока, т.е. нека $f$ е изображение, за което
    за произволен елемент $a \in A$ е изпълнено, че
    \[f(a) = \bigsqcup\{f(c) \mid c \sqsubseteq a\ \&\ c \in K(\A)\}.\]

    Нека първо да проверим, че $f$ е монотонно изображение.
    За целта, нека разгледаме $a \sqsubseteq b$.
    Ясно е, че
    \[\{f(c) \mid c \sqsubseteq a\ \&\ c \in K(\A) \}\subseteq \{f(c) \mid c \sqsubseteq b\ \&\ c \in K(\A) \}.\]
    Оттук директно получаваме, че
    \begin{align*}
      f(a) & = \bigsqcup \{f(c) \mid c \sqsubseteq a\ \&\ c \in K(\A) \}\\
           & \sqsubseteq \bigsqcup\{f(c) \mid c \sqsubseteq b\ \&\ c \in K(\A) \}\\
           & = f(b).
    \end{align*}
    Така, щом $f$ е монотонно изображение, то можем да заключим, че
    за произволна верига $(a_i)^\infty_{i=0}$ е изпълнено
    \[\bigsqcup_i f(a_i) \sqsubseteq f(\bigsqcup_i a_i).\]

    За другата посока, да разгледаме произволна верига $(a_i)^\infty_{i=0}$.
    Тогава ако $c$ е компактен елемент и $c \sqsubseteq \bigsqcup_i a_i$,
    то съществува индекс $i_0$, за който $c \sqsubseteq a_{i_0}$.
    Понеже $f$ е монотонно изображение, то
    \[f(c) \sqsubseteq f(a_{i_0}) \sqsubseteq \bigsqcup_i f(a_i).\]
    Това означава, че елементът $\bigsqcup_i f(a_i)$
    е горна граница на множеството
    \[\{f(c) \mid c \sqsubseteq \bigsqcup_i a_i\ \&\ c \in K(\A)\}.\]
    Оттук заключаваме, че
    \[f(\bigsqcup_i a_i) = \bigsqcup\{f(c) \mid c \sqsubseteq \bigsqcup_i a_i\ \&\ c \in K(\A)\} \sqsubseteq \bigsqcup_i f(a_i).\]
  \end{enumerate}
\end{proof}


Използвайки факта, че $\F_n$ е алгебрична област на Скот, то имаме следната полезна харектеризация.
\begin{cor}
  Следните условия са еквивалентни:
  \begin{enumerate}[(1)]
  \item
    $\Gamma \in \Cont{\F_n}{\F_m}$;
  \item
    $\Gamma(f)(\ov{x}) \simeq y \iff (\exists \theta \subseteq f)[\ \theta\text{ е крайна функция}\ \&\ \Gamma(\theta)(\ov{x}) \simeq y\ ]$.
  \end{enumerate}
\end{cor}

% \begin{prop}
%   \label{pr:compact-monotone1}
%   Нека $\A$ и $\B$ са области на Скот и $f:\A \to \B$ е {\bf компактно изображение}.
%   Тогава $f$ е монотонно върху крайните елементи, т.е. за всеки два крайни елемента $c_1$ и $c_2$,
%   \[c_1 \sqsubseteq c_2 \implies f(c_1) \sqsubseteq f(c_2).\]
% \end{prop}
% \begin{proof}
%   Нека фиксираме крайни елементи $c_1 \sqsubseteq c_2$.
%   Имаме, че $c_1 \sqcup c_2 = c_2$, т.е.
%   $c_1 \sqsubseteq c_2 = c_2 = c_2 = \cdots$ образуват верига,
%   чиято точна горна граница е $c_2$.
%   Понеже $f$ е компактен,
%   \[f(c_1 \sqcup c_2) = f(c_1) \sqcup f(c_2)\]
%   Обединявайки всичко, получаваме:
%   \begin{align*}
%     f(c_1) & \sqsubseteq f(c_1) \sqcup f(c_2)\\
%     & = f(c_1 \sqcup c_2) & \comment{f \text{ е компактен}}\\
%     & = f(c_2).
%   \end{align*}
% \end{proof}

% \begin{prop}
%   \label{pr:compact-monotone2}
%   Нека $\A$ и $\B$ са области на Скот, като $\A$ е алгебрична.
%   Тогава
%   \[\Compact{\A}{\B} \subseteq \Mon{\A}{\B}.\]
% \end{prop}
% \begin{hint}
%   Нека $f \in \Compact{\A}{\B}$ и $a_1 \sqsubseteq a_2$ в $\A$.
%   Ще докажем, че $f(a_1) \sqsubseteq f(a_2)$.
%   Понеже $\A$ е алгебрична област на Скот, 
%   съществуват вериги от крайни елементи $\chain{c}{i}$ и $\chain{d}{i}$,
%   за които $\bigsqcup_k c_k = a_1$ и $\bigsqcup_i d_i = a_2$.
%   Понеже за всяко $k$, $c_k \sqsubseteq a_1 \sqsubseteq a_2 = \bigsqcup_i d_i$,
%   то съществува индекс $i_k$, $c_k \sqsubseteq d_{i_k}$.
%   Тогава от \Prop{compact-monotone1} следва, че $f(c_k) \sqsubseteq f(d_{i_k})$.
%   Оттук следва, че 
%   \begin{equation}
%     \label{eq:mon1}
%     \bigsqcup_k f(c_k) \sqsubseteq \bigsqcup_k f(d_{i_k}) \sqsubseteq \bigsqcup_i f(d_i).
%   \end{equation}
%   Вече сме готови да завършим доказателството:
%   \begin{align*}
%     f(a_1) & = f(\bigsqcup_k c_k) & \comment{a_1 = \bigsqcup_k c_k}\\
%     & = \bigsqcup_k f(c_k) & \comment{f \text{ е компактно}}\\
%     & \sqsubseteq \bigsqcup_i f(d_{i}) & \comment{\text{от Формула (\ref{eq:mon1})}}\\
%     & = f(\bigsqcup_i d_{i}) & \comment{f \text{ е компактно}}\\
%     & = f(a_2) & \comment a_2 = \bigsqcup_i d_i.
%   \end{align*}
% \end{hint}

% \begin{framed}
%   \begin{thm}
%     Нека $\A$ и $\B$ са области на Скот, като $\A$ е {\em алгебрична}.
%     Тогава 
%     \[\Cont{\A}{\B} = \Compact{\A}{\B}.\]
%   \end{thm}  
% \end{framed}
% \begin{hint}
%   $(\Rightarrow)$ Нека $f$ да бъде непрекъснато изображение и да разгледаме верига от крайни елементи $\chain{c}{i}$.
%   Тогава директно от дефиницията на непрекъснато изображение имаме, че
%   \[f(\bigsqcup_ic_i) = \bigsqcup_i f(c_i).\]
%   Следователно, $f$ е компактно изображение.

%   $(\Leftarrow)$ Нека $f$ да бъде компактно изображение и да разгледаме една верига $\chain{a}{i}$ в $\A$.
%   Трябва да докажем, че 
%   \[f(\bigsqcup_i a_i) = \bigsqcup_i f(a_i).\]
%   Понеже от \Prop{compact-monotone2} знаем, че всяко компактно изображение е монотонно, то 
%   прилагаме \Prop{monotone-chain} и получаваме, че
%   \[\bigsqcup_i f(a_i) \sqsubseteq f(\bigsqcup_i a_i).\]
%   Понеже $\A$ е алгебрична област на Скот, то съществува верига от крайни елементи $\chain{c}{k}$,
%   за която $\bigsqcup_k c_k = \bigsqcup_i a_i$.  
%   Понеже $c_k \sqsubseteq \bigsqcup_i a_i$ и $c_k$ е краен, то 
%   съществува индекс $i_k$, за който $c_k \sqsubseteq a_{i_k}$, откъдето $f(c_k) \sqsubseteq f(a_{i_k})$.
  
%   Така получаваме, че 
%   \begin{align*}
%     f(\bigsqcup_i a_i) & = f(\bigsqcup_k c_k) &  \comment{\text{от избора на }\chain{c}{k}}\\
%                        & = \bigsqcup_k f(c_k) & \comment{f \text{ е компактно}}\\
%                        & \sqsubseteq \bigsqcup_k f(a_{i_k}) & \comment{f\text{ е монотонно и } c_k \sqsubseteq a_{i_k}}\\
%                        & \sqsubseteq \bigsqcup_i f(a_i) & \comment{(a_{i_k})^{\infty}_{k=0}\text{ е подредица на }\chain{a}{i}}.
%   \end{align*}
% \end{hint}

% \begin{prop}
%   \label{pr:compact-characterisation}
%   \marginpar{Естествено, можем да говорим и за $f \in \Cont{\A}{\B}$}
%   Нека $\A$ и $\B$ са алгебрични области на Скот.
%   Тогава $f \in \Compact{\A}{\B}$ точно тогава, когато:
%   \begin{enumerate}[(1)]
%   \item 
%     $f \in \Mon{\A}{\B}$;
%   \item
%     за произволен краен елемент $d \sqsubseteq f(a)$
%     съществува краен елемент $c \sqsubseteq a$, за който $d \sqsubseteq f(c)$.
%   \end{enumerate}
% \end{prop}
% \begin{proof}
%   Нека $f$ е компактно изображение. От \Prop{compact-monotone2} имаме, че $f$ е монотонно.
%   За (2), да фиксираме верига от крайни елементи $\chain{c}{i}$, за която $\bigsqcup_i c_i = a$.
%   Тогава от компактността на $f$, $f(\bigsqcup_i c_i) = \bigsqcup_i f(c_i)$.
%   Нека $d$ е краен елемент, за който $d \sqsubseteq f(a)$.
%   Това означава, че съществува индекс $i_0$, за който $d \sqsubseteq f(c_{i_0})$.

%   Нека сега са изпълнени свойства (1) и (2). Ще докажем, че $f$ е компактно изображение.
%   От монотонността на $f$, веднага получаваме, че $\bigsqcup_if(c_i) \sqsubseteq f(\bigsqcup_ic_i)$.
%   За другата посока ще използваме (2).
%   Ще докажем, че $f(\bigsqcup_i c_i) \sqsubseteq \bigsqcup_i f(c_i)$.
%   Нека $a = \bigsqcup_i c_i$ и  $b = f(a)$.
%   От алгебричността на $\B$, съществува верига от крайни елементи $\chain{d}{i}$,
%   за която $\bigsqcup_k d_k = b$. 
%   Тогава от (2) имаме, че същестува краен елемент $c \sqsubseteq a$, за който $d_k \sqsubseteq f(c)$.
%   Понеже $\A$ е алгебрична, същестува индекс $i_k$, за който $c \sqsubseteq c_{i_k}$.
%   От монотонността на $f$, $d_k \sqsubseteq f(c) \sqsubseteq f(c_{i_k})$.
%   Така получаваме, че 
%   \begin{align*}
%     f(\bigsqcup_i c_i) & = f(a) & \comment{a = \bigsqcup_i c_i}\\
%                        & = b\\
%                        & = \bigsqcup_k d_k & \comment{\text{алгебричност на }\B}\\
%                        & \sqsubseteq \bigsqcup_k f(c_{i_k}) & \comment{\text{от }(2)}\\
%                        & \sqsubseteq \bigsqcup_i f(c_i) & \comment{(c_{i_k})^\infty_{k=0}\text{ е подверига на }\chain{c}{i}}.
%   \end{align*}
% \end{proof}



% \begin{cor}
%   \label{pr:operator-compact}
%   \marginpar{Това следствие ще е важно за задачите, които се решават на упражнения}
%   $\Gamma \in \Compact{\F_n}{\F_m}$  точно тогава, когато:
%   \begin{enumerate}[(1)]
%   \item 
%     $\Gamma \in \Mon{\F_n}{\F_m}$, и
%   \item
%     ако $\Gamma(f)(\bar{x}) \simeq y$, то съществува крайна подфункция $\theta \subseteq f$,
%     за която $\Gamma(\theta)(\bar{x}) \simeq y$.
%   \end{enumerate}
% \end{cor}
% \begin{proof}
%   Нека $\Gamma$ е компактен оператор.
%   От (1) на \Prop{compact-characterisation} имаме, че $\Gamma$ е монотонен.
%   Нека $\Gamma(f)(\bar{x}) \simeq y$. Тогава да изберем крайния елемент $\hat\theta \subseteq \Gamma(f)$, 
%   където \[\Graph{\hat\theta} = \{\pair{\bar{x},y}\}.\]
%   Тогава от (2) на \Prop{compact-characterisation} имаме краен елемент $\theta \subseteq f$, за който $\hat\theta \subseteq \Gamma(\theta)$.
%   Това означава, че $\Gamma(\theta)(\bar{x}) \simeq y$.  

%   За обратната посока, единствено трябва да проверим, че е изпълнено (2) на \Prop{compact-characterisation}.
%   Нека $\hat\theta \subseteq \Gamma(f)$, където
%   \[\Graph{\hat\theta} = \{\pair{\bar{x}_i,y_i} \mid i = 1,\dots,k\}.\]
%   Тогава за всяко $i=1,\dots,k$ $\Gamma(f)(\bar{x}_i)\simeq y_i$ и от (2) имаме, че съществува 
%   крайна $\theta_i \subseteq f$, за която $\Gamma(\theta_i)(\bar{x}_i) \simeq y_i$.
%   Накрая, нека $\theta = \bigcup_i \theta_i$.
% \end{proof}

% \begin{cor}
%   \marginpar{Това е дефиницията на компактен оператор дадена в \cite{ditchev-soskov}. Ползата от нашите по-абстрактни разглеждания е, че знем какво представляват компактните оператори и върху по-сложни от $\F_n$ структури.}
%   $\Gamma \in \Compact{\F_n}{\F_m}$  точно тогава, когато
%   \[\Gamma(f)(\bar{x}) \simeq y \iff (\exists \text{ крайна }\theta \subseteq f)[\Gamma(\theta)(\bar{x}) \simeq y].\]
% \end{cor}

% \index{изображение!графика}
% Нека $\A$ и $\B$ са {\em алгебрични} области на Скот и $f \in \Mapping{\A}{\B}$.
% Тогава {\bf графиката} на $f$ е множеството
% \[\Graph{f} \dff \{\pair{c,d} \in K(\A)\times K(\B) \mid  d \sqsubseteq f(c)\}.\]

% \begin{framed}
%   \begin{thm}
%     Нека $\A$ и $\B$ са алгебрични области на Скот.
%     Тогава следните твърдения са еквивалентни:
%     \begin{enumerate}[(1)]
%     \item 
%       $f \in \Cont{\A}{\B}$;
%     \item
%       за всеки елемент $a \in \A$,
%       \[f(a) = \bigsqcup\{d \mid \pair{c,d} \in \Graph{f}\ \&\ c \sqsubseteq a\}.\]    
%     \end{enumerate}
%   \end{thm}
% \end{framed}
% \begin{proof}
%   \marginpar{Тази теорема е важна, когато разглеждаме $\A$ и $\B$ да бъдат областите от лениви списъци или други по-сложни структури}
%   За посоката $(1) \to (2)$, нека за произволно $a \in \A$,
%   \[R_a \dff \{d \mid \pair{c,d} \in \Graph{f}\ \&\ c \sqsubseteq a\}.\]
%   Ще докажем, че $f(a) = \bigsqcup R_a$.

%   Първо, нека $d \in R_a$, т.е. съществува $c \sqsubseteq a$, такова че $d \sqsubseteq f(c)$.
%   Тогава от монотонността на $f$, $d \sqsubseteq f(c) \sqsubseteq f(a)$.
%   Получаваме, че $\bigsqcup R_a \sqsubseteq f(a)$.

%   Второ, от алгебричността на $\B$,
%   $f(a) = \bigsqcup_i d_i$.
%   Понеже $d_i \sqsubseteq f(a)$, от \Prop{compact-characterisation} следва, че
%   съществува $c_i \sqsubseteq a$, за който $d_i \sqsubseteq f(c_i)$.
%   Това означава, че $d_i \in R_a$. 
%   Оттук, $f(a) = \bigsqcup d_i \sqsubseteq \bigsqcup R_a$.

%   Сега ще докажем и посоката $(2) \to (1)$. Ще докажем, че
%   \[\bigsqcup_i f(c_i) = f(\bigsqcup_i c_i).\]
%   Първо, лесно се съобразява, че $f$ е монотонно изображение.
%   Тогава от \Prop{monotone-chain} следва, че
%   \[\bigsqcup_i f(c_i) \sqsubseteq f(\bigsqcup_i c_i).\]
%   Сега нека да обърнем внимание, че според (2), $\bigsqcup R_a$ съществува за всеки елемент $a\in \A$.
%   Освен това, лесно се съобразява, че ако $a_0 \sqsubseteq a_1$, то $R_{a_0} \subseteq R_{a_1}$ и 
%   оттук $\bigsqcup R_{a_0} \sqsubseteq \bigsqcup R_{a_1}$.
  
%   Нека сега да разгледаме $a = \bigsqcup_i c_i$.
%   Знаем, че ако $c \sqsubseteq a$, то съществува индекс $i_0$, за който $c \sqsubseteq c_{i_0}$.
%   От това следва, че множеството $R_a \subseteq \bigcup_i R_{c_i}$.
%   Тогава, ако $d \in R_a$, то $d \in R_{c_i}$, за някое $i$,
%   откъдето следва, че $d \sqsubseteq \bigsqcup R_{c_i}$.
%   Това означава, че
%   \begin{equation}
%     \label{eq:2}
%     \bigsqcup R_a \sqsubseteq \bigsqcup_i (\bigsqcup R_{c_i}).
%   \end{equation}
%   Заключаваме, че 
%   \begin{align*}
%     f(\bigsqcup_i c_i) & = f(a) & \comment{a = \bigsqcup_i c_i}\\
%                        & = \bigsqcup R_a & \comment{\text{ от }(2)}\\
%                        & \sqsubseteq \bigsqcup_i (\bigsqcup R_{c_i}) & \comment{\text{ от }(\ref{eq:2})}\\
%                        & = \bigsqcup_i f(c_i) & \comment{f(c_i) = \bigsqcup R_{c_i} \text{ от }(2)}.
%   \end{align*}
% \end{proof}



%%% Local Variables:
%%% mode: latex
%%% TeX-master: "../sep-notes"
%%% End:
