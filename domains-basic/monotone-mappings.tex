\subsection{Монотонни изображения}

\index{изображение!монотонно}
Да разгледаме областите на Скот $\A_1 = (A_1, \sqsubseteq_1, \bot_1)$ и $\A_2 =(A_2, \sqsubseteq_2, \bot_2)$.
Едно изображение $f:\A_1\to \A_2$ се нарича {\bf монотонно}, ако
\[(\forall a,a'\in\A_1)[a \sqsubseteq_1 a' \implies f(a) \sqsubseteq_2 f(a')].\]
Да въведем означението
\[\Mon{\A_1}{\A_2} \dff \{f: \A_1 \to \A_2 \mid f\text{ е монотонно изобр.}\}.\]


\Stefan{Да се обясни каква е интерпретацията на монотонните изображения в нашия частен случай и защо на практика това е най-естествения клас от изображения, с който ще работим}

\begin{framed}
  \begin{thm}
    \label{th:monotone-is-domain}
    $(\Mon{\A_1}{\A_2},\ \sqsubseteq, \bm{\bot})$ е област на Скот.
  \end{thm}  
\end{framed}
\begin{hint}
  Да фиксираме една верига $\chain{f}{i}$ в $\Mon{\A_1}{\A_2}$. Трябва да докажем, че тази верига притежава точна горна граница,
  която е монотонно изображение.
  Да разгледаме същото изображение $h:\A_1 \to \A_2$ както в доказателството на \Th{all-mappings-is-domain}, като
  \[h(a) \dff \bigsqcup \{f_i(a) \mid i \in \Nat\}.\]
  Оттам знаем, че $h$ е точна горна граница на веригата. 
  Остава да докажем, че $h \in \Mon{\A_1}{\A_2}$.
  Нека $a \sqsubseteq b$. Тогава, за всеки индекс $k$, понеже $f_k$ са монотонни изображения, получаваме следното:
  \begin{align*}
    f_k(a) & \sqsubseteq f_k(b)\\
           & \sqsubseteq \bigsqcup \{f_i(b) \mid i \in \Nat\} \dff h(b).
  \end{align*}
  Това означава, че $h(b)$ е горна граница за веригата $(f_i(a))^{\infty}_{i=0}$.
  Заключаваме, че 
  \begin{align*}
    h(a) & \dff \bigsqcup \{f_i(a) \mid i \in \Nat\}\\
         & \sqsubseteq \bigsqcup \{f_i(b) \mid i \in \Nat\} \dff h(b).    
  \end{align*}
\end{hint}

\begin{cor}
  \label{cr:flat-monotone-is-domain}
  $(\Mon{\Nat^n_\bot}{\Nat_\bot},\ \sqsubseteq, \bm{\bot}^{(n)})$ е област на Скот.
\end{cor}


%%% Local Variables:
%%% mode: latex
%%% TeX-master: "../sep"
%%% End:
