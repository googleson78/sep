\section{Частични наредби}
\index{частична наредба}

Бинарната релация $\sqsubseteq$ върху множеството $A$ се нарича {\bf частична наредба}, ако тя е:
\marginpar{На англ. \emph{partial order}}
\begin{itemize}
\item 
  рефлексивна, т.е. $(\forall a \in A)[a \sqsubseteq a]$;
\item
  транзитивна, т.е. $(\forall a,b,c \in A)[a \sqsubseteq b\ \&\ b \sqsubseteq c \implies a \sqsubseteq c]$;
\item
  антисиметрична, т.е. $(\forall a,b \in A)[a \sqsubseteq b\ \&\ b \sqsubseteq a  \implies a = b]$.
\end{itemize}

Една такава двойка $(A, \sqsubseteq)$ се нарича частично наредено множество.

\begin{example}
  Да означим 
  \[\Partial{\Nat}{\Nat} \dff \{f:\Nat\to\Nat \mid f\text{ е частична функция}\}.\]
  Дефинираме и релацията {\bf включване } между две частични функции по следния начин:
  \begin{align*}
    f \subseteq g \dfff (\forall x\in \Nat)[& f(x)\text{ не е деф.}\ \vee\\
                                            & (f(x)\text{ е деф.}\ \&\ g(x)\text{ е деф.}\ \&\ f(x) = g(x))].
  \end{align*}
  Да дефинираме също {\bf графиката} на частичната функция $f$ като
  \[\Graph{f} \dff \{\pair{x,y} \mid f(x) = y\}.\]
  Тогава лесно се съобразява, че 
  \[f \subseteq g \iff \Graph{f} \subseteq \Graph{g}.\]
  Съобразете, че двойката $(\Partial{\Nat}{\Nat}, \subseteq)$ е частично наредено множество.
\end{example}

Казваме, че $a_0$ е {\bf най-малък елемент} на частично нареденото множество $(A, \sqsubseteq)$,
ако $(\forall a \in A)[a_0 \sqsubseteq a]$. Ако такъв елемент съществува, то той е единствен,
защото релацията $\sqsubseteq$ е антисиметрична.

{\bf Неподвижна точка} на $f:A \to A$ е елемент $a \in A$, такъв че $f(a) = a$.

За по-кратко, монотонно-растящите редици от елементи на $A$,
\[a_0 \sqsubseteq a_1 \sqsubseteq \cdots \sqsubseteq a_n \sqsubseteq \cdots,\]
ще наричаме (растящи) {\bf вериги}. 

Един елемент $b$ е {\bf горна граница} на веригата $\chain{a}{n}$, ако 
$(\forall n)[a_n \sqsubseteq b]$.
Един елемент $b$ е {\bf точна горна граница} на веригата $\chain{a}{n}$, ако са изпълнени свойствата:
\begin{itemize}
\item 
  $(\forall n)[a_n \sqsubseteq b]$, т.е. $b$ е горна граница;
\item
  за всяка друга горна граница $c$ е изпълнено, че $b \sqsubseteq c$, т.е.
  $b$ е най-малкият елемент измежду всички горни граници на веригата $\chain{a}{n}$.
\end{itemize}
Не всяка верига притежава точна горна граница.
Обикновено точната горна граница на вергата $\chain{a}{n}$ ще бележим като $\bigsqcup_n a_n$.

\Stefan{Още тук да се даде пример за верига от частични функции и две функции - една, която е точна горна граница на веригата и една, която е просто горна граница.}

Наредена тройка от вида $\A = (A, \sqsubseteq, \bot)$ се нарича {\bf област на Скот}, ако:
\index{област на Скот}
\marginpar{На англ. {\em Scott domain}}
\begin{itemize}
\item
  $\sqsubseteq$ е бинарна релация върху $A$, която задава частична наредба.
\item
  Всяка растяща верига $\chain{a}{n}$ в $A$ притежава точна горна граница $\bigsqcup_n a_n$.
\item
  $\bot \in A$ е най-малкият елемент на $A$;
\end{itemize}

\marginpar{В Хаскел $\bot$ се означава като \vv{undefined}. Повече за денотационна семантика в Хаскел може да прочетете \href{https://en.wikibooks.org/wiki/Haskell/Denotational_semantics}{тук}}

\begin{example}
  \Stefan{Друго означение вместо $\F_n$ ?}
  Тройката
  \[\F_n \dff (\ \Partial{\Nat^n}{\Nat},\subseteq, \emptyset^{(n)}\ )\] е област на Скот, където:
  \begin{itemize}
  \item
    С $\Partial{\Nat^n}{\Nat}$ означаваме всички частични функции от $\Nat^n$ в $\Nat$.
  \item
     релацията ,,включване'' между функции е дефинирана по следния начин:
     \[f\subseteq g\ \dffff\ (\forall \bar{x})(\forall y)[ f(\bar{x}) \simeq y \implies  g(\bar{x}) \simeq y].\]
   \item
     $\emptyset^{(n)}$ е функцията с празна дефиниционна област, т.е.
     \[(\forall \bar{x} \in \Nat^n)[\neg !\emptyset^{(n)}(\bar{x})].\]
  \end{itemize}
\end{example}

\begin{example}
  Да разгледаме няколко примера, които вече сме срещали.
  \begin{itemize}
  \item
    $(\Ps(\Nat),\subseteq,\emptyset)$ е област на Скот.
  \item
    $(\Nat, \leq, 0)$ не е е област на Скот.
  \item
    $(\Nat\cup\{\infty\}, \leq, 0)$ е също област на Скот, където $0 \leq 1 \leq \cdots \leq \infty$.
  \item
    $(\{0,1\}^\star, \preceq, \varepsilon)$ не е област на Скот, където $\preceq$ е релацията префикс на две думи.
    % Може да образувате безкрайна растяща верига от думи, но тяхната точна горна граница ще бъде безкрайна дума.
  \end{itemize}
\end{example}

\begin{example}
  Да разгледаме множеството 
  \[Bin^\infty = \{\sigma \mid \sigma :\{0,1,2,\dots,n-1\} \to \{0,1\}\ \&\ n \in \Nat\} \cup 
  \{f \mid f:\Nat \to \{0,1\}\}\]
  съставено от всички крайни и безкрайни двоични низове.
  \begin{itemize}
  \item
    Да разгледаме релацията
    \[\sigma \preceq \tau \iff \abs{\sigma} \leq \abs{\tau}\ \&\ (\forall i < |\sigma|)[\sigma(i) = \tau(i)],\]
    $\sigma$ е префикс на $\tau$.    
  \item
    Да означим с $\varepsilon$ единствения двоичен низ с дължина $0$.
  \end{itemize}
  Тогава $Bin^\infty = (Bin^\infty,\preceq,\varepsilon)$ е област на Скот.
\end{example}

Още тук да се дефинира $\Mapping{\Nat_\bot}{\Nat_\bot}$ и да се види, че образува област на Скот.

%%% Local Variables:
%%% mode: latex
%%% TeX-master: "../sep"
%%% End:
