\section{Задачи}

\begin{problem}
  \label{prob:sup-f}
  \marginpar{Достатъчно е всяка верига в $\B$ да се стабилизира}
  Нека $(f_i)^\infty_{i=0}$ е верига от елементи на $\Mapping{\A}{\B}$,
  където $\B$ е {\em плоска} област на Скот. Тогава:
  \begin{itemize}
  \item 
    $(\bigsqcup_i f_i)(a) = \bot \implies (\forall i)[\ f_{i}(a) = b\ ]$;
  \item 
    $(\bigsqcup_i f_i)(a) = b \neq \bot \implies (\exists i_0)(\forall i\geq i_0)[\ f_{i}(a) = b\ ]$.
  \end{itemize}
\end{problem}
% \begin{hint}
%   Първо да отбележем, че доказателството на \Th{all-mappings-is-domain} знаем, че
%   \[(\bigsqcup_i f_i)(\bar{a}) = \bigsqcup_i \{f_i(\bar{a})\}.\]
%   За посоката $(\Leftarrow)$, нека $f_{i_0}(\bar{a}) = b$, за някой индекс $i_0$.
%   Имаме, че
%   \begin{align*}
%     b = f_{i_0}(\bar{a}) & \sqsubseteq \bigsqcup_i \{f_i(\bar{a})\} \\
%                          & = (\bigsqcup_i f_i)(\bar{a}) & \comment{\text{от \Th{all-mappings-is-domain}}}.
%   \end{align*}
  
%   Получихме, че $b \sqsubseteq (\bigsqcup_i f_i)(\bar{a})$,
%   но понеже $\bot \neq b$, то $b = (\bigsqcup_i f_i)(\bar{a})$.

%   За посоката $(\Rightarrow)$, нека $(\bigsqcup_i f_i)(\bar{a}) = b \neq \bot$.
%   Отново използваме \Th{all-mappings-is-domain} и получаваме, че
%   $\bigsqcup_i\{f_i(\bar{a})\} = b$.
%   Очевидно е, че не е възможно $f_i(\bar{a}) = \bot$ за всяко $i$, защото тогава $\bigsqcup_i\{f_i(\bar{a})\} = \bot$.
%   Нека $i_0$ е първия индекс, за който $f_{i_0}(\bar{a}) = c\neq \bot$.
%   Понеже разглеждаме плоската наредба, ясно е, че $(\forall i \geq i_0)[f_i(\bar{a}) = c]$.
%   Тогава $\bigsqcup_i\{f_i(\bar{a})\} = c$, откъдето получаваме, че $c = b$.
% \end{hint}

\begin{problem}
  \label{prob:stab-continuous-finite}
  % Нека $\A_1$ е област на Скот, в която всяка верига се {\em стабилизира},
  % а $\A_2$ е {\em плоска} област на Скот.
  \marginpar{Достатъчно е всяка верига в $\B$ да се стабилизира}
  Нека $f \in \Mon{\A}{\B}$, където $\B$ е {\em плоска} област на Скот.
  Тогава за всяка верига $\chain{a}{i}$ в $\A$ е изпълнено, че:
  \begin{itemize}
  \item 
    $f(\bigsqcup_i a_i) = \bot \implies (\forall i)[\ f(a_i) = \bot\ ]$;
  \item
    $f(\bigsqcup_i a_i) = b \neq \bot \implies (\exists i_0)(\forall i\geq i_0)[\ f(a_i) = b\ ]$.
  \end{itemize}
\end{problem}
% \begin{hint}
%   За посоката $(\Leftarrow)$, от \Prop{continuous-is-monotone} знаем, че $f$ е монотонно изображение.
%   Нека $f(a_{i_0}) = b \neq \bot$.
%   Тогава, понеже $f$ е монотонна и $a_{i_0} \sqsubseteq_1 \bigsqcup_i a_i$,
%   имаме, че $f(a_{i_0}) = b \sqsubseteq_2 f(\bigsqcup_i a_i)$.
%   Щом $\sqsubseteq_2$ е плоската наредба и $b \neq \bot$, то получаваме, че $b = f(\bigsqcup_i a_i)$.

%   За посоката $(\Rightarrow)$, нека $f(\bigsqcup_i a_i) = b$.
%   Ако веригата $\chain{a}{i}$ се стабилизира от $a_{i_0}$ нататък, то $\bigsqcup_i a_i = a_{i_0}$.
%   Тогава $f(\bigsqcup_i a_i) = f(a_{i_0}) = b$.
% \end{hint}

% \begin{cor}
%   Нека $f \in \Cont{\Nat^n_\bot}{\Nat_\bot}$.
%   Тогава за всяка верига $\chain{\bar{a}}{i}$ в $\Nat^n_\bot$, и всяко $b \neq \bot$,
%   \[f(\bigsqcup_i \bar{a}_i) = b \iff (\exists i_0)[f(\bar{a}_{i_0}) = b].\]
% \end{cor}


\begin{problem}
  \label{pr:composition}
  \index{изображения!композиция}
  Ако $f \in \Cont{\A}{B}$ и $g \in \Cont{\B}{\C}$, то $g \circ f \in \Cont{\A}{\C}$,
  където \[(g\circ f)(a) \dff g(f(a)).\]
\end{problem}

\begin{problem}
  Докажете, че ако $\A$ и $\B$ са алгебрични области на Скот, то
  $\A \times \B$ също е алгебрична област на Скот.
\end{problem}

\begin{problem}
  Нека е даден следния оператор $\Gamma:\F^\bot_1\to\F^\bot_1$:
  \begin{align*}
    \Gamma(f)(x) =
    \begin{cases}
      \bot, & |Dom(f)| < \infty\\
      1, & |Dom(f)| = \infty.
    \end{cases}
  \end{align*}
  Проверете дали:
  \begin{enumerate}[a)]
  \item 
    $\Gamma$ е монотонен оператор;
  \item
    $\Gamma$ е компактен оператор.
  \end{enumerate}
\end{problem}
\begin{solution}
  \begin{enumerate}[a)]
  \item 
    Трябва да проверим дали:
    \[(\forall f,g\in\F^\bot_1)[f \sqsubseteq g \implies \Gamma(f) \sqsubseteq \Gamma(g)].\]
    Нека $f \sqsubseteq g$ са произволни елементи от $\F^\bot_1$.
    Ще разгледаме два случая.
    \begin{itemize}
    \item 
      $f$ е крайна функция. Тогава $\Gamma(f) = \lambda x.\bot$ и е очевидно, че 
      \[\Gamma(f) \sqsubseteq \Gamma(g).\]
    \item
      $f$ не е крайна функция. Щом $f \sqsubseteq g$, то $g$ също не е крайна функция.
      Тогава 
      \[(\forall x \in \Nat_\bot)[\Gamma(f)(x) = 1 = \Gamma(g)(x)],\]
      от което следва, че 
      \[\Gamma(f) \sqsubseteq \Gamma(g).\]
    \end{itemize}
    Разгледахме всички възможни случаи за $f$ и във всеки от тях получихме, че $\Gamma(f) \sqsubseteq \Gamma(g)$.
    Следователно, $\Gamma$ е монотонен оператор.
  \item
    Според \Prop{operator-compact}, достатъчно е да докажем, че за произволни елементи $f \in \F^\bot_1$, $x, y \in \Nat_\bot$, 
    \begin{equation}
      \label{eq:compact}
      \Gamma(f)(x) = y\ \implies\ (\exists \theta \sqsubseteq f)[\theta\text{ е крайна }\&\ \Gamma(\theta)(x) = y]].
    \end{equation}
    Нека $f$ е не е крайна функция.
    Тогава е ясно, че за всяко $x \in \Nat_\bot$, $\Gamma(f)(x) = 1$.
    От друга страна, понеже $\theta$ е крайна, $\Gamma(\theta)(x) = \bot$ за всяко $x \in \Nat_\bot$.
    Така видяхме, че ако $f$ не е крайна, то за произволна $\theta \sqsubseteq f$ и произволно $x \in \Nat_\bot$,
    $\Gamma(\theta)(x) \neq 1$.
    От това следва, че Формула (\ref{eq:compact}) не е изпълнена и тогава $\Gamma$ не е компактен оператор.  
  \end{enumerate}
\end{solution}


\begin{problem}
  Нека е даден следния оператор $\Gamma:\F^\bot_2\to\F^\bot_2$:
  \begin{align*}
    \Gamma(f)(x,y) = &
    \begin{cases}
      y, & x = 0\\
      f(x, f(x-1,y)), & x > 0\\
      \bot, & x = \bot.
    \end{cases}
  \end{align*}
  \begin{enumerate}[a)]
  \item 
    Докажете, че $\Gamma$ е компактен оператор.
  \item
    Намерете $\lfp(\Gamma)$.
  \item
    Има ли $\Gamma$ други неподвижни точки ?
  \end{enumerate}
\end{problem}

\begin{problem}
  Монотонен ли е операторът $\Gamma:\Strict{\Nat_\bot}{\Nat_\bot} \to \Nat_\bot$, където:
  \begin{align*}
    \Gamma(f) =
    \begin{cases}
      n, & |Dom(f)| = n\\
      \bot, & |Dom(f)| = \infty\\
    \end{cases}
  \end{align*}
\end{problem}

\begin{problem}
  Какви свойства има оператора $\Gamma:\F^\bot_1\times\F^\bot_1 \to \F^\bot_1$, където:
  \begin{align*}
    \Gamma(f,g)(x) =
    \begin{cases}
      g(x), & f \sqsubseteq g\\
      f(x), & g \sqsubseteq f\ \&\ f \not\sqsubseteq g\\
      \bot, & \text{иначе}.
    \end{cases}
  \end{align*}
\end{problem}

\begin{problem}
  \marginpar{\cite[стр. 122]{nikolova-soskova}}
  Нека разгледаме $f \in \Mon{\Nat_\bot}{\Nat_\bot}$.
  Съобразете, че $\lfp(f) = f(\bot)$.
\end{problem}


\begin{problem}
  Знаем, че всяко изображение $f \in \Cont{\A}{\A}$ притежава най-малка неподвижна точка.
  \begin{itemize}
  \item 
    Вярно ли е, че съществуват изображения от вида $f:\A\to\A$, които са монотонни, {\em не} са непрекъснати, но въпреки това притежават 
    най-малка неподвижна точка?
  \item
    \marginpar{\cite[стр. 131]{nikolova-soskova}}
    Вярно ли е, че съществуват изображения от вида $f:\A\to\A$, които са не са монотонни, но въпреки това притежават най-малка неподвижна точка?
  \end{itemize}
  Дайте примери за такава области на Скот $\A$ и изображения $f:\A \to \A$.
  Обосновете заключенията си.
\end{problem}

%%% Local Variables:
%%% mode: latex
%%% TeX-master: "../sep"
%%% End:
