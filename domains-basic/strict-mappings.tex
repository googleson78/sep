\subsection{Точни изображения}

\marginpar{На англ. се нарича {\em strict}. ,,Стандартната'' семантика на хаскел е {\em non-strict}}
\index{изображение!точно}
Едно изображение $f:\A^n_1 \to \A_2$ се нарича {\bf точно}, ако 
\[(\forall \bar{a}\in\A^n_1)[\bot_1\in\{a_1,\dots,a_n\} \implies f(a_1,\dots,a_n) = \bot_2].\]
% Тук ще разглеждаме съвкупността от точни изображения:
% \[S_n \dff \{f: \Nat^n_\bot \to \Nat_\bot \mid f\text{ е точна}\}.\]
За произволни области на Скот $\A_1$ и $\A_2$, ще означаваме съвкупността от
точните изображения като $\Strict{\A_1}{\A_2}$.

\begin{example}
  Да разгледаме свойствата на няколко прости изображения.
  \begin{itemize}
  \item 
    Изображението $f:\Nat_\bot \to \Nat_\bot$, дефинирано като 
    \[f(x) = 42,\]
    за всяко $x \in \Nat_\bot$,
    {\bf не е точно}, защото $f(\bot) = 42$. Лесно се вижда, че $f$ е монотонно и непрекъснато изображение.
  \item
    От друга страна, изображението $g:\Nat_\bot \to \Nat_\bot$, дефинирано като 
    \[g(x) = \bot,\]
    за всяко $x \in \Nat_\bot$,
    {\bf е точно}, защото $g(\bot) = \bot$. Освен това, $g$ е монотонно и непрекъснато изображение.
  \end{itemize}
\end{example}

Видяхме, че лесно се намират монотонни изображения, които не са точни.
Сега ще разгледаме обратната посока.

\begin{prop}
  \label{pr:strict-is-monotone}
  Всяко точно изображение $f:\Nat^n_\bot \to \Nat_\bot$ е също така и монотонно.
  С други думи, 
  \[\Strict{\Nat^n_\bot}{\Nat_\bot} \subseteq \Mon{\Nat^n_\bot}{\Nat_\bot}.\]
\end{prop}
\begin{proof}
  Нека $f$ е точно и $\bar{a} \sqsubseteq \bar{b}$.
  Ще проверим, че 
  \[f(\bar{a}) = c \sqsubseteq d = f(\bar{b}).\]
  \begin{itemize}
  \item 
    Ако $c = \bot$, то е очевидно, че $c \sqsubseteq d$.
  \item
    Интересният случай е когато $c \neq \bot$. Тогава трябва да видим, че $c = d$.
    Понеже $f$ е точна и $c \neq \bot$, това означава, че $\bot \not\in\{a_1,\dots,a_n\}$.
    Но понеже $\sqsubseteq$ е плоска наредба, и $\bar{a} \sqsubseteq \bar{b}$, това означава, че $\bar{a} = \bar{b}$.
    Следотвателно, наистина $c = d$.
  \end{itemize}
\end{proof}

\begin{framed}
  \begin{thm}
    \label{th:strict-is-domain}
    $(\Strict{\Nat^n_\bot}{\Nat_\bot},\ \sqsubseteq,\ \bm{\bot}^{(n)})$ е област на Скот.
  \end{thm}  
\end{framed}
\begin{hint}
  Да разгледаме веригата $\chain{f}{i}$ от елементи на $\Strict{\Nat^n_\bot}{\Nat_\bot}$.
  Трябва да докажем, че тази верига притежава точна горна граница, която също е точно изображение.
  % Понеже от \Prop{strict-is-monotone} имаме, че всяка точна функция е монотонна, то наготово от 
  От доказателството на \Th{all-mappings-is-domain} знаем, че изображението $h$ дефинирано за всяко $\bar{a} \in \Nat^n_\bot$ като
  \[h(\bar{a}) \dff \bigsqcup \{f_i(\bar{a}) \mid i \in \Nat\}\]
  е точна горна граница на веригата $\chain{f}{i}$.
  Остава да проверим, че $h$ е точно изображение.
  Нека $\bar{a} \in \Nat^n_\bot$ и да приемем, че $\bot \in \{a_1,\dots,a_n\}$.
  Тогава:
  \begin{align*}
    h(\bar{a}) & = \bigsqcup \{f_i(\bar{a}) \mid i \in \Nat\} & \comment{\text{от деф. на }h}\\
               & = \bigsqcup\{\bot,\bot,\dots,\bot,\dots\} & \comment{f_i\text{ са точни}}\\
               & = \bot.
  \end{align*}
\end{hint}

\begin{example}
  \label{ex:simple-non-continuous}
  Нека сега да разгледаме $h:\Nat_\bot \to \Nat_\bot$, където 
  \begin{align*}
    h(x) = &
    \begin{cases}
      0, & \text{ако }x = \bot,\\
      1, & \text{иначе}.
    \end{cases}
  \end{align*}
  Да видим колко ,,лошо'' изображение е $h$:
  \begin{itemize}
  \item 
    $h$ не е точно, защото $h(\bot) \neq \bot$;
  \item
    $h$ не е монотонно, защото $\bot \sqsubseteq 5$, но $h(\bot) = 0 \not\sqsubseteq 1 = h(5)$;
  \item
    $h$ не е непрекъснато, защото $\Cont{\Nat_\bot}{\Nat_\bot} = \Mon{\Nat_\bot}{\Nat_\bot}$.
  \end{itemize}
  Нека да видим, че хаскел не позволява такива ,,лоши'' функции:

  \begin{haskellcode}
ghci> let h(x) = if x == undefined then 0 else 1
ghci> h(5)
*** Exception: Prelude.undefined
ghci> h(undefined)
*** Exception: Prelude.undefined
  \end{haskellcode}
\end{example}

\begin{example}
  \label{ex:non-strict-monotone}
  Да разгледаме $f:\Nat^2_\bot \to \Nat_\bot$ дефинирано като \[f(x,y) = x.\]
  \begin{itemize}
  \item 
    $f$ не е точно, защото $f(0,\bot) = 0$.
  \item
    $f$ е монотонно, защото ако $\pair{x,y} \sqsubseteq \pair{x',y'}$, то $x \sqsubseteq x'$ и
    \[f(x,y) = x \sqsubseteq x' = f(x',y').\]
  \item
    $f$ е непрекъснато, защото от \Prop{stab-continuous} знаем, че 
    $\Mon{\Nat^2_\bot}{\Nat_\bot} = \Cont{\Nat^2_\bot}{\Nat_\bot}$.
  \end{itemize}
\end{example}

Можем да обобщим всичко, което направихме дотук, със следната илюстрация на основните области на Скот, с които ще работим
в \Chapter{rec}.

\begin{framed}
    \begin{align*}
      \Strict{\Nat^n_\bot}{\Nat_\bot} & \subsetneqq \Mon{\Nat^n_\bot}{\Nat_\bot} & \comment{\text{от \Ex{non-strict-monotone} и \Prop{strict-is-monotone}}}\\
      & = \Cont{\Nat^n_\bot}{\Nat_\bot} & \comment{\text{от \Prop{continuous-is-monotone} и \Cor{monotone-is-continuous}}}\\
      & \subsetneqq \Mapping{\Nat^n_\bot}{\Nat_\bot} & \comment{\text{от \Ex{simple-non-continuous}}}.
    \end{align*}
\end{framed}

\begin{example}
  \label{ex:plus}
  Да разгледаме $f:\Nat^2_\bot \to \Nat_\bot$ дефинирана по следния начин:
  \[f(x,y) = 
  \begin{cases}
    \bot, & x = \bot\ \&\ y = \bot\\
    x, & x \neq \bot\ \&\ y = \bot\\
    y, & x = \bot\ \&\ y \neq \bot\\
    x+y, & x \neq \bot\ \&\ y \neq \bot.
  \end{cases}\]
  Изображението $f$ не е точно, защото например $f(\bot,0) \neq \bot$.
  $f$ не е монотонно изображение, защото $\pair{2,\bot} \sqsubseteq \pair{2,3}$, но $2 = f(2,\bot) \not\sqsubseteq f(2,3) = 5$.
  Също така, $f$ не е непрекъснато изображение, защото $\Mon{\Nat^2_\bot}{\Nat_\bot} = \Cont{\Nat^2_\bot}{\Nat_\bot}$.
\end{example}
% \begin{hint}
%   \begin{itemize}
%   \item 
%     $f$ не е точно изображение, защото например
%     $f(\bot,0) \neq \bot$.
%   \item
%     $f$ не е монотонно. Например,
%     \[\pair{\bot,0} \sqsubseteq \pair{1,0}\ \&\ f(\bot,0) = 0 \not\sqsubseteq 1 = f(1,0).\]    
%   \item
%     $f$ не е непрекъснато, защото $\Cont{\Nat^2_\bot}{\Nat_\bot} = \Mon{\Nat^2_\bot}{\Nat_\bot}$.
%   \end{itemize}    
% \end{hint}


% \subsection{Точни продължения}
% \index{изображение!точно продължение}

% Нека $A$ и $B$ са произволни множества, а $\A_\bot$ и $\B_\bot$ са плоските области на Скот получени съответно от $A$ и $B$.
% За еднa частична функция $f:A^n \to B$, определяме точното изображение $f^\star:\A^n_\bot \to \B_\bot$ по следния начин:
% \marginpar{$f^\star \in \Strict{\A^n_\bot}{\B_\bot}$}
% \begin{align*}
%   f^\star(\ov{a}) \dff
%   \begin{cases}
%     f(\ov{a}), & \text{ако }\bot\not\in\{a_1,\dots,a_n\}\ \&\ f(\ov{a})\text{ е деф.}\\
%     \bot, & \text{иначе}.
%   \end{cases}
% \end{align*}
% Изображението $f^\star$ се нарича {\bf точно продължение} на $f$.

% \begin{example}
%   Да разгледаме частичната функция $f:\Nat^2 \to \Nat$, дефинирана като
%   $f(x,y) = x+y$. Тогава точното продължение на $f$ е 
%   \[f^\star(x,y) = 
%   \begin{cases}
%     x+y, & x,y\in\Nat\\
%     \bot, & x = \bot \vee y = \bot.
%   \end{cases}\]
% \end{example}

% \Stefan{За какво ми е точното продължение да е непрекъснато?}
% \noindent Да дефинираме оператор $\Sigma^\star : \Partial{\Nat^n}{\Nat} \to \Strict{\Nat^n_\bot}{\Nat_\bot}$ като
% \[\Sigma^\star(f) = f^\star.\] 
% Ще наричаме $\Sigma^\star$ {\bf продължаващ} оператор, защото на всяка частична функция дава нейното точно продължение.

% \begin{framed}
%   \begin{prop}
%     $\Sigma^\star$ е непрекъснато изображение, т.е.
%     \[\Sigma^\star \in \Cont{\Partial{\Nat^n}{\Nat}}{\Strict{\Nat^n_\bot}{\Nat_\bot}}.\]
%   \end{prop}
% \end{framed}
% \begin{hint}
%   Трябва да докажем, че за произволна верига $(f_i)^{\infty}_{i=0}$ от частични функции, 
%   $\Sigma^\star(\bigcup_i f_i) = \bigsqcup_i \Sigma^\star(f_i)$.
%   \begin{itemize}
%   \item 
%     Лесно се съобразява, че ако $f \subseteq g$, то $\Sigma^\star(f) \sqsubseteq \Sigma^\star(g)$.
%     Оттук следва, че 
%     \[\bigsqcup_i \Sigma^\star(f_i) \sqsubseteq \Sigma^\star(\bigcup_i f_i).\]
%   \item
%     За другата посока,
%     Нека $\Sigma^\star(\bigcup_i f_i)(\ov{x}) = y \neq \bot$.
%     Това означава, че $\ov{x} \in \Nat^n$ и $(\bigcup_i f_i)(\ov{x}) \simeq y$.
%     Оттук следва, че съществува $i_0$, за което $f_{i_0}(\ov{x}) \simeq y$.
%     Тогава $\Sigma^\star(f_{i_0})(\ov{x}) = y$.
%     Понеже $y \neq \bot$, 
%     \[\bigsqcup_i (\Sigma^\star(f_i)(\ov{x})) = (\bigsqcup_i \Sigma^\star(f_i))(\ov{x}) = y.\]
%     Заключаваме, че $\Sigma^\star(\bigcup_i f_i)(\ov{x}) \sqsubseteq \bigsqcup_i (\Sigma^\star(f_i)(\ov{x}))$
%   \end{itemize}
% \end{hint}

% \Stefan{Това означава, че някъде трябва да се каже, че $\F_n$ е област на Скот. Някъде в тази глава трябва да е. Това може да бъде един от първите примери след дефиницията на О.С.}



%%% Local Variables:
%%% mode: latex
%%% TeX-master: "../sep"
%%% End:
