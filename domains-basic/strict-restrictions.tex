\subsection{Точни ограничения}
\index{изображение!точно ограничение}

Нека фиксираме две области на Скот $\A$ и $\B$.
За едно изображение $f \in \Mapping{\A^n}{\B}$, дефинираме точното изображение $f_\star \in \Strict{\A^n}{\B}$ по следния начин:
\[f_\star(\bar{a}) =
\begin{cases}
  f(\bar{a}), & \text{ако } \bot\not\in\{a_1,\dots,a_n\}\\
  \bot, & \text{ако }\bot\in\{a_1,\dots,a_n\}.
\end{cases}\]
Ще казваме, че $f_\star$ е {\bf точно ограничение} на $f$.

\begin{framed}
  \begin{proposition}
    За всяко $f \in \Mapping{\A^n}{\B}$, имаме 
    \[f_\star \sqsubseteq f.\]
  \end{proposition}
\end{framed}

\begin{example}
  Точното ограничение на функцията $f$ от \Ex{plus} е
  \[f_\star(x, y) =
  \begin{cases}
    x + y, & \text{ако } x \neq \bot\ \&\ y \neq \bot\\
    \bot, & \text{иначе}
  \end{cases}
\]
\end{example}

\begin{example}
  Да видим какъв резултат връща хаскел при следните извиквания:
  
  \begin{haskellcode}
ghci> (\x -> 5) 4
5
ghci> (\x -> 5) undefined
5
  \end{haskellcode}
  
  От второто извикване се вижда, че в хаскел, анонимната функция $\lambda x.5$ е дефинирана и върху $\bot$ и връща отново $5$.
  \marginpar{$(\lambda x.5)(4) = 5$, $(\lambda x.5)(\bot) = 5$}
  Това е естествено, понеже хаскел е {\em неточен} език, т.е. една функция може да има като аргумент $\bot$
  и да връща ,,смислен'' резултат. 

  \marginpar{Повече информация за \texttt{seq} може да се намери в \cite[стр. 108]{real-world-haskell}}
  При разглеждането на семантики на езици за програмиране, ще се нуждаем от това да направим една функция {\em точна}.
  Такава възможност има и в хаскел с командата \texttt{seq}.
  
  \begin{haskellcode}
ghci> :t seq
seq :: a -> b -> b
ghci> (\x -> x `seq` 5) 4
5
ghci> (\x -> x `seq` 5) undefined
*** Exception: Prelude.undefined
ghci> :set -XBangPatterns
ghci> (\(!x) -> 5) undefined
*** Exception: Prelude.undefined
\end{haskellcode}
  
  Командата \texttt{seq} приема два аргумента. Тя работи като оценява първия аргумент, като очаква той да се оцени до ,,смислен'' резултат, т.е. нещо различно от $\bot$. След това връща втория.
  Тук първия пример се превежда като $(\lambda x. 5)_\star(4) = 5$, а втория като $(\lambda x.5)_\star(\bot) = \bot$.
\end{example}

За произволно естествено число $n$, да дефинираме изображение
\[\Sigma_\star : \Mapping{\Nat^n_\bot}{\Nat_\bot} \to \Strict{\Nat^n_\bot}{\Nat_\bot},\]
където
\[\Sigma_\star(f) = f_\star.\] 
Ще наричаме $\Sigma_\star$ {\bf ограничаващ} оператор, защото на всяка функция дава нейното точно ограничение.

\begin{framed}
\begin{proposition}
  \label{pr:strict-operator}
  За всяко $n$, $\Sigma_\star$ е непрекъснато изображение, т.е.
  \[\Sigma_\star \in \Cont{\Mapping{\Nat^n_\bot}{\Nat_\bot}}{\Strict{\Nat^n_\bot}{\Nat_\bot}}.\]
\end{proposition}
\end{framed}
\begin{hint}
  Трябва да докажем, че за произволна верига $\chain{f}{i}$ от елементи на $\Mapping{\Nat^n_\bot}{\Nat_\bot}$, имаме
  \[\Sigma_\star(\bigsqcup_i f_i) = \bigsqcup_i \Sigma_\star(f_i).\]
  \begin{enumerate}[(1)]
  \item 
    Лесно се съобразява, че $\Sigma_\star$ е монотонно изображение, т.е.
    \[f \sqsubseteq g \implies \Sigma_\star(f) \sqsubseteq \Sigma_\star(g).\]
    Сега използваме \Prop{monotone-chain}, откъдето следва, че
    \[\bigsqcup_i \Sigma_\star(f_i) \sqsubseteq \Sigma_\star(\bigsqcup_i f_i).\]
  \item
    За другата посока, нека да разгледаме $\ov{a}$, за които
    \marginpar{Случаят $\Sigma_\star(\bigsqcup_i f_i)(\ov{a}) = \bot$ е тривиален.}
    \[\Sigma_\star(\bigsqcup_i f_i)(\ov{a}) = b \neq \bot.\]
    Щом $b \neq \bot$, то ако $\ov{a} = (a_1, \dots, a_n)$, със сигурност $\bot \not\in \{a_1, \dots, a_n\}$.
    Тогава директно имаме, че $(\bigsqcup_i f_i)(\ov{a}) = b \neq \bot$.
    Сега от \Problem{sup-f} получаваме, че съществува $i_0$, за което $f_{i_0}(\ov{a}) = b$.
    Ясно е, че също така $\Sigma_\star(f_{i_0})(\ov{a}) = b$ и оттук
    $(\bigsqcup_i \Sigma_\star(f_i))(\ov{a}) = b$,
    защото $\Sigma_\star(f_{i_0}) \sqsubseteq \bigsqcup_i \Sigma_\star(f_i)$.
    Заключаваме, че
    \[\Sigma_\star(\bigsqcup_i f_i) \sqsubseteq \bigsqcup_i\Sigma_\star(f_i).\]
  \end{enumerate}
\end{hint}

%%% Local Variables:
%%% mode: latex
%%% TeX-master: "../sep"
%%% End:
