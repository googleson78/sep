\documentclass[a4paper,12pt,oneside]{book}

% \renewcommand{\familydefault}{\sfdefault}

\usepackage{cmap}
\usepackage[utf8]{inputenc}
\usepackage[T2A]{fontenc}
\usepackage[bulgarian]{babel}



\usepackage[xindy]{imakeidx}
\def\xindylangopt{-M lang/bulgarian/utf8-lang.xdy}
\makeindex[options = \xindylangopt]
% \makeindex[name=names, title = Именен указател, options = \xindylangopt]
% \makeindex[name=proglangs, title = Изчислителни машини и програмни езици, options = \xindylangopt]
% \makeindex[name=organisations, title = Организации и фирми, options = \xindylangopt]

% \makeindex[options = \xindylangopt]
\makeatletter
\let\original@index\index
\renewcommand{\index}[2][\imki@jobname]{%
  \original@index[#1]{\detokenize{#2}}%
}



\usepackage{amssymb, amsmath, amsthm, latexsym, mathrsfs, bm, mathtools}
\usepackage{stmaryrd}
\usepackage{makeidx}

\usepackage{stackrel}
\usepackage{paralist}
\usepackage{pifont}
\usepackage[shortlabels]{enumitem}
\setlist{leftmargin=*}
\usepackage{algorithm}
\floatname{algorithm}{Алгоритъм}
\usepackage[noend]{algpseudocode}
\usepackage{framed}

\usepackage{bussproofs}
\def\defaultHypSeparation{\hskip .2cm}

\usepackage{wrapfig}
\usepackage{xcolor}
\usepackage{minted}
\usepackage{pdflscape}

\usemintedstyle{friendly}
\definecolor{bg}{rgb}{0.9, 0.9, 0.9}
\newminted{haskell}{mathescape,fontsize=\footnotesize,baselinestretch=1.2,frame=single,framesep=5pt} %,bgcolor=bg}
%\usepackage{microtype} ??????

\usepackage{color, soulutf8}
\newcommand{\newauthor}[2]{
  \definecolor{#1}{rgb}{#2}
  \expandafter\newcommand\csname #1\endcsname[1]{{\sethlcolor{#1}\hl{#1: ,,##1''}}}}
 
\definecolor{codegreen}{rgb}{0,0.6,0}
\definecolor{codegray}{rgb}{0.5,0.5,0.5}
\definecolor{codepurple}{rgb}{0.58,0,0.82}
\definecolor{backcolour}{rgb}{0.95,0.95,0.92}

% \newauthor{Stefan}{0.6,0.6,0.6}


\newcommand{\Stefan}[1]{}

% \newauthor{Stela}{0.8,0.8,0}


\setlist[itemize]{leftmargin=*}

\def\dotminus{\mathbin{\ooalign{\hss\raise1ex\hbox{.}\hss\cr\mathsurround=0pt$-$}}}

%%%%%%%%%%%%%%% TIKZ Package %%%%%%%%%%%%%%%%%%%%%%%
\usepackage{tikz, pgf}
\usetikzlibrary{shapes,arrows,cd,positioning}
%%%%%%%%%%%%%%%%%%%%%%%%%%%%%%%%%%%%%%%%%%%%%%%%%%%%
\usepackage{caption}
\usepackage{subcaption}
% \usepackage{subfigure}

\usepackage[pdfencoding=unicode, colorlinks=true, linkcolor=blue, pdfstartview=FitV, citecolor=green, urlcolor=blue]{hyperref}
\hypersetup{pdfauthor={Стефан Вътев},pdftitle={Семантика на езиците за програмиране},pdftex,unicode}

\theoremstyle{definition}
\newtheorem{theorem}{Теорема}[chapter]
\newtheorem{definition}{Определение}[chapter]
\newtheorem{problem}{{\bf Задача}}[chapter]
\newtheorem{prb}{{\bf Задача}}[chapter]
\newtheorem{prop}{Твърдение}[chapter]
\newtheorem{cor}{Следствие}[chapter]
\newtheorem{lemma}{Лема}[chapter]
\newtheorem{example}{Пример}[chapter]
\newtheorem{remark}{Забележка}[chapter]

\newcommand{\REC}{{\bf REC}}

\newcommand{\NN}{\mathbf{Num}}
\newcommand{\BB}{\mathbf{Bool}}

\usepackage{mysymbols}

\renewenvironment{proof}{\noindent{\bf Доказателство.}\hspace*{1em}}{\qed\par}
\newenvironment{hint}{\noindent{\bf Упътване.}\hspace*{1em}}{\qed\par}
\newenvironment{solution}{\noindent{\bf Решение.}\hspace*{1em}}{\qed\par}
\newcommand{\answer}{\it Отговор.\rm}

\newcommand{\Ex}[1]{{\em Пример }~\ref{ex:#1}}
\newcommand{\Alg}[1]{{\em Алгоритъм}~\ref{alg:#1}}
\newcommand{\Th}[1]{{\em Теорема}~\ref{th:#1}}
\newcommand{\Cor}[1]{{\em Следствие}~\ref{cr:#1}}
\newcommand{\Lem}[1]{{\em Лема}~\ref{lem:#1}}
\newcommand{\Prop}[1]{{\em Твърдение}~\ref{pr:#1}}
\newcommand{\Problem}[1]{{\em Задача}~\ref{prob:#1}}
\newcommand{\Fig}[1]{{\em Фигура}~\ref{fig:#1}}
\newcommand{\Chapter}[1]{{\em Глава}~\ref{ch:#1}}
%%\selectlanguage{bulgarian}

\newcommand{\writedown}{\ding{45}\ }

% \newcommand{\val}[1]{\llbracket{#1}\rrbracket}
\newcommand{\evalv}[1]{\text{eval}^P_V\val{{#1}}}
\newcommand{\evaln}[1]{\text{eval}^P_N\val{{#1}}}
\newcommand{\ifelse}[3]{\texttt{if}\ #1\ \texttt{then}\ #2\ \texttt{else}\ #3} 
\newcommand{\while}[2]{\mathbf{while}\ #1\ \mathbf{do}\ #2}
\newcommand{\rpt}[2]{\mathbf{repeat}\ #1\ \mathbf{until}\ #2}
\newcommand{\op}{\texttt{op}}
\newcommand{\true}{\mathbf{t}}
\newcommand{\false}{\mathbf{f}}
\newcommand{\exit}{\mathbf{exit}}
\newcommand{\skp}{\mathbf{skip}}
\newcommand{\newvar}[3]{\mathbf{newvar}\ #1 := #2\ \mathbf{in}\ #3}
\newcommand{\trycatch}[2]{\mathbf{try}\ #1\ \mathbf{catch}\ #2}
\newcommand{\finally}[3]{\mathbf{try}\ #1\ \mathbf{catch}\ #2\ \mathbf{finally}\ #3}
\newcommand{\for}[4]{\mathbf{for}\ #1 := #2\ \mathbf{to}\ #3\ \mathbf{do}\ #4}
\newcommand{\dfs}{\stackrel{\textnormal{деф}}{=}}
% \newcommand{\dff}{\stackrel{\textnormal{деф}}{=}}
\newcommand{\dfff}{\stackrel{\textnormal{деф}}{\equiv}}
\newcommand{\dffff}{\stackrel{\textnormal{деф}}{\equiv}}
\newcommand{\lfp}{\texttt{lfp}}
\newcommand{\lub}{\texttt{lub}}

\newcommand{\Mapping}[2]{[#1~\to~#2]}
% \newcommand{\Total}[2]{[#1~\stackrel{t}{\to}~#2]}
\newcommand{\Partial}[2]{[#1~\stackrel{\text{ч}}{\to}~#2]}
\newcommand{\Strict}[2]{[#1~\stackrel{\text{т}}{\to}~#2]}
\newcommand{\Mon}[2]{[#1~\stackrel{\text{м}}{\to}~#2]}
\newcommand{\Cont}[2]{[#1~\stackrel{\text{н}}{\to}~#2]}
% \newcommand{\Compact}[2]{[#1~\stackrel{\text{k}}{\to}~#2]}
\newcommand{\DomOpCBN}{\Cont{\Nat^{m_1}_\bot}{\Nat_\bot}\times\cdots\times\Cont{\Nat^{m_k}_\bot}{\Nat_\bot}}
\newcommand{\RanOpCBN}{\Cont{\Nat^{n}_\bot}{\Nat_\bot}}
\newcommand{\DomOpCBV}{\Strict{\Nat^{m_1}_\bot}{\Nat_\bot}\times\cdots\times\Strict{\Nat^{m_k}_\bot}{\Nat_\bot}}
\newcommand{\RanOpCBV}{\Strict{\Nat^{n}_\bot}{\Nat_\bot}}


\newcommand{\Pref}{\texttt{Pref}}                 %% Преднеподвижни точки
\newcommand{\Fenv}{\textbf{Fenv}[\varsf]}
\newcommand{\Venv}{\textbf{Env}[\varsx]}

\newcommand{\Graph}[1]{\texttt{Graph}(#1)}

\newcommand\nil{\texttt{[]}}
% \newcommand{\pair}[1]{\langle{#1}\rangle}
% \newcommand{\abs}[1]{\lvert{#1}\rvert}
% \newcommand{\ov}[1]{\overline{#1}}
% \renewcommand{\iff}{\Longleftrightarrow}
\newcommand{\Cmd}{\mathbf{Cmd}}
\newcommand{\Arith}{\mathbf{Arith}}
\newcommand{\vv}[1]{\texttt{#1}}
\newcommand{\varx}{\mathbf{x}}
\newcommand{\varsx}{\bar{\texttt{x}}}
\newcommand{\varsf}{\bar{\texttt{f}}}

%%%%%%% LISTS %%%%%%%

\newcommand{\FinL}{\texttt{FinL}}
\newcommand{\PartL}{\texttt{PartL}}
\newcommand{\InfL}{\texttt{InfL}}

%%%%%%%%%%%%%%%%%%%%%

\def\monus{\mathbin{\ooalign{\hss\raise1ex\hbox{.}\hss\cr
      \mathsurround=0pt$-$}}}

\setlength{\marginparsep}{1cm}
\setlength{\oddsidemargin}{0.3cm}
\setlength{\hoffset}{-0.75in}
\setlength{\marginparwidth}{110pt}
\setlength{\textwidth}{420pt}
\setlength{\textheight}{670pt}
\let\oldmarginpar\marginpar

\def\dotminus{\mathbin{\ooalign{\hss\raise1ex\hbox{.}\hss\cr
             \mathsurround=0pt$-$}}}


% \renewcommand\marginpar[1]{\-\oldmarginpar[\raggedleft\footnotesize #1]%
% {\raggedright\footnotesize #1}}
% \renewcommand\marginpar[1]{\-\oldmarginpar[\raggedleft\scriptsize #1]
% {\raggedright\scriptsize #1}}
\renewcommand\marginpar[1]{\oldmarginpar{\scriptsize #1}}

\newif\ifhints
%\proofstrue % comment out to hide answers
\hintstrue

\allowdisplaybreaks
% \makeindex


\usepackage[scaled=0.89]{PTSerif}
\usepackage[scaled=0.89]{PTSans}

\author{Стефан Вътев\footnote{\href{mailto:stefanv@fmi.uni-sofia.bg}{stefanv@fmi.uni-sofia.bg}}}
\title{Семантика на езиците за програмиране - записки}

\begin{document}


% \frontmatter
\maketitle
\tableofcontents


% \chapter{Увод}

% Увод няма.


% \mainmatter

% \begin{otherlanguage}{greek}
% \begin{haskellcode}
%   ΤΕΣΤ
% \end{haskellcode}
% \end{otherlanguage}


% \include{basic}
% \include{imp}
% \include{Operators}
\chapter{Области на Скот}\index{област на Скот}

\marginpar{\cite[Глава 5]{models-of-computation}}

В тази глава ще разгледаме понятията, които са ни нужни за дефинирането на понятието денотационна семантика на една програма.

\section{Частични наредби}
\index{частична наредба}

Бинарната релация $\sqsubseteq$ върху множеството $A$ се нарича {\bf частична наредба}, ако тя е:
\marginpar{На англ. \emph{partial order}}
\begin{itemize}
\item 
  рефлексивна, т.е. $(\forall a \in A)[a \sqsubseteq a]$;
\item
  транзитивна, т.е. $(\forall a,b,c \in A)[a \sqsubseteq b\ \&\ b \sqsubseteq c \implies a \sqsubseteq c]$;
\item
  антисиметрична, т.е. $(\forall a,b \in A)[a \sqsubseteq b\ \&\ b \sqsubseteq a  \implies a = b]$.
\end{itemize}
Една такава двойка $(A, \sqsubseteq)$ се нарича частично наредено множество.

\begin{example}
  Да означим 
  \[\Partial{\Nat}{\Nat} \dff \{f:\Nat\to\Nat \mid f\text{ е частична функция}\}.\]
  Дефинираме и релацията {\bf включване } между две частични функции по следния начин:
  \begin{align*}
    f \subseteq g \dfff (\forall x\in \Nat)[& f(x)\text{ не е деф.}\ \vee\\
                                            & (f(x)\text{ е деф.}\ \&\ g(x)\text{ е деф.}\ \&\ f(x) = g(x))].
  \end{align*}
  Да дефинираме също {\bf графиката} на частичната функция $f$ като
  \[\Graph{f} \dff \{\pair{x,y} \mid f(x) = y\}.\]
  Тогава лесно се съобразява, че 
  \[f \subseteq g \iff \Graph{f} \subseteq \Graph{g}.\]
  Съобразете, че двойката $(\Partial{\Nat}{\Nat}, \subseteq)$ е частично наредено множество.
\end{example}

Казваме, че $a_0$ е {\bf най-малък елемент} на частично нареденото множество $(A, \sqsubseteq)$,
ако $(\forall a \in A)[a_0 \sqsubseteq a]$. Ако такъв елемент съществува, то той е единствен,
защото релацията $\sqsubseteq$ е антисиметрична.
{\bf Неподвижна точка} на $f:A \to A$ е елемент $a \in A$, такъв че $f(a) = a$.
За по-кратко монотонно-растящите редици от елементи на $A$,
\[a_0 \sqsubseteq a_1 \sqsubseteq \cdots \sqsubseteq a_n \sqsubseteq \cdots,\]
ще наричаме (растящи) {\bf вериги}. 

Един елемент $b$ е {\bf горна граница} на веригата $\chain{a}{n}$, ако 
$(\forall n)[a_n \sqsubseteq b]$.
Един елемент $b$ е {\bf точна горна граница} на веригата $\chain{a}{n}$, ако са изпълнени свойствата:
\begin{itemize}
\item 
  $(\forall n)[a_n \sqsubseteq b]$, т.е. $b$ е горна граница;
\item
  за всяка друга горна граница $c$ е изпълнено, че $b \sqsubseteq c$, т.е.
  $b$ е най-малкият елемент измежду всички горни граници на веригата $\chain{a}{n}$.
\end{itemize}
Обърнете внимание, че не всяка верига притежава точна горна граница.
Обикновено точната горна граница на вергата $\chain{a}{n}$ ще бележим като $\bigsqcup_n a_n$.

\Stefan{Още тук да се даде пример за верига от частични функции и две функции - една, която е точна горна граница на веригата и една, която е просто горна граница.}


\begin{definition}
  Наредена тройка от вида $\A = (A, \sqsubseteq, \bot)$ се нарича {\bf област на Скот}, ако:
  \index{област на Скот}
  \marginpar{На англ. {\em Scott domain}}
  \begin{itemize}
  \item
    $\sqsubseteq$ е бинарна релация върху $A$, която задава частична наредба.
  \item
    Всяка растяща верига $\chain{a}{n}$ в $A$ притежава точна горна граница $\bigsqcup_n a_n$.
  \item
    $\bot \in A$ е най-малкият елемент на $A$;
  \end{itemize}
\end{definition}


\marginpar{В Хаскел $\bot$ се означава като \vv{undefined}. Повече за денотационна семантика в Хаскел може да прочетете \href{https://en.wikibooks.org/wiki/Haskell/Denotational_semantics}{тук}}

\begin{example}
  \Stefan{Друго означение вместо $\F_n$ ?}
  Тройката
  \[\F_n \dff (\ \Partial{\Nat^n}{\Nat},\ \sqsubseteq,\ \emptyset^{(n)}\ )\] е област на Скот, където:
  \begin{itemize}
  \item
    С $\Partial{\Nat^n}{\Nat}$ означаваме всички частични функции от $\Nat^n$ в $\Nat$.
  \item
     релацията ,,включване'' между функции е дефинирана по следния начин:
     \[f\sqsubseteq g\ \dffff\ \text{Graph}(f) \subseteq \text{Graph}(g).\]
   \item
     $\emptyset^{(n)}$ е функцията с празна дефиниционна област, т.е.
     \[(\forall \bar{x} \in \Nat^n)[\neg !\emptyset^{(n)}(\bar{x})].\]
  \end{itemize}
\end{example}

\begin{example}
  Да разгледаме няколко примера, които вече сме срещали.
  \begin{itemize}
  \item
    $(\Ps(\Nat),\subseteq,\emptyset)$ е област на Скот.
  \item
    $(\Nat, \leq, 0)$ не е е област на Скот.
  \item
    $(\Nat\cup\{\infty\}, \leq, 0)$ е също област на Скот, където наредбата $\leq$ е зададена като
    \[0 \leq 1 \leq \cdots \leq \infty.\]
  \item
    $(\{0,1\}^\star, \preceq, \varepsilon)$ не е област на Скот, където $\preceq$ е релацията префикс на две думи.
    % Може да образувате безкрайна растяща верига от думи, но тяхната точна горна граница ще бъде безкрайна дума.
  \end{itemize}
\end{example}

\begin{example}
  Да разгледаме множеството 
  \[Bin^\infty = \{\sigma \mid \sigma :\{0,1,2,\dots,n-1\} \to \{0,1\}\ \&\ n \in \Nat\} \cup 
  \{f \mid f:\Nat \to \{0,1\}\}\]
  съставено от всички крайни и безкрайни двоични низове.
  \begin{itemize}
  \item
    Да разгледаме релацията
    \[\sigma \preceq \tau \iff \abs{\sigma} \leq \abs{\tau}\ \&\ (\forall i < |\sigma|)[\sigma(i) = \tau(i)],\]
    т.е. $\sigma$ е префикс на $\tau$.    
  \item
    Да означим с $\varepsilon$ единствения двоичен низ с дължина $0$. С други думи, $\varepsilon$ е празната функция.
  \end{itemize}
  Тогава $Bin^\infty = (Bin^\infty,\preceq,\varepsilon)$ е област на Скот.
\end{example}

%%% Local Variables:
%%% mode: latex
%%% TeX-master: "../sep"
%%% End:


\section{Конструкции}

Ще разгледаме няколко конструкции, с които ще видим как можем да строим по-сложни области на Скот.

\subsection{Плоска област на Скот}

\Stefan{По принцип плоската област на Скот се дефинира върху друга област на Скот}

Да фиксираме едно произволно непразно множество $A$ и един елемент $\bot \not \in A$.
Да означим $A_\bot = A \cup \{\bot\}$ и да разгледаме следната бинарна релация $\sqsubseteq$ върху $A_\bot$:
\[a \sqsubseteq b\ \iff\ a = \bot\ \vee\ a = b.\]
Лесно се съобразява, че $\sqsubseteq$ задава {\em частична наредба} върху $A_\bot$:
\marginpar{От деф. на $\sqsubseteq$ следва, че $\bot$ е най-малкият елемент}
\begin{itemize}
\item 
  {\em рефлексивност}: $a \sqsubseteq a$ за всяка $a \in A_\bot$;
\item
  {\em транзитивност}: $a \sqsubseteq b\ \&\ b \sqsubseteq c \implies a\sqsubseteq c$ за всеки $a,b,c \in A_\bot$;
\item
  {\em антисиметричност}: $a \sqsubseteq b\ \&\ b\sqsubseteq a \implies a = b$ за всеки $a,b \in A_\bot$.
\end{itemize}

Наредбата $(A_\bot, \sqsubseteq)$ ще наричаме {\bf плоска наредба}. Тя ще играе важна роля в нашите разглеждания.
\marginpar{$\bot$ се нарича {\em bottom} елемент}
Например, често ще разглеждаме плоската наредба $(\Nat_\bot, \sqsubseteq)$.

\begin{framed}
  \begin{figure}[H]
    \label{fig:flat-nat-1}
    \centering
    \begin{tikzpicture}[shorten >=1pt,->]
      \tikzstyle{vertex}=[circle,minimum size=17pt,inner sep=0pt]
      
      \node[vertex] (bot) at (3,0) {$\bot$};
      \node[vertex] (0) at (0,2) {$0$};
      \node[vertex] (1) at (1,2) {$1$};
      \node[vertex] (2) at (2,2) {$2$};
      \node[vertex] (dots) at (3,2) {$\cdots$};
      \node[vertex] (n) at (4,2) {$n$};
      \node[vertex] (n1) at (5,2) {$n+1$};
      \node[vertex] (ddots) at (6.25,2) {$\cdots$};
      
      \draw (bot) -- node[below left]{$\scriptstyle{\sqsupseteq}$} (0);
      \draw (bot) -- (1);
      \draw (bot) -- (2);
      \draw[dashed] (bot) -- (dots);
      \draw (bot) -- (n);
      \draw (bot) -- (n1);
      \draw[dashed] (bot) -- (ddots);
    \end{tikzpicture}    
    \caption{Графично представяне на плоската наредба $\sqsubseteq$ върху $\Nat_\bot$}
  \end{figure}
  \end{framed}

\begin{prop}
  \index{област на Скот!плоска}
  Нека $A$ е произволно множество и нека елементът $\bot \not \in A$.
  \marginpar{На англ. {\em flat domain}}
  Определяме наредената тройка $\A_\bot = (A_\bot, \sqsubseteq, \bot)$ като:
  \begin{itemize}
  \item 
    $A_\bot = A\cup\{\bot\}$;
  \item
    $\sqsubseteq$ задава {\em плоската наредба} върху $A_\bot$.
  \end{itemize}
  Тогава $\A_\bot$ е област на Скот, която ще наричаме {\bf плоска област на Скот} за множеството $A$.
\end{prop}

%%% Local Variables:
%%% mode: latex
%%% TeX-master: "../sep"
%%% End:


\subsection{Крайно произведение}
\label{subsect:domains:product}
\index{област на Скот!крайно произведение}
\marginpar{\cite[стр. 125]{winskel}}

Да разгледаме тройките $\A_i = (A_i, \sqsubseteq_i, \bot_i)$, където $i = 1,\dots,n$. Дефинираме тройката 
$\prod^n_{i=1}\A_i = (A, \sqsubseteq, \bot)$ като:
\begin{itemize}
\item
  $A \df A_1\times\dots\times A_n$, декартовото произведение на множествата $A_1,\dots,A_n$;
\item 
  $\pair{a_1,\dots,a_n} \sqsubseteq \pair{b_1,\dots,b_n} \dfff a_1 \sqsubseteq_1 b_1\ \&\ \dots\ \&\ a_n \sqsubseteq_n b_n$,
  която ще наричаме {\em поточкова наредба};
\item
  $\bot \df \pair{\bot_1,\dots,\bot_n}$.
\end{itemize}

\begin{framed}
  \begin{proposition}
    \label{pr:cartesian}
    Ако $\A_1,\dots,\A_n$ са области на Скот, то $\prod^n_{i=1}\A_i$ е област на Скот.
  \end{proposition}  
\end{framed}
\begin{hint}
  Лесно се съобразява, че $\sqsubseteq$ е частична наредба и че $\bot$ е най-малкият елемент.
  Да разгледаме една верига $\chain{\ov{a}}{i}$ в $\prod^n_{i=1}\A_i$,
  където $\bar{a}_i = \pair{a^i_1,\dots,a^i_n}$.
  Да положим \[\bar{a} \df \pair{\bigsqcup_i a^i_1,\bigsqcup_i a^i_2,\ldots,\bigsqcup_i a^i_n,}.\]
  Ще докажем, че $\bar{a}$ е точната горна граница на верига $\chain{\ov{a}}{i}$, т.е.
  \[\bigsqcup_i\bar{a}_i = \pair{\bigsqcup_i a^i_1,\bigsqcup_i a^i_2,\ldots,\bigsqcup_i a^i_n,}.\]
  Това ще направим на две стъпки.
  \begin{itemize}
  \item 
    Първо ще докажем, че $\bar{a}$ е горна граница на $\chain{\bar{a}}{i}$.
    За всеки индекс $j$, имаме, че $\bar{a}_j = (a^j_1,\dots,a^j_n)$.
    Да разгледаме $1 \leq k \leq n$.
    Ясно е, че $a^j_k \sqsubseteq_k \bigsqcup_i a^i_k$.
    Оттук веднага следва, че
    \[\bar{a}_j = \pair{a^j_1,\dots,a^j_n} \sqsubseteq \pair{\bigsqcup_i a^i_1,\dots,\bigsqcup_i a^i_n} = \bar{a}.\]
  \item
    Нека $\bar{b} = \pair{b_1,\dots,b_n}$ е произволна горна граница на $\chain{\bar{a}}{i}$.
    Ще докажем, че $\bar{a} \sqsubseteq \bar{b}$.
    Знаем, че за всеки индекс $j$, $\bar{a}_j \sqsubseteq \bar{b}$.
    Да разгледаме произволен индекс $k$, за който $1 \leq k \leq n$.
    Знаем, че в областта на Скот $\A_k$ е изпълнено, че $a^j_k \sqsubseteq_k b_k$, за всеки индекс $j$.
    Оттук следва, че $b_k$ е горна граница за веригата $(a^j_k)^\infty_{j=0}$ в $\A_k$.
    Следователно
    \[\bigsqcup_j a^k_j \sqsubseteq_k b_k.\]
    Заключаваме, че
    \[\bar{a} = \pair{\bigsqcup_i a^i_1,\dots,\bigsqcup_i a^i_n} \sqsubseteq \pair{b_1,\dots,b_n} = \bar{b}.\]
  \end{itemize}
\end{hint}

\begin{framed}
  \begin{figure}[H]
    \centering
    \begin{tikzpicture}[shorten >=1pt,->]
      \tikzstyle{vertex}=[circle,minimum size=17pt,inner sep=0pt]
      
      \node[vertex] (bot) at (5,-1) {$\pair{\bot,\bot}$};
      
      \node[vertex] (0b) at (0,1) {$\pair{0,\bot}$};
      \node[vertex] (db) at (1.25,1) {$\cdots$};
      \node[vertex] (nb) at (2.5,1) {$\pair{n,\bot}$};
      \node[vertex] (ddb) at (4,1) {$\cdots$};
      
      \node[vertex] (b0) at (6,1) {$\pair{\bot,0}$};
      \node[vertex] (bd) at (7.25,1) {$\cdots$};
      \node[vertex] (bn) at (8.5,1) {$\pair{\bot,n}$};
      \node[vertex] (bdd) at (9.5,1) {$\cdots$};
      
      \node[vertex] (00) at (-1,3) {$\pair{0,0}$};
      \node[vertex] (01) at (0.25,3) {$\pair{0,1}$};
      \node[vertex] (10) at (1.5,3) {$\pair{1,0}$};
      \node[vertex] (ddd) at (2.75,3) {$\cdots$};
      \node[vertex] (0n) at (4,3) {$\pair{0,n}$};
      \node[vertex] (dddd) at (5.25,3) {$\cdots$};
      \node[vertex] (n0) at (6.5,3) {$\pair{n,0}$};
      \node[vertex] (ddddd) at (7.75,3) {$\cdots$};
      \node[vertex] (nn) at (9,3) {$\pair{n,n}$};
      \node[vertex] (dddddd) at (10,3) {$\cdots$};

      \draw (bot) -- node[below left]{$\scriptstyle{\sqsupseteq}$} (0b);
      \draw (bot) -- (nb);
      \draw (bot) -- (b0);
      \draw (bot) -- (bn);

      \draw (0b) -- node[below left]{$\scriptstyle{\sqsupseteq}$} (00);
      \draw (0b) -- (01);
      \draw (0b) -- (0n);

      \draw (b0) -- (00);
      \draw (b0) -- (10);
      \draw (b0) -- (n0);
      \draw (nb) -- (n0);
      
      \draw (nb) -- (nn);
      \draw (bn) -- (0n);
      \draw (bn) -- (nn);

    \end{tikzpicture}    
    \caption{Графично представяне на част от $\sqsubseteq$ върху $\Nat^2_\bot$}
    \label{fig:flat-nat-2}
  \end{figure}
\end{framed}

Вижда се от \Fig{flat-nat-2}, че всяка верига в $\Nat^2_\bot$ има дължина най-много $3$.
Лесно се съобразява, че всяка верига в $\Nat^k_\bot$ има дължина най-много $k+1$.
Свойството, че всяка верига в $\Nat^k_\bot$ има само краен брой различни члена
ще се окаже важно по-нататък. Сега ще въведем понятие, което описва това свойство в произволна област на Скот.
\marginpar{ $\Nat^k_\bot = \underbrace{\Nat_\bot \times \cdots \times\Nat_\bot}_{k}$}

Нека $\A$ е област на Скот и да разгледаме една верига $\chain{a}{n}$ в $\A$.
Ще казваме, че $\chain{a}{n}$ се {\bf стабилизира}, ако съществува индекс $n_0$, за който
\[(\forall n \geq n_0)[a_{n_0} = a_{n}],\]
т.е.
\[a_0 \sqsubseteq a_1 \sqsubseteq a_2 \sqsubseteq \cdots \sqsubseteq a_{n_0} = a_{n_0+1} = a_{n_0+2} = \cdots\]
От казаното по-горе следва, че всяка растяща верига в $\Nat^k_\bot$ се стабилизира.

\begin{remark}
  Едни от основните области на Скот, които ще разглеждаме при дефинирането на денотационната семантика
  ще бъдат $\Nat_\bot$ и $\Nat^k_\bot$.
\end{remark}


%%% Local Variables:
%%% mode: latex
%%% TeX-master: "../sep"
%%% End:


\section{Изображения в области на Скот}

\index{област на Скот!изображения}
Нека $\A_i = (A_i,\ \sqsubseteq_i,\ \bot_i)$, за $i = 1,2$, са области на Скот.
Ще въведем няколко основни вида изображения между $\A_1$ и $\A_2$, 
които ще използваме често. След това ще разгледаме някои свойства на тези изображения
и ще видим каква е връзката между тях.
\begin{itemize}
\item
  Всяка тотална функция от вида $f:A_1 \to A_2$ ще наричаме изображение между областите на Скот $\A_1$ и $\A_2$
  и ще записваме $f:\A_1 \to \A_2$.
  Да въведем означението 
  \[\Mapping{\A_1}{\A_2} \dff (\{f \mid f:\A_1 \to \A_2\},\ \sqsubseteq,\ \bm{\bot}),\]
  където имаме следната релация между изображенията $f,g:\A_1 \to \A_2$:
  \[f \sqsubseteq g \dfff (\forall a \in \A_1)[f(a) \sqsubseteq_2 g(a)].\]
  Също така, изображението $\bm{\bot}:\A_1 \to \A_2$ е дефинирано като
  \[(\forall a \in \A_1)[\bm{\bot}(a) = \bot_2].\]
\end{itemize}
На хаскел можем да дефинираме изображението $\bm{\bot}$ по следния начин:
\begin{haskellcode}
ghci> let bottom _ = undefined
\end{haskellcode}

\begin{framed}
  \begin{theorem}
    \label{th:all-mappings-is-domain}
    $\Mapping{\A_1}{\A_2}$ е област на Скот.
  \end{theorem}  
\end{framed}
\begin{proof}
  Нетривиалната част в доказателството е да проверим, че всяка верига $(f_i)^{\infty}_{i=0}$ в $\Mapping{\A_1}{\A_2}$
  притежава точна горна граница.
  \marginpar{За по-кратко пишем $\bar{a}$ вместо $a_1,\dots,a_n$}
  Да разгледаме изображението $h:\A_1 \to \A_2$, където:
  \begin{equation}
    \label{eq:9}
    h(a) \dff \bigsqcup \{f_i(a) \mid i \in \Nat\}.
  \end{equation}
  Ще докажем, че $h$ е тази точна горна граница.
  \begin{itemize}
  \item
    \marginpar{Това задължително трябва да се провери,
      защото например множеството $\{\bot, 0, 3\}$ няма точна горна граница относно плоската наредба в $\Nat_\bot$}
    Първо, трябва да се убедим, че дефиницията на $h$ е ,,смислена'', т.е. $h$ е тотална функция.
    Трябва да докажем, че за всяко $a\in \A_1$,
    \[\bigsqcup\{f_i(a) \mid i \in \Nat\}\] съществува.
    Да фиксираме произволен елемент $a \in \A_1$.
    Получаваме следната верига в $\A_2$:
    \[f_0(a) \sqsubseteq f_1(a) \sqsubseteq f_2(a) \sqsubseteq \cdots \]
    Понеже $\A_2$ е област на Скот, то тази верига притежава точна горна граница в $\A_2$,
    която означачаваме като $\bigsqcup\{f_i(a) \mid i \in \Nat\}$.
    Това означава, че $h(a)$ е тотална функция.
  \item
    Дотук имаме, че $h \in \Mapping{\A_1}{\A_2}$.
    Лесно се съобразява, че $h$ е горна граница на веригата $\chain{f}{i}$, защото за всяки елемент $a \in \A_1$
    и произволен индекс $i$,
    \[f_i(a) \sqsubseteq \bigsqcup\{f_i(a) | i \in \Nat\} \dff h(a).\]
  \item
    Сега остава да проверим, че $h$ е точна горна граница, т.е. $h$ е най-малката измежду всички горни граници на 
    веригата $\chain{f}{i}$.
    Нека $g$ е друга горна граница на $\chain{f}{i}$. Това означава, че за всеки индекс $i$,
    $f_i \sqsubseteq g$. Следователно, за фиксирано $a \in \A_1$,
    $g(a)$ е горна граница за веригата $(f_i(a))^{\infty}_{i=0}$.
    Тогава е ясно, че за разглеждания елемент $a$,
    \[h(a) \dff \bigsqcup\{f_i(a) \mid i\in\Nat\} \sqsubseteq g(a).\]
    Понеже елементът $a$ е прозиволен, получаваме, че $h \sqsubseteq g$.
  \item
    Доказахме, че $h$ е горна граница и че $h$ е най-малката измежду всички горни граници.
    Заключваме, че $h$ е {\em точна горна граница} на веригата $\chain{f}{i}$.
    \marginpar{Получаваме, че \[(\bigsqcup_if_i)(a) = \bigsqcup_i\{f_i(a)\}.\]}
    С други думи,
    \[h = \bigsqcup_i f_i.\]
  \end{itemize}
\end{proof}

\begin{corollary}
  $\Mapping{\Nat^n_\bot}{\Nat_\bot}$ е област на Скот.
\end{corollary}


%%% Local Variables:
%%% mode: latex
%%% TeX-master: "../sep"
%%% End:


\subsection{Монотонни изображения}

\index{изображение!монотонно}
Да разгледаме областите на Скот $\A_1 = (A_1,\ \sqsubseteq_1,\ \bot_1)$ и $\A_2 =(A_2,\ \sqsubseteq_2,\ \bot_2)$.
Едно изображение $f:\A_1\to \A_2$ се нарича {\bf монотонно}, ако
\[(\forall a,a'\in\A_1)[a \sqsubseteq_1 a' \implies f(a) \sqsubseteq_2 f(a')].\]
Да въведем означението
\[\Mon{\A_1}{\A_2} \dff (\{f: \A_1 \to \A_2 \mid f\text{ е мон. изобр.}\},\ \sqsubseteq,\ \bm{\bot}).\]

\begin{framed}
  \begin{theorem}\label{th:monotone-is-domain}
    $\Mon{\A_1}{\A_2}$ е област на Скот.
  \end{theorem}  
\end{framed}
\begin{hint}
  Да фиксираме една верига $\chain{f}{i}$ в $\Mon{\A_1}{\A_2}$. Трябва да докажем, че тази верига притежава точна горна граница,
  която е монотонно изображение.
  Да разгледаме същото изображение $h:\A_1 \to \A_2$ както в доказателството на \Th{all-mappings-is-domain}, като
  \[h(a) \dff \bigsqcup \{f_i(a) \mid i \in \Nat\}.\]
  Оттам знаем, че $h$ е точна горна граница на веригата. 
  Остава да докажем, че $h \in \Mon{\A_1}{\A_2}$.
  Нека $a \sqsubseteq b$. Тогава, за всеки индекс $k$, понеже $f_k$ са монотонни изображения, получаваме следното:
  \begin{align*}
    f_k(a) & \sqsubseteq f_k(b)\\
           & \sqsubseteq \bigsqcup \{f_i(b) \mid i \in \Nat\} \dff h(b).
  \end{align*}
  Това означава, че $h(b)$ е горна граница за веригата ${(f_i(a))}^{\infty}_{i=0}$.
  Заключаваме, че 
  \begin{align*}
    h(a) & \dff \bigsqcup \{f_i(a) \mid i \in \Nat\}\\
         & \sqsubseteq \bigsqcup \{f_i(b) \mid i \in \Nat\} \dff h(b).    
  \end{align*}
\end{hint}

\begin{corollary}\label{cr:flat-monotone-is-domain}
  $\Mon{\Nat^n_\bot}{\Nat_\bot}$ е област на Скот.
\end{corollary}

Удобно е да имаме следващото свойство на монотонните изображения като отделно твърдение за да можем да го използваме наготово по-късно.
\begin{proposition}\label{pr:monotone-chain}
  Нека $f \in \Mon{\A}{\B}$ и $\chain{a}{i}$ е верига от елементи на $\A$.
  Тогава
  \[\bigsqcup_i f(a_i) \sqsubseteq f(\bigsqcup_i a_i).\]
\end{proposition}
\begin{proof}
  Понеже $f$ е монотонно, то ${(f(a_i))}^{\infty}_{i=0}$ е верига от елементи на $\B$
  и тя има точна горна граница.
  Достатъчно е да докажем, че $f(\bigsqcup_i a_i)$ е горна граница на веригата ${(f(a_i))}^{\infty}_{i=0}$.
  Но това е лесно.
  Понеже $a_i \sqsubseteq \bigsqcup_i a_i$ и $f$ е монотонно, то веднага получаваме, че
  $f(a_i) \sqsubseteq f(\bigsqcup_i a_i)$.
  Заключаваме, че $\bigsqcup_i f(a_i) \sqsubseteq f(\bigsqcup_i a_i)$.
\end{proof}

%%% Local Variables:
%%% mode: latex
%%% TeX-master: "../sep"
%%% End:


\subsection{Непрекъснати изображения}\index{изображение!непрекъснато}

\marginpar{На англ. {\em continuous}}
Едно изображение $f:\A_1\to \A_2$ се нарича {\bf непрекъснато}, ако са изпълнени свойствата:
\begin{itemize}
\item
  $f$ е монотонно изображение;
\item
  при всеки избор на верига $\chain{a}{i}$ в $\A_1$, то имаме равенството
  \marginpar{Понеже $\A_1$ е област на Скот знаем, че $\bigsqcup_i a_i \in \A_1$}
  \marginpar{Понеже $f$ е монотонно, то ${(f(a_i))}^{\infty}_{i=0}$ е верига.}
  \marginpar{$\bigsqcup_i f(a_i) \dff \bigsqcup\{f(a_i) \mid i \in \Nat\}$}
  \[f(\bigsqcup_i a_i) = \bigsqcup \{f(a_i) \mid i \in \Nat\}.\]  
\end{itemize}
Да означим
\[\Cont{\A_1}{\A_2} \dff (\{f: \A_1 \to \A_2 \mid f\text{ е непр. изобр.}\},\ \sqsubseteq,\ \bm{\bot}).\]

% \begin{framed}
%   \begin{prop}
%     \label{pr:continuous-is-monotone}
%     За произволни области на Скот $\A_1$ и $\A_2$, всяко непрекъснато изображение $f:\A_1 \to \A_2$ е монотонно, т.е.
%     \[\Cont{\A_1}{\A_2}\ \subseteq\ \Mon{\A_1}{\A_2}.\]
%   \end{prop}
% \end{framed}
% \begin{proof}
%   Нека $f \in \Cont{\A_1}{\A_2}$.
%   Да вземем два произволни елемента в $\A_1$, за които $a \sqsubseteq_1 b$.
%   Ще докажем, че $f(a) \sqsubseteq_2 f(b)$.
%   Да разгледаме веригата $\chain{a}{i}$ в $\A_1$, където:
%   \[\underbrace{a_0}_{a} \sqsubseteq \underbrace{a_1}_{b} = \underbrace{a_2}_{b} = \underbrace{a_3}_{b} = \cdots\]
%   Ясно е, че 
%   \[\bigsqcup\{a_i \mid i \in \Nat\} = \bigsqcup\{a,b\} = b.\]
%   Тогава от непрекъснатостта на $f$ имаме, че
%   \begin{align*}
%     f(a) & = f(a_0) & \comment{\text{защото }a_0 \dff a}\\
%     & \sqsubseteq_2 \bigsqcup\{f(a_i) \mid i \in \Nat\} & \comment{\text{защото }f(a_0) \in \{f(a_i) \mid i\in\Nat\}}\\
%     & = f(\bigsqcup_i a_i) & \comment{\text{защото $f$ е непр.}}\\
%     & = f(\bigsqcup\{a,b,b,b,\dots\}) & \comment{\text{от избора на веригата }\chain{a}{i}}\\
%     & = f(b) & \comment{\text{защото }a \sqsubseteq_1 b}.
%   \end{align*}
%   Така получихме, че за произволни $a,b\in\A_1$, 
%   \[a \sqsubseteq_1 b \implies f(a) \sqsubseteq_2 f(b).\]
% \end{proof}

Ясно е, че всяко непрекъснато изображение е монотонно.
Ествено е да си зададем въпроса дали имаме и обратното включване.
Оказва се, че в общия случай не е вярно, че всяко монотонно изображение е непрекъснато.

\begin{proposition}
  Съществува област на Скот $\A$, за която
  \[\Cont{\A}{\A} \subsetneqq \Mon{\A}{\A}.\]
\end{proposition}
\begin{hint}
  Нека $A = \{a_n \mid n \in \Nat\} \cup \{a_\omega, b_0\}$.
  Да разгледаме областта на Скот $\A = (A, \sqsubseteq, a_0)$, където 
  наредбата между елементите е следната:
  \[a_0 \sqsubseteq a_1 \sqsubseteq \cdots \sqsubseteq a_n \sqsubseteq \cdots \sqsubseteq a_\omega \sqsubseteq b_0. \]
  Нека $f(a_n) = a_{n+1}$, $f(a_{\omega}) = b_0$ и $f(b_0) = b_0$.
  Очевидно е, че $f$ е монотонно изображение.
  Лесно се вижда, че $f$ не е непрекъснато изображение, 
  защото
  \[f(\bigsqcup_n a_n) = f(a_\omega) = b_0,\]
  но 
  \[\bigsqcup_n f(a_n) = \bigsqcup_n a_{n+1} = a_\omega.\]
\end{hint}

Сега да видим един важен за нас случай, при който имаме и обратното включване.

\begin{framed}
  \begin{proposition}\label{pr:stab-continuous}
    Ако всяка верига в $\A_1$ се {\em стабилизира}, то
    \[\Mon{\A_1}{\A_2} \subseteq \Cont{\A_1}{\A_2}.\]
  \end{proposition}
\end{framed}
\begin{hint}
  Да разгледаме една верига $\chain{a}{i}$ в $\A_1$ и $f \in \Mon{\A_1}{\A_2}$.
  Ще докажем, че \[f(\bigsqcup_i a_i) = \bigsqcup_i f(a_i).\]

  \begin{enumerate}[(1)]
  \item
    % \marginpar{Това включване е вярно за произволна област на Скот $\A_1$.}
    % Ясно е, че за всяко монотонно изображение $f$,
    % понеже $a_i \sqsubseteq \bigsqcup_i a_i$, то $f(a_i) \sqsubseteq f(\bigsqcup_i a_i)$.
    % Това означава, че $f(\bigsqcup_i a_i)$ е горна граница на веригата $(f(a_i))^\infty_{i=0}$ в $\A_2$
    % и следователно
    % \[\bigsqcup_i f(a_i) \sqsubseteq f(\bigsqcup_i a_i).\]
    От \Prop{monotone-chain} веднага получаваме, че
    \[\bigsqcup_i f(a_i) \sqsubseteq f(\bigsqcup_i a_i).\]
  \item
    За другата посока ще използваме свойството, че веригата $\chain{a}{i}$ се стабилизира.
    Нека $n_0$ е индекс, такъв че $(\forall k \geq n_0)[a_k = a_{n_0}]$.
    Това означава, че $\bigsqcup_i a_i = a_{n_0}$.
    Тогава
    \[f(\bigsqcup_i a_i) = f(a_{n_0}) \sqsubseteq \bigsqcup_i f(a_i).\]
  \end{enumerate}
  
  От $(1)$ и $(2)$ следва, че $f(\bigsqcup_i a_i) = \bigsqcup_i f(a_i)$.
\end{hint}

Понеже всяка верига в $\Nat^n_\bot$ се {\em стабилизира}, то
получаваме следното важно следствие.
\begin{framed}
\begin{corollary}\label{cr:monotone-is-continuous}
  $\Mon{\Nat^n_\bot}{\Nat_\bot} = \Cont{\Nat^n_\bot}{\Nat_\bot}$.
\end{corollary}  
\end{framed}

% От доказателството на (1) за \Prop{stab-continuous} можем да извлечем следното свойство,
% което ще ни бъде полезно по-нататък.
% \begin{proposition}\label{pr:monotone-chain}
%   За всяко изображение $f \in \Mon{\A}{\B}$ и всяка верига $\chain{a}{i}$, е изпълнено, че
%   \[\bigsqcup_i f(a_i) \sqsubseteq f(\bigsqcup_i a_i).\]
% \end{proposition}

Понеже от \Cor{flat-monotone-is-domain} имаме, че монотонните изображения образуват област на Скот, 
то директно получаваме следната важна теорема.

\begin{framed}
\begin{theorem}
  \label{th:continuous-is-domain}
  $\Cont{\Nat^n_\bot}{\Nat_\bot}$ е област на Скот.
\end{theorem}
\end{framed}
% \begin{proof}
%   От \Cor{monotone-is-continuous} имаме, че 
%   \[\Mon{\Nat^n_\bot}{\Nat_\bot} = \Cont{\Nat^n_\bot}{\Nat_\bot}.\]
%   От \Cor{flat-monotone-is-domain} имаме, че 
%   \[(\Mon{\Nat^n_\bot}{\Nat_\bot}, \sqsubseteq, \Omega^{(n)})\]
%   е област на Скот. 
%   Оттук директно получаваме, че 
%   \[(\Cont{\Nat^n_\bot}{\Nat_\bot}, \sqsubseteq, \Omega^{(n)})\] е област на Скот.
% \end{proof}


% \begin{prop}
%   \label{pr:composition}
%   \index{изображения!композиция}
%   Ако $f \in \Cont{\A}{B}$ и $g \in \Cont{\B}{\C}$, то $g \circ f \in \Cont{\A}{\C}$,
%   където \[(g\circ f)(a) \dff g(f(a)).\]
% \end{prop}
% \begin{hint}
%   Нека $\chain{a}{i}$ е верига в $\A$.
%   Да обърнем внимание, че понеже $f \in \Cont{\A}{\B}$,
%   то $f$ е монотонно изображение и тогава $(f(a_i))^\infty_{i=0}$ е верига в $\B$.
%   Тогава:
%   \begin{align*}
%     (g \circ f)(\bigsqcup_i a_i) & = g(f(\bigsqcup_i a_i)) & \comment{\text{от деф.}}\\
%     & = g(\bigsqcup_i f(a_i)) & \comment{f \text{ е непр.}}\\
%     & = \bigsqcup_i g(f(a_i)) & \comment{g \text{ е непр.}}
%   \end{align*}
% \end{hint}

%%% Local Variables:
%%% mode: latex
%%% TeX-master: "../sep"
%%% End:


% \subsection{Точни изображения}

\marginpar{На англ. се нарича {\em strict}. ,,Стандартната'' семантика на хаскел е {\em non-strict}}
\index{изображение!точно}
Едно изображение $f:\A^n_1 \to \A_2$ се нарича {\bf точно}, ако 
\[(\forall \bar{a}\in\A^n_1)[\bot_1\in\{a_1,\dots,a_n\} \implies f(a_1,\dots,a_n) = \bot_2].\]
% Тук ще разглеждаме съвкупността от точни изображения:
% \[S_n \dff \{f: \Nat^n_\bot \to \Nat_\bot \mid f\text{ е точна}\}.\]
За произволни области на Скот $\A_1$ и $\A_2$, ще означаваме съвкупността от
точните изображения като $\Strict{\A_1}{\A_2}$.

\begin{example}
  Да разгледаме свойствата на няколко прости изображения.
  \begin{itemize}
  \item 
    Изображението $f:\Nat_\bot \to \Nat_\bot$, дефинирано като 
    \[f(x) = 42,\]
    за всяко $x \in \Nat_\bot$,
    {\bf не е точно}, защото $f(\bot) = 42$. Лесно се вижда, че $f$ е монотонно и непрекъснато изображение.
  \item
    От друга страна, изображението $g:\Nat_\bot \to \Nat_\bot$, дефинирано като 
    \[g(x) = \bot,\]
    за всяко $x \in \Nat_\bot$, {\bf е точно}, защото $g(\bot) = \bot$.
    Освен това, $g$ е монотонно и непрекъснато изображение.
  \end{itemize}
\end{example}

Видяхме, че лесно се намират монотонни изображения, които не са точни.
Сега ще разгледаме обратната посока.

\begin{proposition}
  \label{pr:strict-is-monotone}
  Всяко точно изображение $f:\Nat^n_\bot \to \Nat_\bot$ е също така и монотонно.
  С други думи, 
  \[\Strict{\Nat^n_\bot}{\Nat_\bot} \subseteq \Mon{\Nat^n_\bot}{\Nat_\bot}.\]
\end{proposition}
\begin{proof}
  Нека $f$ е точно и $\bar{a} \sqsubseteq \bar{b}$.
  Ще проверим, че 
  \[f(\bar{a}) = c \sqsubseteq d = f(\bar{b}).\]
  \begin{itemize}
  \item 
    Ако $c = \bot$, то е очевидно, че $c \sqsubseteq d$.
  \item
    Интересният случай е когато $c \neq \bot$. Тогава трябва да видим, че $c = d$.
    Понеже $f$ е точна и $c \neq \bot$, това означава, че $\bot \not\in\{a_1,\dots,a_n\}$.
    Но понеже $\sqsubseteq$ е плоска наредба, и $\bar{a} \sqsubseteq \bar{b}$, това означава, че $\bar{a} = \bar{b}$.
    Следотвателно, наистина $c = d$.
  \end{itemize}
\end{proof}

\begin{framed}
  \begin{theorem}
    \label{th:strict-is-domain}
    $(\Strict{\Nat^n_\bot}{\Nat_\bot},\ \sqsubseteq,\ \bm{\bot}^{(n)})$ е област на Скот.
  \end{theorem}
\end{framed}
\begin{hint}
  Да разгледаме веригата $\chain{f}{i}$ от елементи на $\Strict{\Nat^n_\bot}{\Nat_\bot}$.
  Трябва да докажем, че тази верига притежава точна горна граница, която също е точно изображение.
  % Понеже от \Prop{strict-is-monotone} имаме, че всяка точна функция е монотонна, то наготово от 
  От доказателството на \Th{all-mappings-is-domain} знаем, че изображението $h$ дефинирано за всяко $\bar{a} \in \Nat^n_\bot$ като
  \[h(\bar{a}) \dff \bigsqcup \{f_i(\bar{a}) \mid i \in \Nat\}\]
  е точна горна граница на веригата $\chain{f}{i}$.
  Остава да проверим, че $h$ е точно изображение.
  Нека $\bar{a} \in \Nat^n_\bot$ и да приемем, че $\bot \in \{a_1,\dots,a_n\}$.
  Тогава:
  \begin{align*}
    h(\bar{a}) & = \bigsqcup \{f_i(\bar{a}) \mid i \in \Nat\} & \comment{\text{от деф. на }h}\\
               & = \bigsqcup\{\bot,\bot,\dots,\bot,\dots\} & \comment{f_i\text{ са точни}}\\
               & = \bot.
  \end{align*}
\end{hint}

\begin{example}
  \label{ex:simple-non-continuous}
  Нека сега да разгледаме $h:\Nat_\bot \to \Nat_\bot$, където 
  \begin{align*}
    h(x) = &
    \begin{cases}
      0, & \text{ако }x = \bot,\\
      1, & \text{иначе}.
    \end{cases}
  \end{align*}
  Да видим колко ,,лошо'' изображение е $h$:
  \begin{itemize}
  \item 
    $h$ не е точно, защото $h(\bot) \neq \bot$;
  \item
    $h$ не е монотонно, защото $\bot \sqsubseteq 5$, но $h(\bot) = 0 \not\sqsubseteq 1 = h(5)$;
  \item
    $h$ не е непрекъснато, защото $\Cont{\Nat_\bot}{\Nat_\bot} = \Mon{\Nat_\bot}{\Nat_\bot}$.
  \end{itemize}
  Нека да видим, че хаскел не позволява такива ,,лоши'' функции:

  \begin{haskellcode}
ghci> let h(x) = if x == undefined then 0 else 1
ghci> h(5)
*** Exception: Prelude.undefined
ghci> h(undefined)
*** Exception: Prelude.undefined
  \end{haskellcode}
\end{example}

\begin{example}
  \label{ex:non-strict-monotone}
  Да разгледаме $f:\Nat^2_\bot \to \Nat_\bot$ дефинирано като \[f(x,y) = x.\]
  \begin{itemize}
  \item 
    $f$ не е точно, защото $f(0,\bot) = 0$.
  \item
    $f$ е монотонно, защото ако $\pair{x,y} \sqsubseteq \pair{x',y'}$, то $x \sqsubseteq x'$ и
    \[f(x,y) = x \sqsubseteq x' = f(x',y').\]
  \item
    $f$ е непрекъснато, защото от \Cor{monotone-is-continuous} знаем, че 
    $\Mon{\Nat^2_\bot}{\Nat_\bot} = \Cont{\Nat^2_\bot}{\Nat_\bot}$.
  \end{itemize}
\end{example}

Можем да обобщим всичко, което направихме дотук, със следната илюстрация на основните области на Скот, с които ще работим
в \Chapter{rec}.

\begin{framed}
    \begin{align*}
      \Strict{\Nat^n_\bot}{\Nat_\bot} & \subsetneqq \Mon{\Nat^n_\bot}{\Nat_\bot} & \comment{\text{от \Ex{non-strict-monotone} и \Prop{strict-is-monotone}}}\\
      & = \Cont{\Nat^n_\bot}{\Nat_\bot} & \comment{\text{от \Prop{continuous-is-monotone} и \Cor{monotone-is-continuous}}}\\
      & \subsetneqq \Mapping{\Nat^n_\bot}{\Nat_\bot} & \comment{\text{от \Ex{simple-non-continuous}}}.
    \end{align*}
\end{framed}

\begin{example}
  \label{ex:plus}
  Да разгледаме $f:\Nat^2_\bot \to \Nat_\bot$ дефинирана по следния начин:
  \[f(x,y) = 
  \begin{cases}
    \bot, & x = \bot\ \&\ y = \bot\\
    x, & x \neq \bot\ \&\ y = \bot\\
    y, & x = \bot\ \&\ y \neq \bot\\
    x+y, & x \neq \bot\ \&\ y \neq \bot.
  \end{cases}\]
  Изображението $f$ не е точно, защото например $f(\bot,0) \neq \bot$.
  $f$ не е монотонно изображение, защото $\pair{2,\bot} \sqsubseteq \pair{2,3}$, но $2 = f(2,\bot) \not\sqsubseteq f(2,3) = 5$.
  Също така, $f$ не е непрекъснато изображение, защото $\Mon{\Nat^2_\bot}{\Nat_\bot} = \Cont{\Nat^2_\bot}{\Nat_\bot}$.
\end{example}
% \begin{hint}
%   \begin{itemize}
%   \item 
%     $f$ не е точно изображение, защото например
%     $f(\bot,0) \neq \bot$.
%   \item
%     $f$ не е монотонно. Например,
%     \[\pair{\bot,0} \sqsubseteq \pair{1,0}\ \&\ f(\bot,0) = 0 \not\sqsubseteq 1 = f(1,0).\]    
%   \item
%     $f$ не е непрекъснато, защото $\Cont{\Nat^2_\bot}{\Nat_\bot} = \Mon{\Nat^2_\bot}{\Nat_\bot}$.
%   \end{itemize}    
% \end{hint}


% \subsection{Точни продължения}
% \index{изображение!точно продължение}

% Нека $A$ и $B$ са произволни множества, а $\A_\bot$ и $\B_\bot$ са плоските области на Скот получени съответно от $A$ и $B$.
% За еднa частична функция $f:A^n \to B$, определяме точното изображение $f^\star:\A^n_\bot \to \B_\bot$ по следния начин:
% \marginpar{$f^\star \in \Strict{\A^n_\bot}{\B_\bot}$}
% \begin{align*}
%   f^\star(\ov{a}) \dff
%   \begin{cases}
%     f(\ov{a}), & \text{ако }\bot\not\in\{a_1,\dots,a_n\}\ \&\ f(\ov{a})\text{ е деф.}\\
%     \bot, & \text{иначе}.
%   \end{cases}
% \end{align*}
% Изображението $f^\star$ се нарича {\bf точно продължение} на $f$.

% \begin{example}
%   Да разгледаме частичната функция $f:\Nat^2 \to \Nat$, дефинирана като
%   $f(x,y) = x+y$. Тогава точното продължение на $f$ е 
%   \[f^\star(x,y) = 
%   \begin{cases}
%     x+y, & x,y\in\Nat\\
%     \bot, & x = \bot \vee y = \bot.
%   \end{cases}\]
% \end{example}

% \Stefan{За какво ми е точното продължение да е непрекъснато?}
% \noindent Да дефинираме оператор $\Sigma^\star : \Partial{\Nat^n}{\Nat} \to \Strict{\Nat^n_\bot}{\Nat_\bot}$ като
% \[\Sigma^\star(f) = f^\star.\] 
% Ще наричаме $\Sigma^\star$ {\bf продължаващ} оператор, защото на всяка частична функция дава нейното точно продължение.

% \begin{framed}
%   \begin{prop}
%     $\Sigma^\star$ е непрекъснато изображение, т.е.
%     \[\Sigma^\star \in \Cont{\Partial{\Nat^n}{\Nat}}{\Strict{\Nat^n_\bot}{\Nat_\bot}}.\]
%   \end{prop}
% \end{framed}
% \begin{hint}
%   Трябва да докажем, че за произволна верига $(f_i)^{\infty}_{i=0}$ от частични функции, 
%   $\Sigma^\star(\bigcup_i f_i) = \bigsqcup_i \Sigma^\star(f_i)$.
%   \begin{itemize}
%   \item 
%     Лесно се съобразява, че ако $f \subseteq g$, то $\Sigma^\star(f) \sqsubseteq \Sigma^\star(g)$.
%     Оттук следва, че 
%     \[\bigsqcup_i \Sigma^\star(f_i) \sqsubseteq \Sigma^\star(\bigcup_i f_i).\]
%   \item
%     За другата посока,
%     Нека $\Sigma^\star(\bigcup_i f_i)(\ov{x}) = y \neq \bot$.
%     Това означава, че $\ov{x} \in \Nat^n$ и $(\bigcup_i f_i)(\ov{x}) \simeq y$.
%     Оттук следва, че съществува $i_0$, за което $f_{i_0}(\ov{x}) \simeq y$.
%     Тогава $\Sigma^\star(f_{i_0})(\ov{x}) = y$.
%     Понеже $y \neq \bot$, 
%     \[\bigsqcup_i (\Sigma^\star(f_i)(\ov{x})) = (\bigsqcup_i \Sigma^\star(f_i))(\ov{x}) = y.\]
%     Заключаваме, че $\Sigma^\star(\bigcup_i f_i)(\ov{x}) \sqsubseteq \bigsqcup_i (\Sigma^\star(f_i)(\ov{x}))$
%   \end{itemize}
% \end{hint}

% \Stefan{Това означава, че някъде трябва да се каже, че $\F_n$ е област на Скот. Някъде в тази глава трябва да е. Това може да бъде един от първите примери след дефиницията на О.С.}



%%% Local Variables:
%%% mode: latex
%%% TeX-master: "../sep"
%%% End:


% \subsection{Точни ограничения}
\index{изображение!точно ограничение}

Нека фиксираме две области на Скот $\A$ и $\B$.
За едно изображение $f \in \Mapping{\A^n}{\B}$, дефинираме точното изображение $f_\star \in \Strict{\A^n}{\B}$ по следния начин:
\[f_\star(\bar{a}) =
\begin{cases}
  f(\bar{a}), & \text{ако } \bot\not\in\{a_1,\dots,a_n\}\\
  \bot, & \text{ако }\bot\in\{a_1,\dots,a_n\}.
\end{cases}\]
Ще казваме, че $f_\star$ е {\bf точно ограничение} на $f$.

\begin{framed}
  \begin{prop}
    За всяко $f \in \Mapping{\A^n}{\B}$, имаме 
    \[f_\star \sqsubseteq f.\]
  \end{prop}
\end{framed}

\begin{example}
  Точното ограничение на функцията $f$ от \Ex{plus} е
  \[f_\star(x, y) =
  \begin{cases}
    x + y, & \text{ако } x \neq \bot\ \&\ y \neq \bot\\
    \bot, & \text{иначе}
  \end{cases}
\]
\end{example}

\begin{example}
  Да видим какъв резултат връща хаскел при следните извиквания:
  
  \begin{haskellcode}
ghci> (\x -> 5) 4
5
ghci> (\x -> 5) undefined
5
  \end{haskellcode}
  
  От второто извикване се вижда, че в хаскел, анонимната функция $\lambda x.5$ е дефинирана и върху $\bot$ и връща отново $5$.
  \marginpar{$(\lambda x.5)(4) = 5$, $(\lambda x.5)(\bot) = 5$}
  Това е естествено, понеже хаскел е {\em неточен} език, т.е. една функция може да има като аргумент $\bot$
  и да връща ,,смислен'' резултат. 

  \marginpar{Повече информация за \texttt{seq} може да се намери в \cite[стр. 108]{real-world-haskell}}
  При разглеждането на семантики на езици за програмиране, ще се нуждаем от това да направим една функция {\em точна}.
  Такава възможност има и в хаскел с командата \texttt{seq}.
  
  \begin{haskellcode}
ghci> :t seq
seq :: a -> b -> b
ghci> (\x -> x `seq` 5) 4
5
ghci> (\x -> x `seq` 5) undefined
*** Exception: Prelude.undefined
ghci> :set -XBangPatterns
ghci> (\(!x) -> 5) undefined
*** Exception: Prelude.undefined
\end{haskellcode}
  
  Командата \texttt{seq} приема два аргумента. Тя работи като оценява първия аргумент, като очаква той да се оцени до ,,смислен'' резултат, т.е. нещо различно от $\bot$. След това връща втория.
  Тук първия пример се превежда като $(\lambda x. 5)_\star(4) = 5$, а втория като $(\lambda x.5)_\star(\bot) = \bot$.
\end{example}

За произволно естествено число $n$, да дефинираме изображение
\[\Sigma_\star : \Mapping{\Nat^n_\bot}{\Nat_\bot} \to \Strict{\Nat^n_\bot}{\Nat_\bot},\]
където
\[\Sigma_\star(f) = f_\star.\] 
Ще наричаме $\Sigma_\star$ {\bf ограничаващ} оператор, защото на всяка функция дава нейното точно ограничение.

\begin{framed}
\begin{prop}
  \label{pr:strict-operator}
  За всяко $n$, $\Sigma_\star$ е непрекъснат оператор, т.е.
  \[\Sigma_\star \in \Cont{\Mapping{\Nat^n_\bot}{\Nat_\bot}}{\Strict{\Nat^n_\bot}{\Nat_\bot}}.\]
\end{prop}  
\end{framed}
\begin{hint}
  Трябва да докажем, че за произволна верига $\chain{f}{i}$ от елементи на $\Mapping{\Nat^n_\bot}{\Nat_\bot}$, имаме
  \[\Sigma_\star(\bigsqcup_i f_i) = \bigsqcup_i \Sigma_\star(f_i).\]
  \begin{enumerate}[(1)]
  \item 
    Лесно се съобразява, че $\Sigma_\star$ е монотонно изображение, т.е.
    \[f \sqsubseteq g \implies \Sigma_\star(f) \sqsubseteq \Sigma_\star(g).\]
    Сега използваме \Prop{monotone-chain}, откъдето следва, че
    \[\bigsqcup_i \Sigma_\star(f_i) \sqsubseteq \Sigma_\star(\bigsqcup_i f_i).\]
  \item
    За другата посока, нека да разгледаме $\ov{a}$, за които
    \marginpar{Случаят $\Sigma_\star(\bigsqcup_i f_i)(\ov{a}) = \bot$ е тривиален.}
    \[\Sigma_\star(\bigsqcup_i f_i)(\ov{a}) = b \neq \bot.\]
    Щом $b \neq \bot$, то ако $\ov{a} = (a_1, \dots, a_n)$, със сигурност $\bot \not\in \{a_1, \dots, a_n\}$.
    Тогава директно имаме, че $(\bigsqcup_i f_i)(\ov{a}) = b \neq \bot$.
    Сега от \Problem{sup-f} получаваме, че съществува $i_0$, за което $f_{i_0}(\ov{a}) = b$.
    Ясно е, че също така $\Sigma_\star(f_{i_0})(\ov{a}) = b$ и оттук
    $(\bigsqcup_i \Sigma_\star(f_i))(\ov{a}) = b$,
    защото $\Sigma_\star(f_{i_0}) \sqsubseteq \bigsqcup_i \Sigma_\star(f_i)$.
    Заключаваме, че
    \[\Sigma_\star(\bigsqcup_i f_i) \sqsubseteq \bigsqcup_i\Sigma_\star(f_i).\]
  \end{enumerate}
\end{hint}

%%% Local Variables:
%%% mode: latex
%%% TeX-master: "../sep-notes"
%%% End:


\section{Област на Скот от непрекъснати изображения}

Следващата теорема е важна, защото с нейна помощ се доказват много свойства на непрекъснатите изображения от по-висок ред.

\begin{theorem}\label{th:double-chain}
  \marginpar{\cite[стр. 127]{winskel}}
  \marginpar{\cite[стр. 178]{models-of-computation}}
  Нека $\A = (A,\sqsubseteq,\bot)$ да бъде област на Скот и нека множеството 
  \[\{a_{m,n} \mid m,n \in \Nat\}\] от елементи на $A$ притежава свойството, че 
  \[n \leq n^\prime\ \&\ m \leq m^\prime\ \Rightarrow\ a_{n,m} \sqsubseteq a_{n^\prime,m^\prime}.\]
  Тогава са изпълнени равенствата
  \[\bigsqcup_m(\bigsqcup_n a_{n,m}) = \bigsqcup_n(\bigsqcup_{m} a_{n,m}) = \bigsqcup_n a_{n,n}.\]
\end{theorem}
\begin{proof}
  Първо ще въведем някои означения.
  \begin{itemize}
  \item 
    Да фиксираме произволно $m$. Тогава множеството $\{a_{n,m} \mid n \in \Nat\}$ образува верига:
    \[a_{0,m} \sqsubseteq a_{1,m} \sqsubseteq a_{2,m} \sqsubseteq \cdots\]
    \marginpar{По дефиниция, всяка монотонно растяща редица в област на Скот притежава точна горна граница.}
    Следователно тя има точна горна граница $b_m \dff \bigsqcup \{a_{n,m} \mid n \in \Nat\}$.
  \item
    Аналогично, при фиксирано $n$, множеството $\{a_{n,m} \mid m \in \Nat\}$ образува верига:
    \[a_{n,0} \sqsubseteq a_{n,1} \sqsubseteq a_{n,2} \sqsubseteq \ldots,\]
    която притежава точна горна граница $c_n \dff \bigsqcup \{a_{n,m} \mid m \in \Nat\}$.
  \end{itemize}
  Това означава, че трябва да докажем следното:
  \[\bigsqcup_m b_m = \bigsqcup_n c_n = \bigsqcup_n a_{n,n}.\]
  \begin{enumerate}[1)]
  \item 
    Първо да съобразим, че множеството $\{b_m \mid m \in \Nat\}$ образува верига в $\A$ и следователно притежава точна горна граница $\bigsqcup_m b_m$.
    Нека да разгледаме произволни $m \leq m^\prime$.
    Тогава \[(\forall n)[a_{n,m} \sqsubseteq a_{n,m^\prime} \sqsubseteq \bigsqcup_k a_{k,m^\prime} = b_{m^\prime}].\]
    Следователно $b_{m^\prime}$ е горна граница на веригата $(a_{n,m})^{\infty}_{n=0}$ и понеже $b_m$ е точна горна граница на $(a_{n,m})^{\infty}_{n=0}$, то получаваме, че \[b_m \sqsubseteq b_{m^\prime}.\]
    Това означава, че $\chain{b}{m}$ е верига в $\A$ и тя притежава точна горна граница $\bigsqcup_m b_m$.  
  \item
    С подобни разсъждения можем да докажем, че множеството $\{c_n \mid n \in \Nat\}$ образува верига в $\A$, която притежава точна горна граница $\bigsqcup_n c_n$.
  \item
    Сега ще докажем, че \[\bigsqcup_m b_m = \bigsqcup_n c_n.\]
    Имаме, че 
    \[(\forall m)(\forall n)[a_{n,m} \sqsubseteq \bigsqcup_{i}a_{i,m} = b_m \sqsubseteq \bigsqcup_i b_i],\]
    което е еквивалентно на 
    \[(\forall n)(\forall m)[a_{n,m} \sqsubseteq b_m \sqsubseteq \bigsqcup_i b_i].\]
    Да фиксираме произволно $n$.
    Тогава $\bigsqcup_i b_i$ е горна граница на веригата $(a_{n,i})^\infty_{i=0}$.
    Следователно, $c_n = \bigsqcup_i a_{n,i} \sqsubseteq \bigsqcup_i b_i$.
    Така получаваме, че $\bigsqcup_i b_i$ е горна граница и на веригата $\chain{c}{n}$
    и тогава \[\bigsqcup_n c_n \sqsubseteq \bigsqcup_i b_i.\]
    С аналогични разсъждения можем да докажем също, че 
    \[\bigsqcup_m b_m \sqsubseteq \bigsqcup_n c_n.\]
    Така доказахме, че \[\bigsqcup_m b_m = \bigsqcup_n c_n.\]
  \item
    Достатъчно е още да докажем, че
    \[\bigsqcup_n a_{n,n} = \bigsqcup_n c_n.\]
    Ясно е, че $a_{n,n}$ е елемент на веригата ${(a_{n,m})}^{\infty}_{m=0}$ и следователно
    $a_{n,n} \sqsubseteq \bigsqcup_m a_{n,m} = c_n \sqsubseteq \bigsqcup_n c_n$.
    Получаваме, че $\bigsqcup_n c_n$ е горна граница на веригата ${(a_{n,n})}^{\infty}_{n=0}$
    и следователно $\bigsqcup_n a_{n,n} \sqsubseteq \bigsqcup_n c_n$.
    
    
    За другата посока, да разгледаме произволен елемент $a_{n,m}$.
    Нека $k = \max\{n,m\}$.
    Ясно е, че $a_{n,m} \sqsubseteq a_{k,k} \sqsubseteq \bigsqcup_n a_{n,n}$.
    Следователно, $\bigsqcup_n a_{n,n}$ е горна граница на верига ${(a_{m,n})}^{\infty}_{m=0}$
    и оттук получаваме, че за всяко $n$, $c_n \sqsubseteq \bigsqcup_n a_{n,n}$.
    Получаваме, че $\bigsqcup_n a_{n,n}$ е горна граница на веригата $\chain{c}{n}$ и следователно
    $\bigsqcup_n c_n \sqsubseteq \bigsqcup_n a_{n,n}$.
  \end{enumerate}
  С това доказателството на теоремата е завършено.
\end{proof}

\begin{framed}
  \begin{lemma}
    Нека $\A$ и $\B$ са области на Скот.
    Нека $\chain{f}{k}$ е верига от елементи на $\Cont{\A}{\B}$.
    Да дефинираме изображението $h$ на $\A$ в $\B$ по следния начин
    \[h(a) \dff \bigsqcup\{f_k(a) \mid k \in \Nat\}.\]
    Изображението $h$ е {\em непрекъснато} и е {\em точна горна граница} на веригата $\chain{f}{k}$,
    т.е. $h = \bigsqcup_k f_k$.
  \end{lemma}
\end{framed}
\marginpar{Ако $b_k = f_k(a)$, то $h(a)$ е точната горна граница на веригата $\chain{b}{k}$ в $\B$}
\begin{proof}
  \ifhints
  Доказателството, че $h$ е точна горна граница на веригата $\chain{f}{k}$ е лесно.
  \begin{itemize}
  \item 
    Да разгледаме произволен елемент $a \in A$.
    Лесно се вижда, че понеже $\chain{f}{k}$ е верига, то $(f_k(a))^\infty_{k=0}$ също е верига.
    Това е така, защото всяко непрекъснато изображение е също така и монотонно.

    \marginpar{$\bigsqcup_n f_n(a)$ е съкратен запис за $\bigsqcup\{f_n(a) \mid n \in \Nat\}$.}
    Получаваме, че за всяко $k$, $f_k(a) \sqsubseteq^\B \bigsqcup_n f_n(a) \dff h(a)$.
    Понеже това е вярно за произволно $a \in A$, $(\forall k)[f_k \sqsubseteq h]$,
    което означава, че $h$ е горна граница на веригата.
  \item
    Да разгледаме произволно изображение $g$, което е горна граница на веригата $\chain{f}{k}$.
    За произволен елемент $a \in A$, 
    \[(\forall k)[f_k(a) \sqsubseteq^\B g(a)].\]
    Това означава, че $g(a)$ е горна граница на веригата ${(f_k(a))}^\infty_{k=0}$.
    Понеже $h(a) = \bigsqcup_k \{f_k(a)\}$ е точната горна граница на веригата ${(f_k(a))}^\infty_{k=0}$,
    то $h(a) \sqsubseteq^\B g(a)$.
    Оттук следва, че $h \sqsubseteq g$.
  \end{itemize}
  \fi
  По-сложната част на доказателството е проверката, че $h$ е непрекъснато изображение.
  Да вземем една монотонно растяща редица $\chain{a}{k}$ от елементи на $A$.
  \marginpar{За момента дори не е ясно дали $\{h(a_k) \mid k \in \Nat\}$ е верига в $\B$}
  Ще докажем, че \[h(\bigsqcup_k a_k) = \bigsqcup_k \{h(a_k)\}.\]
  Нека $e_{n,m} \dff f_n(a_m)$.
  Понеже всяко $f_n$ е непрекъснато и следователно монотонно изображение, то имаме
  \[n \leq n^\prime\ \&\ m \leq m^\prime\ \Rightarrow\ e_{n,m} \sqsubseteq^{\B} e_{n^\prime,m^\prime}.\]
  Следователно,
  \begin{align*}
    h(\bigsqcup_m a_m) & = \bigsqcup_n(f_n(\bigsqcup_m a_m)) & \comment{\text{от деф. на }h}\\
                       & = \bigsqcup_n(\bigsqcup_m f_n(a_m)) & \comment{\text{ защото } f_n \text{ е непр.}}\\
                       & = \bigsqcup_n(\bigsqcup_m e_{n,m}) = \bigsqcup_m(\bigsqcup_n e_{n,m}) & \comment{\text{от \Th{double-chain}}}\\
                       & = \bigsqcup_m(\bigsqcup_n f_n(a_m)) & \comment{\text{от деф. на }e_{n.m}}\\
                       & = \bigsqcup_m \{h(a_m)\}. & \comment{\text{от деф. на }h}
  \end{align*}
\end{proof}

Да напомним, че релацията $\sqsubseteq$ между две изображения е дефинирана като
\[f \sqsubseteq g \dfff (\forall a\in A)[f(a) \sqsubseteq^\B g(a)].\]
\begin{framed}
  \begin{theorem}
    \label{th:continuous-domain}
    Ако $\A$ и $\B$ са области на Скот, то $\Cont{\A}{\B}$ е област на Скот.
  \end{theorem}
\end{framed}

Нека $\A_1,\dots,\A_n$ и $\A$ са области на Скот и да разгледаме $f: \A_1\times \dots \times \A_n \to \A$.
Казваме, че $f$ е {\bf непрекъснато изображение по $i$-тия аргумент}, ако 
за всяка верига $\chain{a}{k}$ в $\A_i$, то
\[f(b_1,\dots, b_{i-1}, \bigsqcup_k a_k, b_{i+1},\dots,b_n) = \bigsqcup_k f(b_1,\dots, b_{i-1}, a_k, b_{i+1},\dots,b_n).\]

\begin{proposition}
  \label{pr:continuous-arguments}
  \marginpar{\cite[стр. 184]{models-of-computation}}
  Нека $\A_1,\dots,\A_n$ и $\A$ са области на Скот. Едно изображение
  \[f: \A_1\times \dots \times \A_n \to \A\]
  е непрекъснато точно тогава, когато $f$ е непрекъснато по всеки от аргументите си.
\end{proposition}
\begin{proof}
  \marginpar{\writedown Обобщете това твърдение за $n > 2$.}
  За по-просто изложение, да разгледаме случая $n = 2$.

  $(\Rightarrow)$ Лесно се съобразява, че ако $f$ е непрекъснато изображение, то $f$ е непрекъснато по всеки от аргументите си.
  Да видим например защо $f$ е непрекъснато по първия аргумент.
  Да разгледаме веригата ${(\pair{a_i,b})}^\infty_{i=0}$ от елементи на $\A\times\B$, за някое фиксирано $b$.
  Знаем, че $\bigsqcup_i \pair{a_i,b} = \pair{\bigsqcup_ia_i,b}$.
  Тогава
  \marginpar{Формално погледнато, правилно е да пишем $f(\pair{a,b})$ вместо $f(a,b)$.}
  \begin{align*}
    f(\bigsqcup_i a_i,b) & = f(\bigsqcup_i\pair{a_i,b}) \\
                        & = \bigsqcup_i f(a_i,b) & \comment{f\text{ е непрекъснато}}.
  \end{align*}
    
  $(\Leftarrow)$ Нека сега $f$ е непрекъснато по всеки от аргументите си. Ще докажем, че $f$ е непрекъснато.
  Нека ${\{\pair{a_n,b_n}\}}^\infty_{n=0}$ е верига в $\A_1\times \A_2$.
  Понеже от \Prop{cartesian} знаем, че
  \[\bigsqcup_n\pair{a_n,b_n} = \pair{\bigsqcup_n a_n,\bigsqcup_n b_n},\]
  ще докажем, че 
  \[\bigsqcup_n f(a_n,b_n) = f(\bigsqcup_n a_n,\bigsqcup_n b_n).\]
  Да положим $e_{n,m} = f(a_n,b_m)$.
  Понеже $f$ е непрекъснато по всеки от аргументите си, лесно се вижда, че $f$
  е монотонно изображение по всеки от аргументите си. Следователно, 
  \[n \leq n^\prime\ \&\ m \leq m^\prime\ \Rightarrow\ e_{n,m} \sqsubseteq e_{n^\prime,m^\prime}.\]  
  Получаваме, че
  \begin{align*}
    \bigsqcup_n f(a_n,b_n) & = \bigsqcup_n e_{n,n} & \comment{\text{от опр. на }e_{n,m}}\\
                           & = \bigsqcup_n (\bigsqcup_m e_{n,m}) & \comment{\text{от \Th{double-chain}}}\\
                           & = \bigsqcup_n (\bigsqcup_m f(a_n,b_m)) & \comment{\text{от опр. на }e_{n,m}}\\
                           & = \bigsqcup_n f(a_n,\bigsqcup_m b_m) & \comment{f \text{ е непр. по втория си аргумент}}\\
                           & = f(\bigsqcup_n a_n,\bigsqcup_m b_m) & \comment{f \text{ е непр. по първия си аргумент}}.
  \end{align*}
\end{proof}


%%% Local Variables:
%%% mode: latex
%%% TeX-master: "../sep"
%%% End:


\section{Най-малки неподвижни точки}
\index{най-малка неподвижна точка}

\begin{itemize}
\item\index{неподвижна точка}
  Да фиксираме произволна област на Скот $\A = (A, \sqsubseteq, \bot)$ и да разгледаме едно изображение $f:\A\to\A$.
  Казваме, че $a \in \A$ е {\bf неподвижна точка} на $f$, ако $f(a) = a$.
\item\index{най-малка неподвижна точка}
  Казваме, че $a \in \A$ е {\bf най-малката неподвижна точка} на $f$, ако:
  \begin{itemize}
  \item 
    $a$ е неподвижна точка, т.е. $f(a) = a$;
  \item
    за всяко $b \in \A$ със свойството, че $f(b) = b$ имаме $a \sqsubseteq b$.
  \item
    \marginpar{least fixed point}
    Ще означаваме най-малката неподвижна точка на $f$ като $\lfp(f)$.
  \end{itemize}
\end{itemize}

\begin{framed}
\begin{theorem}[Клини]
  \label{th:knaster-tarski}
  \index{Клини}
  Нека $\A$ е област на Скот.
  Всяко $f \in \Cont{\A}{\A}$ притежава най-малка неподвижна точка.
\end{theorem}
\end{framed}
\begin{proof}
  \marginpar{В \cite{ditchev-soskov} се нарича теорема на Кнастер-Тарски. Според \href{https://en.wikipedia.org/wiki/Kleene_fixed-point_theorem}{уикипедия} е теорема на Клини}
  Определяме монотонно растяща редица от елементи на $\A$ по следния начин:
  \begin{align*}
    & a_0 \dff \bot & \comment = f^0(\bot)\\
    & a_{n+1} \dff f(a_n) & \comment = f^{n+1}(\bot).
  \end{align*}

  Първо ще докажем с индукция по $n$, че $\chain{a}{n}$ е верига.
  Ясно е, че $a_0 \sqsubseteq a_1$.
  Да приемем, че $a_n \sqsubseteq a_{n+1}$. Тогава, понеже всяко непрекъснато
  изображение е монотонно, то имаме, че
  \[\underbrace{f(a_n)}_{a_{n+1}} \sqsubseteq \underbrace{f(a_{n+1})}_{a_{n+2}}.\]

  Нека $a \dff \bigsqcup_i a_i$. Тогава 
  \begin{align*}
    f(a) & = f(\bigsqcup_i a_i) & \comment a \dff \bigsqcup_i a_i\\
         & = \bigsqcup_i f(a_i) & \comment f \text{ е непрекъсната}\\
         & = \bigsqcup_i a_{i+1} & \comment a_{i+1} = f(a_i)\\
         & = \bigsqcup_i a_i & \comment \text{защото }\chain{a}{i}\text{ е верига}\\
         & = a.
  \end{align*}
  Така доказахме, че $a$ е \emph{ неподвижна точка} на $f$.
  Остана да видим, че е най-малката неподвижна точка на $f$.

  Нека $b = f(b)$. С индукция по $n$ ще докажем, $(\forall n)[a_n \sqsubseteq b]$.
  \begin{itemize}
  \item 
    За $n = 0$ е очевидно.
  \item
    Да приемем, че $a_n \sqsubseteq b$.
    Тогава $a_{n+1} \dff f(a_n) \sqsubseteq f(b) = b$, защото $f$ е монотонно изображение.    
  \end{itemize}
  Така доказахме, че $b$ е горна граница на веригата $\chain{a}{n}$.
  Заключаваме, че $a \dff \bigsqcup_n a_n \sqsubseteq b$.
  Следователно, $a$ е \emph{ най-малката неподвижна точка} на $f$,
  т.е. $a = \lfp(f)$.
\end{proof}

\begin{problem}
  Покажете, че съществува област на Скот $\A$ и $f \in \Mon{\A}{\A}$, което притежава най-малка неподвижна точка, но тя не е $\bigsqcup_n f^n(\bot^\A)$.
\end{problem}
\ifhints\begin{hint}
  Да разгледаме $\A = (A,\ \sqsubseteq,\ a_0)$, където елементите на $A$ са подредени по следния начин:
  \[A = \{ a_0 \sqsubset a_1 \sqsubset \cdots \sqsubset a_n \sqsubset \cdots \sqsubset a_\omega \sqsubset b \}.\]
  т.е. $A$ е съставена от веригата ${(a_n)}^\infty_{n=0}$ веднага следвана от елементите $a_\omega$ и $b$.
  Да обърнем внимание, че $\bigsqcup_n a_n = a_\omega$.
  Сега да разгледаме изображението $f:\A\to\A$, където за всяко $n$,
  \begin{align*}
    & f(a_n) = a_{n+1}\\
    & f(a_\omega) = b\\
    & f(b) = b.
  \end{align*}
  Лесно се вижда, че това изображение е монотонно.
  Обаче $f$ не е непрекъснато изображение, защото $\bigsqcup_n a_n = a_\omega$, и тогава:
  \[f(\bigsqcup_n a_n) = f(a_\omega) = b \neq a_\omega = \bigsqcup_n a_{n+1} = \bigsqcup_n \{f(a_n)\}.\]
  Според дефиницията на изображението $f$, единствената неподвижна точка на $f$ е елементът $b$.
  Това означава, че $b$ е също и най-малката неподвижна точка.
  \marginpar{$f^n(a_0) = a_n$.}
  Това е пример за монотонно изображение, което не е непрекъснато, но притежава най-малка неподвижна точка,
  макар и тя да не е $\bigsqcup_n f^n(a_0)$.
\end{hint}
\fi

% \begin{problem}
%   Да разгледаме $\Gamma:\Partial{\Nat}{\Nat} \to \Partial{\Nat}{\Nat}$, където
%   \begin{enumerate}[a)]
%   \item
%     $\Gamma(f)(x) = 5$, за всяко $x \in \Nat$;
%   \item
%     $\Gamma(f)(x)$ не е деф. за всяко $x \in \Nat$;
%   \item
%     $\Gamma(f) = f$;
%   \item
%     $\Gamma(f) = f\circ f$;
%   \item
%     $\Gamma(f)(x) = x * f(x+1)$;
%     \item
%     $\Gamma(f)(x) =
%     \begin{cases}
%       \text{не е деф.}, & \text{ ако }x = 0\\
%       x * f(x-1), & \text{ ако }x > 0.
%     \end{cases}$    
%   \item
%     $\Gamma(f)(x) =
%     \begin{cases}
%       0, & \text{ ако }x = 0\\
%       x * f(x-1), & \text{ ако }x > 0.
%     \end{cases}$
%     \item
%     $\Gamma(f)(x) =
%     \begin{cases}
%       1, & \text{ ако }x = 0\\
%       x * f(x-1), & \text{ ако }x > 0.
%     \end{cases}$    
%   \end{enumerate}
% \end{problem}


\begin{example}
  Да разгледаме следното изображение $\Gamma:\Partial{\Nat}{\Nat} \to \Partial{\Nat}{\Nat}$, където
  \[\Gamma(f)(x) =
    \begin{cases}
      0, & \text{ ако }x = 0\\
      x + f(x-1), & \text{ ако }x > 0.
    \end{cases}\]
  Първо да видим, че $\Gamma$ е монотонно изображение.
  Нека $f \sqsubseteq g$.
  Трябва да докажем, че за всяко $x$, ако $\Gamma(f)(x) = y$, то $\Gamma(g)(x) = y$.
  \begin{itemize}
  \item
    Нека $x = 0$.
    Тогава $\Gamma(f)(0) = 0 = \Gamma(g)(0)$.
  \item
    Нека $x > 0$ и да приемем, че $\Gamma(f)(x) = x + f(x-1) = y$.
    Това означава, че $f(x-1) = z$, за някое $z$, и $x + z = y$.
    От $f \sqsubseteq g$ следва, че имаме също и $g(x-1) = z$.
    Тогава е ясно, че $\Gamma(g)(x) = x + g(x-1) = x+z = y$.
  \end{itemize}
  Разгледахме всички възмножни случаи за естественото число $x$ и
  \marginpar{$\texttt{Graph}(\Gamma(f)) \subseteq \texttt{Graph}(\Gamma(g))$.}
  видяхме, че за произволно $x$, ако $\Gamma(f)(x) = y$, то $\Gamma(g)(x) = y$.
  Заключаваме, че $\Gamma(f) \sqsubseteq \Gamma(g)$, т.е. $\Gamma$ е монотонно изображение.

  Нека сега да видим, че $\Gamma$ е непрекъснато изображение.
  Да разгледаме произволна верига $\chain{f}{i}$ от частични функции.
  Трябва да докажем, че
  \[\Gamma(\bigsqcup_i f_i) = \bigsqcup_i \Gamma(f_i).\]
  Щом $\Gamma$ е монотонно, от \Prop{monotone-chain} вече знаем, че
  \[\bigsqcup_i \Gamma(f_i) \sqsubseteq \Gamma(\bigsqcup_i f_i).\]
  Остава да докажем обратната посока. И така, нека първо да вземем $x = 0$.
  Тогава $\Gamma(\bigsqcup_i f_i)(0) = 0$. От дефиницията от $\Gamma$ знаем, че
  за всяко $i$, $\Gamma(f_i)(0) = 0$ и оттук $(\bigsqcup_i \Gamma(f_i))(0) = 0$.
  Следователно,
  \[\Gamma(\bigsqcup_i f_i)(0) = 0 = (\bigsqcup_i \Gamma(f_i))(0).\]
  Нека сега $x > 0$. Тогава
  \marginpar{Да напомним, че $(\bigsqcup_i f_i)(u) = v$ точно тогава, когато съществува индекс $i$, за който $f_i(u) = v$.}
  \[\Gamma(\bigsqcup_i f_i)(x) = x + (\bigsqcup_i f_i)(x-1) = y.\]
  Ясно е, че съществува $z$, за което $(\bigsqcup_i f_i)(x-1) = z$ и $x + z = y$.
  Знаем, че съществува индекс $i$, за който $f_i(x-1) = z$.
  Тогава, понеже $\Gamma(f_i)(x) = x + f_i(x-1) = x+z = y$, то следва, че $(\bigsqcup_i \Gamma(f_i))(x) = y$.

  Сега вече можем да намерим $\lfp(\Gamma) = \bigsqcup_n \Gamma^n(\bm{\bot})$.
  Ще докажем, че
  \[\lfp(\Gamma)(x) = \frac{x(x+1)}{2}\] за всяко естествено число $x$.
  Нека за улеснение да означим $g_n = \Gamma^n(\bm{\bot})$.
  Ще докажем, че за всяко $n$,
  \[g_n(x) = \begin{cases}
      \sum^x_{i=1}i, & \text{ ако } x < n\\
      \text{не е деф.}, & \text{ ако } x \geq n.
    \end{cases}\]
  Ясно е, че $g_0 = \bm{\bot}$, което може да се запише и така:
  \marginpar{Да напомним, че $\Gamma^0(g) = g$ и $\Gamma^{n+1}(g) = \Gamma(\Gamma^n(g))$.}
  \[g_0(x) = \begin{cases}
      \sum^x_{i=1}i, & \text{ ако } x < 0\\
      \text{не е деф.}, & \text{ ако } x \geq 0.
    \end{cases}\]
  Да приемем, че нашето твърдение е изпълнено за $g_n$.
  Ще докажем, че то е изпълнено и за $g_{n+1}$. И така,
  \begin{align*}
    g_{n+1}(x) & = \Gamma(g_n)(x)\\
               &  = \begin{cases}
                 0, & \text{ ако } x = 0\\
                 x + g_n(x-1), & \text{ ако }x > 0\\
               \end{cases}\\
               & \stackrel{\text{И.П.}}{=} \begin{cases}
                 0, & \text{ ако } x = 0\\
                 x + \sum^{x-1}_{i=1}i , & \text{ ако } 0 \leq x-1 < n\\
                 \text{не е деф.}, & \text{ ако }x-1 \geq n\\
               \end{cases}\\
               & = \begin{cases}
                 0, & \text{ ако } x = 0\\
                 \sum^{x}_{i=1}i , & \text{ ако } 1 \leq x < n+1\\
                 \text{не е деф.}, & \text{ ако }x \geq n+1\\
               \end{cases}\\
               & = \begin{cases}
                 \sum^{x}_{i=1}i , & \text{ ако } x < n+1\\
                 \text{не е деф.}, & \text{ ако }x \geq n+1.
               \end{cases}
  \end{align*}
  Сега можем да заключим, че за всяко естествено число $x$,
  \[g_{x+1}(x) = \sum^x_{i=1}i = \frac{x(x+1)}{2}.\]
  Тогава
  \[\texttt{lfp}(\Gamma)(x) = (\bigsqcup_i g_i)(x) = \frac{x(x+1)}{2}.\]
\end{example}


\begin{proposition}\label{pr:prefix-point}
  За всяко $f \in \Cont{\A}{\A}$ е изпълнено, че 
  \[(\forall a \in \Pref(f))[\lfp(f) \sqsubseteq a],\]
  където
  \index{преднеподвижна точка}
  \[\Pref(f) \dff \{a \in \A \mid f(a) \sqsubseteq a\}\]
  е множеството от всички преднеподвижни точки на $f$.
  Това означава, че $\lfp(f)$ е най-малката преднеподвижна точка на $f$.
\end{proposition}
\begin{proof}
  Знаем от \hyperref[th:knaster-tarski]{Теоремата на Клини}, че $\lfp(f) = \bigsqcup_n f^n(\bot)$.
  Също така знаем, че $\chain{b}{n}$ е верига, където за улеснение сме положили $b_n \dff f^n(\bot)$. 
  Ясно е също, че $\texttt{Pref}(f) \neq \emptyset$, защото $\lfp(f) \in \texttt{Pref}(f)$.
  Да фиксираме прозиволен елемент $a\in \texttt{Pref}(f)$.
  С индукция по $n$ ще докажем, че $b_n \sqsubseteq a$ за всяко $n$.
  \begin{itemize}
  \item 
    За $n = 0$ е очевидно, защото тогава $b_0 \dff \bot \sqsubseteq a$.
  \item
    Да приемем, че $b_n \sqsubseteq a$.
    Ще докажем, че $b_{n+1} \sqsubseteq a$.
    Но това е лесно.
    \begin{align*}
      b_{n+1} & = f(b_n) & \comment \text{от деф. на }b_{n+1}\\
      & \sqsubseteq f(a) & \comment b_n \sqsubseteq a\ \&\ f\text{ е мон.}\\
      & \sqsubseteq a & \comment a \in \texttt{Pref}(f).
    \end{align*}
  \end{itemize}
  Така доказахме, че за всяко $n$, $b_n \sqsubseteq a$,
  откъдето следва, че $a$ е горна граница за веригата $\chain{b}{n}$, откъдето директно получаваме, че
  \[\lfp(f) = \bigsqcup_n b_n \sqsubseteq a.\]
\end{proof}

%%% Local Variables:
%%% mode: latex
%%% TeX-master: "../sep"
%%% End:


\section{Оператор за най-малка неподвижна точка}

\begin{problem}
  Нека $\chain{f}{i}$ е верига от елементи на $\Mon{\A}{\A}$.
  Докажете, че:
  \begin{enumerate}[a)]
  \item
    $\chain{f^n}{i}$ е верига за произволно $n$;
  \item
    ${(f^n_i)}^\infty_{n=0}$ е верига за произволно $i$;
  \item
    ${(\bigsqcup_n f^n_i)}^\infty_{i=0}$ е верига;
  \item
    ${(\bigsqcup_i f^n_i)}^\infty_{n=0}$ е верига.
  \end{enumerate}
\end{problem}


\begin{theorem}\label{th:Y}\index{Y}
  \marginpar{Доказателството в \cite[стр. 188]{models-of-computation} е малко по-различно.}
  % \index{$Y_\A$}
  Нека $\A$ е област на Скот и нека $f \in \Cont{\A}{\A}$.
  \marginpar{Знаем от \Th{knaster-tarski}, че най-малката неподвижна точка на $f$ е елемента $\bigsqcup_n f^n(\bot^\A)$,
    т.е. $\lfp(f) = \bigsqcup_n f^n(\bot^\A)$.}
  Тогава изображението $Y : \Cont{\A}{\A} \to \A$, определено като
  \[Y(f) = \lfp(f),\]
  е непрекъснато, т.е.
  $Y \in \Cont{\Cont{\A}{\A}}{\A}$.
\end{theorem}
\begin{proof}
  Нека да вземем една верига $\chain{f}{n}$ от непрекъснати изображения.
  Нашата цел е да докажем, че
  \[Y(\bigsqcup_n f_n) = \bigsqcup_n Y(f_n).\]
  \marginpar{Записът ще стане много тромав, ако вместо $h$ използваме означението $\bigsqcup_n f_n$.}
  Да означим с $h$ точната горна граница на $\chain{f}{n}$.
  Знаем, че $h(a) = \bigsqcup_n \{f_n(a)\}$, а от \Th{continuous-domain} знаем, че $h$ е непрекъснато изображение.
  \begin{proposition}
    За всяко $k \geq 0$ и за всеки елемент $a \in \A$,
    \[h^k(a) = \bigsqcup_n \{f^k_n(a)\}.\]
  \end{proposition}
  \begin{proof}
    Ще докажем твърдението с индукция по $k$, като случая $k = 0$ е ясен, защото $h^0(a) = f^0_n(a) = a$.
    Нека приемем, че твърдението е вярно за произволно $k$.
    Ще докажем, че твърдението е вярно за $k+1$.
    \begin{align*}
      h^{k+1}(a) & = h(h^k(a)) & \\
                 & = h(\bigsqcup_n f^k_n(a))& \comment{\text{ от инд. предположение}}\\
                 & = \bigsqcup_n h(f^k_n(a))& \comment{h \text{ е непрекъснато изображение}}\\
                 & = \bigsqcup_n (\bigsqcup_m f_m(f^k_n(a))). & 
    \end{align*}
    
    Да положим $b_n = f^k_n(a)$, за всяко $n$.
    Понеже $f_n \sqsubseteq f_{n^\prime}$, лесно се съобразява, че за $n \leq n^\prime$
    имаме $b_n \sqsubseteq^\A b_{n^\prime}$.

    Сега да положим $e_{m,n} = f_m(b_n)$.
    Отново, понеже $\chain{b}{n}$ и $\chain{f}{m}$ са вериги, имаме 
    \[m \leq m^\prime\ \&\ n\leq n^\prime\ \Rightarrow\ e_{m,n} \sqsubseteq^\A e_{m^\prime,n^\prime}.\]
    Получаваме, че
    \begin{align*}
      h^{k+1}(a) & = \bigsqcup_n (\bigsqcup_m f_m(f^k_n(a))) & \comment{\text{ от горното равенство за } h^{k+1}}\\
                 & = \bigsqcup_n (\bigsqcup_m e_{m,n}) & \comment{\text{ от определението на }e_{m,n}}\\
                 & = \bigsqcup_n e_{n,n} & \comment{\text{ от \Th{double-chain}}}\\
                 & = \bigsqcup_n f_n(f^k_n(a))  = \bigsqcup_n f^{k+1}_n(a) & 
    \end{align*}
    С това твърдението е доказано.
  \end{proof}
  Сега вече сме готови да докажем непрекъснатостта на $Y$.
  Имаме, че:
  \begin{align*}
    Y(\bigsqcup_n f_n) & = Y(h) & \comment{\text{ от опр. на }h}\\
                       & = \bigsqcup_m h^m(\bot^\A) & \comment{\text{ от опр. на }Y }\\
                       & = \bigsqcup_m (\bigsqcup_n f^m_n(\bot^\A)) & \comment{\text{ от горното твърдение}}.
  \end{align*}
  
  Да положим $e_{m,n} = f^m_n(\bot^\A)$.
  Отново лесно се съобразява, че 
  \[m \leq m^\prime\ \&\ n\leq n^\prime\ \Rightarrow\ e_{m,n} \sqsubseteq^\A e_{m^\prime,n^\prime}.\]
  Получаваме, че
  \begin{align*}
    Y(\bigsqcup_n f_n) & = \bigsqcup_m (\bigsqcup_n f^m_n(\bot^\A)) & \comment{\text{ от горното равенство}}\\
                          & = \bigsqcup_m (\bigsqcup_n e_{m,n}) & \comment{\text{ от опр. на }e_{m,n}}\\
                          & = \bigsqcup_n(\bigsqcup_m e_{m,n}) & \comment{\text{ от \Th{double-chain}}}\\
                          & = \bigsqcup_n (\bigsqcup_m f^m_n(\bot^\A)) = \bigsqcup_n Y(f_n). & \comment{\text{ от опр. на }Y}.
  \end{align*}
\end{proof}

\marginpar{Добре е да погледнете това: \url{https://en.wikibooks.org/wiki/Haskell/Fix_and_recursion}}

\begin{haskellcode}
ghci> fact x = if x == 0 then 1 else x * fact(x-1)
ghci> fact 5
120
ghci> fix f = x where x = f x
ghci> :t f
fix :: (t -> t) -> t
ghci> fact' = fix \f x -> if x == 0 then 1 else x * f(x-1)
ghci> fact' 5
120
ghci> gamma f = \x -> if x == 0 then 1 else x * f(x-1)
ghci> :t gamma
(t -> t) -> t -> t
ghci> fact'' = fix gamma
ghci> fact'' 5
120
ghci> fix' f = x where x = f(f(f(x)))
ghci> fact''' = fix' gamma
ghci> fact''' 5
120
\end{haskellcode}


%%% Local Variables:
%%% mode: latex
%%% TeX-master: "../sep"
%%% End:


\section{Изображения запазващи непрекъснатостта}

\begin{prop}
  \label{pr:composition}
  \index{изображения!композиция}
  Ако $f \in \Cont{\A}{B}$ и $g \in \Cont{\B}{\C}$, то $g \circ f \in \Cont{\A}{\C}$,
  където \[(g\circ f)(a) \dff g(f(a)).\]
\end{prop}
\begin{hint}
  Нека $\chain{a}{i}$ е верига в $\A$.
  Да обърнем внимание, че понеже $f \in \Cont{\A}{\B}$,
  то $f$ е монотонно изображение и тогава $(f(a_i))^\infty_{i=0}$ е верига в $\B$.
  Тогава:
  \begin{align*}
    (g \circ f)(\bigsqcup_i a_i) & = g(f(\bigsqcup_i a_i)) & \comment{\text{от деф.}}\\
    & = g(\bigsqcup_i f(a_i)) & \comment{f \text{ е непр.}}\\
    & = \bigsqcup_i g(f(a_i)) & \comment{g \text{ е непр.}}
  \end{align*}
\end{hint}


%%% Local Variables:
%%% mode: latex
%%% TeX-master: "../sep"
%%% End:


\begin{definition}\label{def:if}\index{if}
  Нека $\A$ е област на Скот. Дефинираме следното изображение
  \begin{align*}
    & \texttt{if}:\Nat_\bot \times \A \times \A \to \A \text{, където}\\
    & \texttt{if}(b, a_1,a_2) =
      \begin{cases}
        a_1, & \text{ако } b \in \Nat^+\\
        a_2, & \text{ако } b = 0\\
        \bot, & \text{ако } b = \bot.
      \end{cases}
  \end{align*}
\end{definition}

\begin{problem}\label{prob:if}
  Докажете, че $\texttt{if}$ е непрекъснато изображение, т.е.
  \[\texttt{if} \in \Cont{\Nat_\bot \times \A \times \A}{\A}.\]
\end{problem}
\begin{hint}
  Докажете, че $\texttt{if}$ е непрекъснато изображение по всеки от аргументите си поотделно.
\end{hint}

%%% Local Variables:
%%% mode: latex
%%% TeX-master: "../sep"
%%% End:


\begin{definition}\label{def:eval}\index{eval}
  Нека $\D$ и $\E$ са области на Скот. Дефинираме изображението 
  \[\texttt{eval}: \Cont{\D}{\E} \times \D \to \E,\]
  по следния начин:
  \[\texttt{eval}(f,d) \dff f(d).\]  
\end{definition}

\begin{problem}\label{prob:eval}
  \marginpar{\cite[стр. 186]{models-of-computation}}
  Докажете, че $\texttt{eval}$ е непрекъснато изображение, т.е.
  \[\texttt{eval} \in \Cont{\Cont{\D}{\E} \times \D}{\E}.\]
\end{problem}
\begin{proof}
  Според \Prop{continuous-arguments}, достатъчно е да докажем, че $\texttt{eval}$ е непрекъснато
  изображение по всеки от двата си аргумента поотделно.
  
  Първо, нека $\chain{f}{n}$ е верига от елементи на $\Cont{\D}{\E}$ и $d$ е произволен елемент на $\D$.
  Тогава
  \[\texttt{eval}(\bigsqcup_n f_n,d) = (\bigsqcup_n f_n)(d) = \bigsqcup_n \{f_n(d)\} = \bigsqcup_n \texttt{eval}(f_n,d),\]
  т.е. изображението $\texttt{eval}$ е непрекъснато по първия си аргумент.
  
  Нека сега $\chain{d}{n}$ е верига от елементи на $\D$.
  Тогава за произволен елемент $f$ на $\Cont{\D}{\E}$ получаваме, че
  \[\texttt{eval}(f,\bigsqcup_n d_n) = f(\bigsqcup_n d_n) = \bigsqcup_n \{f(d_n)\} = \bigsqcup_n \texttt{eval}(f,d_n).\]
\end{proof}

%%% Local Variables:
%%% mode: latex
%%% TeX-master: "../sep"
%%% End:


\begin{problem}
  Докажете, че изображението
  \[\texttt{curry}:\Cont{\A\times \B}{\C} \to \Cont{\A}{\Cont{\B}{\C}},\]
  дефинирано като
  \[\texttt{curry}(f)(a)(b) \dff f(a,b),\]
  е непрекъснато.
\end{problem}

%%% Local Variables:
%%% mode: latex
%%% TeX-master: "../sep"
%%% End:


\section{Най-малко решение на система от уравнения}\index{система}

\begin{problem}
  % \label{pr:cartesian-continuous}
  Нека $f \in \Cont{\A}{\B}$ и $g \in \Cont{\A}{\C}$.
  Докажете, че $h \in \Cont{\A}{\B\times\C}$, където
  \[h(a) \dff \pair{f(a),g(a)}.\]
  В такъв случай ще означаваме $h = f \times g$.
\end{problem}
\begin{proof}
  Нека ${(a_i)}^{\infty}_{i=0}$ е верига в $\A$. Тогава:
  \begin{align*}
    h(\bigsqcup_i a_i) & = \pair{f(\bigsqcup_i a_i), g(\bigsqcup_i a_i)} & \comment{\text{от деф.}}\\
    & = \pair{\bigsqcup_i f(a_i), \bigsqcup_i g(a_i)} & \comment{\text{$f$ и $g$ са непр.}}\\
    & = \bigsqcup_i \pair{f(a_i),g(a_i)} & \comment{\text{от \Prop{cartesian}}}\\
    & = \bigsqcup_i h(a_i) & \comment{\text{от деф.}}
  \end{align*}
\end{proof}

Обърнете внимание на следващото твърдение, защото ще го използваме често по-късно.
То представлява обобщение на предишната задача и има сходно доказателство.
% Първо да въведем следното означение за произволно области на Скот $\B_1,\dots,\B_n$,
% \[\prod^n_{i=1}\B_i \dff \B_1\times \B_2 \times \cdots \B_n,\]
% което също е област на Скот, дефинирана в {\em Раздел \ref{subsect:domains:product}}.

\begin{proposition}\label{pr:product-continuous}
  Нека $f_i \in \Cont{\A}{\B_i}$, за $i = 1,\dots,n$.
  \marginpar{\writedown Докажете сами!}
  Тогава
  \[g \in \Cont{\A}{\prod^n_{i=1}\B_i},\]
  където
  \[g(a) \dff \pair{f_1(a),f_2(a),\dots,f_n(a)}.\]
  В такъв случай ще означаваме $g = f_1\times f_2 \cdots \times f_n$.
\end{proposition}
Нека $\A_1,\dots,\A_n$ са области на Скот 
и да разгледаме изображенията
\[f_i:\prod^n_{k=1}\A_k \to \A_i,\] за $i = 1,\dots,n$.
\index{решение на система}
Казваме, че $\bar{a} = \pair{a_1,\dots,a_n}$ е {\bf решение на системата}

\begin{align*}
  \bigstar = 
  \begin{cases}
    &f_1(x_1,\dots,x_n) = x_1\\
    & \ \vdots\\
    & f_n(x_1,\dots,x_n) = x_n,
  \end{cases}
\end{align*}

ако са в сила равенствата 
\begin{align*}
  & f_1(a_1,\dots,a_n) = a_1\\
  & \ \vdots\\
  & f_n(a_1,\dots,a_n) = a_n.
\end{align*}
\index{система!най-малко решение}

Казваме, че $\bar{a}$ е {\bf най-малкото решение} на системата $\bigstar$, ако за всяко друго решение $\bar{b}$
е изпълнено, че $\bar{a} \sqsubseteq \bar{b}$.

\begin{framed}
\begin{theorem}
  \label{th:sep:min-solution-system}
  За произволни изображения $f_i \in \Cont{\prod^n_{k=1}\A_k}{\A_i}$, за $i = 1,\dots,n$, системата
  \begin{align*}
    & f_1(x_1,\dots,x_n) = x_1\\
    & \vdots\\
    & f_n(x_1,\dots,x_n) = x_n,
  \end{align*} 
  притежава най-малко решение.
\end{theorem}
\end{framed}
\begin{proof}
  Първо да дефинираме както в \Prop{product-continuous} непрекъснатото изображение 
  \[g \dff f_1\times\dots\times f_n : \prod^n_{k=1}\A_k \to  \prod^n_{k=1}\A_k,\]
  като 
  \[g(\bar{a}) \dff \pair{f_1(\bar{a}),\dots,f_n(\bar{a})}.\]
  От \hyperref[th:knaster-tarski]{Теоремата на Клини} знаем, че $g$ притежава най-малка неподвижна точка
  $\bar{a} = \pair{a_1,\dots,a_n}$. Ще проверим, че $\ov{a}$ е най-малкото решение на системата.
  \begin{itemize}
  \item 
    Понеже $\ov{a}$ е неподвижна точка на $g$, то
    \begin{align*}
      g(a_1,\dots,a_n) & = \pair{f_1(\ov{a}),\dots,f_n(\ov{a})} & \comment{\text{от деф. на }g}\\
                       & = \pair{a_1,\dots,a_n} & \comment{\ov{a}\text{ е неподвижна точка}}.
    \end{align*}
    Оттук директно следва, че $f_i(\bar{a}) = a_i$, за $i = 1, \dots, n$, и следователно $\ov{a}$ е решение на системата.
  \item
    Нека $\ov{b} = \pair{b_1,\dots,b_n}$ е друго решение на системата, т.е. 
    $f_i(\ov{b}) = b_i$, за $i = 1, \dots, n$. Тогава 
    $g(\ov{b}) = \pair{f_1(\ov{b}),\dots,f_n(\ov{b})} = \ov{b}$.
    Следователно $\bar{b}$ е неподвижна точна на $g$.
    Понеже $\ov{a} = \lfp(g)$, то $\ov{a} \sqsubseteq \ov{b}$.
  \end{itemize}
  Така достигнахме до извода, че $\ov{a}$ е най-малкото решение на системата.
\end{proof}

\begin{remark}
  Да разгледаме изображенията $f\in\Cont{\A\times\B}{\A}$, $g \in \Cont{\B}{\B}$ и системата
  \begin{align*}
    & f(x,y) = x\\
    & g(y) = y.
  \end{align*}
  За да можем директно да се позовем на \Th{sep:min-solution-system} и да твърдим, че тази система има най-малко решение,
  ние трябва да разгледаме следната модификация на системата:
  \begin{align*}
    & f(x,y)       = x\\
    & \hat{g}(x,y) = y,
  \end{align*}
  където $\hat{g}(x,y) = g(y)$, т.е. добавяме един фиктивен аргумент, защото искаме всички изображения да имат равен брой аргументи.
\end{remark}


Ще завършим този раздел с две твърдения, които ще улеснят нашите разсъждения при 
решаването на задачи.

\begin{framed}
  \begin{proposition}\label{pr:system:independent}
    Да разгледаме две изображения
    \begin{align*}
      & f \in \Cont{\A\times\B}{\A}\\
      & g \in \Cont{\B}{\B},
    \end{align*}
    за които имаме системата от уравнения
    \begin{align*}
      \bigstar = 
      \begin{cases}
        & f(x,y) = x\\
        & g(y) = y.
      \end{cases}
    \end{align*}  
    Нека $b_0 = \lfp(g)$ и $a_0 = \lfp(\hat{f})$, където $\hat{f}(a) \df f(a,b_0)$.
    Тогава $\pair{a_0,b_0}$ е най-малкото решение на системата $\bigstar$.
  \end{proposition}
\end{framed}
\begin{proof}
  \begin{itemize}
  \item
    Първо, понеже $b_0 = \lfp(g)$, то очевидно $g(b_0) = b_0$.
    Освен това, $a_0 = \lfp(\hat{f})$, откъдето следва, че $a_0 = f(a_0,b_0)$.
    Ясно е, че $\pair{a_0,b_0}$ е решение на системата $\bigstar$.
  \item
    Сега нека $\pair{a,b}$ е произволно решение на системата $\bigstar$.
    Да видим, че $\pair{a_0,b_0} \sqsubseteq \pair{a,b}$.
    \begin{itemize}
    \item 
      Първо, ясно е, че $b = g(b)$. Понеже $b_0 = \lfp(g)$, то $b_0 \sqsubseteq b$.
    \item
      Второ, ясно е, че 
      \begin{align*}
        a & = f(a,b) & \comment{a \text{ е решение на }\bigstar}\\
          & \sqsupseteq f(a,b_0) & \comment{b \sqsupseteq b_0}\\
          & = \hat{f}(a) & \comment{\text{от деф.}}
      \end{align*}
      Получихме, че $a \in \texttt{Pref}(\hat{f})$.
      От \Prop{prefix-point} знаем, че 
      \[a_0 \dff \lfp(\hat{f}) \sqsubseteq a.\]
    \end{itemize}
    Заключваваме, че $\pair{a_0,b_0} \sqsubseteq \pair{a,b}$.
  \end{itemize}
\end{proof}

Нещата започнаха да стават прекалено абстрактни, затова нека да видим един прост пример, който показва,
че всъщност горното твърдение е близо до нашата интуиция.
\Stefan{По-долу в примера трябва да се цитира горното твърдение по някакъв разбираем начин.}

\begin{example}
  Нека да разгледаме следната програма на езика \texttt{haskell}:
\begin{haskellcode}
ghci> let g(x,y) = if x == 0 then 0 else g(x-1,y) + y
ghci> let f(x) = if x == 0 then 1 else g(x,f(x-1))
\end{haskellcode}
Лесно се съобразява, че всъщност
\[g(x,y) = x * y.\]
Това означава, че можем да пренапишем дефиницията на $f$ по следния начин:
\begin{haskellcode}
ghci> let f(x) = if x == 0 then 1 else x * f(x - 1)
\end{haskellcode}
Сега лесно се съобразява, че $f(x) = x!$, за $x \in \Nat$.
Да не забравяме, че в {\texttt haskell} имаме и константатa {\texttt undefined}.
Това означава, че ако се ограничим до $\Nat_\bot$, то по горния начин сме дефинирали следните две функции:
\begin{align}
  \label{eq:4}
  f(x) = & 
  \begin{cases}
    x!,   & \text{ако }x \in \Nat\\
    \bot, & \text{ако }x = \bot
  \end{cases}
  \\
  \label{eq:5}
  g(x,y) = &
  \begin{cases}
    x\cdot y, & \text{ако }x,y \in \Nat\\
    \bot,     & \text{ако }\bot \in \{x,y\}.
  \end{cases}  
\end{align}

Ясно е, че тези функции са точни, а следователно и непрекъснати.
Целта на \Chapter{rec} е да формализираме разсъжданията, които направихме по-горе.
Ще видим, че на тази програма можем да съпоставим система от {\em непрекъснати} изображения.

\marginpar{$x + \bot \dff \bot$}
\marginpar{В \Chapter{rec} ще видим, че на всяка програма съпоставяме система от {\em непрекъснати} изображения. В конкретния пример можем директно да докажем, че $\Gamma$ и $\Delta$ са непрекъснати изображения.}
\begin{align*}
  \Gamma(f,g)(x) =
  \begin{cases}
    1, & x = 0\\
    g(x, f(x-1)), & x > 0\\
    \bot, & x = \bot\\
  \end{cases}
  \\
  \Delta(g)(x,y) = 
  \begin{cases}
    0, & x = 0\\
    g(x-1,y) + y, & x > 0\\
    \bot, & x = \bot.
  \end{cases}
\end{align*}

Да видим как можем да дефинираме тези изображения на {\texttt haskell}
и как можем получим редицата от апроксимации на най-малките неподвижни точки по Теоремата на Клини.

\begin{haskellcode}
ghci> let gamma(f, g)(x) = if x == 0 then 1 else g(x, f(x - 1))
ghci> let delta(g)(x, y) = if x == 0 then 0 else g(x - 1, y) + y
-- Започваме да строим редицата от апроксимации по Теоремата на Клини
ghci> let g1 = delta( \(x,y) -> undefined )
ghci> let g2 = delta(g1)
ghci> let g3 = delta(g2)
ghci> g3(2,4)
8
ghci> g3(3,4)
*** Exception: Prelude.undefined
-- Можем да подходим и по-мързеливо, като направо дефинираме безкрайния
-- списък от тези апроксимации.
ghci> let approx = (\(x,y) -> undefined):[delta(g) | g <- approx]
ghci> let g9 = approx !! 9
ghci> g9(8,100)
800
ghci> g9(9,100)
*** Exception: Prelude.undefined
-- най-малката неподвижна точка на delta
ghci> let psi(x) = (approx !! (x+1))(x) 
\end{haskellcode}

Горният пример ни подсказва, че с индукция по $k$, можем да докажем, че 
ако имаме редицата
\begin{align*}
  & g_0 = \bm{\bot}^{(2)}\\
  & g_{k+1} = \Delta(g_k),
\end{align*}
то, за произволнен индекс $k$, имаме
\[g_k(x,y) =
\begin{cases}
  x \cdot y, & \text{ако }x < k\text{ и }y \in \Nat\\
  \bot, & \text{иначе}.
\end{cases}\]
Тогава с помощта на Теоремата на Клини можем да докажем, че
\[\lfp(\Delta)(x,y) =
\begin{cases}
  x \cdot y, & \text{ако }x,y\in\Nat\\
  \bot,      & \text{ако }\bot \in \{x,y\}.
\end{cases}\]

Нека сега да разгледаме изображението
\[\hat\Gamma(f)(x) \dff \Gamma(f, \lfp(\Delta))(x) = 
\begin{cases}
  1,              & \text{ако }x = 0\\
  x \cdot f(x-1), & \text{ако }x > 0\\
  \bot,           & \text{ако }x = \bot.
\end{cases}\]

Нека отново да видим как можем да дефинираме това изображение на {\texttt haskell}
и как можем получим редицата от апроксимации на най-малките неподвижни точки по Теоремата на Клини.
\begin{haskellcode}
ghci> let gamma(f,g)(x) = if x == 0 then 1 else g(x,f(x-1))
ghci> let gamma'(f) = gamma(f, \(x, y) -> x * y)
ghci> :t gamma'
gamma' :: (a -> a) -> a -> a
ghci> let approx' = (\x -> undefined):[gamma'(f) | f <- approx']
ghci> let f9 = approx' !! 9
ghci> f9(8)
40320
ghci> f9(9)
*** Exception: Prelude.undefined 
ghci> let phi(x) = (approx' !! (x+1))(x)
ghci> phi(8)    -- phi е най-малмата неподвижна точка на gamma'
40320           -- лесно се съобразява, че phi(x) == x!
\end{haskellcode}

Горният пример ни подсказва, че с индукция по $k$, можем да докажем,
че ако имаме редицата
\begin{align*}
  & f_0 = \bm{\bot}^{(1)}\\
  & f_{k+1} = \hat\Gamma(f_k),
\end{align*}
то, за произволен индекс $k$, имаме
\[f_k(x) =
\begin{cases}
  x!, & \text{ако }x < k\\
  \bot, & \text{иначе}.
\end{cases}\]

\noindent Отново по Теоремата на Клини, 
\[\lfp(\hat\Gamma)(x) =
\begin{cases}
  x!, & \text{ако }x \in \Nat\\
  \bot, & \text{ако }x = \bot.
\end{cases}\]
            
От \Prop{system:independent} знаем, че двойката $(\lfp(\hat\Gamma)),\lfp(\Delta))$ е най-малкото решение на системата,
което е точно двойката изображения $(f,g)$ с дефиниции (\ref{eq:4}) и (\ref{eq:5}).
\end{example}

\begin{framed}
  \begin{proposition}\label{pr:system:definition}
    Да разгледаме изображенията $f \in \Cont{\A}{\B}$ и $g \in \Cont{\A}{\A}$
    и системата:
    \begin{align*}
      \bigstar = 
      \begin{cases}
        & f(y) = x\\
        & g(y) = y.
      \end{cases}
    \end{align*}  
    Нека $a_0 = \lfp(g)$.
    Тогава най-малкото решение на системата $\bigstar$ е
    \[\pair{f(a_0), a_0}.\]
  \end{proposition}
\end{framed}
\begin{proof}
  \begin{itemize}
  \item 
    Лесно се съобразява, че $\pair{f(a_0), a_0}$ е решение на системата $\bigstar$.
  \item
    Нека $\pair{c,d}$ е решение на системата $\bigstar$.
    Тогава $g(d) = d$ и понеже $a_0 = \lfp(g)$, то $a_0 \sqsubseteq d$.
    Освен това, $c = f(d) \sqsupseteq f(a_0)$.
    Получихме, че $\pair{f(a_0), a_0} \sqsubseteq \pair{c,d}$.
  \end{itemize}
  Заключаваме, че $\pair{f(a_0), a_0}$ е най-малкото решение на системата $\bigstar$.
\end{proof}

\Stefan{Тук пак трябва да се обясни как горното твърдение се използва в долния пример.}

\begin{example}
Да разгледаме следната програма:
  \begin{haskellcode}
ghci> :{  -- използване на multiline дефиниции
ghci> let g(x, y, z) = if x == y + z then z 
ghci|                    else if z == x + 1 then 0 
ghci|                      else g(x, y, z + 1)
ghci| :}
ghci> let f(x, y) = g(x, y, 0)
  \end{haskellcode}

Лесно се съобразява, че 
\[g(x,y) = 
\begin{cases}
  x - y, & \text{ако }x \geq y\\
  0, & \text{ако }x < y.
\end{cases}\]

Тази функция ще я означаваме като $x \monus y$.
На горната програма можем да съпоставим системата от непрекъснати изображения:

\begin{align*}
  \Gamma(g)(x,y) & = g(x,y,0)\\
  \Delta(g)(x,y,z) & = \begin{cases}
    z, & \text{ако } x = y+z\\
    0, & \text{ако } z = x + 1\\
    g(x,y,z+1), & \text{ иначе и }x,y,z\in\Nat\\
    \bot, & \bot \in \{x,y,z\}.
  \end{cases}
\end{align*}


\begin{haskellcode}
ghci> :{  -- Multiline
ghci> let delta(g)(x, y, z) = if x == y + z then z 
ghci|                           else if z == x + 1 then 0 
ghci|                             else g(x, y, z + 1)
ghci| :}
ghci> :t delta
delta :: ((t, t, t) -> t) -> (t, t, t) -> t
ghci> let approx = (\(x,y,z) -> undefined):[delta(g) | g <- approx]
ghci> let g9 = approx !! 9
ghci> g9(20,11,1)  -- 20-11 \in [1, 10)
9
ghci> g9(20,1,11) -- 20-1 \in [11, 20)
19
ghci> g9(2,11,4)  -- 2+1 \not\in [4, 13)
*** Exception: Prelude.undefined
\end{haskellcode}

Горният пример ни подсказва, че с индукция по $k$, можем да докажем,
че ако имаме редицата
\begin{align*}
  & g_0 = \bm{\bot}^{(3)}\\
  & g_{k+1} = \Delta(g_k),
\end{align*}
то, за произволно $k$, имаме
\[g_k(x,y,z) =
\begin{cases}
  0,   & x + 1\in [z,z+k)\\
  x-y, & x \geq y\ \&\ x-y \in [z,z+k)\\
  \bot, & \text{иначе}.
\end{cases}\]
Тогава можем да приложим Теоремата на Клини и да докажем, че
\[\lfp(\Delta)(x,y,z)  =
\begin{cases}
  x \monus y, & z \leq x+1\\
  \bot, & z > x+1\text{ или } \bot \in \{x,y,z\}.
\end{cases}\]
Тогава от \Prop{system:definition} следва, че
\[\lfp(\Gamma)(x,y) = \lfp(\Delta)(x,y,0) =
\begin{cases}
  x \monus y, & x,y\in\Nat\\
  \bot, & \bot \in \{x,y\}.
\end{cases}\]

Съобразете, че $\lfp(\Gamma) = \bigsqcup_k f_k$,
където $f_k(x,y) = g_k(x,y,0)$.

\end{example}



%%% Local Variables:
%%% mode: latex
%%% TeX-master: "../sep"
%%% End:

% \newpage
% \section{Алгебрични области на Скот}

\marginpar{\cite{abramsky94}, \cite{barendregt-handbook}}

\begin{itemize}
\item 
  \index{компактен елемент}
  Нека $\A$ е област на Скот.
  Казваме, че елементът $c$ е {\bf компактен}, ако 
  за всяка верига $\chain{a}{i}$, за която $c \sqsubseteq \bigsqcup_i a_i$,
  съществува индекс $i_0$, за който $c \sqsubseteq a_{i_0}$.
  Компактните елементи на $\A$ ще означаваме с $K(\A)$.
\item
  \index{област на Скот!алгебрична}
  Ще казваме, че областта на Скот $\A$ е {\bf алгебрична}, ако за всеки елемент $a \in \A$,
  съществува верига от {\em компактни} елементи $\chain{c}{i}$ в $\A$, за която $a = \bigsqcup_i c_i$.
\end{itemize}

\begin{example}
  Нека да разгледаме $\A = \pair{A,\sqsubseteq, \bot}$,
  където 
  \[A = \{a_0,a_1,\dots,\} \cup \{a_\omega, b\},\]
  релацията $\sqsubseteq$ е представена на \Fig{noncompact-element}.
  Лесно се съобразява, че $\A$ е област на Скот.
  Всеки от елементите на $A$ е компактен, с изключение на $a_\omega$ и $b$.
  Също така е ясно, че $\A$ {\em не е} алгебрична област на Скот, защото 
  няма верига от крайни елементи, чиято точна горна граница да е елементът $b$.
  \begin{framed}
    \begin{figure}[H]
    \centering
    \begin{tikzpicture}[shorten >=1pt,->]
      \tikzstyle{vertex}=[circle,minimum size=17pt,inner sep=0pt]
      
      \node[vertex] (omega) at (0,5) {$a_\omega$};
      \node[vertex] (2) at (0,3) {$a_3$};
      \node[vertex] (1) at (0,2) {$a_2$};
      \node[vertex] (0) at (0,1) {$a_1$};
      \node[vertex] (bot) at (0,0) {$a_0$};
      
      \node[vertex] (a) at (3,3) {$b$};

      \draw (bot) -- (a);
      \draw (a)   -- (omega);
      \draw (bot) -- (0);
      \draw (0)   -- (1);
      \draw (1)   -- (2);
      \draw[dashed] (2) -- (omega);
    \end{tikzpicture}    
    \caption{Графично представяне на $\sqsubseteq$ върху $\A$}
    \label{fig:noncompact-element}
  \end{figure}
\end{framed}
\end{example}

\begin{example}
  Областта на Скот $\F_n$ е алгебрична.
  Компактните елементи са тези функции $f$, за които $|Dom(f)| < \infty$, т.е.
  крайните функции.   
\end{example}
\Stefan{Да се даде някакво обяснение!}


\begin{framed}
  \begin{thm}
    \label{th:compact-operator}
    Нека $\A$ и $\B$ са области на Скот, където $\A$ е {\em алгебрична}.
    Тогава $f \in \Cont{\A}{\B}$ точно тогава, когато за произволен елемент $a \in A$,
    \[f(a) = \bigsqcup\{f(c) \mid c \sqsubseteq a\ \&\ c \in K(\A)\}.\]
  \end{thm}
\end{framed}
\begin{proof}
  \marginpar{\cite[стр. 17]{barendregt-handbook}}
  \begin{enumerate}[(1)]
  \item
    Нека $f \in \Cont{\A}{\B}$ и да разгледаме произволен елемент $a \in A$.
    Нека $c$ е комапктен елемент, за който $c \sqsubseteq a$.
    \marginpar{Всяко непрекъснато изображение е монотонно}
    Тогава $f(c) \sqsubseteq f(a)$, защото $f$ е монотонно изображение.
    Това означава, че $f(a)$ е горна граница на множеството
    $\{f(c) \mid c \sqsubseteq a\ \&\ c\in K(\A)\}$.
    
    Нека сега $b$ е друга горна граница на $\{f(c) \mid c \sqsubseteq a\ \&\ c\in K(\A)\}$.
    Ще докажем, че $f(a) \sqsubseteq b$.

    Понеже $\A$ е алгебрична област на Скот, то $a = \bigsqcup_i c_i$, за някоя вергига $\chain{c}{i}$ от компактни елементи.
    Знаем, че $f(c_i) \sqsubseteq b$ за всеки компактен елемент $c_i \sqsubseteq a$.
    \marginpar{$\chain{f(c_i)}{i}$ образуват верига и следователно притежава точна горна граница}
    Тогава $\bigsqcup_i f(c_i) \sqsubseteq b$ и следователно
    $f(\bigsqcup_i c_i) \sqsubseteq b$, защото $f$ е непрекъснато изображение.
    Понеже $a = \bigsqcup_i c_i$, то получаваме, че $f(a) \sqsubseteq b$.

    От всичко това следва, че
    \[f(a) = \bigsqcup \{f(c) \mid c \sqsubseteq a\ \&\ c\in K(\A)\}.\]
  \item
    Сега да разгледаме обратната посока, т.е. нека $f$ е изображение, за което
    за произволен елемент $a \in A$ е изпълнено, че
    \[f(a) = \bigsqcup\{f(c) \mid c \sqsubseteq a\ \&\ c \in K(\A)\}.\]

    Нека първо да проверим, че $f$ е монотонно изображение.
    За целта, нека разгледаме произволни елементи $a,b\in A$, за които $a \sqsubseteq b$.
    Ясно е, че
    \[\{f(c) \mid c \sqsubseteq a\ \&\ c \in K(\A) \}\subseteq \{f(c) \mid c \sqsubseteq b\ \&\ c \in K(\A) \}.\]
    Оттук директно получаваме, че
    \begin{align*}
      f(a) & = \bigsqcup \{f(c) \mid c \sqsubseteq a\ \&\ c \in K(\A) \}\\
           & \sqsubseteq \bigsqcup\{f(c) \mid c \sqsubseteq b\ \&\ c \in K(\A) \}\\
           & = f(b).
    \end{align*}
    Така, щом $f$ е монотонно изображение, то можем да заключим, че
    за произволна верига $(a_i)^\infty_{i=0}$ е изпълнено
    \[\bigsqcup_i f(a_i) \sqsubseteq f(\bigsqcup_i a_i).\]

    За другата посока, да разгледаме произволна верига $(a_i)^\infty_{i=0}$.
    Тогава ако $c$ е компактен елемент и $c \sqsubseteq \bigsqcup_i a_i$,
    то съществува индекс $i_0$, за който $c \sqsubseteq a_{i_0}$.
    Понеже $f$ е монотонно изображение, то
    \[f(c) \sqsubseteq f(a_{i_0}) \sqsubseteq \bigsqcup_i f(a_i).\]
    Това означава, че елементът $\bigsqcup_i f(a_i)$
    е горна граница на множеството
    \[\{f(c) \mid c \sqsubseteq \bigsqcup_i a_i\ \&\ c \in K(\A)\}.\]
    Оттук заключаваме, че
    \[f(\bigsqcup_i a_i) = \bigsqcup\{f(c) \mid c \sqsubseteq \bigsqcup_i a_i\ \&\ c \in K(\A)\} \sqsubseteq \bigsqcup_i f(a_i).\]
  \end{enumerate}
\end{proof}


Използвайки факта, че $\F_n$ е алгебрична област на Скот, то имаме следната полезна харектеризация.
\begin{framed}
  \begin{cor}
    Следните условия са еквивалентни:
    \begin{enumerate}[(1)]
    \item
      $\Gamma \in \Cont{\F_n}{\F_m}$;
    \item
      $\Gamma(f)(\ov{x}) \simeq y \iff (\exists \theta \subseteq f)[\ \theta\text{ е крайна функция}\ \&\ \Gamma(\theta)(\ov{x}) \simeq y\ ]$.
    \end{enumerate}
  \end{cor}
\end{framed}

% \begin{proof}
%   $(1) \to (2)$. Да напомним, че
%   \[h = \bigcup\{\Gamma(\theta) \mid \theta \subseteq f\ \&\ \theta\text{ е крайна функция}\}\]
%     е функция, защото $\Gamma$ е монотонно изображение, и
%   \[h(\ov{x}) \simeq y \iff \exists \theta \subseteq f\ \&\ \theta\text{ е крайна функция}\ \&\ \Gamma(\theta)(\ov{x}) \simeq y.\]
  
%   От \Theorem{compact-operator} имаме, че
%   \[\Gamma(f) = h.\]
%   Тогава е ясно, че
%   \[\Gamma(f)(\ov{x}) \simeq y \iff (\exists \theta \subseteq f)[\ \theta\text{ е крайна функция}\ \&\ \Gamma(\theta)(\ov{x}) \simeq y\ ].\]
% \end{proof}

\marginpar{Понякога се оказва, че за проверката дали даден оператор $\Gamma$ е непрекъснат е по-лесно да се провери условието (2). Особено на упражнение ще се използва често характеризацията (2).}

\Stefan{Да се даде поне един пример!}

%%% Local Variables:
%%% mode: latex
%%% TeX-master: "../sep"
%%% End:

% \newpage
% \section{Задачи}

\begin{problem}
  \label{prob:sup-f}
  \marginpar{Достатъчно е всяка верига в $\B$ да се стабилизира}
  Нека $(f_i)^\infty_{i=0}$ е верига от елементи на $\Mapping{\A}{\B}$,
  където $\B$ е {\em плоска} област на Скот. Тогава:
  \begin{itemize}
  \item 
    $(\bigsqcup_i f_i)(a) = \bot \implies (\forall i)[\ f_{i}(a) = b\ ]$;
  \item 
    $(\bigsqcup_i f_i)(a) = b \neq \bot \implies (\exists i_0)(\forall i\geq i_0)[\ f_{i}(a) = b\ ]$.
  \end{itemize}
\end{problem}
% \begin{hint}
%   Първо да отбележем, че доказателството на \Th{all-mappings-is-domain} знаем, че
%   \[(\bigsqcup_i f_i)(\bar{a}) = \bigsqcup_i \{f_i(\bar{a})\}.\]
%   За посоката $(\Leftarrow)$, нека $f_{i_0}(\bar{a}) = b$, за някой индекс $i_0$.
%   Имаме, че
%   \begin{align*}
%     b = f_{i_0}(\bar{a}) & \sqsubseteq \bigsqcup_i \{f_i(\bar{a})\} \\
%                          & = (\bigsqcup_i f_i)(\bar{a}) & \comment{\text{от \Th{all-mappings-is-domain}}}.
%   \end{align*}
  
%   Получихме, че $b \sqsubseteq (\bigsqcup_i f_i)(\bar{a})$,
%   но понеже $\bot \neq b$, то $b = (\bigsqcup_i f_i)(\bar{a})$.

%   За посоката $(\Rightarrow)$, нека $(\bigsqcup_i f_i)(\bar{a}) = b \neq \bot$.
%   Отново използваме \Th{all-mappings-is-domain} и получаваме, че
%   $\bigsqcup_i\{f_i(\bar{a})\} = b$.
%   Очевидно е, че не е възможно $f_i(\bar{a}) = \bot$ за всяко $i$, защото тогава $\bigsqcup_i\{f_i(\bar{a})\} = \bot$.
%   Нека $i_0$ е първия индекс, за който $f_{i_0}(\bar{a}) = c\neq \bot$.
%   Понеже разглеждаме плоската наредба, ясно е, че $(\forall i \geq i_0)[f_i(\bar{a}) = c]$.
%   Тогава $\bigsqcup_i\{f_i(\bar{a})\} = c$, откъдето получаваме, че $c = b$.
% \end{hint}

\begin{problem}
  \label{prob:stab-continuous-finite}
  % Нека $\A_1$ е област на Скот, в която всяка верига се {\em стабилизира},
  % а $\A_2$ е {\em плоска} област на Скот.
  \marginpar{Достатъчно е всяка верига в $\B$ да се стабилизира}
  Нека $f \in \Mon{\A}{\B}$, където $\B$ е {\em плоска} област на Скот.
  Тогава за всяка верига $\chain{a}{i}$ в $\A$ е изпълнено, че:
  \begin{itemize}
  \item 
    $f(\bigsqcup_i a_i) = \bot \implies (\forall i)[\ f(a_i) = \bot\ ]$;
  \item
    $f(\bigsqcup_i a_i) = b \neq \bot \implies (\exists i_0)(\forall i\geq i_0)[\ f(a_i) = b\ ]$.
  \end{itemize}
\end{problem}
% \begin{hint}
%   За посоката $(\Leftarrow)$, от \Prop{continuous-is-monotone} знаем, че $f$ е монотонно изображение.
%   Нека $f(a_{i_0}) = b \neq \bot$.
%   Тогава, понеже $f$ е монотонна и $a_{i_0} \sqsubseteq_1 \bigsqcup_i a_i$,
%   имаме, че $f(a_{i_0}) = b \sqsubseteq_2 f(\bigsqcup_i a_i)$.
%   Щом $\sqsubseteq_2$ е плоската наредба и $b \neq \bot$, то получаваме, че $b = f(\bigsqcup_i a_i)$.

%   За посоката $(\Rightarrow)$, нека $f(\bigsqcup_i a_i) = b$.
%   Ако веригата $\chain{a}{i}$ се стабилизира от $a_{i_0}$ нататък, то $\bigsqcup_i a_i = a_{i_0}$.
%   Тогава $f(\bigsqcup_i a_i) = f(a_{i_0}) = b$.
% \end{hint}

% \begin{cor}
%   Нека $f \in \Cont{\Nat^n_\bot}{\Nat_\bot}$.
%   Тогава за всяка верига $\chain{\bar{a}}{i}$ в $\Nat^n_\bot$, и всяко $b \neq \bot$,
%   \[f(\bigsqcup_i \bar{a}_i) = b \iff (\exists i_0)[f(\bar{a}_{i_0}) = b].\]
% \end{cor}


\begin{problem}
  \label{pr:composition}
  \index{изображения!композиция}
  Ако $f \in \Cont{\A}{B}$ и $g \in \Cont{\B}{\C}$, то $g \circ f \in \Cont{\A}{\C}$,
  където \[(g\circ f)(a) \dff g(f(a)).\]
\end{problem}

\begin{problem}
  Докажете, че ако $\A$ и $\B$ са алгебрични области на Скот, то
  $\A \times \B$ също е алгебрична област на Скот.
\end{problem}

\begin{problem}
  Нека е даден следния оператор $\Gamma:\F^\bot_1\to\F^\bot_1$:
  \begin{align*}
    \Gamma(f)(x) =
    \begin{cases}
      \bot, & |Dom(f)| < \infty\\
      1, & |Dom(f)| = \infty.
    \end{cases}
  \end{align*}
  Проверете дали:
  \begin{enumerate}[a)]
  \item 
    $\Gamma$ е монотонен оператор;
  \item
    $\Gamma$ е компактен оператор.
  \end{enumerate}
\end{problem}
\begin{solution}
  \begin{enumerate}[a)]
  \item 
    Трябва да проверим дали:
    \[(\forall f,g\in\F^\bot_1)[f \sqsubseteq g \implies \Gamma(f) \sqsubseteq \Gamma(g)].\]
    Нека $f \sqsubseteq g$ са произволни елементи от $\F^\bot_1$.
    Ще разгледаме два случая.
    \begin{itemize}
    \item 
      $f$ е крайна функция. Тогава $\Gamma(f) = \lambda x.\bot$ и е очевидно, че 
      \[\Gamma(f) \sqsubseteq \Gamma(g).\]
    \item
      $f$ не е крайна функция. Щом $f \sqsubseteq g$, то $g$ също не е крайна функция.
      Тогава 
      \[(\forall x \in \Nat_\bot)[\Gamma(f)(x) = 1 = \Gamma(g)(x)],\]
      от което следва, че 
      \[\Gamma(f) \sqsubseteq \Gamma(g).\]
    \end{itemize}
    Разгледахме всички възможни случаи за $f$ и във всеки от тях получихме, че $\Gamma(f) \sqsubseteq \Gamma(g)$.
    Следователно, $\Gamma$ е монотонен оператор.
  \item
    Според \Prop{operator-compact}, достатъчно е да докажем, че за произволни елементи $f \in \F^\bot_1$, $x, y \in \Nat_\bot$, 
    \begin{equation}
      \label{eq:compact}
      \Gamma(f)(x) = y\ \implies\ (\exists \theta \sqsubseteq f)[\theta\text{ е крайна }\&\ \Gamma(\theta)(x) = y]].
    \end{equation}
    Нека $f$ е не е крайна функция.
    Тогава е ясно, че за всяко $x \in \Nat_\bot$, $\Gamma(f)(x) = 1$.
    От друга страна, понеже $\theta$ е крайна, $\Gamma(\theta)(x) = \bot$ за всяко $x \in \Nat_\bot$.
    Така видяхме, че ако $f$ не е крайна, то за произволна $\theta \sqsubseteq f$ и произволно $x \in \Nat_\bot$,
    $\Gamma(\theta)(x) \neq 1$.
    От това следва, че Формула (\ref{eq:compact}) не е изпълнена и тогава $\Gamma$ не е компактен оператор.  
  \end{enumerate}
\end{solution}


\begin{problem}
  Нека е даден следния оператор $\Gamma:\F^\bot_2\to\F^\bot_2$:
  \begin{align*}
    \Gamma(f)(x,y) = &
    \begin{cases}
      y, & x = 0\\
      f(x, f(x-1,y)), & x > 0\\
      \bot, & x = \bot.
    \end{cases}
  \end{align*}
  \begin{enumerate}[a)]
  \item 
    Докажете, че $\Gamma$ е компактен оператор.
  \item
    Намерете $\lfp(\Gamma)$.
  \item
    Има ли $\Gamma$ други неподвижни точки ?
  \end{enumerate}
\end{problem}

\begin{problem}
  Монотонен ли е операторът $\Gamma:\Strict{\Nat_\bot}{\Nat_\bot} \to \Nat_\bot$, където:
  \begin{align*}
    \Gamma(f) =
    \begin{cases}
      n, & |Dom(f)| = n\\
      \bot, & |Dom(f)| = \infty\\
    \end{cases}
  \end{align*}
\end{problem}

\begin{problem}
  Какви свойства има оператора $\Gamma:\F^\bot_1\times\F^\bot_1 \to \F^\bot_1$, където:
  \begin{align*}
    \Gamma(f,g)(x) =
    \begin{cases}
      g(x), & f \sqsubseteq g\\
      f(x), & g \sqsubseteq f\ \&\ f \not\sqsubseteq g\\
      \bot, & \text{иначе}.
    \end{cases}
  \end{align*}
\end{problem}

\begin{problem}
  \marginpar{\cite[стр. 122]{nikolova-soskova}}
  Нека разгледаме $f \in \Mon{\Nat_\bot}{\Nat_\bot}$.
  Съобразете, че $\lfp(f) = f(\bot)$.
\end{problem}


\begin{problem}
  Знаем, че всяко изображение $f \in \Cont{\A}{\A}$ притежава най-малка неподвижна точка.
  \begin{itemize}
  \item 
    Вярно ли е, че съществуват изображения от вида $f:\A\to\A$, които са монотонни, {\em не} са непрекъснати, но въпреки това притежават 
    най-малка неподвижна точка?
  \item
    \marginpar{\cite[стр. 131]{nikolova-soskova}}
    Вярно ли е, че съществуват изображения от вида $f:\A\to\A$, които са не са монотонни, но въпреки това притежават най-малка неподвижна точка?
  \end{itemize}
  Дайте примери за такава области на Скот $\A$ и изображения $f:\A \to \A$.
  Обосновете заключенията си.
\end{problem}

%%% Local Variables:
%%% mode: latex
%%% TeX-master: "../sep"
%%% End:

\section{Изоморфни области на Скот}
\index{изоморфизъм}

Нека $\A_1 = (A_1,~\sqsubseteq_1,~\bot_1)$ и $\A_2 = (A_2,~\sqsubseteq_2~,~\bot_2)$ 
са области на Скот.
Ще казваме, че $\A_1$ е {\bf изоморфна} на $\A_2$, което ще означаваме като 
\[\A_1 \cong \A_2,\]
ако съществува {\em биективна} функция $F:A_1 \to A_2$ със свойството:
\[(\forall a,b\in A_1)[\ a \sqsubseteq_1 b \iff F(a) \sqsubseteq_2 F(b)\ ].\]
В такъв случай ще казваме, че $F$ задава изоморфизъм между $\A_1$ и $\A_2$.

Когато искаме да означим, че $\A_1$ е изоморфна на $\A_2$ чрез $F$,
то понякога ще пишем $\A_1 \cong_F \A_2$.

\begin{problem}
  Докажете, че ако $\A_1 \cong_F \A_2$, то $F(\bot_1) = \bot_2$.
\end{problem}


\begin{proposition}
  \label{pr:isomorphism-is-continuous}
  Ако $\A_1 \cong_F \A_2$ , то $F \in \Cont{\A_1}{\A_2}$.
\end{proposition}
\begin{hint}
  Да разгледаме произволна верига $\chain{a}{i}$ от елементи на $\A_1$.
  Ще докажем, че 
  \[F(\bigsqcup_i a_i) = \bigsqcup_iF(a_i).\]
  
  \begin{itemize}
  \item 
    Първо, от дефиницията веднага следва, че $F$ е монотонно изображение, защото
    \[a \sqsubseteq_1 b \implies F(a) \sqsubseteq_2 F(b).\]
    Това означава, че $(F(a_i))^\infty_{i=0}$ е монотонно растяща верига от елементи на $\A_2$.
    От \Prop{monotone-chain} получаваме, че 
    \[\bigsqcup_i F(a_i) \sqsubseteq_2 F(\bigsqcup_i a_i).\]
  \item
    За другата посока, нека $b \in \A_2$ е горна граница на веригата $(F(a_i))^\infty_{i=0}$, т.е. 
    \[(\forall i)[\ F(a_i) \sqsubseteq_2 b\ ].\]
    Ще докажем, че $F(\bigsqcup_i a_i) \sqsubseteq_2 b$.
    Понеже $F$ е {\em върху} $A_2$, то съществува елемент $a \in A_1$, такъв че $F(a) = b$.
    Тогава:
    \begin{align*}
      (\forall i)[\ F(a_i) \sqsubseteq_2 F(a)\ ] & \implies (\forall i)[\ a_i \sqsubseteq_1 a\ ] & \comment{F \text{ е изоморфизъм }}\\
                                                 & \implies \bigsqcup_i a_i \sqsubseteq_1 a & \comment{a\text{ е горна граница}}\\
                                                 & \implies F(\bigsqcup_i a_i) \sqsubseteq_1 F(a). & \comment{F\text{ е изоморфизъм }}
    \end{align*}
    Понеже $b = F(a)$, заключаваме, че
    \[F(\bigsqcup_i a_i) \sqsubseteq_2 b.\]
  \end{itemize}
\end{hint}

\begin{proposition}
  \label{pr:isomorphic-pair}
  Нека $f \in \Mon{\A_1}{\A_2}$ и $g \in \Mon{\A_2}{\A_1}$,
  като 
  \begin{itemize}
  \item 
    $f \circ g = \texttt{id}_2$;
  \item
    $g \circ f = \texttt{id}_1$.
  \end{itemize}
  \marginpar{$\texttt{id}_i(a) \dff a$ за вс. $a \in \A_i$}
  Тогава са изпълнени свойствата:
  \begin{enumerate}[(1)]
  \item
    $\A_1 \cong_f \A_2$;
  \item
    $\A_2 \cong_g \A_1$;
  \end{enumerate}
\end{proposition}
% \begin{hint}
%   За Свойство $(1)$ трябва да проверим, че $f$ отговаря на дефиницията за изоморфизъм.
%   \begin{itemize}
%   \item
%     Ще докажем, че $f$ е инективна като покажем, че за произволни $a, b\in A_1$,
%     ако $f(a) = f(b)$, то $a = b$.
%     Но това е лесно, защото
%     \[a = \texttt{id}_1(a) = g(f(a)) = g(f(b)) = \texttt{id}_1(b) = b.\]
%   \item
%     Нека сега $b \in \A_2$.
%     Знаем, че $f(g(b)) = \texttt{id}_2(b) = b$. Това означава, че $f$ е {\em сюрективна},
%     защото за всеки елемент $b \in A_2$ съществува елемент $a \in A_1$, а именно $a = g(b)$,
%     за който $f(a) = b$.
%   \item
%     Понеже $f$ е монотонно изображение, то директно имаме, че
%     \[a \sqsubseteq_1 b \implies f(a) \sqsubseteq_2 f(b).\]
%   \item
%     Нека $f(a) \sqsubseteq_2 f(b)$.
%     Сега пък понеже $g$ е монотонно изображение, 
%     \[a = \texttt{id}_1(a) = g(f(a)) \sqsubseteq_1 g(f(b)) = \texttt{id}_1(b) = b.\]
%     Така показахме, че
%     \[f(a) \sqsubseteq_2 f(b)\ \implies\ a \sqsubseteq_1 b.\]
%   \end{itemize}
%   Доказахме Свойство $(1)$, т.е. $\A_1 \cong_f \A_2$.
%   Разсъжденията за Свойство $(2)$ са аналогични.
% \end{hint}


\begin{proposition}
  \label{pr:isomorphic-higher-order}
  Нека $\A_1 \cong_F \A_2$. Тогава:
  \begin{enumerate}[(1)]
  \item 
    $\Cont{\A_1}{\A_1} \cong_G \Cont{\A_2}{\A_2}$, където 
    \[G(f) \dff F \circ f \circ F^{-1};\]
    Графично това може да се изобрази така:

    \shorthandoff{"}%
    \begin{center}
    \begin{tikzcd}[sep=large]
      \A_1 \arrow[r, "f"] & \A_1 \arrow[d, "F"]\\
      \A_2 \arrow[u, "F^{-1}"]\arrow[r, dashed, "G(f)"] & \A_2 
    \end{tikzcd}
    \end{center}
    \shorthandon{"}%
  \item
    ако $f \in \Cont{\A_1}{\A_1}$, то 
    \[F(\lfp(f)) = \lfp(G(f)).\]
  \end{enumerate}
\end{proposition}
\begin{hint}
  Ще докажем $(1)$ като използвме \Prop{isomorphic-pair}.

  \begin{itemize}
  \item 
    Ще докажем, че $G$ е монотонно изображение.
    Нека $f,h \in \Cont{\A_1}{\A_1}$ и $f \sqsubseteq h$, т.е.
    \[(\forall a \in \A_1)[\ f(a) \sqsubseteq_1 h(a)\ ].\]
    Ще докажем, че $G(f) \sqsubseteq G(h)$, т.е.
    \[(\forall b \in \A_2)[\ G(f)(b) \sqsubseteq_1 G(h)(b)\ ].\]
    Да разгледаме произволен елемент $b \in \A_2$. 
    Понеже $F$ е биекция, то съществува елемент $a \in A_1$, такъв че $F(a) = b$,
    т.е. $F^{-1}(b) = a$. Тогава:
    \begin{align*}
      G(f)(b) & \dff F(f(F^{-1}(b)))\\
              & = F(f(a)) & \comment{F^{-1}(b) = a}\\
              & \sqsubseteq_2 F(h(a)) & \comment{f(a) \sqsubseteq h(a)\text{ и $F$ е изом.}}\\
              & = F(h(F^{-1}(b))) & \comment{F^{-1}(b) =a}\\
              & \dff G(h)(b).
    \end{align*}
  \item
    Нека $G(f) \sqsubseteq G(h)$. Ще докажем, че $f \sqsubseteq h$.
    За целта, нека $a \in A_1$.
    Понеже $F$ е сюрективна, то съществува $b \in A_2$, за който $F^{-1}(b) =a$.
    Понеже
    \[G(f) \dff F \circ f \circ F^{-1} \sqsubseteq F \circ h \circ F^{-1} = G(h),\]
    то получаваме, че
    \[F(f(F^{-1}(b))) \sqsubseteq_2 F(h(F^{-1}(b))).\]
    Оттук,
    \[F(f(a)) \sqsubseteq_2 F(h(a)) \implies f(a) \sqsubseteq_1 h(a),\]
    защото $F$ е изоморфизъм.
  \end{itemize}
  Сега преминаваме към доказателството на $(2)$.
  Да напомним, че за $f \in \Cont{\A_1}{\A_1}$, означаваме
  \begin{align*}
    f^0  & = \lambda x. \bot_1\\
    f^{n+1} & = f \circ f^n.
  \end{align*}
  Понеже $f$ е непрекъснато изображение е ясно, че $(f^n(\bot_1))^{\infty}_{n=0}$ е верига.
  Също така знаем, че
  \[\lfp(f) = \bigsqcup_n f^n(\bot_1).\]
  След аналогични разсъждения можем да съобразим, че
  \[\lfp(G(f)) = \bigsqcup_n G(f)^n(\bot_2).\]
  Първо ще докажем с индукция по $n$, че 
  \begin{equation}
    \label{eq:2}
    (\forall n)[\ (G(f))^n = G(f^n)\ ].
  \end{equation}
  \begin{itemize}
  \item 
    За $n = 0$ имаме, че за произволен елемент $b \in \A_2$,
    \begin{align*}
      (G(f))^{0}(b) & \dff \bot_2\\
                    & = F(\bot_1) & \comment{F \text{ е изом.}}\\
                    & = F(f^{0}(F^{-1}(b))) & \comment{f^{0}(F^{-1}(b)) \dff \bot_1}\\
                    & = (F \circ f^{0} \circ F^{-1})(b) \\
                    & \dff G(f^{0})(b).
    \end{align*} 
  \item
    Нека да приемем, че твърдението е вярно за $n$.
    Тогава за $n+1$ имаме, че:
    \begin{align*}
      (G(f))^{n+1} & \dff G(f) \circ (G(f))^n\\
                   & = G(f) \circ G(f^n) & \comment{\text{ от И.П.}}\\
                   & \dff (F \circ f \circ F^{-1}) \circ (F\circ f^n \circ F^{-1})\\
                   & = F \circ f \circ (F^{-1} \circ F)\circ f^n \circ F^{-1} \\
                   & = F \circ f \circ f^{n} \circ F^{-1} & \comment{F^{-1}\circ F = id}\\
                   & = F \circ f^{n+1} \circ F^{-1} & \comment{f\circ f^n = f^{n+1}}\\
                   & \dff G(f^{n+1}).
    \end{align*}
  \end{itemize}
  Тогава:
  \begin{align*}
    F(\lfp(f)) & = F(\bigsqcup_n f^n(\bot_1)) & \comment{\lfp(f) = \bigsqcup_n f^n(\bot_1)}\\
               & = \bigsqcup_n F(f^n(\bot_1))& \comment{F\text{ е непр.}}\\
               & = \bigsqcup_n F(f^n(F^{-1}(\bot_2))) & \comment{F^{-1}(\bot_2) = \bot_1}\\
               & = \bigsqcup_n (F \circ f^n \circ F^{-1})(\bot_2) \\
               & \dff \bigsqcup_n G(f^n)(\bot_2)\\
               & = \bigsqcup_n G(f)^n(\bot_2) & \comment{\text{от }(\ref{eq:2})}\\
               & = \lfp(G(f)).
  \end{align*}
\end{hint}

\begin{framed}
  \begin{proposition}
    За произволни области на Скот $\A$, $\B$ и $\C$ е изпълнено, че
    \[\Cont{\A}{\Cont{\B}{\C}}\ \cong\ \Cont{\A\times\B}{\C}.\]
  \end{proposition}  
\end{framed}
\begin{hint}
  % \begin{itemize}
  % \item 
    Докажете, че изображението
    \[\texttt{curry}:\Cont{\A\times \B}{\C} \to \Cont{\A}{\Cont{\B}{\C}},\]
    където
    \[\texttt{curry}(f)(a)(b) \dff f(a,b)\]
    задава изоморфизъм.
  % \item
  %   Докажете, че изображението
  %   \[\texttt{uncurry}:\Cont{\A}{\Cont{\B}{\C}} \to \Cont{\A\times \B}{\C},\]
  %   където
  %   \[\texttt{uncurry}(f)(a,b) \dff f(a)(b)\]
  %   е монотонно.
  % \item
  %   Лесно се съобразява, че
  %   \[\texttt{curry} \circ \texttt{uncurry} = \texttt{id}\]
  %   \[\texttt{uncurry} \circ \texttt{curry} = \texttt{id}.\]    
  % \item
  %   Приложете \Prop{isomorphic-pair}.
  % \end{itemize}
\end{hint}

\begin{remark}
  Когато на хаскел пишем типовата сигнатура на някоя функция като 
  \mint{haskell}|f :: a -> b -> c|
  в действителност се има предвид следното
  \mint{haskell}|f :: a -> (b -> c)|
  
  На практика тези две задачи ни казват, че няма значение дали използваме {\em curried}
  или {\em uncurried} версията на една функция. На хаскел е по-удобно да използваме {\em curried}
  версията, защото като фиксираме първия аргумент на една функция получаваме нова функция наготово.
  Например, 
  \begin{haskellcode}
ghci> let plus x y = x + y
ghci> :t plus
plus :: Num a => a -> a -> a
ghci> let plus1 = plus 1
ghci> :t plus1
plus1 :: Num a => a -> a
   \end{haskellcode}

  Всъщност, хаскел има функциите \texttt{curry} и \texttt{uncurry} вградени в стандартната библиотека:
  \begin{haskellcode}
ghci> :t curry
curry :: ((a, b) -> c) -> a -> b -> c
ghci> :t uncurry
uncurry :: (a -> b -> c) -> (a, b) -> c
  \end{haskellcode}
\end{remark}

Нека да дефинираме
\[\emptyset_\bot = (\{\bot\}, \sqsubseteq, \bot).\]
\begin{problem}
  Докажете, че за произволна област на Скот $\A$ е изпълнено:
  \begin{align*}
    & \Mapping{\emptyset_\bot}{\A} \cong \A\\
    & \Mapping{\A}{\emptyset_\bot} \cong \emptyset_\bot.
  \end{align*}
\end{problem}

\begin{problem}
  Докажете, че съществуват области на Скот $\A$, $\B$ и $\C$, за които
  \[\Cont{\Cont{\A}{\B}}{\C} \not\cong \Cont{\A}{\Cont{\B}{\C}}.\]  
\end{problem}
\begin{hint}
  Нека изберем $\A = \C = \Nat_\bot$, а $\B = \emptyset_\bot$. Тогава
  \[\Cont{\Cont{\Nat_\bot}{\emptyset_\bot}}{\Nat_\bot} \cong \Cont{\emptyset_\bot}{\Nat_\bot} \cong \Nat_\bot,\]
  но
  \[\Cont{\Nat_\bot}{\Cont{\emptyset_\bot}{\Nat_\bot}} \cong \Cont{\Nat_\bot}{\Nat_\bot}.\]
\end{hint}

%%% Local Variables:
%%% mode: latex
%%% TeX-master: "../sep"
%%% End:

\newpage
\section{Допълнителен материал}
\section{Допълнителен материал}
\subsection{Регулярни езици}\index{език!регулярен}

Да фиксираме азбуката $\Sigma = \{a_1,\dots,a_k\}$.
Да дефинираме полиномите над $\Sigma$ като
\[\tau ::= \emptyset\ |\ \varepsilon\ |\ a_i \cdot X_j\ |\ \tau_1 + \tau_2.\]
където $i = 1, \dots,k$, а $X$ е променлива.
За всеки полином $\tau[X_1,\dots,X_n]$ дефинираме оператора 
\[\val{\tau}: \mathcal{P}(\Sigma^\star)^n \to \mathcal{P}(\Sigma^\star)\]
 по следния начин:
\begin{itemize}
\item
    $\val{\emptyset}(L_1,\dots,L_n) = \emptyset$.
\item 
  $\val{\varepsilon}(L_1,\dots,L_n) = \varepsilon$.
\item 
  $\val{a_i \cdot X_j}(L_1,\dots,L_n) = \{a_i\} \cdot L_j$.
\item
  $\val{\tau_1 + \tau_2}(L_1,\dots,L_n) = \val{\tau_1}(L_1,\dots,L_n) \cup \val{\tau_2}(L_1,\dots,L_n)$.
\end{itemize}

\begin{problem}
  Докажете, че за всеки полином $\tau$ имаме, че $\val{\tau}$ е непрекъснато изображение в областта на Скот
  $\mathcal{S} = ( \mathcal{P}(\Sigma^\star),\subseteq, \emptyset)$.
\end{problem}


\begin{example}
  Да разгледаме системата 
  \marginpar{$\tau_1[X_1,X_2] \equiv b \cdot X_1 + a \cdot X_2$}
  \marginpar{$\tau_2[X_1,X_2] \equiv \varepsilon$}
  \begin{align*}
    & X_1 = b \cdot X_1 + a\cdot X_2\\
    & X_2 = \varepsilon.
  \end{align*}

  % Понеже $\val{\tau}$ е непрекъснат оператор, то той има най-малка неподвижна точка.
  Дефинираме непрекъснатия оператор 
  \[\Gamma:\mathcal{P}(\Sigma^\star)^2 \to \mathcal{P}(\Sigma^\star)^2,\]
  където:
  \[\Gamma(L_1,L_2) = (\val{\tau_1}(L_1,L_2), \val{\tau_2}(L_1,L_2)).\]

  От Теоремата на Клини ние знам как можем да намерим най-малката неподвижна точка на $\Gamma$,
  която ще бъде и най-малкото решение на горната система.

  \begin{itemize}
  \item 
    $(L_0,M_0) \df (\emptyset,\emptyset)$;
  \item
    $(L_1,M_1) \df \Gamma(L_0,M_0) = (\val{\tau_1}(L_0,M_0), \val{\tau_2}(L_0,M_0)) = (\emptyset, \varepsilon)$;
  \item
    $(L_2,M_2) \df \Gamma(L_1,M_1) = (\val{\tau_1}(L_1,M_1), \val{\tau_2}(L_1,M_1)) = (\{a\},\varepsilon)$;
  \item
    $(L_3,M_3) \df \Gamma(L_2,M_2) = (\val{\tau_1}(L_2,M_2), \val{\tau_2}(L_2,M_2)) = (\{ba,a\},\varepsilon)$;
  \item
    $(L_4,M_4) \df \Gamma(L_3,M_3) =(\val{\tau_1}(L_3,M_3), \val{\tau_2}(L_3,M_3)) = (\{bba, ba,a\},\varepsilon)$;
  \item
    $(L_5,M_5) \df \Gamma(L_4,M_4) = ( \val{\tau_1}(L_4,M_4), \val{\tau_2}(L_4,M_4)) = (\{bba, bba, ba,a\},\varepsilon)$.
  \end{itemize}
  Лесно се съобразява, че $L_n = \{ b^ka \mid k < n\}$.
  Тогава
  \[\lfp( \Gamma ) = (\bigcup_n L_n, \{\varepsilon\}) = (b^\star a, \{\varepsilon\} ).\]
\end{example}


\begin{problem}
  Докажете, че най-малкото решение на системата 
  \begin{align*}
    & X_1 = a \cdot X_1 + b \cdot X_2 + \varepsilon\\
    & X_2 = b \cdot X_2 + \varepsilon
  \end{align*}
  е двойката $(a^\star b^\star, b^\star)$.
\end{problem}

\begin{problem}
  Да разгледаме системата от оператори
  \begin{align*}
    & \val{\tau_1}(L_1,\dots,L_n) = L_1\\
    & \ \ \vdots\\
    & \val{\tau_n}(L_1,\dots,L_n) = L_n.
  \end{align*}
  Знаем, че тя притежава най-малко решение $(\hat{L}_1,\dots,\hat{L}_n)$.
  Докажете, че всеки от езиците $\hat{L}_i$ е регулярен.

  Докажете, че всеки регулярен език е елемент от най-малкото решение 
  на някоя система от оператори от горния вид.
\end{problem}

\subsection{Безконтекстни езици}\index{език!безконтекстен}

Да фиксираме азбуката $\Sigma = \{a_1,\dots,a_n\}$.
Да дефинираме термове от тип 1 като
\[\tau ::= X_i\ |\ a_j\ |\ \varepsilon\ |\ \emptyset\ |\ \tau_1 \cdot \tau_2\ |\ (\tau_1 + \tau_2),\]
където $j = 1, \dots,n$, а $X_i$ са изброимо безкрайна редица от променливи.
За всеки терм $\tau[X_1,\dots,X_n]$ дефинираме оператора 
\[\val{\tau}: (\mathcal{P}(\Sigma^\star))^n \to \mathcal{P}(\Sigma^\star)\]
 по следния начин:
\begin{itemize}
\item 
  $\val{X_i}(L_1,\dots,L_n) = L_i$.
\item 
  $\val{a_j}(L_1,\dots,L_n) = \{a_j\}$.
\item 
  $\val{\varepsilon}(L_1,\dots,L_n) = \varepsilon$.
\item 
  $\val{\emptyset}(L_1,\dots,L_n) = \emptyset$.
\item 
  $\val{\tau_1 \cdot \tau_2}(L_1,\dots,L_n) = \val{\tau_1}(L_1,\dots,L_n) \cdot \val{\tau_2}(L_1,\dots,L_n)$.
\item
  $\val{\tau_1 + \tau_2}(L_1,\dots,L_n) = \val{\tau_1}(L_1,\dots,L_n) \cup \val{\tau_2}(L_1,\dots,L_n)$.
\end{itemize}

\begin{problem}
  Докажете, че за всеки терм $\tau$, $\val{\tau}$ е непрекъснато изображение в областта на Скот
  $\mathcal{S} = ( \mathcal{P}(\Sigma^\star),\subseteq, \emptyset)$.
\end{problem}

\begin{problem}
  Докажете, че $\{a^nb^n \mid n\in \Nat\} = \lfp(\val{\tau})$, където 
  \[\tau[X] \equiv \varepsilon + a \cdot X \cdot b.\]
  С други думи, $\{a^nb^n \mid n \in \Nat\}$ е най-малкото решение на уравнението
  \[X = a \cdot X \cdot b + \varepsilon.\]
\end{problem}

Нека сега да разгледаме термовете $\tau_1[X_1,\dots,X_n], \dots, \tau_n[X_1,\dots,X_n]$.

\begin{problem}
  Да разгледаме системата от оператори
  \begin{align*}
    & \val{\tau_1}(L_1,\dots,L_n) = L_1\\
    & \ \ \vdots\\
    & \val{\tau_n}(L_1,\dots,L_n) = L_n.
  \end{align*}
  Знаем, че тя притежава най-малко решение $(\hat{L}_1,\dots,\hat{L}_n)$.
  Докажете, че всеки от езиците $\hat{L}_i$ е безконтекстен.

  Докажете, че всеки безконтекстен език е елемент от най-малкото решение 
  на някоя система от оператори от горния вид.
\end{problem}

\begin{problem}
  \marginpar{Това е аналог на нормалната форма на Чомски}
  Да дефинираме термове от тип 2 като
  \[\tau ::= a_j\ |\ \varepsilon\ |\ \emptyset\ |\ X_i \cdot X_k\ |\ (\tau_1 + \tau_2),\]
  където $j = 1, \dots,n$, а $X_i$ са изброимо безкрайна редица от променливи.
  Докажете горното твърдение, като замените термовете от тип 1 с тези от тип 2.
\end{problem}

\begin{example}
  Да разгледаме системата
  \begin{align*}
    & X_1 = X_3 \cdot X_2 + \varepsilon\\
    & X_2 = X_1 \cdot X_4\\
    & X_3 = a\\
    & X_4 = b.
  \end{align*}


  % \begin{align*}
  %   & \val{\varepsilon + X_3 \cdot X_2}(L_1, L_2, L_3, L_4) = L_1\\
  %   & \val{X_1 \cdot X_4}(L_1, L_2, L_3, L_4) = L_2\\
  %   & \val{a}(L_1, L_2, L_3, L_4) = L_3\\
  %   & \val{b}(L_1, L_2, L_3, L_4) = L_4\\
  % \end{align*}
  Нека $(\hat{L}_1, \hat{L}_2, \hat{L}_3, \hat{L}_4)$ е най-малкото решение на системата.
  Докажете, че $\hat{L}_1 = \{a^nb^n\mid n \in \Nat\}$ $\hat{L}_2 = \{a^nb^{n+1}\mid n \in \Nat\}$,
  $\hat{L}_3 = \{a\}$ и $\hat{L}_4 = \{b\}$.
\end{example}

%%% Local Variables:
%%% mode: latex
%%% TeX-master: "../sep"
%%% End:




%%% Local Variables:
%%% mode: latex
%%% TeX-master: "../sep"
%%% End:

% \chapter{Допълнителни свойства на областите на Скот}

\section{Област на Скот от непрекъснати изображения}

Следващата теорема е важна, защото с нейна помощ се доказват много свойства на непрекъснатите изображения.

\begin{thm}
  \label{th:double-chain}
  \marginpar{\cite[стр. 127]{winskel}}
  \marginpar{\cite[стр. 178]{models-of-computation}}
  Нека $\A = (A,\sqsubseteq,\bot)$ да бъде област на Скот и нека множеството 
  \[E = \{a_{m,n} \mid m,n \in \Nat\}\]
  от елементи на $A$ притежава свойството, че 
  \[n \leq n^\prime\ \&\ m \leq m^\prime\ \Rightarrow\ a_{n,m} \sqsubseteq a_{n^\prime,m^\prime}.\]
  Тогава множеството $E$ има точна горна граница, която означаваме с $\bigsqcup E$, и са изпълнени равенствата
  \[\bigsqcup E = \bigsqcup_m(\bigsqcup_n a_{n,m}) = \bigsqcup_n(\bigsqcup_{m} a_{n,m}) = \bigsqcup_n a_{n,n}.\]
\end{thm}
\begin{proof}
  Първо ще въведем някои означения.
  \begin{itemize}
  \item 
    Да фиксираме произволно $m$. Тогава можем да подредим елементите на множеството $\{a_{n,m} \mid n \in \Nat\}$ във възходящ ред: 
    \[a_{0,m} \sqsubseteq a_{1,m} \sqsubseteq a_{2,m} \sqsubseteq \cdots\]
    \marginpar{По дефиниция, всяка монотонно растяща редица в област на Скот притежава точна горна граница.}
    Следователно тя има точна горна граница $b_m \dff \bigsqcup \{a_{n,m} \mid n \in \Nat\}$.
  \item
    Аналогично, при фиксирано $n$, можем да подредим елементите на множеството $\{a_{n,m} \mid m \in \Nat\}$ в монотонно растяща редица:
    \[a_{n,0} \sqsubseteq a_{n,1} \sqsubseteq a_{n,2} \sqsubseteq \ldots,\]
    която притежава точна горна граница $c_n \dff \bigsqcup \{a_{n,m} \mid m \in \Nat\}$.
  \end{itemize}
  Това означава, че трябва да докажем следното:
  \[\bigsqcup E = \bigsqcup_mb_m = \bigsqcup_n c_n = \bigsqcup_n a_{n,n}.\]
  \begin{enumerate}[1)]
  \item 
    Първо да съобразим, че множеството $\{b_m \mid m \in \Nat\}$ образува верига в $\A$ и следователно притежава точна горна граница $\bigsqcup_m b_m$.
    Нека да разгледаме произволни $m \leq m^\prime$.
    Тогава \[(\forall n)[a_{n,m} \sqsubseteq a_{n,m^\prime} \sqsubseteq \bigsqcup_k a_{k,m^\prime} = b_{m^\prime}].\]
    Следователно $b_{m^\prime}$ е горна граница на веригата $(a_{n,m})^{\infty}_{n=0}$ и понеже $b_m$ е точна горна граница на $(a_{n,m})^{\infty}_{n=0}$, то получаваме, че \[b_m \sqsubseteq b_{m^\prime}.\]
    Това означава, че $\chain{b}{m}$ е верига в $\A$ и тя притежава точна горна граница $\bigsqcup_m b_m$.  
  \item
    С подобни разсъждения можем да докажем, че множеството $\{c_n \mid n \in \Nat\}$ образува верига в $\A$, която притежава точна горна граница $\bigsqcup_n c_n$.
  \item
    Сега ще докажем, че \[\bigsqcup_m b_m = \bigsqcup_n c_n.\]
    Имаме, че 
    \[(\forall m)(\forall n)[a_{n,m} \sqsubseteq \bigsqcup_na_{n,m} = b_m \sqsubseteq \bigsqcup_m b_m],\]
    което е еквивалентно на 
    \[(\forall n)(\forall m)[a_{n,m} \sqsubseteq b_m \sqsubseteq \bigsqcup_m b_m].\]
    Да фиксираме произволно $n$.
    Тогава $\bigsqcup_m b_m$ е горна граница на веригата $(a_{n,m})^\infty_{m=0}$.
    Следователно, $c_n = \bigsqcup_m a_{n,m} \sqsubseteq \bigsqcup_m b_m$.
    Така получаваме, че $\bigsqcup_m b_m$ е горна граница и на веригата $\chain{c}{n}$
    и тогава \[\bigsqcup_n c_n \sqsubseteq \bigsqcup_m b_m.\]
    С аналогични разсъждения можем да докажем също, че 
    \[\bigsqcup_m b_m \sqsubseteq \bigsqcup_n c_n.\]
    Така доказахме, че \[\bigsqcup_m b_m = \bigsqcup_n c_n.\]
  \item
    Сега ще докажем, че \[\bigsqcup E = \bigsqcup_m b_m.\]
    За целта ще проверим следното:
    \begin{enumerate}[a)]
    \item 
      $\bigsqcup_m b_m$ е горна граница на $E$.
    \item
      Ако $d$ е друга горна граница на $E$, то $\bigsqcup_m b_m \sqsubseteq d$.
    \end{enumerate}
    \begin{itemize}
    \item 
      $\bigsqcup_m b_m$ е горна граница на $E$, защото
      \[(\forall m)(\forall n)[a_{n,m} \sqsubseteq b_m \sqsubseteq \bigsqcup_m b_m].\]
    \item
      Нека $d$ е друга горна граница на $E$, т.е.
      \[(\forall m)(\forall n)[a_{n,m} \sqsubseteq d].\]
      Да фиксираме произволно $m$.
      Тогава $d$ е горна граница на веригата $(a_{n,m})^\infty_{n=0}$.
      Това означава, че $b_m = \bigsqcup_n a_{n,m} \sqsubseteq d$.
      Получаваме, че $d$ е горна граница на веригата $\chain{b}{m}$,
      откъдето следва, че $\bigsqcup_m b_m \sqsubseteq d$.
    \end{itemize}
    Заключаваме, че $\bigsqcup_m b_m$ е точната горна граница на $E$.
    Обобщавайки всичко от по-горе, следва, че:
    \[\bigsqcup E = \bigsqcup_m b_m = \bigsqcup_n c_n.\]
  \item
    Остана да видим, че 
    \[\bigsqcup E = \bigsqcup_n a_{n,n}.\]
    \begin{itemize}
    \item 
      Да разгледаме произволен елемент $a_{n,m} \in E$.
      Нека $k = \max\{n,m\}$.
      Ясно е, че $a_{n,m} \sqsubseteq a_{k,k} \sqsubseteq \bigsqcup_na_{n,n}$.
      Следователно, $\bigsqcup_n a_{n,n}$ е горна граница на $E$, откъдето получаваме
      \[\bigsqcup E \sqsubseteq \bigsqcup_n a_{n,n}.\]
    \item
      Нека $d$ е горна граница на $E$.
      Тогава $(\forall n)(\forall m)[a_{n,m} \sqsubseteq d]$
      и в частност, $(\forall n)[a_{n,n} \sqsubseteq d]$.
      Сега можем да заключим, че $\bigsqcup_n a_{n,n} \sqsubseteq d$.    
    \end{itemize}
    Така доказахме, че $\bigsqcup_n a_{n,n}$ е точна горна граница на $E$.
    \end{enumerate}
  С това доказателството на теоремата е завършено.
\end{proof}

\begin{framed}
  \begin{lemma}
    Нека $\A$ и $\B$ са области на Скот.
    Нека $\chain{f}{k}$ е верига от елементи на $\Cont{\A}{\B}$.
    Да дефинираме изображението $h$ на $\A$ в $\B$ по следния начин
    \[h(a) \dff \bigsqcup\{f_k(a) \mid k \in \Nat\}.\]
    Изображението $h$ е {\em непрекъснато} и е {\em точна горна граница} на веригата $\chain{f}{k}$,
    т.е. $h = \bigsqcup_k f_k$.
  \end{lemma}
\end{framed}
\marginpar{Ако $b_k = f_k(a)$, то $h(a)$ е точната горна граница на веригата $\chain{b}{k}$ в $\B$}
\begin{proof}
  \ifhints
  Доказателството, че $h$ е точна горна граница на веригата $\chain{f}{k}$ е лесно.
  \begin{itemize}
  \item 
    Да разгледаме произволен елемент $a \in A$.
    Лесно се вижда, че понеже $\chain{f}{k}$ е верига, то $(f_k(a))^\infty_{k=0}$ също е верига.
    Това е така, защото всяко непрекъснато изображение е също така и монотонно.

    \marginpar{$\bigsqcup_n f_n(a)$ е съкратен запис за $\bigsqcup\{f_n(a) \mid n \in \Nat\}$.}
    Получаваме, че за всяко $k$, $f_k(a) \sqsubseteq^\B \bigsqcup_n f_n(a) \dff h(a)$.
    Понеже това е вярно за произволно $a \in A$, $(\forall k)[f_k \sqsubseteq h]$,
    което означава, че $h$ е горна граница на веригата.
  \item
    Да разгледаме произволно изображение $g$, което е горна граница на веригата $\chain{f}{k}$.
    За произволен елемент $a \in A$, 
    \[(\forall k)[f_k(a) \sqsubseteq^\B g(a)].\]
    Това означава, че $g(a)$ е горна граница на веригата $(f_k(a))^\infty_{k=0}$.
    Понеже $h(a) = \bigsqcup_k \{f_k(a)\}$ е точната горна граница на веригата $(f_k(a))^\infty_{k=0}$,
    то $h(a) \sqsubseteq^\B g(a)$.
    Оттук следва, че $h \sqsupseteq g$.
  \end{itemize}
  \fi
  По-сложната част на доказателството е проверката, че $h$ е непрекъснато изображение.
  Да вземем една монотонно растяща редица $\chain{a}{k}$ от елементи на $A$.
  \marginpar{За момента дори не е ясно дали $\{h(a_k) \mid k \in \Nat\}$ е верига в $\B$}
  Ще докажем, че \[h(\bigsqcup_k a_k) = \bigsqcup_k \{h(a_k)\}.\]
  Нека $e_{n,m} \dff f_n(a_m)$.
  Понеже всяко $f_n$ е непрекъснато и следователно монотонно изображение, то имаме
  \[n \leq n^\prime\ \&\ m \leq m^\prime\ \Rightarrow\ e_{n,m} \sqsubseteq^{\B} e_{n^\prime,m^\prime}.\]
  Следователно,
  \begin{align*}
    h(\bigsqcup_m a_m) & = \bigsqcup_n(f_n(\bigsqcup_m a_m)) & \comment{\text{от деф. на }h}\\
                       & = \bigsqcup_n(\bigsqcup_m f_n(a_m)) & \comment{\text{ защото } f_n \text{ е непр.}}\\
                       & = \bigsqcup_n(\bigsqcup_m e_{n,m}) = \bigsqcup_m(\bigsqcup_n e_{n,m}) & \comment{\text{от \Th{double-chain}}}\\
                       & = \bigsqcup_m(\bigsqcup_n f_n(a_m)) & \comment{\text{от деф. на }e_{n.m}}\\
                       & = \bigsqcup_m \{h(a_m)\}. & \comment{\text{от деф. на }h}
  \end{align*}
\end{proof}

Да напомним, че релацията $\sqsubseteq$ между две изображения е дефинирана като
\[f \sqsubseteq g \dfff (\forall a\in A)[f(a) \sqsubseteq^\B g(a)].\]
\begin{framed}
  \begin{cor}
    $(\Cont{\A}{\B}, \sqsubseteq, \bm{\bot} )$ е област на Скот.
  \end{cor}
\end{framed}



% \begin{remark}
%   Интересен въпрос е при какви условия областта на Скот $\Cont{\A_1}{\A_2}$ е алгебрична.
%   Ние по-късно ще разгледаме този въпрос в един частен, но достатъчно общ, случай, 
%   а именно когато $\A_1$ и $\A_2$ са областта на Скот от лениви списъци.
% \end{remark}

% \begin{problem}
%   Нека $\A = (A, \sqsubseteq^\A, \bot^\A)$ е област на Скот, а $X$ е произволно непразно множество.
%   Определяме $\A^X = (A^X,\sqsubseteq^X,\bot^X)$ като:
%   \begin{itemize}
%   \item 
%     $A^X \dff \{f:X \to A\mid f\mbox{ е тотална}\}$;
%   \item
%     $f \sqsubseteq^X g\ \iff\ (\forall x \in X)[f(x) \sqsubseteq^\A g(x)]$;
%   \item
%     $\bot^X \in A^X$, като $(\forall x \in X)[\bot^X(x) = \bot^\A]$.
%   \end{itemize}
%   Докажете, че $\A^X$ е област на Скот.
% \end{problem}
% \begin{proof}
%   \begin{enumerate}[1)]
%   \item 
%     Лесно се вижда, че $\sqsubseteq^X$ е частична наредба.
%   \item
%     Също така, $\bot^X$ е най-малкият елемент на $A^X$ относно $\sqsubseteq^X$.
%   \item
%     Нека $(f_n)^\infty_{n=0}$ е верига от елементи на $A^X$. 
%     Ще докажем, че функцията $h$, дефинирана като $h(x) = \bigsqcup_n f_n(x)$,
%     е точна горна граница на $(f_n)^\infty_{n=0}$.
%     Но преди това, първо да съобразим, че $h \in A^X$.
%     Това следва от факта, че $\A$ е област на Скот и за всяко $x\in X$,
%     $(f_n(x))^\infty_{n=0}$ е верига в $\A$ 
%     и следователно има точна горна граница, която е равна на $h(x)$.
    
%     Сега ще покажем, че $h$ е горна граница на $(f_n)^\infty_{n=0}$.
%     Имаме, че:
%     \begin{align*}
%       (\forall x\in X)(\forall n)[f_n(x) \sqsubseteq^\A \bigsqcup_n f_n(x) = h(x)] & \iff (\forall n)(\forall x\in X)[f_n(x) \sqsubseteq^\A h(x)] \\
%       & \iff (\forall n)[f_n \sqsubseteq^\A h].
%     \end{align*}
    
%     Остана да видим, че $h$ е най-малката измежду горните граници на редицата $(f_n)^\infty_{n=0}$.
%     Нека $g$ е произволна горна граница на $(f_n)^\infty_{n=0}$. Това означава, че:
%     \begin{align*}
%       (\forall n)[f_n \sqsubseteq^X g] & \iff (\forall n)(\forall x\in X)[f_n(x) \sqsubseteq^\A g(x)]\\
%       & \iff (\forall x\in X)(\forall n)[f_n(x) \sqsubseteq^\A g(x)]
%     \end{align*}
%     Да разгледаме произволно $x \in X$. Тогава:
%     \[(\forall n)[f_n(x) \sqsubseteq^\A \bigsqcup_n f_n(x) \sqsubseteq^\A g(x)].\]
%     Понеже $h(x) = \bigsqcup_n f_n(x)$, получаваме, че за всяко $x \in X$,
%     \[h(x) \sqsubseteq^\A g(x),\]
%     което означава, че
%     $h \sqsubseteq^X g$.
%   \end{enumerate}
% \end{proof}

Нека $\A_1,\dots,\A_n$ и $\A$ са области на Скот и да разгледаме $f: \A_1\times \dots \times \A_n \to \A$.
Казваме, че $f$ е {\bf непрекъснато изображение по $i$-тия аргумент}, ако 
за всяка верига $\chain{a}{k}$ в $\A_i$, то
\[f(b_1,\dots, b_{i-1}, \bigsqcup_k a_k, b_{i+1},\dots,b_n) = \bigsqcup_kf(b_1,\dots, b_{i-1}, a_k, b_{i+1},\dots,b_n).\]

\begin{prop}
  Нека $\A_1,\dots,\A_n$ и $\A$ са области на Скот.
  Едно изображение $f: \A_1\times \dots \times \A_n \to \A$ 
  е непрекъснато точно тогава, когато $f$ е непрекъснато по всеки от аргументите си.
\end{prop}
\Stefan{Интересно е, че това твърдение не е вярно за непрекъснати свойства. Добре е да се обясни някъде.}

\begin{proof}
  \marginpar{\writedown Обобщете това твърдение за $n > 2$.}
  За по-просто изложение, да разгледаме случая $n = 2$.

  $(\Rightarrow)$ Лесно се съобразява, че ако $f$ е непрекъснато изображение, то $f$ е непрекъснато по всеки от аргументите си.
  \Stefan{Все пак е добре тук да се напише нещо.}
  
  $(\Leftarrow)$ Нека сега $f$ е непрекъснато по всеки от аргументите си. Ще докажем, че $f$ е непрекъснато.
  Нека $\{\pair{a_n,b_n}\}^\infty_{n=0}$ е верига в $\A_1\times \A_2$.
  Понеже от \Prop{cartesian} знаем, че
  \[\bigsqcup_n\pair{a_n,b_n} = \pair{\bigsqcup_na_n,\bigsqcup_n b_n},\]
  ще докажем, че 
  \[\bigsqcup_n f(a_n,b_n) = f(\bigsqcup_n a_n,\bigsqcup_n b_n).\]
  Да положим $e_{n,m} = f(a_n,b_m)$.
  Понеже $f$ е непрекъснато по всеки от аргументите си, лесно се вижда, че $f$
  е монотонно изображение по всеки от аргументите си. Следователно, 
  \[n \leq n^\prime\ \&\ m \leq m^\prime\ \Rightarrow\ e_{n,m} \sqsubseteq e_{n^\prime,m^\prime}.\]  
  Получаваме, че
  \begin{align*}
    \bigsqcup_n f(a_n,b_n) & = \bigsqcup_n e_{n,n} & \comment{\text{от опр. на }e_{n,m}}\\
                           & = \bigsqcup_n(\bigsqcup_m e_{n,m}) & \comment{\text{от \Th{double-chain}}}\\
                           & = \bigsqcup_n(\bigsqcup_m f(a_n,b_m)) & \comment{\text{от опр. на }e_{n,m}}\\
                           & = \bigsqcup_nf(a_n,\bigsqcup_m b_m) & \comment{f \text{ е непр. по втория си аргумент}}\\
                           & = f(\bigsqcup_n a_n,\bigsqcup_m b_m) & \comment{f \text{ е непр. по първия си аргумент}}.
  \end{align*}
\end{proof}


%%% Local Variables:
%%% mode: latex
%%% TeX-master: "../sep"
%%% End:


 \section{Оператор за най-малка неподвижна точка}

\Stefan{Това да се остави само с упътване.}

\begin{thm}
  % \index{$Y_\A$}
  Нека $\A$ е област на Скот и нека $f \in \Cont{\A}{\A}$.
  \marginpar{Знаем от \Th{knaster-tarski}, че най-малката неподвижна точка на $f$ е елемента $\bigsqcup_n f^n(\bot^\A)$.}
  Тогава изображението $Y_\A : \Cont{\A}{\A} \to \A$, определено като
  \[Y_\A(f) = \lfp(f),\]
  е непрекъснато, т.е.
  $Y_\A \in \Cont{\Cont{\A}{\A}}{\A}$.
\end{thm}
\begin{proof}
  Нека да вземем една верига $(f_n)^\infty_{n=0}$ от непрекъснати изображения.
  Нашата цел е да докажем, че
  \[Y_\A(\bigsqcup_n f_n) = \bigsqcup_n Y_\A(f_n).\]
  Да означим с $h$ точната горна граница на $(f_n)^\infty_{n=0}$.
  Знаем, че $h(a) = \bigsqcup_n f_n(a)$.
  \begin{prop}
    За всяко $k \geq 1$, $h^k(a) = \bigsqcup_n f^k_n(a)$.
  \end{prop}
  \begin{proof}
    Ще докажем твърдението с индукция по $k$, като случая $k = 1$ следва от дефиницията на $h$.
    Нека приемем, че твърдението е вярно за произволно $k \geq 1$.
    Ще докажем, че твърдението е вярно за $k+1$.
    \begin{align*}
      h^{k+1}(a) & = h(h^k(a)) & \\
      & = h(\bigsqcup_n f^k_n(a))& \comment{\text{ от инд. предположение}}\\
      & = \bigsqcup_n h(f^k_n(a))& \comment{h \text{ е непрекъснато изображение}}\\
      & = \bigsqcup_n (\bigsqcup_m f_m(f^k_n(a))). & 
    \end{align*}
    
    Да положим $b_n = f^k_n(a)$, за всяко $n$.
    Понеже $f_n \sqsubseteq f_{n^\prime}$, лесно се съобразява, че за $n \leq n^\prime$
    имаме $b_n \sqsubseteq^\A b_{n^\prime}$.

    Сега да положим $e_{m,n} = f_m(b_n)$.
    Отново, понеже $(b_n)^\infty_{n=0}$ и $(f_m)^\infty_{m=0}$ са вериги, имаме 
    \[m \leq m^\prime\ \&\ n\leq n^\prime\ \Rightarrow\ e_{m,n} \sqsubseteq^\A e_{m^\prime,n^\prime}.\]
    Това означава, че можем да приложим \Th{double-chain} за множеството $E = \{e_{m,n} \mid m,n \in \Nat\}$.
    Получаваме, че
    \begin{align*}
      h^{k+1}(a) & = \bigsqcup_n (\bigsqcup_m f_m(f^k_n(a))) & \comment{\text{ от горното равенство}}\\
      & = \bigsqcup_n (\bigsqcup_m e_{m,n}) & \comment{\text{ от определението на }e_{m,n}}\\
      & = \bigsqcup_n e_{n,n} & \comment{\text{ от \Th{double-chain}}}\\
      & = \bigsqcup_n f_n(f^k_n(a))  = \bigsqcup_n f^{k+1}_n(a) & 
    \end{align*}
    С това твърдението е доказано.
  \end{proof}
  Сега вече сме готови да докажем непрекъснатостта на $Y_\A$.
  Имаме, че:
  \begin{align*}
    Y_\A(\bigsqcup_n f_n) & = Y_\A(h) & \comment{\text{ от опр. на }h}\\
    & = \bigsqcup_m h^m(\bot^\A) & \comment{\text{ от опр. на }Y_\A }\\
    & = \bigsqcup_m (\bigsqcup_n f^m_n(\bot^\A)) & \comment{\text{ от горното твърдение}}.
  \end{align*}
  
  Да положим $e_{m,n} = f^m_n(\bot^\A)$.
  Отново лесно се съобразява, че 
  \[m \leq m^\prime\ \&\ n\leq n^\prime\ \Rightarrow\ e_{m,n} \sqsubseteq^\A e_{m^\prime,n^\prime}.\]
  Получаваме, че
  \begin{align*}
    Y_\A(\bigsqcup_n f_n) & = \bigsqcup_m (\bigsqcup_n f^m_n(\bot^\A)) & \comment{\text{ от горното равенство}}\\
    & = \bigsqcup_m (\bigsqcup_n e_{m,n}) & \comment{\text{ от опр. на }e_{m,n}}\\
    & = \bigsqcup_n(\bigsqcup_m e_{m,n}) & \comment{\text{ от \Th{double-chain}}}\\
    & = \bigsqcup_n (\bigsqcup_m f^m_n(\bot^\A)) = \bigsqcup_n Y_\A(f_n). & \comment{\text{ от опр. на }Y_\A}.
  \end{align*}
\end{proof}



%%% Local Variables:
%%% mode: latex
%%% TeX-master: "../sep"
%%% End:


\section{Изоморфни области на Скот}
\index{изоморфизъм}

Нека $\A_1 = (A_1,~\sqsubseteq_1,~\bot_1)$ и $\A_2 = (A_2,~\sqsubseteq_2~,~\bot_2)$ 
са области на Скот.
Ще казваме, че $\A_1$ е {\bf изоморфна} на $\A_2$, което ще означаваме като 
\[\A_1 \cong \A_2,\]
ако съществува {\em биективна} функция $F:A_1 \to A_2$ със свойството:
\marginpar{Ясно е, че $F(\bot_1) = \bot_2$}
\[(\forall a,b\in A_1)[\ a \sqsubseteq_1 b \iff F(a) \sqsubseteq_2 F(b)\ ].\]
В такъв случай ще казваме, че $F$ задава изоморфизъм между $\A_1$ и $\A_2$.

Когато искаме да означим, че $\A_1$ е изоморфна на $\A_2$ чрез $F$,
то понякога ще пишем $\A_1 \cong_F \A_2$.

\begin{prop}
  \label{pr:isomorphism-is-continuous}
  Ако $\A_1 \cong_F \A_2$ , то $F \in \Cont{\A_1}{\A_2}$.
\end{prop}
\begin{hint}
  Да разгледаме произволна верига $\chain{a}{i}$ от елементи на $\A_1$.
  Ще докажем, че 
  \[F(\bigsqcup_i a_i) = \bigsqcup_iF(a_i).\]
  
  \begin{itemize}
  \item 
    Първо, от дефиницията веднага следва, че $F$ е монотонно изображение,
    защото $a \sqsubseteq_1 b \implies F(a) \sqsubseteq_2 F(b)$.
    Това означава, че $(F(a_i))^\infty_{i=0}$ е монотонно растяща верига от елементи на $\A_2$.
    От \Prop{monotone-chain} получаваме, че 
    \[\bigsqcup_i F(a_i) \sqsubseteq_2 F(\bigsqcup_i a_i).\]
  \item
    За другата посока, нека $b \in \A_2$ е горна граница на веригата $(F(a_i))^\infty_{i=0}$, т.е. 
    \[(\forall i)[\ F(a_i) \sqsubseteq_2 b\ ].\]
    Ще докажем, че $F(\bigsqcup_i a_i) \sqsubseteq_2 b$.
    Понеже $F$ е {\em върху} $A_2$, то съществува елемент $a \in A_1$, такъв че $F(a) = b$.
    Тогава:
    \begin{align*}
      (\forall i)[\ F(a_i) \sqsubseteq_2 F(a)\ ] & \implies (\forall i)[\ a_i \sqsubseteq_1 a\ ] & \comment{F \text{ е изоморфизъм }}\\
                                                 & \implies \bigsqcup_i a_i \sqsubseteq_1 a & \comment{a\text{ е горна граница}}\\
                                                 & \implies F(\bigsqcup_i a_i) \sqsubseteq_1 F(a). & \comment{F\text{ е изоморфизъм }}
    \end{align*}
    Понеже $b = F(a)$, заключаваме, че
    \[F(\bigsqcup_i a_i) \sqsubseteq_2 b.\]
  \end{itemize}
\end{hint}

\begin{prop}
  \label{pr:isomorphic-pair}
  Нека $f \in \Mon{\A_1}{\A_2}$ и $g \in \Mon{\A_2}{\A_1}$,
  като 
  \begin{itemize}
  \item 
    $f \circ g = \texttt{id}_2$;
  \item
    $g \circ f = \texttt{id}_1$.
  \end{itemize}
  \marginpar{$\texttt{id}_i(a) \dff a$ за вс. $a \in \A_i$}
  Тогава са изпълнени свойствата:
  \begin{enumerate}[(1)]
  \item
    $\A_1 \cong_f \A_2$;
  \item
    $\A_2 \cong_g \A_1$;
  \end{enumerate}
\end{prop}
% \begin{hint}
%   За Свойство $(1)$ трябва да проверим, че $f$ отговаря на дефиницията за изоморфизъм.
%   \begin{itemize}
%   \item
%     Ще докажем, че $f$ е инективна като покажем, че за произволни $a, b\in A_1$,
%     ако $f(a) = f(b)$, то $a = b$.
%     Но това е лесно, защото
%     \[a = \texttt{id}_1(a) = g(f(a)) = g(f(b)) = \texttt{id}_1(b) = b.\]
%   \item
%     Нека сега $b \in \A_2$.
%     Знаем, че $f(g(b)) = \texttt{id}_2(b) = b$. Това означава, че $f$ е {\em сюрективна},
%     защото за всеки елемент $b \in A_2$ съществува елемент $a \in A_1$, а именно $a = g(b)$,
%     за който $f(a) = b$.
%   \item
%     Понеже $f$ е монотонно изображение, то директно имаме, че
%     \[a \sqsubseteq_1 b \implies f(a) \sqsubseteq_2 f(b).\]
%   \item
%     Нека $f(a) \sqsubseteq_2 f(b)$.
%     Сега пък понеже $g$ е монотонно изображение, 
%     \[a = \texttt{id}_1(a) = g(f(a)) \sqsubseteq_1 g(f(b)) = \texttt{id}_1(b) = b.\]
%     Така показахме, че
%     \[f(a) \sqsubseteq_2 f(b)\ \implies\ a \sqsubseteq_1 b.\]
%   \end{itemize}
%   Доказахме Свойство $(1)$, т.е. $\A_1 \cong_f \A_2$.
%   Разсъжденията за Свойство $(2)$ са аналогични.
% \end{hint}


\begin{prop}
  \label{pr:isomorphic-higher-order}
  Нека $\A_1 \cong_F \A_2$. Тогава:
  \begin{enumerate}[(1)]
  \item 
    $\Cont{\A_1}{\A_1} \cong_G \Cont{\A_2}{\A_2}$, където 
    \[G(f) \dff F \circ f \circ F^{-1};\]
    Графично това може да се изобрази така:

    \shorthandoff{"}%
    \begin{center}
    \begin{tikzcd}[sep=large]
      \A_1 \arrow[r, "f"] & \A_1 \arrow[d, "F"]\\
      \A_2 \arrow[u, "F^{-1}"]\arrow[r, dashed, "G(f)"] & \A_2 
    \end{tikzcd}
    \end{center}
    \shorthandon{"}%
  \item
    ако $f \in \Cont{\A_1}{\A_1}$, то 
    \[F(\lfp(f)) = \lfp(G(f)).\]
  \end{enumerate}
\end{prop}
\begin{hint}
  Ще докажем $(1)$ като използвме \Prop{isomorphic-pair}.

  \begin{itemize}
  \item 
    Ще докажем, че $G$ е монотонно изображение.
    Нека $f,h \in \Cont{\A_1}{\A_1}$ и $f \sqsubseteq h$, т.е.
    \[(\forall a \in \A_1)[\ f(a) \sqsubseteq_1 h(a)\ ].\]
    Ще докажем, че $G(f) \sqsubseteq G(h)$, т.е.
    \[(\forall b \in \A_2)[\ G(f)(b) \sqsubseteq_1 G(h)(b)\ ].\]
    Да разгледаме произволен елемент $b \in \A_2$. 
    Понеже $F$ е биекция, то съществува елемент $a \in A_1$, такъв че $F(a) = b$,
    т.е. $F^{-1}(b) = a$. Тогава:
    \begin{align*}
      G(f)(b) & \dff F(f(F^{-1}(b)))\\
              & = F(f(a)) & \comment{F^{-1}(b) = a}\\
              & \sqsubseteq_2 F(h(a)) & \comment{f(a) \sqsubseteq h(a)\text{ и $F$ е изом.}}\\
              & = F(h(F^{-1}(b))) & \comment{F^{-1}(b) =a}\\
              & \dff G(h)(b).
    \end{align*}
  \item
    Нека $G(f) \sqsubseteq G(h)$. Ще докажем, че $f \sqsubseteq h$.
    За целта, нека $a \in A_1$.
    Понеже $F$ е сюрективна, то съществува $b \in A_2$, за който $F^{-1}(b) =a$.
    Понеже
    \[G(f) \dff F \circ f \circ F^{-1} \sqsubseteq F \circ h \circ F^{-1} = G(h),\]
    то получаваме, че
    \[F(f(F^{-1}(b))) \sqsubseteq_2 F(h(F^{-1}(b))).\]
    Оттук,
    \[F(f(a)) \sqsubseteq_2 F(h(a)) \implies f(a) \sqsubseteq_1 h(a),\]
    защото $F$ е изоморфизъм.
  \end{itemize}
  Сега преминаваме към доказателството на $(2)$.
  Да напомним, че за $f \in \Cont{\A_1}{\A_1}$, означаваме
  \begin{align*}
    f^0  & = \lambda x. \bot_1\\
    f^{n+1} & = f \circ f^n.
  \end{align*}
  Понеже $f$ е непрекъснато изображение е ясно, че $(f^n(\bot_1))^{\infty}_{n=0}$ е верига.
  Също така знаем, че
  \[\lfp(f) = \bigsqcup_n f^n(\bot_1).\]
  След аналогични разсъждения можем да съобразим, че
  \[\lfp(G(f)) = \bigsqcup_n G(f)^n(\bot_2).\]
  Първо ще докажем с индукция по $n$, че 
  \begin{equation}
    \label{eq:2}
    (\forall n)[\ (G(f))^n = G(f^n)\ ].
  \end{equation}
  \begin{itemize}
  \item 
    За $n = 0$ имаме, че за произволен елемент $b \in \A_2$,
    \begin{align*}
      (G(f))^{0}(b) & \dff \bot_2\\
                    & = F(\bot_1) & \comment{F \text{ е изом.}}\\
                    & = F(f^{0}(F^{-1}(b))) & \comment{f^{0}(F^{-1}(b)) \dff \bot_1}\\
                    & = (F \circ f^{0} \circ F^{-1})(b) \\
                    & \dff G(f^{0})(b).
    \end{align*} 
  \item
    Нека да приемем, че твърдението е вярно за $n$.
    Тогава за $n+1$ имаме, че:
    \begin{align*}
      (G(f))^{n+1} & \dff G(f) \circ (G(f))^n\\
                   & = G(f) \circ G(f^n) & \comment{\text{ от И.П.}}\\
                   & \dff (F \circ f \circ F^{-1}) \circ (F\circ f^n \circ F^{-1})\\
                   & = F \circ f \circ (F^{-1} \circ F)\circ f^n \circ F^{-1} \\
                   & = F \circ f \circ f^{n} \circ F^{-1} & \comment{F^{-1}\circ F = id}\\
                   & = F \circ f^{n+1} \circ F^{-1} & \comment{f\circ f^n = f^{n+1}}\\
                   & \dff G(f^{n+1}).
    \end{align*}
  \end{itemize}
  Тогава:
  \begin{align*}
    F(\lfp(f)) & = F(\bigsqcup_n f^n(\bot_1)) & \comment{\lfp(f) = \bigsqcup_n f^n(\bot_1)}\\
               & = \bigsqcup_n F(f^n(\bot_1))& \comment{F\text{ е непр.}}\\
               & = \bigsqcup_n F(f^n(F^{-1}(\bot_2))) & \comment{F^{-1}(\bot_2) = \bot_1}\\
               & = \bigsqcup_n (F \circ f^n \circ F^{-1})(\bot_2) \\
               & \dff \bigsqcup_n G(f^n)(\bot_2)\\
               & = \bigsqcup_n G(f)^n(\bot_2) & \comment{\text{от }(\ref{eq:2})}\\
               & = \lfp(G(f)).
  \end{align*}
\end{hint}

\begin{framed}
  \begin{prop}
    За произволни области на Скот $\A$, $\B$ и $\C$ е изпълнено, че
    \[\Cont{\A}{\Cont{\B}{\C}}\ \cong\ \Cont{\A\times\B}{\C}.\]
  \end{prop}  
\end{framed}
\begin{hint}
  % \begin{itemize}
  % \item 
    Докажете, че изображението
    \[\texttt{curry}:\Cont{\A\times \B}{\C} \to \Cont{\A}{\Cont{\B}{\C}},\]
    където
    \[\texttt{curry}(f)(a)(b) \dff f(a,b)\]
    задава изоморфизъм.
  % \item
  %   Докажете, че изображението
  %   \[\texttt{uncurry}:\Cont{\A}{\Cont{\B}{\C}} \to \Cont{\A\times \B}{\C},\]
  %   където
  %   \[\texttt{uncurry}(f)(a,b) \dff f(a)(b)\]
  %   е монотонно.
  % \item
  %   Лесно се съобразява, че
  %   \[\texttt{curry} \circ \texttt{uncurry} = \texttt{id}\]
  %   \[\texttt{uncurry} \circ \texttt{curry} = \texttt{id}.\]    
  % \item
  %   Приложете \Prop{isomorphic-pair}.
  % \end{itemize}
\end{hint}

\begin{remark}
  Когато на хаскел пишем типовата сигнатура на някоя функция като 
  \mint{haskell}|f :: a -> b -> c|
  в действителност се има предвид следното
  \mint{haskell}|f :: a -> (b -> c)|
  
  На практика тези две задачи ни казват, че няма значение дали използваме {\em curried}
  или {\em uncurried} версията на една функция. На хаскел е по-удобно да използваме {\em curried}
  версията, защото като фиксираме първия аргумент на една функция получаваме нова функция наготово.
  Например, 
  \begin{haskellcode}
ghci> let plus x y = x + y
ghci> :t plus
plus :: Num a => a -> a -> a
ghci> let plus1 = plus 1
ghci> :t plus1
plus1 :: Num a => a -> a
   \end{haskellcode}

  Всъщност, хаскел има функциите \texttt{curry} и \texttt{uncurry} вградени в стандартната библиотека:
  \begin{haskellcode}
ghci> :t curry
curry :: ((a, b) -> c) -> a -> b -> c
ghci> :t uncurry
uncurry :: (a -> b -> c) -> (a, b) -> c
  \end{haskellcode}
\end{remark}

\begin{problem}
  Докажете, че съществуват области на Скот $\A$, $\B$ и $\C$, за които
  \[\Cont{\Cont{\A}{\B}}{\C} \not\cong \Cont{\A}{\Cont{\B}{\C}}.\]  
\end{problem}

\subsection*{Точни продължения на частични функции}

За еднa частична функция $f:\Nat^n \to \Nat$, определяме изображението $f^\star \in \Strict{\Nat^n_\bot}{\Nat_\bot}$ по следния начин:
\begin{align*}
  f^\star(\ov{a}) \dff
  \begin{cases}
    f(\ov{a}), & \text{ако }\bot\not\in\{a_1,\dots,a_n\}\ \&\ f(\ov{a})\text{ е деф.}\\
    \bot, & \text{иначе}.
  \end{cases}
\end{align*}
Изображението $f^\star$ се нарича {\bf точно продължение} на $f$.

\begin{example}
  Да разгледаме частичната функция $f:\Nat^2 \to \Nat$, дефинирана като
  $f(x,y) = x+y$. Тогава точното продължение на $f$ е 
  \[f^\star(x,y) = 
  \begin{cases}
    x+y, & x,y\in\Nat\\
    \bot, & x = \bot \vee y = \bot.
  \end{cases}\]
\end{example}

\noindent Да дефинираме оператор $\Sigma^\star_n : \Partial{\Nat^n}{\Nat} \to \Strict{\Nat^n_\bot}{\Nat_\bot}$ като
\[\Sigma^\star_n(f) = f^\star.\] 
Ще наричаме $\Sigma^\star_n$ {\bf продължаващ} оператор, защото на всяка частична функция дава нейното точно продължение.


\begin{thm}
  За всяко естествено число $n$,
  \[\Partial{\Nat^n}{\Nat}\ \cong\ \Strict{\Nat^n_\bot}{\Nat_\bot}\]
  чрез изображението $\Sigma^\star_n$.
\end{thm}
\begin{hint}
  \begin{itemize}
  \item 
    Лесно се съобразява, че $\Sigma^\star_n$ е биективна функция на $\Partial{\Nat^n}{\Nat}$ върху $\Strict{\Nat^n_\bot}{\Nat_\bot}$.
  \item
    Лесно се проверява също, че $f \subseteq g \iff \Sigma^\star_n(f) \sqsubseteq \Sigma^\star_n(g)$.    
  \end{itemize}
\end{hint}

\begin{cor}
  $\Sigma^\star_n$ е непрекъснато изображение, т.е.
  \[\Sigma^\star \in \Cont{\Partial{\Nat^n}{\Nat}}{\Strict{\Nat^n_\bot}{\Nat_\bot}}.\]
\end{cor}

\begin{example}
  Да разгледаме следната рекурсивна програма $\vv{P}$:

  \begin{haskellcode}
f(x, y) = if x == y then 0
            else 1 + f(x, y + 1)
   \end{haskellcode}

  Ако искаме да намерим $\D_V\val{\vv{P}}(x,y)$,
  то формално погледнато трябва да намерим най-малката неподвижна точка на оператора
  \[\Delta \in \Cont{\Strict{\Nat^2_\bot}{\Nat_\bot}}{\Strict{\Nat^2_\bot}{\Nat_\bot}},\]
  където
  \[\Delta(g)(x,y) = 
  \begin{cases}
    0, & \text{ако } x,y\in\Nat\ \&\ x = y\\
    1 + g(x,y+1), & \text{ако } x,y,\in\Nat\ \&\ x \neq y\\
    \bot, & \text{ако }x = \bot\ \lor\ y = \bot.
  \end{cases}\]
  Лесно се съобразява, че
  \[\D_V\val{\vv{P}}(a,b) = \lfp(\Delta)(a,b) = 
  \begin{cases}
    a - b, & \text{ако }a,b\in\Nat\ \&\ a \geq b\\
    \bot, & \text{иначе}
  \end{cases}\]

В някои отношения е по-лесно да подходим по друг начин.
Да разглеждаме оператора
\[\Gamma \in \Cont{\Partial{\Nat^2}{\Nat}}{\Partial{\Nat^2}{\Nat}},\]
където
\[\Gamma(f)(x,y) \simeq
\begin{cases}
  0, & \text{ако }x = y\\
  1 + f(x,y+1), & \text{ако } x \neq y.
\end{cases}\]
Лесно се съобразява, че
\begin{align*}
  \lfp(\Gamma)(n,k) \simeq
  \begin{cases}
    n - k, & \text{ако }n \geq k\\
    \neg !, & \text{иначе}.
  \end{cases}
\end{align*}

Понеже $\Sigma^\star_2$ задава изоморфизъм между областите на Скот $\Partial{\Nat^2}{\Nat}$ и $\Strict{\Nat^2_\bot}{\Nat_\bot}$,
то от \Prop{isomorphic-higher-order} имаме, че
\[\Cont{\Partial{\Nat^2}{\Nat}}{\Partial{\Nat^2}{\Nat}} \cong_{\mathcal{G}} \Cont{\Strict{\Nat^2_\bot}{\Nat_\bot}}{\Strict{\Nat^2_\bot}{\Nat_\bot}},\]
където
\[\mathcal{G}(\Gamma) = \Sigma^\star_2 \circ \Gamma \circ (\Sigma^\star_2)^{-1}.\]
Графично това може да се представи така:
\shorthandoff{"}%
\begin{center}
  \begin{tikzcd}[sep=large]
    \Partial{\Nat^n}{\Nat} \arrow[r, "\Gamma"] & \Partial{\Nat^n}{\Nat} \arrow[d, "\Sigma^\star_2"]\\
    \Strict{\Nat^n_\bot}{\Nat_\bot} \arrow[u, "(\Sigma^\star_2)^{-1}"]\arrow[r, dashed, "\mathcal{G}(\Gamma)"] & \Strict{\Nat^n_\bot}{\Nat_\bot}
  \end{tikzcd}
\end{center}
\shorthandon{"}%
Сега директно получаваме, че
\begin{align*}
  \D_V\val{\vv{P}} & \dff \lfp(\Delta)\\
                   & = \Sigma^\star_2(\lfp(\Gamma)),
\end{align*}
защото $\Delta = \mathcal{G}(\Gamma)$ и от \Prop{isomorphic-higher-order} знаем, че
\[\lfp(\mathcal{G}(\Gamma)) = \Sigma^\star_2(\lfp(\Gamma)).\]
\end{example}

%%% Local Variables:
%%% mode: latex
%%% TeX-master: "../sep"
%%% End:


\section{Задачи}  

Тук с $\A$, $\B$ и $\C$ ще означаваме области на Скот.

\marginpar{Много от задачите са от \cite[стр. 31]{abramsky94}}

\begin{problem}
  Да разгледаме операторите \[\Gamma,\Delta \in \Cont{\Cont{\A}{\A}}{\Cont{\A}{\A}}.\]
  Знаем, че операторът $\Gamma \circ \Delta$ е непрекъснат, където
  \[(\Gamma\circ\Delta)(f) \dff \Gamma(\Delta(f)).\]
  Вярно ли е, че
  \[\lfp(\Gamma \circ \Delta) \sqsubseteq \lfp(\Gamma) \circ \lfp(\Delta)?\]
  Обосновете се!
\end{problem}
\ifhints
\begin{hint}
  Нека $\A = \Nat_\bot$.
  Нека например да разгледаме
  \begin{align*}
    & \Delta(f)(x) \dff f(x+1)\\
    & \Gamma(f)(x) \dff
      \begin{cases}
        0, & x \neq \bot\\
        \bot, & x = \bot.
      \end{cases}
  \end{align*}
  Да положим $f_\Gamma \dff \lfp(\Gamma)$ и $f_\Delta \dff \lfp(\Delta)$.
  Ясно е, че 
  \begin{align*}
    & f_\Delta(x) = \bot\\
    & f_\Gamma(x) =
    \begin{cases}
      0, & x \neq \bot\\
      \bot, & x = \bot.
    \end{cases}  
  \end{align*}
  Тогава за произволно $x \in \Nat_\bot$,
  \[(f_\Gamma\circ f_\Delta)(x) = f_\Gamma(f_\Delta(x)) = f_\Gamma(\bot)  = \bot.\]
  От друга страна, понеже $(\Gamma \circ \Delta)(f) = \Gamma(\Delta(f))$, то 
  \begin{align*}
    & (\Gamma \circ \Delta)(f)(x) = \Gamma(\Delta(f))(x) = 
      \begin{cases}
        0, & x \neq \bot\\
        \bot, & x = \bot.
      \end{cases}
  \end{align*}
  Лесно се съобразява, че 
  \[\lfp(\Gamma \circ \Delta)(x) =
  \begin{cases}
    0, & x \neq \bot\\
    \bot, & x = \bot.
  \end{cases}\]
  Заключаваме, че 
  \[\lfp(\Gamma \circ \Delta) \sqsupset \lfp(\Gamma) \circ \lfp(\Delta).\]
\end{hint}
\fi

\begin{problem}
  Да разгледаме операторите \[\Gamma,\Delta \in \Cont{\Cont{\A}{\A}}{\Cont{\A}{\A}}.\]
  Знаем, че операторът $\Gamma \circ \Delta$ е непрекъснат, където
  \[(\Gamma\circ\Delta)(f) \dff \Gamma(\Delta(f)).\]
  Вярно ли е, че
  \[\lfp(\Gamma \circ \Delta) \sqsupseteq \lfp(\Gamma) \circ \lfp(\Delta)?\]
  Обосновете се!
\end{problem}
\ifhints
\begin{hint}
  Нека $\A = \Nat_\bot$.
  Нека например да разгледаме
  \begin{align*}
    & \Delta(f)(x) \dff 0\\
    & \Gamma(f)(x) \dff
      \begin{cases}
        0, & x = 0\\
        \bot, & \text{ иначе}.
      \end{cases}
  \end{align*}
  Да положим $f_\Gamma \dff \lfp(\Gamma)$ и $f_\Delta \dff \lfp(\Delta)$.
  Ясно е, че 
  \begin{align*}
    & f_\Delta(x) = 0\\
    & f_\Gamma(x) =
    \begin{cases}
      0, & x = 0\\
      \bot, & \text{ иначе}.
    \end{cases}  
  \end{align*}
  Тогава за произволно $x \in \Nat_\bot$,
  \[(f_\Gamma\circ f_\Delta)(x) = f_\Gamma(f_\Delta(x)) = f_\Gamma(0)  = 0.\]
  От друга страна, понеже $(\Gamma \circ \Delta)(f) = \Gamma(\Delta(f))$, то 
  \begin{align*}
    & (\Gamma \circ \Delta)(f)(x) = \Gamma(\Delta(f))(x) = 
      \begin{cases}
        0, & x = 0\\
        \bot, & \text{ иначе}.
      \end{cases}
  \end{align*}
  Лесно се съобразява, че 
  \[\lfp(\Gamma \circ \Delta)(x) =
  \begin{cases}
    0, & x = 0\\
    \bot, & \text{ иначе}.
  \end{cases}\]
  Заключаваме, че 
  \[\lfp(\Gamma \circ \Delta) \sqsubset \lfp(\Gamma) \circ \lfp(\Delta).\]
\end{hint}
\fi

\begin{problem}
  Нека $f_0 \sqsubseteq f_1 \sqsubseteq f_2 \sqsubseteq \cdots$
  е верига от елементи на $\Cont{\A}{\A}$.
  Да положим $h = \bigsqcup_n f_n$.
  Вярно ли е, че 
  \[h \circ h = \bigsqcup_n (f\circ f)?\]
  Обосновете се!
\end{problem}

\begin{problem}
  Да разгледаме едно изображение $f: \A \times \B \to \C$.
  За произволно $a \in \A$, дефинираме изображението $g_a: \B \to \C$, където
  \[g_a(b) \dff f(a,b).\]
  Аналогично, за произволно $b \in \B$, дефинираме изображението $h_b: \A \to \C$, където
  \[h_b(a) \dff f(a,b).\]
  Докажете, че следните твърдения са еквивалентни:
  \begin{enumerate}[1)]
  \item 
    $f$ е непрекъснато изображение;
  \item
    $g_a$ и $h_b$ са непрекъснати изображения, за всяко $a \in \A$ и $b \in \B$.
  \end{enumerate}
\end{problem}

\begin{problem}
  Да разгледаме $f \in \Cont{\A \times \B}{\C}$.
  За произволно $a \in \A$, дефинираме изображението $g_a: \B \to \C$, където
  \[g_a(b) \dff f(a,b).\]
  Вече знаем, че $g_a \in \Cont{\B}{\C}$, за всяко $a \in \A$.
  Да разгледаме изображението $h:\A \to \Cont{\B}{\C}$, където
  \[h(a) \dff g_a.\]
  Докажете, че $h$ е непрекъснато изображение.
\end{problem}

\begin{problem}
  \marginpar{\cite[стр. 129]{nikolova-soskova}}
  Да разгледаме $f \in \Cont{\A \times \B}{\B}$.
  За произволно $a \in \A$, дефинираме изображението $g_a: \B \to \B$, където
  \[g_a(b) \dff f(a,b).\]
  Вече знаем, че $g_a \in \Cont{\B}{\B}$, за всяко $a \in \A$,
  следователно $\lfp(g_a)$ съществува.
  Да разгледаме изображението $h:\A \to \B$, където
  \[h(a) \dff \lfp(g_a).\]
  Докажете, че $h$ е непрекъснато изображение.
\end{problem}


\begin{problem}
  Да разгледаме $f \in \Cont{\A}{\Cont{\B}{\C}}$.
  За произволно $a \in \A$, дефинираме изображението $g_a \in \Cont{\B}{\C}$, където
  \[g_a(b) \dff f(a).\]
  Да разгледаме изображението $h:\A\times \B \to \C$, където
  \[h(a,b) \dff g_a(b).\]
  Докажете, че $h$ е непрекъснато изображение.
\end{problem}

% \begin{problem}
%   Нека са дадени областите на Скот $\D$ и $\E$ и изображението 
%   \[\texttt{eval}: \Cont{\D}{\E} \times \D \to \E,\]
%   където 
%   \[\texttt{eval}(f,d) \dff f(d).\]
%   Докажете, че $\texttt{eval}$ е непрекъснато изображение.
% \end{problem}
% \ifhints
% \begin{hint}
%   Понеже $\bigsqcup_n(f_n,d_n) = (\bigsqcup_m f_m,\bigsqcup_n d_n)$, 
%   ще докажем, че \[\texttt{eval}(\bigsqcup_m f_m, \bigsqcup_n d_n) = \bigsqcup_n \texttt{eval}(f_n,d_n).\]
%   Знаем, че
%   \begin{align*}
%     \texttt{eval}(\bigsqcup_m f_m, \bigsqcup_n d_n) & = (\bigsqcup_m f_m)(\bigsqcup_n d_n) & (\mbox{от опр. на }\texttt{eval})\\
%     & = \bigsqcup_m (f_m(\bigsqcup_n d_n)) & (\mbox{от опр. на }\bigsqcup_mf_m)\\
%     & = \bigsqcup_m (\bigsqcup_n (f_m(d_n)) & (\mbox{всяка }f_m\mbox{ е непр.} )\\
%   \end{align*}
%   Нека да положим $e_{m,n} = f_m(d_n)$.
%   Лесно се съобразява, че
%   \[m \leq m^\prime\ \&\ n \leq n^\prime\ \Rightarrow\ e_{m,n} \sqsubseteq^\E e_{m^\prime,n^\prime}.\]
%   Така получаваме, че 
%   \begin{align*}
%     \texttt{eval}(\bigsqcup_m f_m, \bigsqcup_n d_n) & = \bigsqcup_m (\bigsqcup_n (f_m(d_n)) & (\mbox{от по-горе})\\
%     & = \bigsqcup_{m,n} e_{m,n} = \bigsqcup_{n} e_{n,n} & (\mbox{от \Th{double-chain}})\\
%     & = \bigsqcup_{n} f_n(d_n) & (\text{ от опр. на }e_{m,n})\\
%     & = \bigsqcup_n \texttt{eval}(f_n,d_n).
%   \end{align*}
% \end{hint}
% \fi

\begin{problem}
  Нека са дадени областите на Скот $\D$ и $\E$ и изображението 
  \[\texttt{eval}: \Cont{\D}{\E} \times \D \to \E,\]
  където 
  \[\texttt{eval}(f,d) \dff f(d).\]
  Докажете, че $\texttt{eval}$ е непрекъснато изображение.
\end{problem}
\ifhints
\begin{hint}
  Понеже $\bigsqcup_n(f_n,d_n) = (\bigsqcup_m f_m,\bigsqcup_n d_n)$, 
  ще докажем, че \[\texttt{eval}(\bigsqcup_m f_m, \bigsqcup_n d_n) = \bigsqcup_n \texttt{eval}(f_n,d_n).\]
  Знаем, че
  \begin{align*}
    \texttt{eval}(\bigsqcup_m f_m, \bigsqcup_n d_n) & = (\bigsqcup_m f_m)(\bigsqcup_n d_n) & \comment\text{от опр. на }\texttt{eval}\\
                                                    & = \bigsqcup_m (f_m(\bigsqcup_n d_n)) & \comment\text{от опр. на }\bigsqcup_mf_m\\
                                                    & = \bigsqcup_m (\bigsqcup_n (f_m(d_n)) & \comment\text{всяка }f_m\mbox{ е непр.}\\
  \end{align*}
  Нека да положим $e_{m,n} = f_m(d_n)$.
  Лесно се съобразява, че
  \[m \leq m^\prime\ \&\ n \leq n^\prime\ \Rightarrow\ e_{m,n} \sqsubseteq^\E e_{m^\prime,n^\prime}.\]
  Така получаваме, че 
  \begin{align*}
    \texttt{eval}(\bigsqcup_m f_m, \bigsqcup_n d_n) & = \bigsqcup_m (\bigsqcup_n (f_m(d_n)) & \comment\text{от по-горе}\\
    & = \bigsqcup_{m,n} e_{m,n} = \bigsqcup_{n} e_{n,n} & \comment\text{от \Th{double-chain}}\\
    & = \bigsqcup_{n} f_n(d_n) & \comment\text{ от опр. на }e_{m,n}\\
    & = \bigsqcup_n \texttt{eval}(f_n,d_n).
  \end{align*}
\end{hint}
\fi

%%% Local Variables:
%%% mode: latex
%%% TeX-master: "../sep"
%%% End:

\begin{problem}
  Нека изображението \[\texttt{comp}:(\Cont{\B}{\C} \times \Cont{\A}{\B}) \to \Cont{\A}{\C}\]
  е определено като 
  \[\texttt{comp}(g,f) \dff g\circ f.\]
  \marginpar{$(g \circ f)(a) = g(f(a))$}
  Докажете, че $\texttt{comp}$ е непрекъснато изображение.
\end{problem}
\ifhints
\begin{hint}
  \marginpar{\cite[стр. 124]{reynolds}}
  Трябва да докажем, че за всяка монотонно растяща редица $\{(g_n,f_n)\}_{n\in\Nat}$,
  \[\Gamma(\bigsqcup_n(g_n,f_n))(a) = \bigsqcup_n\Gamma(g_n,f_n)(a),\]
  за произволно $a \in A$.
  Да фиксираме едно $a\in A$ и да положим $g_n(f_k(a)) = e_{n,k}$.
  Лесно се вижда, че 
  \[n\leq n^\prime\ \&\ k \leq k^\prime\ \Rightarrow\ e_{n,k} \sqsubseteq e_{n^\prime,k^\prime}.\]
  Тогава:
  \begin{align*}
    \Gamma(\bigsqcup_n(g_n,f_n))(a) & = \Gamma(\bigsqcup_n g_n, \bigsqcup_k f_k)(a) & \\
                                    & = (\bigsqcup_n g_n)(\bigsqcup_k f_k(a)) & \comment\text{ по деф. на }\Gamma\\
                                    & = (\bigsqcup_n g_n)(\bigsqcup_k b_k) & \comment\text{ полагаме }b_k = f_k(a)\\
                                    & = \bigsqcup_k (\bigsqcup_n g_n)(b_k) & \comment\bigsqcup_n g_n\text{ е непр.}\\
                                    & = \bigsqcup_k(\bigsqcup_n g_n(b_k)) & \comment\text{ по деф. на }\bigsqcup_n g_n\\
                                    & = \bigsqcup_k (\bigsqcup_n g_n(f_k(a))) & \comment\text{ полагаме }e_{n,k} = g_n(f_k(a))\\
                                    & = \bigsqcup_k\bigsqcup_n e_{n,k} = \bigsqcup_n e_{n,n} & \comment\text{ от \Th{double-chain}}\\
                                    & = \bigsqcup_n g_n(f_n(a)) = \bigsqcup_n \Gamma(g_n, f_n)(a).
  \end{align*}
\end{hint}
\fi

\begin{remark}
  \marginpar{Когато на хаскел пишем $(.)$, означава, че операцията е инфиксна}
  В хаскел има операция композиция на функции.
  \begin{haskellcode}
ghci> :t (.)
(.) :: (b -> c) -> (a -> b) -> a -> c
  \end{haskellcode}
\end{remark}


%%% Local Variables:
%%% mode: latex
%%% TeX-master: "../sep"
%%% End:


% \begin{problem}
%   Нека изображението \[\texttt{comp}:(\Cont{\B}{\C} \times \Cont{\A}{\B}) \to \Cont{\A}{\C}\]
%   е определено като 
%   \[\texttt{comp}(g,f) \dff g\circ f.\]
%   \marginpar{$(g \circ f)(a) = g(f(a))$}
%   Докажете, че $\texttt{comp}$ е непрекъснато изображение.
% \end{problem}
% \ifhints
% \begin{hint}
%   \marginpar{\cite[стр. 124]{reynolds}}
%   Трябва да докажем, че за всяка монотонно растяща редица $\{(g_n,f_n)\}_{n\in\Nat}$,
%   \[\Gamma(\bigsqcup_n(g_n,f_n))(a) = \bigsqcup_n\Gamma(g_n,f_n)(a),\]
%   за произволно $a \in A$.
%   Да фиксираме едно $a\in A$ и да положим $g_n(f_k(a)) = e_{n,k}$.
%   Лесно се вижда, че 
%   \[n\leq n^\prime\ \&\ k \leq k^\prime\ \Rightarrow\ e_{n,k} \sqsubseteq e_{n^\prime,k^\prime}.\]
%   Тогава:
%   \begin{align*}
%     \Gamma(\bigsqcup_n(g_n,f_n))(a) & = \Gamma(\bigsqcup_n g_n, \bigsqcup_k f_k)(a) & \\
%     & = (\bigsqcup_n g_n)(\bigsqcup_k f_k(a)) & (\text{ по деф. на }\Gamma )\\
%     & = (\bigsqcup_n g_n)(\bigsqcup_k b_k) & (\text{ полагаме }b_k = f_k(a))\\
%     & = \bigsqcup_k (\bigsqcup_n g_n)(b_k) & (\bigsqcup_n g_n\text{ е непр.})\\
%     & = \bigsqcup_k(\bigsqcup_n g_n(b_k)) & (\text{ по деф. на }\bigsqcup_n g_n)\\
%     & = \bigsqcup_k (\bigsqcup_n g_n(f_k(a))) & (\text{ полагаме }e_{n,k} = g_n(f_k(a)))\\
%     & = \bigsqcup_k\bigsqcup_n e_{n,k} = \bigsqcup_n e_{n,n} & (\text{ от \Th{double-chain}})\\
%     & = \bigsqcup_n g_n(f_n(a)) = \bigsqcup_n \Gamma(g_n, f_n)(a).
%   \end{align*}
% \end{hint}
% \fi

% \begin{remark}
%   \marginpar{Когато пишем $(.)$ означава, че операцията е инфиксна}
%   В хаскел има операция композиция на функции.
%   \begin{haskellcode}
% ghci> :t (.)
% (.) :: (b -> c) -> (a -> b) -> a -> c
%   \end{haskellcode}
% \end{remark}

\begin{problem}
  \marginpar{\cite[стр. 131]{nikolova-soskova}}
  Нека $f \in \Cont{\A}{\B}$ и $g \in \Cont{\B}{\A}$.
  Докажете, че 
  \begin{itemize}
  \item 
    $\lfp(g \circ f) \sqsubseteq g(\lfp(f \circ g))$;
  \item
    $f(\lfp(g \circ f)) \sqsubseteq \lfp(f \circ g)$.
  \end{itemize}
  Оттук заключете, че 
  \[\lfp(g \circ f) = g(\lfp(f \circ g)) \text{ и }f(\lfp(g \circ f)) = \lfp(f \circ g).\]
\end{problem}


% \begin{problem}[Кантор-Шрьодер-Бернщайн]
%   \marginpar{\cite[стр. 639]{hanbook-cs}}
%   Нека имаме две инективни функции $f:A \to B$ и $g:B \to A$.
%   Тогава съществува биективна функция $h: A \to B$.  
% \end{problem}
% \begin{hint}
%   За множеството $B$, да дефинираме областта на Скот 
%   \[\D_B = (\Ps(B),\subseteq,\emptyset).\]
%   \begin{enumerate}[a)]
%   \item 
%     За дадените от условието инективни функции $f$ и $g$,
%     да разгледаме изображението $\Gamma:\D_B \to \D_B$ зададено като
%     \marginpar{Озн. $h(X) = \{h(x) \mid x\in X\}$, $h^{-1}(X) = \{z \mid h(z) \in X\}$}
%     \[\Gamma(X) = B\setminus f(A)\cup f(g(X)).\]
%     Докажете, че $F$ е непрекъснато изображение.
%   \item
%     \Stefan{Използвам, че $X_0$ е неподвижна точка, но не виждам къде използвам, че е най-малката.}
%     Нека $X_0 = \lfp(\Gamma)$. Тогава $X_0 = B\setminus f(A) \cup f(g(X_0))$.
%     Докажете, че 
%     \[B \setminus X_0 = f(A \setminus g(X_0)).\]
%   \item
%     Дефинираме функцията $h:A \to B$ по следния начин:
%     \begin{align*}
%       h(a) = 
%       \begin{cases}
%         g^{-1}(a), & a \in g(X_0)\\
%         f(a), & a \in A \setminus g(X_0).
%       \end{cases}
%     \end{align*}
%     Докажете, че $h$ е биекция.
%   \end{enumerate}
% \end{hint}

\begin{problem}% Gunter textbook
  Нека $f \in \Cont{\A}{\A}$.
  Да разгледаме множеството 
  \[B = \{a \in \A \mid f(a) = a\}.\]
  Вярно ли е, че 
  \[\B = (B, \sqsubseteq^\A, \lfp(f))\] е област на Скот?
  Обосновете се!
\end{problem}

\begin{problem}% Gunter textbook
  Нека $f \in \Cont{\A}{\A}$.
  Да разгледаме множеството 
  \[B = \{a \in \A \mid f(a) \sqsubseteq a\}.\]
  Вярно ли е, че 
  \[\B = (B, \sqsubseteq^\A, \lfp(f))\] е област на Скот?
  Обосновете се!
\end{problem}


\begin{problem} % Gunter textbook
  Да разгледаме множеството
  \[B = \{f \in \Mon{\A}{\A} \mid f\circ f = f\}.\]
  Вярно ли е, че 
  \[\B = (B,\ \sqsubseteq,\ \lambda x.\bot^\A)\] е област на Скот,
  където 
  \[f \sqsubseteq g \dfff (\forall a\in\A)[f(a) \sqsubseteq^\A g(a)] ?\]
  Обосновете се!
\end{problem}

% \begin{problem} % Gunter textbook
%   Да разгледаме множеството
%   \[B = \{f \in \Strict{\A}{\A} \mid f\circ f = f\}.\]
%   Вярно ли е, че 
%   \[\B = (B,\ \sqsubseteq,\ \lambda x.\bot^\A)\] е област на Скот,
%   където 
%   \[f \sqsubseteq g \dfff (\forall a\in\A)[f(a) \sqsubseteq^\A g(a)] ?\]
%   Обосновете се!
% \end{problem}

\begin{problem} % Gunter textbook
  Да разгледаме множеството
  \[B = \{f \in \Strict{\Nat_\bot}{\Nat_\bot} \mid f\circ f = f\}.\]
  Вярно ли е, че 
  \[\B = (B,\ \sqsubseteq,\ \lambda x.\bot)\] е област на Скот,
  където 
  \[f \sqsubseteq g \dfff (\forall a\in\Nat_\bot)[f(a) \sqsubseteq g(a)] ?\]
  Обосновете се!
\end{problem}

\begin{problem}
  % \marginpar{\cite[стр. 124]{reynolds}}
  Нека $f \in \Mon{\A}{\B}$ и $g \in \Mon{\B}{\A}$ имат свойствата:
  \begin{itemize}
  \item 
    $f\circ g = id_\B$;
  \item
    $g \circ f = id_\A$.
  \end{itemize}
  Докажете, че $f$ и $g$ са точни и непрекъснати.
\end{problem}


\begin{problem}
  % \marginpar{задачата е \href{http://www.cl.cam.ac.uk/teaching/exams/pastpapers/y2008p8q14.pdf}{оттук} и \href{http://www.cl.cam.ac.uk/teaching/exams/pastpapers/y1998p9q10.pdf}{оттук}}
  Да разгледаме областта на Скот 
  \[\O = (\{\bot,\top\},\sqsubseteq, \bot),\]
  където $\bot \sqsubseteq \top$.
  За произволна област на Скот $\A$ и елемент $a \in \A$, $a \neq \bot$, дефинираме изображенията:
  \begin{enumerate}[a)]
  \item
    $f_a:\A \to \O$, където
    \[f_a(x) \dff
    \begin{cases}
      \top, & a \sqsubseteq x\\
      \bot, & a \not\sqsubseteq x.
    \end{cases}\]
    Вярно ли е, че $f_a$ е точно непрекъснато изображение? Обосновете се!
  \item
    $\hat{f}_a:\A \to \O$, където
    \[\hat{f}_a(x) \dff
    \begin{cases}
      \bot, & a \sqsubseteq x\\
      \top, & a \not\sqsubseteq x.
    \end{cases}\]
    Вярно ли е, че $\hat{f}_a$ е точно непрекъснато изображение? Обосновете се!
  \item 
    $g_a:\A \to \O$, където
    \[g_a(x) \dff
    \begin{cases}
      \bot, & x \sqsubseteq a\\
      \top, & x \not\sqsubseteq a.
    \end{cases}\]
    Вярно ли е, че $g_a$ е точно непрекъснато изображение? Обосновете се!
  \item 
    $\hat{g}_a:\A \to \O$, където
    \[\hat{g}_a(x) \dff
    \begin{cases}
      \top, & x \sqsubseteq a\\
      \bot, & x \not\sqsubseteq a.
    \end{cases}\]
    Вярно ли е, че $\hat{g}_a$ е точно непрекъснато изображение? Обосновете се!
  \item
    Докажете, че 
    \[f \in \Cont{\D}{\A} \iff (\forall a \in \A)[g_a \circ f \in \Cont{\D}{\O}].\]
  \end{enumerate}
\end{problem}

\begin{problem}
  Да разгледаме изображението
  \[\Gamma: \Cont{\A}{\B} \times \Cont{\A}{\C} \to \Cont{\A}{\B\times\C},\]
  където $\Gamma(f,g)(a) \dff \pair{f(a),g(b)}$.
  \begin{itemize}
  \item
    Докажете, че $\Gamma$ е добре дефинирано изображение, т.е. за всеки непрекъснати $f$ и $g$,
    $\Gamma(f,g)$ е непрекъснато изображение.
  \item 
    Докажете, че $\Gamma$ е непрекъснато изображение.
  \end{itemize}
\end{problem}

\begin{problem}
  \marginpar{задачата е \href{http://www.cl.cam.ac.uk/teaching/exams/pastpapers/y2005p9q15.pdf}{оттук}}
  Докажете, че изображението
  \[\texttt{uncurry}:\Cont{\A}{\Cont{\B}{\C}} \to \Cont{\A\times \B}{\C},\]
  дефинирано като
  \[\texttt{uncurry}(f)(a,b) \dff f(a)(b),\]
  е непрекъснато.
\end{problem}

% \begin{problem}
%   Докажете, че изображението
%   \[\texttt{curry}:\Cont{\A\times \B}{\C} \to \Cont{\A}{\Cont{\B}{\C}},\]
%   дефинирано като
%   \[\texttt{curry}(f)(a)(b) \dff f(a,b),\]
%   е непрекъснато.
% \end{problem}

\begin{problem}
  Докажете, че изображението
  \[\texttt{curry}:\Cont{\A\times \B}{\C} \to \Cont{\A}{\Cont{\B}{\C}},\]
  дефинирано като
  \[\texttt{curry}(f)(a)(b) \dff f(a,b),\]
  е непрекъснато.
\end{problem}

%%% Local Variables:
%%% mode: latex
%%% TeX-master: "../sep"
%%% End:


\newpage
\subsection{Регулярни езици}

Да фиксираме азбуката $\Sigma = \{a_1,\dots,a_k\}$.
Да дефинираме полиномите над $\Sigma$ като
\[\tau ::= \emptyset\ |\ \varepsilon\ |\ a_i \cdot X_j\ |\ \tau_1 + \tau_2.\]
където $i = 1, \dots,k$, а $X$ е променлива.
За всеки полином $\tau[X_1,\dots,X_n]$ дефинираме оператора 
\[\val{\tau}: \mathcal{P}(\Sigma^\star)^n \to \mathcal{P}(\Sigma^\star)\]
 по следния начин:
\begin{itemize}
\item
    $\val{\emptyset}(L_1,\dots,L_n) = \emptyset$.
\item 
  $\val{\varepsilon}(L_1,\dots,L_n) = \varepsilon$.
\item 
  $\val{a_i \cdot X_j}(L_1,\dots,L_n) = \{a_i\} \cdot L_j$.
\item
  $\val{\tau_1 + \tau_2}(L_1,\dots,L_n) = \val{\tau_1}(L_1,\dots,L_n) \cup \val{\tau_2}(L_1,\dots,L_n)$.
\end{itemize}

\begin{problem}
  Докажете, че за всеки полином $\tau$ имаме, че $\val{\tau}$ е непрекъснато изображение в областта на Скот
  $\mathcal{S} = ( \mathcal{P}(\Sigma^\star),\subseteq, \emptyset)$.
\end{problem}


\begin{example}
  Да разгледаме системата 
  \marginpar{$\tau_1[X_1,X_2] \equiv b \cdot X_1 + a \cdot X_2$}
  \marginpar{$\tau_2[X_1,X_2] \equiv \varepsilon$}
  \begin{align*}
    & X_1 = b \cdot X_1 + a\cdot X_2\\
    & X_2 = \varepsilon.
  \end{align*}

  % Понеже $\val{\tau}$ е непрекъснат оператор, то той има най-малка неподвижна точка.
  Дефинираме непрекъснатия оператор 
  \[\Gamma:\mathcal{P}(\Sigma^\star)^2 \to \mathcal{P}(\Sigma^\star)^2,\]
  където:
  \[\Gamma(L_1,L_2) = (\val{\tau_1}(L_1,L_2), \val{\tau_2}(L_1,L_2)).\]

  От Теоремата на Клини ние знам как можем да намерим най-малката неподвижна точка на $\Gamma$,
  която ще бъде и най-малкото решение на горната система.

  \begin{itemize}
  \item 
    $(L_0,M_0) \df (\emptyset,\emptyset)$;
  \item
    $(L_1,M_1) \df \Gamma(L_0,M_0) = (\val{\tau_1}(L_0,M_0), \val{\tau_2}(L_0,M_0)) = (\emptyset, \varepsilon)$;
  \item
    $(L_2,M_2) \df \Gamma(L_1,M_1) = (\val{\tau_1}(L_1,M_1), \val{\tau_2}(L_1,M_1)) = (\{a\},\varepsilon)$;
  \item
    $(L_3,M_3) \df \Gamma(L_2,M_2) = (\val{\tau_1}(L_2,M_2), \val{\tau_2}(L_2,M_2)) = (\{ba,a\},\varepsilon)$;
  \item
    $(L_4,M_4) \df \Gamma(L_3,M_3) =(\val{\tau_1}(L_3,M_3), \val{\tau_2}(L_3,M_3)) = (\{bba, ba,a\},\varepsilon)$;
  \item
    $(L_5,M_5) \df \Gamma(L_4,M_4) = ( \val{\tau_1}(L_4,M_4), \val{\tau_2}(L_4,M_4)) = (\{bba, bba, ba,a\},\varepsilon)$.
  \end{itemize}
  Лесно се съобразява, че $L_n = \{ b^ka \mid k < n\}$.
  Тогава
  \[\lfp( \Gamma ) = (\bigcup_n L_n, \{\varepsilon\}) = (b^\star a, \{\varepsilon\} ).\]
\end{example}


\begin{problem}
  Докажете, че най-малкото решение на системата 
  \begin{align*}
    & X_1 = a \cdot X_1 + b \cdot X_2 + \varepsilon\\
    & X_2 = b \cdot X_2 + \varepsilon
  \end{align*}
  е двойката $(a^\star b^\star, b^\star)$.
\end{problem}

\begin{problem}
  Да разгледаме системата от оператори
  \begin{align*}
    & \val{\tau_1}(L_1,\dots,L_n) = L_1\\
    & \ \ \vdots\\
    & \val{\tau_n}(L_1,\dots,L_n) = L_n.
  \end{align*}
  Знаем, че тя притежава най-малко решение $(\hat{L}_1,\dots,\hat{L}_n)$.
  Докажете, че всеки от езиците $\hat{L}_i$ е регулярен.

  Докажете, че всеки регулярен език е елемент от най-малкото решение 
  на някоя система от оператори от горния вид.
\end{problem}


% \begin{problem}
%   Докажете, че всеки регулярен език е елемент на най-малкото решение на някоя система от $n$
%   полинома с $n$ променливи за някое $n$.
% \end{problem}

% \begin{problem}
%   Докажете, че всяко най-малко решение на система от $n$ полинома с $n$ променливи представлява $n$-орка от 
%   регулярни езици.
% \end{problem}

% \begin{problem}
%   Опишете алгоритъм, по който може от система от $n$ полинома с $n$ променливи да се построи 
%   краен автомат с $n$ състояния.
% \end{problem}

% \begin{problem}
%   Опишете алгоритъм, по който може от краен автомат с $n$ състояния може да се построи 
% \end{problem}



\subsection{Безконтекстни езици}

Да фиксираме азбуката $\Sigma = \{a_1,\dots,a_n\}$.
Да дефинираме термове от тип 1 като
\[\tau ::= X_i\ |\ a_j\ |\ \varepsilon\ |\ \emptyset\ |\ \tau_1 \cdot \tau_2\ |\ (\tau_1 + \tau_2),\]
където $j = 1, \dots,n$, а $X_i$ са изброимо безкрайна редица от променливи.
За всеки терм $\tau[X_1,\dots,X_n]$ дефинираме оператора 
\[\val{\tau}: (\mathcal{P}(\Sigma^\star))^n \to \mathcal{P}(\Sigma^\star)\]
 по следния начин:
\begin{itemize}
\item 
  $\val{X_i}(L_1,\dots,L_n) = L_i$.
\item 
  $\val{a_j}(L_1,\dots,L_n) = \{a_j\}$.
\item 
  $\val{\varepsilon}(L_1,\dots,L_n) = \varepsilon$.
\item 
  $\val{\emptyset}(L_1,\dots,L_n) = \emptyset$.
\item 
  $\val{\tau_1 \cdot \tau_2}(L_1,\dots,L_n) = \val{\tau_1}(L_1,\dots,L_n) \cdot \val{\tau_2}(L_1,\dots,L_n)$.
\item
  $\val{\tau_1 + \tau_2}(L_1,\dots,L_n) = \val{\tau_1}(L_1,\dots,L_n) \cup \val{\tau_2}(L_1,\dots,L_n)$.
\end{itemize}

\begin{problem}
  Докажете, че за всеки терм $\tau$, $\val{\tau}$ е непрекъснато изображение в областта на Скот
  $\mathcal{S} = ( \mathcal{P}(\Sigma^\star),\subseteq, \emptyset)$.
\end{problem}

\begin{problem}
  Докажете, че $\{a^nb^n \mid n\in \Nat\} = \lfp(\val{\tau})$, където 
  \[\tau[X] \equiv \varepsilon + a \cdot X \cdot b.\]
  С други думи, $\{a^nb^n \mid n \in \Nat\}$ е най-малкото решение на уравнението
  \[X = a \cdot X \cdot b + \varepsilon.\]
\end{problem}

Нека сега да разгледаме термовете $\tau_1[X_1,\dots,X_n], \dots, \tau_n[X_1,\dots,X_n]$.

\begin{problem}
  Да разгледаме системата от оператори
  \begin{align*}
    & \val{\tau_1}(L_1,\dots,L_n) = L_1\\
    & \ \ \vdots\\
    & \val{\tau_n}(L_1,\dots,L_n) = L_n.
  \end{align*}
  Знаем, че тя притежава най-малко решение $(\hat{L}_1,\dots,\hat{L}_n)$.
  Докажете, че всеки от езиците $\hat{L}_i$ е безконтекстен.

  Докажете, че всеки безконтекстен език е елемент от най-малкото решение 
  на някоя система от оператори от горния вид.
\end{problem}

\begin{problem}
  \marginpar{Това е аналог на нормалната форма на Чомски}
  Да дефинираме термове от тип 2 като
  \[\tau ::= a_j\ |\ \varepsilon\ |\ \emptyset\ |\ X_i \cdot X_k\ |\ (\tau_1 + \tau_2),\]
  където $j = 1, \dots,n$, а $X_i$ са изброимо безкрайна редица от променливи.
  Докажете горното твърдение, като замените термовете от тип 1 с тези от тип 2.
\end{problem}

\begin{example}
  Да разгледаме системата
  \begin{align*}
    & X_1 = X_3 \cdot X_2 + \varepsilon\\
    & X_2 = X_1 \cdot X_4\\
    & X_3 = a\\
    & X_4 = b.
  \end{align*}


  % \begin{align*}
  %   & \val{\varepsilon + X_3 \cdot X_2}(L_1, L_2, L_3, L_4) = L_1\\
  %   & \val{X_1 \cdot X_4}(L_1, L_2, L_3, L_4) = L_2\\
  %   & \val{a}(L_1, L_2, L_3, L_4) = L_3\\
  %   & \val{b}(L_1, L_2, L_3, L_4) = L_4\\
  % \end{align*}
  Нека $(\hat{L}_1, \hat{L}_2, \hat{L}_3, \hat{L}_4)$ е най-малкото решение на системата.
  Докажете, че $\hat{L}_1 = \{a^nb^n\mid n \in \Nat\}$ $\hat{L}_2 = \{a^nb^{n+1}\mid n \in \Nat\}$,
  $\hat{L}_3 = \{a\}$ и $\hat{L}_4 = \{b\}$.
\end{example}



%%% Local Variables:
%%% mode: latex
%%% TeX-master: "../sep"
%%% End:




% \subsection{Правило на Скот}

% \begin{itemize}
% \item 
%   Нека приемем, че имаме една програма, която работи върху списъци.
%   Ако искаме да докажем, че дадено свойство е вярно за тази програма, когато аргументите са {\em крайни} списъци,
%   то можем да подходим с обикновена индукция по структурата на списъците.
% \item
%   Трябва да сме по-внимателни, ако искаме да докажем, че дадено свойство е вярно за програмата, когато
%   допускаме като аргументи прозволни списъци, т.е. позволяваме частични и безкрайни списъци.
%   Понеже един безкраен списък $l = \bigsqcup_n l_n$, където $l_n $
% \item
%   Да разгледаме едно {\em непрекъснато свойство} $P \subseteq \PartL \cup \InfL$.
%   Тогава имаме правилото:
%   \begin{prooftree}
%     \AxiomC{$P(\bot)$}
%     \AxiomC{$(\forall a\in\Nat)(\forall l \in \PartL)[P(l) \implies P(\pair{a,l})]$}
%     \BinaryInfC{$(\forall l \in \PartL \cup \InfL)[P(l)]$}
%   \end{prooftree}
  
%   Да видим защо това правило е изпълнено.
%   Да разгледаме един произволен безкраен списък $l$.
%   Знаем, че $l = \bigsqcup_n (l\upharpoonright n)$.
%   Знаем, че $P(l \upharpoonright 0)$ и с индукция по естествените числа имаме, че за всяко $n$,
%   $P(l \upharpoonright n)$.
%   Тогава, понеже $P$ е непрекъснато свойство, следва 
%   \[P(\bigsqcup_n (l \upharpoonright n)).\]
% \end{itemize}

% \begin{problem}
%   % Нека е дадена програмата на
  
%   % \begin{haskellcode}
%   %   rev :: [a] -> [a]
%   %   rev x = f(x, []) where 
%   %     f([], y) = y
%   %     f(x:xs, y) = f(xs, x:y)
%   % \end{haskellcode}

%   Да разгледаме оператора:

%   \[
%   \Gamma(f)(l_1,l_2) = 
%   \begin{cases}
%     \bot, & l_1 = \bot\\
%     l_2, & l_1 = \nil\\
%     f(l'_1, \pair{a,l_2}), & l_1 = \pair{a,l'_1}.
%   \end{cases}
%   \]

%   Докажете, че $\Gamma$ е непрекъснат оператор.
%   Да положим 
%   \[\texttt{rev}(l) = \lfp(\Gamma)(l,\nil).\]
%   Докажете, че:
%   \begin{enumerate}[a)]
%   \item 
%     $(\forall l \in L)[\texttt{rev}(l) \neq \bot \iff l \in \FinL]$.
%   \item
%     $(\forall l \in \FinL)[\texttt{rev}(\texttt{rev}(l)) = l]$.
%   \item
%     $(\forall l \in \FinL)[\texttt{rev}(l) = l^R]$.
%   \item
%     $(\forall l \in \PartL \cup \InfL)[rev(l) = \bot]$.
%   \end{enumerate}
% \end{problem}
% \begin{hint}
%   % Ще използваме следното правило:
%   % \begin{prooftree}
%   %   \AxiomC{$P(\nil)$}
%   %   \AxiomC{$(\forall x \in \Sigma^\star)[x\neq\nil\ \&\ P(cdr(x)) \to P(x)]$}
%   %   \RightLabel{\scriptsize(1)}
%   %   \BinaryInfC{$(\forall x\in\Sigma^\star)[P(x)]$}
%   % \end{prooftree}
%   % % \item
%   %   Докажете валидността на правилото $(1)$. % е еквивалентно на структурна индукция върху фундираната наредба
%     % $(\Sigma^\star,\prec)$, където $x \prec y \iff (\exists z\in\Sigma^\star)[z\cdot x = y]$, т.е.
%     % $x$ е суфикс на $y$.
%   % Да разгледаме фундираната наредба $L = (\Sigma^\star, \prec)$, където
%   % $x \prec y \iff \abs{x} < \abs{y}$.
%   \begin{enumerate}[a)]
%   \item
%     Докажете, че $P$ е непрекъснато свойство, където
%     $P(f) \dfff (\forall l \in L)[f(l) \neq \bot \iff l \in \FinL]$.
    
%     % Да разгледаме свойството 
%     % \[P(x) \equiv (\forall y\in \Sigma^\star)[f(x,y)\text{ е дефинирана}].\]
%     % Докажете със структурна индукция по $L$, че $(\forall x\in\Sigma^\star)[P(x)]$.
%   \item
%     Да разгледаме свойството 
%     \[P(x) \equiv (\forall y\in \FinL)[\texttt{rev}(f(x,y)) = f(y,x)].\]
%     Докажете със структурна индукция по $L$, че $(\forall x\in\Sigma^\star)[P(x)]$.
%     Тогава в частния случай $y = \nil$, 
%     \[\texttt{rev}(\texttt{rev}(x)) = \texttt{rev}(f(x,\nil)) = f(\nil,x) = x.\]
%   \item
%     Разгледайте свойството
%     \[P(x) \equiv (\forall y\in L)[f(x,y) = x^R \cdot y].\]
%   \end{enumerate}
% \end{hint}

% \begin{problem}
%   Нека е дадена програмата на езика хаскел:

%   \begin{haskellcode}
% conc :: [Int] -> [Int] -> [Int]
% conc [] y = y
% conc (x:xs) y = x:conc xs y
%   \end{haskellcode}

% Да видим няколко примера:

% \begin{haskellcode}
% ghci> conc([1..10],undefined)  -- $ == \pair{1,2,3,4,5,6,7,8,9,10,\bot}$
% [1,2,3,4,5,6,7,8,9,10*** Exception: Prelude.undefined
% ghci> conc(undefined, [1..10]) -- $== \bot$
% *** Exception: Prelude.undefined
% \end{haskellcode}

%   На нея съответсва оператора

%   \[\Gamma(f)(x,y) = 
%   \begin{cases}
%     \bot, & x = \bot\\
%     y, & x = \nil\\
%     \pair{a, f(x',y)}, & x = \pair{a,x'}.
%   \end{cases}\]
  
%   Да положим $\texttt{conc}(x,y) \dff \lfp(\Gamma)(x,y)$.
  
%   \noindent Проверете дали са изпълнени свойствата:
%   \begin{enumerate}[a)]
%   \item
%     $(\forall x\in \FinL)(\forall y \in L)[\texttt{conc}(x, y) = x \cdot y]$;
%   \item
%     $(\forall x\in \PartL \cup \InfL)(\forall y \in L)[\texttt{conc}(x, y) = x]$;
%   \item 
%     $(\forall x,y,z\in L)[\texttt{conc}(\texttt{conc}(x, y), z) = \texttt{conc}(x, \texttt{conc}(y, z))]$;
%   \item
%     $(\forall x,y \in L)[\texttt{rev}(\texttt{conc}(x, y)) = \texttt{conc}(\texttt{rev}(y), \texttt{rev}(x))]$;
%   \end{enumerate}
% \end{problem}
% \begin{hint}
%   \begin{enumerate}[a)]
%   \item
%     Да разгледаме фундираната наредба $(\FinL, \prec)$, където
%     $x \prec y \iff \abs{x} < \abs{y}$.
%     Разгледайте свойството
%     \[P(x) \equiv (\forall y\in L)[\texttt{conc}(x, y) = x \cdot y].\]
%   \item
%     Разгледайте свойствотото:
%     \[P(x) \equiv (\forall y \in L)[\texttt{conc}(x, y) = x].\]
%     Съобразете, че това е непрекъснато свойство.
%     Докажете, че $P(\bot)$ и че $(\forall a \in \Nat)(\forall l \in \PartL \cup \InfL)[P(l) \implies P(\pair{a,l})]$.
%     Оттук, по правилото на Скот следва, че $P(l)$ за всяко $l \in \PartL \cup \InfL$.
%   \end{enumerate}
% \end{hint}

% \begin{problem}
%   На следната програмата на хаскел:
%   \begin{haskellcode}
% merge :: [Int] -> [Int] -> [Int]
% merge []     y       = y
% merge x      []      = x
% merge (x:xs) (y:ys)  = if x <= y then x:merge xs (y:ys)
%                          else y:merge (x:xs) ys
%   \end{haskellcode}
% отговаря операторът
% \[
% \Gamma(f)(x,y) =
% \begin{cases}
%   \bot, & x = \bot\ \vee\ y = \bot\\
%   y, & x = \nil \\
%   x, & y = \nil \\
%   \pair{a,f(x',y)}, & x = \pair{a,x'}\ \&\ y = \pair{b,y'}\ \&\ a \leq b\\
%   \pair{b,f(x,y')}, & x = \pair{a,x'}\ \&\ y = \pair{b,y'}\ \&\ a > b
% \end{cases}
% \]

% \begin{enumerate}[a)]
% \item 
%   Вярно ли е, че
%   \[(\forall x,y,z \in L)[\texttt{merge}(x,\texttt{merge}(y,z)) = \texttt{merge}(\texttt{merge}(x,y),z)]?\]
% \end{enumerate}
% \end{problem}


% \begin{problem}
%   Да разгледаме следната програма:
  
%   \begin{haskellcode}
% map :: (Int -> Int) -> [Int] -> [Int]
% map f []     = []
% map f (x:xs) = (f x):map f xs
%   \end{haskellcode}

%   Да разгледаме оператора
%   \[\Gamma(g)(f,l) = 
%   \begin{cases}
%     \bot, & l = \bot\\
%     \nil, & l = \nil\\
%     \pair{f(a),g(f,l')}, & l = \pair{a,l'}.
%   \end{cases}\]
%   Лесно се вижда, че този оператор е непрекъснат.
%   Нека $\texttt{map} \dff \lfp(\Gamma)$.

%   \begin{enumerate}[a)]
%   \item 
%     $\texttt{map}(f,\texttt{map}(g, l)) = \texttt{map}(f \circ g, l)$.
%   \end{enumerate}
% \end{problem}

% \begin{problem}
%   Да разгледаме следната програма:
  
%   \begin{haskellcode}
% foldr :: (Int -> Int -> Int) -> Int -> [Int] -> [Int]
% foldr f e []     = e
% foldr f e (x:xs) = f x (foldr f e xs)
%   \end{haskellcode}

%   Да разгледаме оператора
%   \[\Gamma(g)(f,z,l) = 
%   \begin{cases}
%     \bot, & l = \bot\\
%     z, & l = \nil\\
%     f(x, g(f,z,l')), & l = \pair{a,l'}.
%   \end{cases}\]
%   Лесно се вижда, че този оператор е непрекъснат.
%   Нека $\texttt{foldr} \dff \lfp(\Gamma)$.

%   \begin{enumerate}[a)]
%   \item 
%     Нека имаме свойствата 
%     \begin{itemize}
%     \item 
%       $f(\bot) = \bot$
%     \item
%       $f(g(x,y)) = h(x, f(y))$      
%     \end{itemize}
%     Тогава
%     $f(\texttt{foldr}(g, a, l)) = \texttt{foldr}(h, f(a), l)$
%   \end{enumerate}
% \end{problem}


% \begin{problem}
%   Да разгледаме следната програма:
  
%   \begin{haskellcode}
% foldl :: (Int -> Int -> Int) -> Int -> [Int] -> [Int]
% foldl f e []     = e
% foldl f e (x:xs) = foldl (f e x) xs
%   \end{haskellcode}

%   Да разгледаме оператора
%   \[\Gamma(g)(f,e,l) = 
%   \begin{cases}
%     \bot, & l = \bot\\
%     \nil, & l = \nil\\
%     g(f, f(e,a), l'), & l = \pair{a,l'}.
%   \end{cases}\]
%   Лесно се вижда, че този оператор е непрекъснат.
%   Нека $\texttt{foldl} \dff \lfp(\Gamma)$.
% \end{problem}

% \begin{problem}
%   Вярно ли е, че следните две програмите $\texttt{evens}$ и $\texttt{evens'}$ дефинират една и съща функция ?
%   \begin{haskellcode}
%     evens []       = []
%     evens [x]      = []
%     evens (_:x:xs) = x:evens xs
    
%     evens' xs = f xs [] where
%     f [] ys       = ys
%     f [x] ys      = ys
%     f (_:x:xs) ys = f xs (x:ys)  
%   \end{haskellcode}
% \end{problem}


% \subsection{Точни и неточни изображения}

% Една функция $f(x,y)$ е точна, ако $f(x,\bot) = \bot$ и $f(\bot,y) = \bot$,
% за произволни $x$ и $y$.

% \begin{haskellcode}
% -- Да видим, че стандартната операцията дизюнкция в хаскел е почти неточна
% ghci> :t (||)
% (||) :: Bool -> Bool -> Bool
% ghci> True || undefined
% True
% ghci> False || undefined
% *** Exception: Prelude.undefined
% ghci> undefined || True
% *** Exception: Prelude.undefined
% ghci> foldr (||) False [False, True, undefined, True]
% True
% ghci> foldr (||) False [False, undefined, True]
% *** Exception: Prelude.undefined
% ghci> :set +s
% ghci> foldr (||) False (map (>0) [-999999..])
% True
% (0.18 secs, 188,851,600 bytes)
% \end{haskellcode}

% Да разгледаме следната програма:
% \begin{haskellcode}
% mult x y = if x == 0 then 0
%           else mult (x-1) y + y
% \end{haskellcode}

% На тази програма съответства операторът:
% \begin{align*}
%   \Gamma(f)(x,y) = 
%   \begin{cases}
%     0, & x = 0\ \&\ y \in \Nat_\bot\\
%     f(x-1,y)+y, & x > 0\ \&\ y \in \Nat_\bot\\
%     \bot, & x = \bot\ \&\ y \in \Nat_\bot.
%   \end{cases}
% \end{align*}

% \marginpar{Тук използваме, че $\bot + x = x + \bot = \bot$}

% Тогава най-малката неподвижна точка на $\Gamma$, или еквивалентно, 
% семантиката на програмата $\texttt{mult}$ е функцията $f_\Gamma$, където
% \[f_\Gamma(x,y) =
% \begin{cases}
%   0, & x = 0\ \&\ y = \bot\\
%   x \cdot y, & x \in \Nat\ \& \ y \in \Nat\\
%   \bot, & \text{иначе}
% \end{cases}\]
% \marginpar{Това не е напълно коректно, защото на хаскел няма тип на естествените числа, а само на целите}
% Това означава, че $f_\Gamma$ е точна по първия си аргумент и неточна по втория си аргумент.

% \begin{haskellcode}
% ghci> mult 0 undefined
% 0
% ghci> mult undefined 0
% *** Exception: Prelude.undefined
% ghci> :set +s
% ghci> foldr mult 1 [0..]
% 0
% (0.02 secs, 8,959,824 bytes)
% ghci> foldr (*) 1 [0..]
% Interrupted. -- забива
% ghci> foldr mult 1 [1000,999..] -- [1000,999..] == [1000,999,998,997,..]
% 0
% (0.65 secs, 230,882,664 bytes)
% \end{haskellcode}

% Нека сега да разгледаме точен вариант на дизюнкцията.

% \begin{haskellcode}
% or' :: Bool -> Bool -> Bool
% or' !x !y = if x then x
%             else if y then y
%                  else False
% \end{haskellcode}

% Да разгледаме сега пак примера отгоре.
% \begin{haskellcode}
% ghci> True `or'` undefined
% undefined
% ghci> False `or'` undefined
% *** Exception: Prelude.undefined
% ghci> undefined `or'` True
% *** Exception: Prelude.undefined
% ghci> foldr or' False [False, True, undefined, True]
% *** Exception: Prelude.undefined
% ghci> foldr or' False [False, undefined, True]
% *** Exception: Prelude.undefined
% ghci> foldr or' False (map (>0) [-9999..])
% C-c C-cInterrupted.
% \end{haskellcode}

% \begin{example}
%   Защо не можем да използваме операция $+$, за която $x + \bot = x$?
%   Да разгледаме изображението $g:\Nat^2_\bot \to \Nat_\bot$, където
%   \[g(x,y) = 
%   \begin{cases}
%     x, & x \in \Nat_\bot\ \&\ y = \bot\\
%     x + y, & x\in \Nat\ \&\ y \in \Nat\\
%     \bot, & x = \bot\ \&\ y \in \Nat.
%   \end{cases}\]
%   Лесно се вижда, че $g$ не е монотонно, откъдето следва, че $g$ не е непрекъснато.
%   Например, $\pair{1,\bot} \sqsubseteq \pair{1,1}$, 
%   но $g(1,\bot) = 1 \not\sqsubseteq 2 = g(1,1)$.
%   Това означава, че е безсмислено да разглеждаме тази версия събирането, защото
%   ние се интересуваме само от непрекъснати изображения.
% \end{example}


% \section*{Бележки}

% \begin{itemize}
% \item
%   Много от теоретичните задачи мога да се намерят в \cite[Глава 2]{domains-book}.
% \item
%   На \cite[стр. 64]{bird-haskell} са дефинирани крайни, частични и безкрайни списъци в хаскел.
% \item
%   \cite[Глава 9]{bird-haskell} е посветена на безкрайни списъци.
% \item
%   Срещаната в литературата дефиниция на алгебрична област на Скот е по-обща от тази,
%   която ние разглеждаме. Тук се ограничаваме само до точни горни граници на вериги, а по-общата дефиниция е да се 
%   разглеждат точни горни граници на насочени множества.
% \item 
%   Ние се интересуваме основно само от алгебрични области на Скот.
% \item
%   Ситуацията с ленивите списъци в хаскел е малко по-сложна, защото примерно можем да 
%   разглеждаме списъци от вида $\pair{\bot,\bot,\bot,\nil}$.

%   \begin{haskellcode}
% ghci> length [undefined,undefined,undefined]
% 3
%   \end{haskellcode}
% \end{itemize}

% \begin{haskellcode}
% data LazyList a = Nil | Cons a (LazyList a) deriving (Show)

% data EagerList a = Nil' | Cons' !a !(EagerList a) deriving (Show)

% -- Еквиваленти на функцията take от Prelude

% grab 0 _ = Nil
% grab n Nil = Nil
% grab n (Cons x y) = Cons x (grab (n-1) y)

% grab' 0 _ = Nil'
% grab' n Nil' = Nil'
% grab' n (Cons' x y) = Cons' x (grab' (n-1) y)

% -- Безкрайни списъци започващи от x

% inf x = Cons x (inf (x+1))

% inf' x = Cons' x (inf' (x+1))

% h n = grab n (inf 0)

% h' n = grab' n (inf' 0)

% -- Какъв резултата от h inf ?
% -- Какъв резултата от h' inf' ?

% len Nil = 0
% len (Cons _ tail) = 1 + (len tail)

% len' Nil' = 0
% len' (Cons' _ tail) = 1 + (len' tail)

% x = Cons undefined (Cons undefined (Cons undefined Nil))
% x' = Cons' undefined (Cons' undefined (Cons' undefined Nil'))

% -- Какъв е резултата от len x ?
% -- Какъв е резултата от len' x' ?

% \end{haskellcode}


%%% Local Variables:
%%% mode: latex
%%% TeX-master: "../sep"
%%% End:

% \chapter{Езикът {\bf REC}}
\label{ch:rec}

\marginpar{Основно следваме \cite[Глава 9]{winskel}}
\section{Синтаксис}
Ще разглеждаме един много прост език за функционално програмиране.
\begin{itemize}
\item
  \index{константа}
  \marginpar{За разлика от \cite{ditchev-soskov}, няма да въвеждаме термове от тип $\BB$}
  \marginpar{Константите не са числа! Константите са синтактични обекти, докато числата са семантични обекти}
  константи $\vv{n}\in\NN_\bot$, за всяко $n \in \Nat_\bot$;
  \Stefan{Да ги нарека обектови константи. По тази логика, операциите стават функционални константи}
\item
  \index{променлива!обектова}
  \index{променлива!нулев тип}
  \marginpar{Удобно е в нашия език още на синтактично ниво да правим разлика между двата типа променливи, които имаме в езика}
  изброимо много променливи от тип 0 (обектови променливи) $\vv{x}, \vv{y}, \vv{z}, \dots$, евентуално с индекси;
\item
  \index{променлива!функционална}
  изброимо много променливи от тип 1 (или функционални променливи) $\vv{f},\vv{g},\vv{h},\dots$, евентуално с индекси. 
  Формално погледнато, трябва на всяка функционална применлива $\vv{f}$
  да съпоставим число - брой аргументи, които приема. Нека да означим с $\#\vv{f}$ броят аргументи на $f$.
  Обикновено броят аргументи на $\vv{f}$ ще е ясен от контекста.
\item
  Термовете $\tau_0,\tau_1,\dots$ в {\bf REC} имат следния синтаксис:
  \marginpar{Тук $m_i = \#\vv{f}_i$}
  \[\tau ::= \vv{n} \mid \vv{x} \mid \tau_1 + \tau_2 \mid \tau_1\ \vv{==}\ \tau_2 \mid \ifelse{\tau_1}{\tau_2}{\tau_3} \mid \vv{f}_i(\tau_1,\dots,\tau_{m_i}).\]
\item
  Ще записваме $\tau[\vv{x}_1,\dots,\vv{x}_n,\vv{f}_1,\dots,\vv{f}_k]$, когато искаме да означим, че променливите
  на терма $\tau$ са {\em измежду} посочените.
\item
  С $\tau[\vv{x}/\mu]$ ще означаваме терма получен от $\tau$, в който всяко срещане на обектовата променлива $\vv{x}$
  е заменена с терма $\mu$.
\end{itemize}

\marginpar{\cite[стр. 141]{winskel}}

\index{рекурсивна програма}
Една {\bf рекурсивна програма} $\vv{P}$ на езика {\bf REC} има следния общ вид:
\marginpar{Една програма е просто текст със специален формат. Важният въпрос е каква функция (семантика) отговаря на този текст (синтаксис)}
\marginpar{Може да си мислите, че $\vv{f}_0$ е $\vv{main}$ функцията на нашата програма}
\begin{align*}
  \vv{P} = 
  \begin{cases}
    & \vv{f}_0(\vv{x}_1,\dots,\vv{x}_{m_0}) = \tau_0[\vv{x}_1,\dots,\vv{x}_{m_0},\vv{f}_0,\vv{f}_1,\dots,\vv{f}_k]\ \texttt{where}\\
    & \quad\vv{f}_1(\vv{x}_1,\dots,\vv{x}_{m_1}) = \tau_1[\vv{x}_1,\dots,\vv{x}_{m_1},\vv{f}_0,\vv{f}_1,\dots,\vv{f}_k]\\
    & \quad \vdots\\
    & \quad \vv{f}_k(\vv{x}_1,\dots,\vv{x}_{m_k}) = \tau_k[\vv{x}_1,\dots,\vv{x}_{m_k},\vv{f}_0,\vv{f}_1,\dots,\vv{f}_k]
  \end{cases}
\end{align*}

В такъв случай казваме, че термът $\tau_i$ задава {\em дефиницията} на фунционалната променлива $\vv{f}_i$.

\begin{example}
  \label{ex:minus}
  Да разгледаме програмата $\vv{P}$:
  \begin{haskellcode}
h(x) = f(x, 1) where
  f(x,y) = if x == y then 0 
             else f(x, y+1) + 1
  \end{haskellcode}
  Да положим
  \begin{align*}
    & \tau_0[\vv{x},\vv{f}] \dfff \vv{f}(\vv{x},\vv{1})\\
    & \tau_1[\vv{x},\vv{y},\vv{f}] \dfff \ifelse{\vv{x == y}}{\vv{0}}{\vv{f(x,y+1) + 1}}.
  \end{align*}
  Тогава програмата $\vv{P}$ приема следния вид:
  \begin{align*}
    \vv{h}(&\vv{x}) = \tau_0[\vv{x},\vv{f}] \vv{ where }\\
    & \vv{f}(\vv{x},\vv{y}) = \tau_1[\vv{x},\vv{y},\vv{f}].
  \end{align*}
\end{example}

%%% Local Variables:
%%% mode: latex
%%% TeX-master: "../sep-notes"
%%% End:


\section{Денотационна семантика}

\subsection{Стойност на терм}

Нека първо да дефинираме следните {\em изображения}
\begin{align*}
  & \texttt{plus} : \Nat^2_\bot \to \Nat_\bot\text{, където}\\
  & \texttt{plus}(a,b) =
    \begin{cases}
      a+b, & \text{ако }a,b \in \Nat\\
      \bot, & \text{ако }\bot \in \{a,b\}
    \end{cases}\\
  & \texttt{eq} : \Nat^2_\bot \to \Nat_\bot\text{, където}\\
  & \texttt{eq}(a,b) =
    \begin{cases}
      1, & \text{ако }a = b\ \&\ a,b \in \Nat\\
      0, & \text{ако }a \neq b\ \&\ a,b \in \Nat\\
      \bot, & \text{ако }\bot \in \{a,b\}
    \end{cases}\\
  & \texttt{if}:\Nat^3_\bot \to \Nat_\bot\text{, където}\\
  & \texttt{if}(a,b,c) =
    \begin{cases}
      b, & \text{ако } a \in \Nat^+\\
      c, & \text{ако } a = 0\\
      \bot, & \text{ако } a = \bot.
  \end{cases}
\end{align*}

\marginpar{Спестяваме си труда от въвеждането на булевия тип променливи}
\marginpar{Озн. $\Nat^+ \dff \Nat \setminus \{0\}$}

\begin{problem}
  \label{prob:rec:if:continuous}
  Докажете, че изображенията $\texttt{plus}$, $\texttt{eq}$, $\texttt{if}$ са непрекъснати.
\end{problem}

\index{денотационна семантика!по име}
\marginpar{\cite[стр. 155]{winskel}}
\index{терм!стойност}
За всеки терм $\tau[\vv{x}_1,\dots,\vv{x}_{n},\vv{f}_1,\dots,\vv{f}_k]$,
ще разгледаме изображението със сигнатура
\[\val{\tau}:\Mapping{\Nat^{m_1}_\bot}{\Nat_\bot}\times\cdots\times\Mapping{\Nat^{m_k}_\bot}{\Nat_\bot} \to \Mapping{\Nat^{n}_\bot}{\Nat_\bot},\]
което ще дефинираме с индукция по построението на термовете.

\begin{itemize}
\item
  ако $\tau \equiv \vv{c}$, за някоя константа, то 
  \[\val{\vv{c}}(\ov{\varphi})(\ov{a}) \dff c.\]
\item
  ако $\tau \equiv \vv{x}_i$, за някоя обектова променлива, то 
  \[\val{\vv{x}_i}(\ov{\varphi})(\ov{a}) \dff a_i.\]
\item
  ако $\tau \equiv \tau_1 + \tau_2$, то
  \[\val{\tau_1 + \tau_2}(\ov{\varphi})(\ov{a}) \dff \texttt{plus}(\val{\tau_1}(\ov{\varphi})(\ov{a}), \val{\tau_2}(\ov{\varphi})(\ov{a})).\]
\item
  \marginpar{Знаем, че изображенията $\texttt{plus}$ и $\texttt{eq}$ са непрекъснати}
  ако $\tau \equiv \tau_1\ \vv{==}\ \tau_2$, то
  \[\val{\tau_1\ \vv{==}\ \tau_2}(\ov{\varphi})(\ov{a}) \dff \texttt{eq}(\val{\tau_1}(\ov{\varphi})(\ov{a}), \val{\tau_2}(\ov{\varphi})(\ov{a})).\]
\item
  ако $\tau \equiv\ \ifelse{\tau_1}{\tau_2}{\tau_3}$, то
  \[\val{\ifelse{\tau_1}{\tau_2}{\tau_3}}(\ov{\varphi})(\ov{a}) \dff \texttt{if}(\val{\tau_1}(\ov{\varphi})(\ov{a}), \val{\tau_2}(\ov{\varphi})(\ov{a}), \val{\tau_3}(\ov{\varphi})(\ov{a})).\]
\item
  ако $\tau \equiv \vv{f}_i(\tau_1,\dots,\tau_{m_i})$, то
  \marginpar{Термовете $\tau_j$ може да имат различен брой променливи. Ако се наложи, разширяваме ги с фиктивни променливи}
  \[\val{\vv{f}_i(\tau_1,\dots,\tau_{m_i})}(\ov{\varphi})(\ov{a}) \dff \varphi_i(\val{\tau_1}(\ov{\varphi})(\ov{a}),\dots,\val{\tau_{m_i}}(\ov{\varphi})(\ov{a})).\]
\end{itemize}

\begin{example}
  Да разгледаме следния терм:
  \[\tau[\vv{x},\vv{y},\vv{z}] \dfff \ifelse{\vv{x}\ \vv{==}\ 5}{\vv{y}}{\vv{z}}.\]
  Да видим каква е негова стойност при произволни стойности $a,b,c \in \Nat_\bot$:
  \begin{align*}
    \val{\tau}(a,b,c) & =
                        \begin{cases}
                          \val{\vv{y}}(a,b,c), & \text{ако }\val{x \ \vv{==}\ 5}(a,b,c) \in \Nat^+\\
                          \val{\vv{z}}(a,b,c), & \text{ако }\val{x \ \vv{==}\  5}(a,b,c) = 0\\
                          \bot,                & \text{ако }\val{x\ \vv{==}\ 5}(a,b,c) = \bot
                        \end{cases}\\
                      & =
                        \begin{cases}
                          b,    & \text{ако }a = 5\\
                          c,    & \text{ако }a \in \Nat\ \&\ a \neq 5\\
                          \bot, & \text{ако }a = \bot.
                        \end{cases}
  \end{align*}
\end{example}
Нека сега да видим следния пример на хаскел.
\begin{haskellcode}
ghci> let f(x) = if x == undefined then 0 else 1
ghci> f(2)
*** Exception: Prelude.undefined
\end{haskellcode}
Това означава, че функцията \texttt{==} е точна, т.е. не можем да сравняваме с $\bot$, което съответства на нашата денотационна семантика.

\begin{framed}
  \begin{lemma}[Лема за замяната]
    \label{lem:rec:substitution}
    Да разгледаме терма $\tau[\vv{x}_1,\dots,\vv{x}_n,\vv{f}_1,\dots,\vv{f}_{k}]$ и {\em функционалните} термове 
    $\mu_1[\vv{f}_1,\dots,\vv{f}_k],\dots, \mu_n[\vv{f}_1,\dots,\vv{f}_k]$.
    Тогава % за произволни $\varphi_1,\dots,\varphi_k$,
    \[\val{\tau[\bar{\vv{x}}/\bar{\mu}]}(\bar{\varphi}) = \val{\tau}(\ov{\varphi})(\val{\mu_1}(\bar{\varphi}),\dots,\val{\mu_n}(\bar{\varphi}))\]
    за произволни $\varphi_i \in \Mapping{\Nat^{m_i}_\bot}{\Nat_\bot}$, за $i = 1, \dots, k$.
  \end{lemma}
\end{framed}
\begin{proof}
  Доказателството се провежда с {\em индукция по построението на терма $\tau$.}
  \marginpar{Важно е, че нямаме обектови променливи в $\mu_1,\dots,\mu_n$}
  \marginpar{Substitution Lemma, \cite[стр. 149]{winskel}}
  \marginpar{Това твърдение е без д-во в \cite[стр. 188]{ditchev-soskov}}
  \begin{itemize}
  \item
    \marginpar{$\vv{c}$ е константа, докато $c \in \Nat_\bot$}
    Нека да започнем с най-лесния случай.
    Нека $\tau \equiv \vv{c}$, за някоя константа.
    Тогава $\tau[\ov{\vv{x}}/\ov{\mu}] \equiv \vv{c}$.
    Това означава, че 
    \[\val{\tau[\ov{\vv{x}}/\ov{\mu}]}(\ov{\varphi}) = \val{\vv{c}}(\ov{\varphi}) \dff c.\]
    От друга страна, имаме също, че 
    \[\val{\vv{c}}(\ov{\varphi})(\val{\mu_1}(\ov{\varphi}),\dots,\val{\mu_n}(\ov{\varphi})) \dff c.\]
  \item
    Нека $\tau \equiv \vv{x}_i$. Тогава $\vv{x}_i[\varsx/\bar{\mu}] \equiv \mu_i$
    и следователно 
    \[\val{\vv{x}_i[\varsx/\bar{\mu}]}(\bar{\varphi}) = \val{\mu_i}(\bar{\varphi}).\]
    От друга страна, по дефиниция на стойност на терм, 
    \[\val{\vv{x}_i}(\ov{\varphi})(\val{\mu_1}(\bar{\varphi}),\dots,\val{\mu_n}(\bar{\varphi})) = \val{\mu_i}(\bar{\varphi}).\]
  \item
    Нека $\tau \equiv \tau_1 + \tau_2$.
    От {\bf И.П.} имаме, че за $j = 1,2$ е изпъленено следното:
    \[\val{\tau_j[\varsx/\ov{\mu}]}(\ov{\varphi}) = \val{\tau_j}(\ov{\varphi})(\underbrace{\val{\mu_1}(\ov{\varphi})}_{b_1},\dots,\underbrace{\val{\mu_n}(\ov{\varphi})}_{b_n}).\]
    Тогава
    \begin{align*}
      \val{\tau[\varsx/\ov{\mu}]}(\ov{\varphi}) & = \val{\tau_1[\varsx/\ov{\mu}] + \tau_2[\varsx/\ov{\mu}]}(\ov{\varphi})\\
                                                & \dff \texttt{plus}(\val{\tau_1[\varsx/\ov{\mu}]}(\ov{\varphi}), \val{\tau_2[\varsx/\ov{\mu}]}(\ov{\varphi}))\\
                                                % & = \texttt{plus}(\val{\tau_1}(\ov{\varphi})(\val{\mu_1}(\ov{\varphi}),\dots,\val{\mu_n}(\ov{\varphi})),\val{\tau_2}(\ov{\varphi})(\val{\mu_1}(\ov{\varphi}),\dots,\val{\mu_n}(\ov{\varphi}))) & \comment{\text{от {\bf И.П.}}}\\
                                                & = \texttt{plus}(\val{\tau_1}(\ov{\varphi})(b_1,\dots,b_n),\val{\tau_2}(\ov{\varphi})(b_1,\dots,b_n)) & \comment{\text{от {\bf И.П.}}}\\
                                                & \dff \val{\tau}(\ov{\varphi})(b_1,\dots,b_n)\\
                                                & = \val{\tau}(\ov{\varphi})(\val{\mu_1}(\bar{\varphi}),\dots,\val{\mu_n}(\ov{\varphi})). & \comment{b_j \dff \val{\mu_j}(\ov{\varphi})}
    \end{align*}
  \item
    \marginpar{\writedown Докажете сами останалите два случая. Те не крият изненади.}
    Нека $\tau \equiv \tau_1\ \vv{==}\  \tau_2$.
  \item
    Нека $\tau \equiv \ifelse{\tau_0}{\tau_1}{\tau_2}$.
  \item 
    Нека $\tau \equiv \vv{f}_i(\tau_1,\dots,\tau_{m_i})$.
    Имаме, че 
    \begin{equation}
      \label{eq:1}
      \tau[\varsx/\bar{\mu}] \equiv \vv{f}_i(\tau_1[\varsx/\bar{\mu}],\dots,\tau_{m_i}[\varsx/\bar{\mu}]).
    \end{equation}
    Нека за улеснение да означим $b_i \dff \val{\mu_i}(\bar{\varphi})$, за $i = 1,\dots,n$.
    Прилагаме {\bf И.П.} за термовете $\tau_1,\dots,\tau_{m_i}$ и получаваме за $j = 1, \dots, m_i$,
    \begin{align*}
      \val{\tau_j[\varsx /\bar{\mu}]}(\bar{\varphi}) & = \val{\tau_j}(\ov{\varphi})(\underbrace{\val{\mu_1}(\bar{\varphi})}_{b_1},\dots,\underbrace{\val{\mu_{n}}(\bar{\varphi})}_{b_{n}}) & \comment{\text{от {\bf И.П.}}}\\
      & = \val{\tau_j}(\ov{\varphi})(b_1,\dots,b_{n}) & \comment{b_i \dff \val{\mu_i}(\bar{\varphi})}.
    \end{align*}
    Следователно,
    \begin{align*}
      \val{\tau[\varsx/\ov{\mu}]}(\ov{\varphi}) & = \val{\vv{f}_i(\tau_1[\varsx/\ov{\mu}],\dots,\tau_{m_i}[\varsx/\ov{\mu}])}(\ov{\varphi}) & \comment{\text{от (\ref{eq:1})}}\\
                                                & \dff \varphi_i(\val{\tau_1[\varsx/\ov{\mu}]}(\ov{\varphi}),\dots,\val{\tau_{m_i}[\varsx/\ov{\mu}]}(\ov{\varphi}))\\
                                                & = \varphi_i(\val{\tau_1}(\ov{\varphi})(\ov{b}),\dots,\val{\tau_{m_i}}(\ov{\varphi})(\ov{b})) & \comment{\text{от {\bf И.П.}}} \\
                                                & \dff \val{\vv{f}_i(\tau_1,\dots,\tau_{m_i})}(\ov{\varphi})(\ov{b})\\
                                                & = \val{\tau}(\ov{\varphi})(\ov{b}) & \comment{\tau \equiv \vv{f}_i(\tau_1,\dots,\tau_{m_i})}\\
      & = \val{\tau}(\ov{\varphi})(\val{\mu_1}(\ov{\varphi}),\dots,\val{\mu_n}(\ov{\varphi})) & \comment{b_i \dff \val{\mu_i}(\ov{\varphi})}.
    \end{align*}
  \end{itemize}
\end{proof}

\begin{remark}
  В частния случай, когато функционалните термове $\mu_1,\dots, \mu_n$ са константите $\vv{c}_1, \dots, \vv{c}_n$, получаваме, че 
  \[\val{\tau[\ov{\vv{x}}/\ov{\vv{c}}]}(\ov{\varphi}) = \val{\tau}(\ov{\varphi})(\ov{c}).\]  
\end{remark}

%%% Local Variables:
%%% mode: latex
%%% TeX-master: "../sep"
%%% End:


\subsection{Термални оператори}

Вече дефинирахме как на всеки терм $\tau[\vv{x}_1,\dots,\vv{x}_n,\vv{f}_1,\dots,\vv{f}_k]$
съпоставяме изображението $\val{\tau}$ със сигнатура
\[\val{\tau}:\Mapping{\Nat^{m_1}_\bot}{\Nat_\bot}\times\cdots\times\Mapping{\Nat^{m_k}_\bot}{\Nat_\bot} \to \Mapping{\Nat^n_\bot}{\Nat_\bot}\]
%  по следния начин:
% \[\Xi_\tau(\bar{\varphi})(\bar{a}) \dff \val{\tau}(\bar{a},\bar{\varphi}).\]
Едно от основните свойства, които трябва да притежават тези изображения е да бъдат {\em непрекъснати} относно областите на Скот,
за които са дефинирани.
Причината за това е, че искаме да дефинираме семантиката на една програма като използваме най-малкото решение 
на система от такива оператори, а според \hyperref[th:knaster-tarski]{Теоремата на Клини}, за да можем да направим това, трябва да работим
с непрекъснати оператори. Следващият пример ни показва, че трябва да сме по-внимателни върху какви елементи разглеждаждаме тези изображения.

\begin{example}
  \label{ex:non-continuous}
  Да разгледаме терма 
  \[\tau[\vv{x},\vv{f},\vv{g}] \dfff \vv{f}(\vv{g}(\vv{x})),\]
  и следните две изображения от тип $\Mapping{\Nat_\bot}{\Nat_\bot}$:
  \marginpar{Обърнете внимание, че $\varphi$ не е монотонна функция. Тя дори не е точна. 
    Функцията $\psi$ не е точна, но е монотонна. Знаем, че $\varphi$ не е непрекъсната}
  \begin{align*}
    & \varphi(x) \dff
    \begin{cases}
      42,   & \text{ако }x = \bot\\
      \bot, & \text{ако }x \in \Nat
    \end{cases}\\
    & \psi(x) \dff 42 \text{, за всяко }x \in \Nat_\bot.
  \end{align*}
  Лесно се вижда, че $\pair{\varphi,\varphi} \sqsubseteq \pair{\varphi,\psi}$, но
  \[\val{\tau}(\varphi,\varphi) \not\sqsubseteq \val{\tau}(\varphi,\psi),\]
  защото за произволно $a \in \Nat$,
  \begin{align*}
    \val{\tau}(\varphi,\varphi)(a) & = \varphi(\varphi(a)) & \comment{\text{стойност на терм}}\\
                                 & = 42  \\
                                 & \not\sqsubseteq \bot & \comment{\text{плоска наредба}}\\
                                 & = \varphi(\psi(a)) & \comment{\text{защото }\psi(a) \neq \bot}\\
                                 & = \val{\tau}(\varphi,\psi)(a) & \comment{\text{от деф. на }\tau}\\
  \end{align*}

  \marginpar{От \Prop{continuous-is-monotone} знаем, че всеки непрекъснат оператор е монотонен. 
    Щом $\val{\tau}$ не е монотоненно, то той със сигурност не е непрекъснато.}
  Това означава, че за конкретния терм $\tau$, изображението $\val{\tau}$ не е монотонно, откъдето следва, че то не е непрекъснато.
\end{example}

% Горният пример ни казва, че за $\val{\tau}$, дефинирани върху произволни елементи от $\Mapping{\Nat^{m_i}_\bot}{\Nat_\bot}$
% {\em не можем да гарантираме свойството непрекъснатост}. Ще видим, че ако се ограничим само до непрекъснатите изображения,
% то тогава операторите ще бъдат непрекъснати.
Всъщност на нас е необходимо да можем да подаваме като аргументи на $\val{\tau}$ само такива $\varphi$,
които можем да дефинираме на езика \REC.
Според семантиката на термовете, която дефинирахме по-горе, не е ясно дали можем да дефинираме терм на езика ${\bf REC}$, чиято семантика да бъде изображението $\varphi$. Естествено е да разгледаме терма 
\[\tau[\vv{x}] \dfff \ifelse{\vv{x}\ \vv{==}\ \bot}{42}{\bot}.\]
\marginpar{Това {\em не} е доказателство, че не можем да дефинираме функцията $\varphi$ на езика \REC. Тук ние твърдим само, че очевидният опит за дефиниция на $\varphi$ се проваля. По-нататък ще видим, че наистина не може да дефинираме $\varphi$ на нашия език, защото тя не е непрекъсната функция.}
 Каква е семантиката на този терм? Следваме дефиницията на стойност на терм, за произволно $a \in \Nat_\bot$, и получаваме:
 \begin{align*}
   \val{\tau}(a) & =
                   \begin{cases}
                     42,   & \text{ако }\val{\vv{x}\ \vv{==}\ \bot}(a) \in \Nat^+\\
                     \bot, & \text{ако }\val{\vv{x}\ \vv{==}\ \bot}(a) = 0\\
                     \bot, & \text{ако }\val{\vv{x}\ \vv{==}\ \bot}(a) = \bot\\
                   \end{cases}\\
                 & = 
                   \begin{cases}
                     42,   & \text{ако }\underbrace{\val{\vv{x}}(a)}_{\in\Nat} = \underbrace{\val{\bot}(a)}_{\in \Nat}\\
                     \bot, & \text{ако }\underbrace{\val{\vv{x}}(a)}_{\in\Nat} \neq \underbrace{\val{\bot}(a)}_{\in \Nat}\\
                     \bot, & \text{ако }\val{\vv{x}}(a) = \bot\text{ или }\val{\bot}(a) = \bot\\
                   \end{cases}\\
                 & = \bot.
 \end{align*}
 
Да видим, какво ще стане ако преведем директно горния пример на \vv{хаскел}.
\begin{haskellcode}
ghci> let phi(x) = if x == undefined then 42 else undefined
ghci> let psi(x) = 42
ghci> phi(0)
*** Exception: Prelude.undefined
ghci> phi(undefined)
*** Exception: Prelude.undefined
ghci> psi(0)
42
\end{haskellcode}

Виждаме, че тук \texttt{хаскел} следва нашата семантика за езика {\bf REC}.
Направените по-горе бележки ни насочват към следната дефиниция на {\bf термален оператор}.

\begin{framed}
  \begin{dfn}
    \index{термален оператор}
    За всеки терм от вида $\tau[\vv{x}_1,\dots,\vv{x}_n, \vv{f}_1,\dots,\vv{f}_k]$
    дефинираме оператора 
    \[\Gamma_\tau: \DomOpCBN \to \RanOpCBN,\] % като
    където
    \[\Gamma_\tau(\bar{\varphi})(\bar{a}) \dff \val{\tau}(\ov{\varphi})(\ov{a}).\]
  \end{dfn}  
\end{framed}
\marginpar{Обърнете внимание, че все още не е ясно защо $\Gamma_\tau(\ov{\varphi}) \in \RanOpCBN$}
В следващия раздел ще се концентрираме върху доказателството на следния основен резултат:

\begin{framed}
  За всеки терм $\tau[\varsx, \varsf]$, операторът $\Gamma_\tau$ е непрекъснат.
\end{framed}

\subsection{Непрекъснатост на термалните оператори}

Първата ни задача ще бъде да проверим, че операторът $\Gamma_\tau$ е коректно дефиниран.
Това означава, че трябва да се уверим, че винаги когато подадем като аргументи на $\Gamma_\tau$ непрекъснати изображения,
то $\Gamma_\tau$ връща непрекъснато изображение.

\begin{remark}
  В този раздел всички доказателства се провеждат с индукция по построението на термовете.
\end{remark}

\begin{prop}
  \label{pr:term-monotone}
  Да разгледаме един терм $\tau[\vv{x}_1,\dots,\vv{x}_n, \vv{f}_1,\dots,\vv{f}_k]$.
  \marginpar{Озн. $\ov{a}_i = (a^1_i,\dots,a^n_i)$}
  Нека $\ov{a} \sqsubseteq \ov{b}$.
  Тогава 
  \[\val{\tau}(\ov{\varphi})(\ov{a}) \sqsubseteq \val{\tau}(\ov{\varphi})(\ov{b}),\]
  където
  $\varphi_i \in \Cont{\Nat^{m_i}_\bot}{\Nat_\bot}$, за $i = 1,\dots,k$.
\end{prop}
\begin{hint}
  Индукция по построението на терма $\tau$.
  \begin{itemize}
  \item
    Нека $\tau \equiv \vv{c}$.
  \item
    Нека $\tau \equiv \vv{x}_i$. Тогава
    \begin{align*}
      \val{\tau}(\ov{\varphi})(\ov{a}) & = \val{\vv{x}_i}(\ov{\varphi})(\ov{a})\\
                                       & = a_i & \comment{\text{стойност на терм}}\\
                                       & \sqsubseteq b_i & \comment{\ov{a} \sqsubseteq \ov{b}}\\
                                       &= \val{\vv{x}_i}(\ov{\varphi})(\ov{b}) & \comment{\text{стойност на терм}}\\
                                       & = \val{\tau}(\ov{\varphi})(\ov{b}).
    \end{align*}
  \item
    Нека $\tau \equiv \tau_1 + \tau_2$. 
    Тук ще използваме, че от {\bf И.П.}, за $j = 1,2$, имаме
    \begin{equation}
      \label{eq:10}
      \val{\tau_j}(\ov{\varphi})(\ov{a}) \sqsubseteq \val{\tau_j}(\ov{\varphi})(\ov{b}).
    \end{equation}
    Освен това, понеже изображението $\texttt{plus}$ е непрекъснато, то то е и монотонно.
    \begin{align*}
      \val{\tau}(\ov{\varphi})(\ov{a}) & = \val{\tau_1 + \tau_2}(\ov{\varphi})(\ov{a})\\
                                       & \dff \texttt{plus}(\val{\tau_1}(\ov{\varphi})(\ov{a}), \val{\tau_2}(\ov{\varphi})(\ov{a}))\\
                                       & \sqsubseteq \texttt{plus}(\val{\tau_1}(\ov{\varphi})(\ov{b}), \val{\tau_2}(\ov{\varphi})(\ov{b})) & \comment{\text{от (\ref{eq:10}) и мон.}}\\
      & \dff \val{\tau_1 + \tau_2}(\ov{\varphi})(\ov{b}).
    \end{align*}
  \item
    \marginpar{\writedown За домашно!}
    Нека $\tau \equiv \tau_1\ \vv{==}\ \tau_2$.
  \item
    Нека $\tau \equiv \ifelse{\tau_0}{\tau_1}{\tau_2}$.  
  \item
    Нека $\tau \equiv \vv{f}_i(\tau_1,\dots,\tau_{m_i})$. 
    Тук ще използваме, че от {\bf И.П.}, за $j = 1,\dots,m_i$, имаме
    \begin{equation}
      \label{eq:6}
      \val{\tau_j}(\ov{\varphi})(\ov{a}) \sqsubseteq \val{\tau_j}(\ov{\varphi})(\ov{b}).
    \end{equation}
    Тогава
    \begin{align*}
      \val{\tau}(\ov{\varphi})(\ov{a}) & = \val{\vv{f}_i(\tau_1,\dots,\tau_{m_i})}(\ov{\varphi})(\ov{a})\\
                                      & \dff \varphi_i(\val{\tau_1}(\ov{\varphi})(\ov{a}), \dots, \val{\tau_{m_i}}(\ov{\varphi})(\ov{a}))\\
                                      & \sqsubseteq \varphi_i(\val{\tau_1}(\ov{\varphi})(\ov{b}), \dots, \val{\tau_{m_i}}(\ov{\varphi})(\ov{b})) & \comment{\text{от (\ref{eq:6}) и $\varphi_i$ е мон.}}\\
                                      & \dff \val{\tau}(\ov{\varphi})(\ov{b}).
    \end{align*}
  \end{itemize}
\end{hint}

\begin{cor}
  \label{cr:tau-preserves-continuous}
  Да разгледаме един терм $\tau[\vv{x}_1,\dots,\vv{x}_n, \vv{f}_1,\dots,\vv{f}_k]$.
  Тогава за произволна верига $\chain{\bar{a}}{i}$ в $\Nat^{n}_\bot$,
  \[\val{\tau}(\ov{\varphi})(\bigsqcup_i\ov{a}_i) = \bigsqcup_i\val{\tau}(\ov{\varphi})(\ov{a}_i),\]
  където
  $\varphi_i \in \Cont{\Nat^{n_i}_\bot}{\Nat_\bot}$, за $i = 1,\dots,k$.
\end{cor}
\begin{proof}
  За терма $\tau$ и $\ov{\varphi}$, да разгледаме изображението
  \[\psi(\ov{a}) \dff \val{\tau}(\ov{\varphi})(\ov{a}).\]
  От \Prop{term-monotone} следва, че $\psi \in \Mon{\Nat^n_\bot}{\Nat_\bot}$.
  Понеже всяка верига в $\Nat^n_\bot$ се стабилизира, то директно от \Prop{stab-continuous}
  следва, че $\psi \in \Cont{\Nat^n_\bot}{\Nat_\bot}$,
  което означава, че за произволна верига $\chain{\ov{a}}{i}$,
  \[\psi(\bigsqcup_i \ov{a}_i) = \bigsqcup_i \psi(\ov{a}_i),\]
  което означава, че 
  \[\val{\tau}(\ov{\varphi})(\bigsqcup_i\ov{a}_i) = \bigsqcup_i\val{\tau}(\ov{\varphi})(\ov{a}_i).\]
\end{proof}


\begin{cor}
  \label{cr:gamma-preserves-continuous}
  Да разгледаме произволен терм $\tau[\vv{x}_1,\dots,\vv{x}_n, \vv{f}_1,\dots,\vv{f}_k]$.
  Тогава за произволни $\varphi_i \in \Cont{\Nat^{m_i}}{\Nat_\bot}$, за $i = 1,\dots k$,
  имаме, че
  \[\Gamma_\tau(\bar\varphi) \in \RanOpCBN.\]
\end{cor}

Щом термалните оператори $\Gamma_\tau$ са добре дефинирани, ще 
проверим, че са непрекъснати. Първата стъпка ще бъде проверката, че те са монотонни.

\begin{lemma}
  \label{lem:rec:functional:term:continuous}
  Нека $\mu[\vv{f}_1,\dots,\vv{f}_k]$ е {\em функционален} терм.
  \marginpar{Да напомним, че $m_i \dff \#\vv{f}_i$}
  Да разгледаме произволна верига $\chain{\ov{\varphi}}{r}$
  от елементи на
  \[\DomOpCBN.\]
  Тогава $\val{\mu}$ е непрекъснато изображение, т.е.
  \[\val{\mu}(\bigsqcup_r\ov{\varphi}_r) = \bigsqcup_r \val{\mu}(\ov{\varphi}_r).\]
\end{lemma}
\begin{proof}
  Индукция по построението на терма $\mu$.
  \begin{itemize}
  \item
    Нека $\mu \equiv \vv{c}$. Този случай е очевиден.
  \item
    Нека $\mu \equiv \mu_1 + \mu_2$. Ще използваме, че $\texttt{plus}$ е непрекъснато изображение.
    \begin{align*}
      \val{\mu}(\bigsqcup_r \ov{\varphi}_r) & \dff \texttt{plus}(\val{\mu_1}(\bigsqcup_r\ov{\varphi}_r), \val{\mu_2}(\bigsqcup_r\ov{\varphi}_r))\\
      & = \texttt{plus}(\bigsqcup_r\val{\mu_1}(\ov{\varphi}_r), \bigsqcup_r \val{\mu_2}(\ov{\varphi}_r)) & \comment{\text{от И.П.}}\\
      & = \bigsqcup_r \texttt{plus}(\val{\mu_1}(\ov{\varphi}_r), \val{\mu_2}(\ov{\varphi}_r)) & \comment{\texttt{plus}\text{ е непр.}}\\
      & \dff \bigsqcup_r \val{\mu}(\ov{\varphi}_r).
    \end{align*}
  \item
    Нека $\mu \equiv \mu_1\ \vv{==}\ \mu_2$.
    Използвайте, че $\texttt{eq}$ е непрекъснато изображение.
  \item
    Нека $\mu \equiv \ifelse{\mu_0}{\mu_1}{\mu_2}$.
    Използвайте \Problem{rec:if:continuous}.
  \item
    Нека $\mu \equiv \vv{f}_i(\mu_1,\dots,\mu_{m_i})$. 
    От И.П. знаем, че $\val{\mu_j}$ са непрекъснати изображения за $j = 1,\dots,m_i$, а от \Prop{continuous-is-monotone}
    следва, че $\val{\mu_j}$ са монотонни. Тогава
    за произволни индекси $n \leq n'$ и $r \leq r'$ имаме, че
    \marginpar{$\ov{\varphi}_n \dff (\varphi^1_n,\dots,\varphi^{k}_n)$}
    \begin{align*}
      \varphi^i_{n}(\val{\mu_1}(\ov{\varphi}_r),\dots,\val{\mu_{m_i}}(\ov{\varphi}_r)) & \sqsubseteq \varphi^i_{n}(\val{\mu_1}(\ov{\varphi}_{r'}),\dots,\val{\mu_{m_i}}(\ov{\varphi}_{r'}))\\
                                                                                       & \sqsubseteq \varphi^i_{n'}(\val{\mu_1}(\ov{\varphi}_{r'}),\dots,\val{\mu_{m_i}}(\ov{\varphi}_{r'})).
    \end{align*}
    Това означава, че ако положим
    \[e_{n,r} \dff \varphi^i_{n}(\val{\mu_1}(\ov{\varphi}_r),\dots,\val{\mu_{m_i}}(\ov{\varphi}_r)),\]
    то
    \[n \leq n'\ \&\ r \leq r' \implies e_{n,r} \sqsubseteq e_{n',r'}.\]
    Сега можем да приложим \Th{double-chain}, според която
    \begin{equation}
      \label{eq:8}
      \bigsqcup_n(\bigsqcup_r e_{n,r}) = \bigsqcup_n e_{n,n}.
    \end{equation}

    Тогава 
    \begin{align*}
      \val{\mu}(\bigsqcup_n\ov{\varphi}_n) & = \val{\vv{f}_i(\mu_1,\dots,\mu_{m_i})}(\bigsqcup_n\ov{\varphi}_n)\\
                                           & \dff (\bigsqcup_n\varphi^i_n)(\val{\mu_1}(\bigsqcup_r\ov{\varphi}_r),\dots,\val{\mu_{m_i}}(\bigsqcup_r\ov{\varphi}_r))\\
                                           & = \bigsqcup_n \{\varphi^i_n(\val{\mu_1}(\bigsqcup_r\ov{\varphi}_r),\dots,\val{\mu_{m_i}}(\bigsqcup_r\ov{\varphi}_r))\} & \comment{\text{от \Th{monotone-is-domain}}}\\
                                           & =  \bigsqcup_n \{\varphi^i_n(\bigsqcup_r \val{\mu_1}(\ov{\varphi}_r),\dots,\bigsqcup_r \val{\mu_{m_i}}(\ov{\varphi}_r))\} & \comment{\text{от И.П. за }\mu_j}\\
                                           & =  \bigsqcup_n \{\bigsqcup_r \underbrace{\varphi^i_n(\val{\mu_1}(\ov{\varphi}_r),\dots,\val{\mu_{m_i}}(\ov{\varphi}_r))}_{e_{n,r}}\} & \comment{\varphi^i_n \text{ е непр.}}\\
                                           & =  \bigsqcup_n \underbrace{\varphi^i_n(\val{\mu_1}(\ov{\varphi}_n),\dots,\val{\mu_{m_i}}(\ov{\varphi}_n))}_{e_{n,n}} & \comment{\text{от (\ref{eq:8})}}\\
                                           & \dff \bigsqcup_n \val{\vv{f}_i(\mu_1,\dots,\mu_{m_i})}(\ov{\varphi}_n)\\
                                           & = \bigsqcup_n \val{\mu}(\ov{\varphi}_n).
    \end{align*}
  \end{itemize}
\end{proof}

\begin{cor}
  \label{cr:rec:term:continuous}
  Нека $\tau[\varsx,\varsf]$ е терм.
  Да разгледаме произволни елементи $\ov{a}$ на $\Nat^n_\bot$ и 
  произволна верига $\chain{\ov{\varphi}}{r}$
  от елементи на
  \[\DomOpCBN.\]
  Тогава 
  \[\val{\tau}(\bigsqcup_r\ov{\varphi}_r)(\ov{a}) = \bigsqcup_r \val{\tau}(\ov{\varphi}_r)(\ov{a}).\]  
\end{cor}
\begin{proof}
  \begin{align*}
    \val{\tau}(\bigsqcup_r\ov{\varphi}_r)(\ov{a}) & = \val{\tau[\varsx/\ov{\vv{a}}]}(\bigsqcup_r \ov{\varphi}_r) & \comment{\text{от \hyperref[lem:rec:substitution]{Лема за замяната}}}\\
                                                  & = \bigsqcup_r \val{\tau[\varsx/\ov{\vv{a}}]}(\ov{\varphi}_r) & \comment{\text{от \Lem{rec:functional:term:continuous}}}\\
                                                  & = \bigsqcup_r \val{\tau}(\ov{\varphi}_r)(\ov{a}) &  \comment{\text{от \hyperref[lem:rec:substitution]{Лема за замяната}}}\\
  \end{align*}
\end{proof}

Вече имаме всичко необходимо за да се убедим, че термалните оператори са непрекъснати.

\begin{framed}
\begin{thm}
  \label{th:gamma-is-continuous}
  За всеки терм $\tau[\vv{x}_1,\dots,\vv{x}_n, \vv{f}_1,\dots,\vv{f}_k]$, операторът 
  \[\Gamma_\tau: \DomOpCBN \to \RanOpCBN,\]
  дефиниран като
  \[\Gamma_\tau(\bar{\varphi})(\bar{a}) = \val{\tau}(\ov{\varphi})(\bar{a}),\]
  е непрекъснат.
\end{thm}
\end{framed}
\begin{proof}
  От \Cor{gamma-preserves-continuous} имаме, че $\Gamma_\tau$ е коректно дефиниран.
  Сега директно се позоваваме на \Cor{rec:term:continuous}.
\end{proof}

%%% Local Variables:
%%% mode: latex
%%% TeX-master: "../sep"
%%% End:


\subsection{Предаване на параметрите по име}

Нека е дадена една рекурсивна програма $\vv{P}[\varsx,\varsf]$, където:
\marginpar{Можем да си мислим, че $\vv{f}_1$ е нещо като \texttt{main} функция за програмата $\vv{P}$.}
\begin{align*}
  \vv{P} = & 
             \begin{cases}
               & \vv{f}_1(\vv{x}_1,\dots,\vv{x}_{m_1}) = \tau_1[\vv{x}_1,\dots,\vv{x}_{m_1},\vv{f}_1,\dots,\vv{f}_k]\\
               & \vv{f}_2(\vv{x}_1,\dots,\vv{x}_{m_2}) = \tau_2[\vv{x}_1,\dots,\vv{x}_{m_2},\vv{f}_1,\dots,\vv{f}_k]\\
               & \vdots\\
               & \vv{f}_k(\vv{x}_1,\dots,\vv{x}_{m_k}) = \tau_k[\vv{x}_1,\dots,\vv{x}_{m_k},\vv{f}_1,\dots,\vv{f}_k]
             \end{cases}
\end{align*}
Нека $\ov{\gamma} \in \DomOpCBN$
е {\em най-малкото решение} на системата
\begin{align*}
  & \Gamma_{\tau_1}(\varphi_1,\dots,\varphi_k) = \varphi_1\\
  & \ \vdots \\
  & \Gamma_{\tau_k}(\varphi_1,\dots,\varphi_k) = \varphi_k.
\end{align*}
От \hyperref[th:knaster-tarski]{Теоремата на Клини} знаем, че такова съществува.

\index{денотационна семантика!по име}
\begin{framed}
  За дадената рекурсивна програма $\vv{P}[\varsx,\varsf]$, 
  определяме {\bf денотационната семантика с предаване на параметрите по име} 
  като изображението $\D_N\val{\vv{P}} \in \Cont{\Nat^{m_1}_\bot}{\Nat_\bot}$, където:
  \[\D_N\val{\vv{P}}(a_1,\dots,a_{m_1}) \dff \gamma_1(a_1,\dots,a_{m_1}).\]
\end{framed}


%%% Local Variables:
%%% mode: latex
%%% TeX-master: "../sep"
%%% End:


\subsection{Предаване на параметрите по стойност}
\index{денотационна семантика!по стойност}

Основната разлика между денотационната семантика по име и по стойност е, че при семантиката по стойност
изискваме да работим {\em само с точни функции}.

\begin{remark}
  \marginpar{Сигурни сме единствено, че термалните оператори запазват непрекъснатостта}
  Да обърнем внимание, че термалните оператори $\Gamma_\tau$ не запазват точните функции, т.е.
  възможно е $\varphi \in \Strict{\Nat^n_\bot}{\Nat_\bot}$, но $\Gamma_\tau(\varphi) \not \in \Strict{\Nat^m_\bot}{\Nat_\bot}$.
  Например, да разгледаме терма
  \[\tau[\vv{x},\vv{y},\vv{f}] \equiv \vv{x},\]
  т.е. изобщо не се обръщаме към функционалния аргумент.
  Тогава ако $a \neq \bot$, но $b = \bot$, то $\val{\tau}(\varphi)(a,b) \neq \bot$.

  Да се уверим, че хаскел ,,смята'' по подобен начин.
  \begin{haskellcode}
ghci> let f(x,y,g) = x
ghci> let h(x,y) = f(x,y,(\z -> z))
ghci> h(1,undefined)
1
  \end{haskellcode}
  Вижда се, че макар и да подаваме точна функция като аргумент на $\vv{f}$, то новополучената функция $\vv{h}$ не е точна.
\end{remark}

Поради тази причина, за да дефинираме семантика с предаване на параметрите по стойност, ще използваме най-малки неподвижни точки
на други оператори, {\em които винаги връщат точни функции}.

\begin{framed}
  \begin{theorem}
    За всеки терм $\tau[\varsx,\varsf]$, операторът 
    \[\Delta_\tau:\DomOpCBV\to \RanOpCBV,\]
    дефиниран като
    \[\Delta_\tau \dff \Sigma_\star \circ \Gamma_\tau,\]
    е непрекъснат.
  \end{theorem}
\end{framed}
\begin{proof}
  Първо да съобразим, че $\Delta_\tau$ е коректно дефиниран оператор, т.е. 
  за $\bar{\varphi} \in \DomOpCBV$ имаме, че $\Delta_\tau(\bar{\varphi}) \in \RanOpCBV$.
  Това е ясно, защото $\Sigma_\star$ превръща всяко изображение в точно.
  
  От \Prop{strict-operator} знаем, че $\Sigma_\star$ е непрекъснат оператор, а от \Prop{composition}
  знаем, че композиция на непрекъснати оператори е също непрекъснат оператор.
  Заключаваме, че $\Delta_\tau$ е непрекъснат оператор.    
\end{proof}

\marginpar{Тук отново можем да си мислим за $\vv{f}_1$ като за \texttt{main} функцията на програмата $\vv{P}$.}
Нека разгледаме рекурсивната програма $\vv{P}[\varsx,\varsf]$ на езика $\REC$, където:
\begin{align*}
  \vv{P} = & 
             \begin{cases}
               & \vv{f}_1(\vv{x}_1,\dots,\vv{x}_{m_1}) = \tau_1[\vv{x}_1,\dots,\vv{x}_{m_1},\vv{f}_1,\dots,\vv{f}_k]\\
               & \vv{f}_2(\vv{x}_1,\dots,\vv{x}_{m_2}) = \tau_2[\vv{x}_1,\dots,\vv{x}_{m_2},\vv{f}_1,\dots,\vv{f}_k]\\
               & \vdots\\
               & \vv{f}_k(\vv{x}_1,\dots,\vv{x}_{m_k}) = \tau_k[\vv{x}_1,\dots,\vv{x}_{m_k},\vv{f}_1,\dots,\vv{f}_k]
             \end{cases}
\end{align*}
Нека $\bar{\delta} \in \DomOpCBV$ е {\em най-малкото решение} на системата,
която съответства на програмата $\texttt{P}$:
\begin{align*}
  & \Delta_{\tau_1}(\psi_1,\dots,\psi_k) = \psi_1\\
  & \ \vdots \\
  & \Delta_{\tau_k}(\psi_1,\dots,\psi_k) = \psi_k.
\end{align*}
Понеже $\Delta_{\tau_i}$ са непрекъснати оператори, то \hyperref[th:knaster-tarski]{Теоремата на Клини} знаем, че такива точни изображения $\ov{\delta}$ съществуват.

\index{денотационна семантика!по стойност}
\begin{framed}
  За рекурсивната програма $\vv{P}[\varsx,\varsf]$, определяме {\bf денотационната семантика с предаване на параметрите по стойност} 
  като изображението $\D_V\val{\vv{P}} \in \Strict{\Nat^{m_1}_\bot}{\Nat_\bot}$, където:
  \[\D_V\val{\vv{P}}(a_1,\dots,a_{m_1}) \dff \delta_1(a_1,\dots,a_{m_1}).\]
\end{framed}

%%% Local Variables:
%%% mode: latex
%%% TeX-master: "../sep"
%%% End:


\subsection{Сравнение между двете семантики}


\begin{example}
  Да разгледаме програмата $\vv{P}$ на езика $\REC$:
  \begin{haskellcode}
f(x) = g(f(x))
g(x) = 0  
  \end{haskellcode}
Да проверим, че
$\D_V\val{\vv{P}} \neq \D_N\val{\vv{P}}$.
Имаме, че
\begin{align*}
  \tau_1[\vv{x},\vv{f},\vv{g}] & \equiv \vv{g}(\vv{f}(\vv{x}))\\
  \tau_2[\vv{x},\vv{f},\vv{g}] & \equiv \vv{0}.
\end{align*}
За $i = 1,2$ дефинираме операторите 
\[\Delta_{\tau_i} \in \Cont{\Strict{\Nat_\bot}{\Nat_\bot} \times \Strict{\Nat_\bot}{\Nat_\bot}}{\Strict{\Nat_\bot}{\Nat_\bot}},\]
  които имат следните дефиниции:  
  \begin{align*}
    \Delta_{\tau_1}(\varphi,\psi)(x) & \dff
    \begin{cases}
      \psi(\varphi(x)), & \text{ако }x \neq \bot\\
      \bot, & \text{ако }x = \bot\\
    \end{cases}\\
    \Delta_{\tau_2}(\varphi,\psi)(x) & \dff
    \begin{cases}
      0, & \text{ако }x \neq \bot,\\
      \bot, & \text{ако }x = \bot.
    \end{cases}
  \end{align*}
  Нека $\Delta \dff \Delta_{\tau_1} \times \Delta_{\tau_2}$. За да намерим най-малкото
  решение на системата, трябва да намерим най-малката неподвижна точка на $\Delta$.
  Ще получим това най-малко решение като точна горна граница на редица от двойки функции $(\varphi_i, \psi_i)$.
  Имаме, че $\varphi_0(x) \dff \lambda x.\bot$ и $\psi_0(x) \dff \lambda x.\bot$.
  \begin{align*}
    \varphi_1(x) & \dff \Delta_{\tau_1}(\varphi_0,\psi_0)(x)\\
                 & =
                   \begin{cases}
                     \psi_0(\varphi_0(x)), & \text{ако } x \neq \bot\\
                     \bot, & \text{ако }x = \bot
                   \end{cases}\\
                 & = \bot\\
    \psi_1(x) & \dff \Delta_{\tau_2}(\varphi_0,\psi_0)(x)\\
                 & = 
                   \begin{cases}
                     0, & \text{ако }x \neq \bot\\
                     \bot, & \text{ако }x = \bot\\
                   \end{cases}
  \end{align*}
  Това означава, че $\Delta(\varphi_0,\psi_0) = (\varphi_1,\psi_1)$.
  При следващата итерация получаваме следното:
  \begin{align*}
    \varphi_2(x) & \dff \Delta_{\tau_1}(\varphi_1,\psi_1)(x)\\
                 & =
                   \begin{cases}
                     \psi_1(\varphi_1(x)) & \text{ако } x \neq \bot\\
                     \bot, & \text{ако } x = \bot\\
                   \end{cases}\\
                 & = 
                   \begin{cases}
                     \psi_1(\bot) & \text{ако } x \neq \bot\\
                     \bot, & \text{ако } x = \bot\\
                   \end{cases}\\
                 & = \bot\\
    \psi_2(x) & \dff \Delta_{\tau_2}(\varphi_1,\psi_1)(x)\\
                 & = \begin{cases}
                   0, & \text{ако }x \neq \bot\\
                   \bot, & \text{ако }x = \bot\\
                 \end{cases}\\
                 & = \psi_1(x).
  \end{align*}
  Получихме, че $\Delta(\varphi_1,\psi_1) = (\varphi_1,\psi_1)$ и
  следователно $\lfp(\Delta) = (\varphi_1,\psi_1)$.
  Тогава за всяко $a \in \Nat_\bot$,
  \[\D_V\val{\vv{P}}(a) = \varphi_1(a) = \bot.\]

  Сега да намерим семантика по име на горната програма.
За $i = 1,2$ дефинираме операторите 
  \[\Gamma_{\tau_i} \in \Cont{\Cont{\Nat_\bot}{\Nat_\bot} \times \Cont{\Nat_\bot}{\Nat_\bot}}{\Cont{\Nat_\bot}{\Nat_\bot}},\]
  които имат следните дефиниции
  \begin{align*}
    \Gamma_{\tau_1}(\varphi,\psi)(x) & \dff \psi(\varphi(x))\\
    \Gamma_{\tau_2}(\varphi,\psi)(x) & \dff 0.
  \end{align*}
  Търсим най-малката неподвижна точна на оператора 
  \[\Gamma(\varphi,\psi) \dff (\Gamma_{\tau_1}(\varphi, \psi), \Gamma_{\tau_2}(\varphi,\psi)).\]
  \marginpar{т.е. това са функциите, които винаги връщат $\bot$}
  Имаме, че $\varphi_0(x) \dff \lambda x.\bot$ и $\psi_0(x) \dff \lambda x.\bot$.
  Сега, за произволно $x \in \Nat_\bot$, получаваме:
  \begin{align*}
    \varphi_1(x) & \dff \Gamma_{\tau_1}(\varphi_0,\psi_0)(x) = \psi_0(\varphi_0(x)) = \bot\\
    \psi_1(x) & \dff \Gamma_{\tau_2}(\varphi_0,\psi_0)(x) = 0.
  \end{align*}
  Това означава, че $\Gamma(\varphi_0,\psi_0) = (\varphi_1,\psi_1)$.
  Итерираме процеса още веднъж, и получаваме:
  \marginpar{Единствената разлика е, че тук не разглеждаме отделен случай дали $x = \bot$}
  \begin{align*}
    \varphi_2(x) & \dff \Gamma_{\tau_1}(\varphi_1,\psi_1)(x) = \psi_1(\varphi_1(x)) = 0\\
    \psi_2(x) & \dff \Gamma_{\tau_2}(\varphi_1,\psi_1)(x) = 0.
  \end{align*}
  Това означава, че $\Gamma(\varphi_1,\psi_1) = (\varphi_2,\psi_2)$.
  Отново итерираме процеса:
    \begin{align*}
      \varphi_3(x) & \dff \Gamma_{\tau_1}(\varphi_2,\psi_2)(x) = \psi_2(\varphi_2(x)) = 0 = \varphi_2(x)\\
      \psi_3(x) & \dff \Gamma_{\tau_2}(\varphi_2,\psi_2)(x) = 0 = \psi_2(x).
  \end{align*}
  Получихме, че $\Gamma(\varphi_2,\psi_2) = (\varphi_2,\psi_2)$ и
  следователно $\lfp(\Gamma) = (\varphi_2,\psi_2)$. 
  Тогава, за всяко $a \in \Nat_\bot$,
  \[\D_N\val{\vv{P}}(a) = \varphi_2(a) = 0.\]
  Това означава, че денотационната семантика по име на програмата $\vv{P}$ е константната функция, която винаги връща $0$.
  Обърнете внимание, че също $\D_N\val{\vv{P}}(\bot) = 0$.
\end{example}

Можем да проверим нашите изчисления за най-малките неподвижни точки на операторите $\Gamma$ и $\Delta$
от горния пример като преведем техните дефиниции на \texttt{хаскел}.

\begin{haskellcode}
ghci> let gamma1(f,g)(x) = g(f(x))
ghci> let gamma2(f,g)(x) = 0
ghci> let gamma(f,g) = (gamma1(f,g), gamma2(f,g))
ghci> let omega = \x -> undefined
ghci> let approxCBN = (omega, omega) : [gamma(f,g) | (f,g) <- approxCBN ]
ghci> let (f3, g3) = approxCBN !! 3
ghci> f3(undefined)
0
ghci> f3(55)
0
ghci> :set -XBangPatterns
ghci> let delta1(f,g)(!x) = g(f(x))
ghci> let delta2(f,g)(!x) = 0
ghci> let delta(f,g) = (delta1(f,g), delta2(f,g))
ghci> let approxCBV = (omega, omega) : [delta(f,g) | (f,g) <- approxCBV ]
ghci> let (f3', g3') = approxCBV !! 3
ghci> f3'(undefined)
undefined
ghci> f3'(0)
undefined
ghci> f3'(55)
undefined
\end{haskellcode}



  Горният пример показва, че съществуват програми $\vv{P}$, 
  за които двете денотационни семантики са различни, но все пак $\D_V\val{\vv{P}} \sqsubseteq \D_N\val{\vv{P}}$.
  Ще видим, че това свойство е вярно за всяка програма $\vv{P}$ на езика \REC.


  \begin{prop}
    \label{pr:delta-in-gamma}
    Да разгледаме произволен терм $\tau[\vv{x}_1,\dots,\vv{x}_n, \vv{f}_1,\dots,\vv{f}_k]$.
    Тогава всеки 
    \[\bar{\varphi} \in \Mapping{\Nat^{m_1}_\bot}{\Nat_\bot}\times\cdots\times\Mapping{\Nat^{m_k}_\bot}{\Nat_\bot}\]
    е изпълнено, че
    \[\Delta_\tau(\bar{\varphi}) \sqsubseteq \Gamma_\tau(\bar{\varphi}).\]
  \end{prop}
  \begin{hint}
    \marginpar{Обърнете внимание, че ако $\varphi$ е точна функция, то $\Gamma_\tau(\varphi)$ е непрекъсната, но може да не е точна.}
    Понеже за всяка $\varphi \in \Mapping{\Nat^n_\bot}{\Nat_\bot}$,
    \[\Sigma_n(\varphi) \sqsubseteq \varphi,\]
    по получаваме, че
    \[\Delta_\tau(\bar{\varphi}) \dff \Sigma_n(\Gamma_\tau(\bar{\varphi})) \sqsubseteq \Gamma_\tau(\bar{\varphi}).\]
\end{hint}

\begin{framed}
  \begin{thm}
    За всяка рекурсивна програма \vv{P} е изпълнено, че
    \[\D_V\val{\vv{P}} \sqsubseteq \D_N\val{\vv{P}}.\]
  \end{thm}
\end{framed}
\begin{proof}
  Първо с индукция по $k$ ще докажем, че
  \[(\forall k)[\ \bar{\delta}_k \sqsubseteq \bar{\gamma}_k\ ].\]
  За $k = 0$ е ясно, защото $\delta^i_0 = \Omega^{(m_i)} = \gamma^i_0$.
  \marginpar{Озн. $\ov{\delta}_k = (\delta^1_k,\dots,\delta^n_k)$}
  \begin{align*}
    \delta^i_{k+1} & \dff \Delta_{\tau_i}(\bar{\delta}_k)\\
                   & \sqsubseteq \Gamma_{\tau_i}(\bar{\delta}_k) & \comment{\text{\Prop{delta-in-gamma}}}\\
                   & \sqsubseteq \Gamma_{\tau_i}(\bar{\gamma}_k) & \comment{\text{$\Gamma_{\tau_i}$ е мон. и от {\bf И.П.} имаме, че } \ov{\delta}_k \sqsubseteq \ov{\gamma}_k}\\
                   & \dff \gamma^i_{k+1}.
  \end{align*}
  Оттук следва, че 
  \begin{equation}
    \label{eq:12}
    \bar{\delta} = \bigsqcup_i\bar{\delta}_i \sqsubseteq \bigsqcup_i\bar{\gamma}_i = \bar{\gamma}.
  \end{equation}
  Сега, за произволни елементи $\ov{a}$,
  \begin{align*}
    \D_V\val{\vv{h}}(\bar{a}) & \dff \delta_1(\ov{a})\\
                              & \sqsubseteq \gamma_1(\ov{a})  & \comment{\text{от (\ref{eq:12}) имаме, че }\bar{\delta}\sqsubseteq\bar{\gamma}}\\
                              & \dff \D_N\val{\vv{h}}(\ov{a}).
  \end{align*}
\end{proof}


%%% Local Variables:
%%% mode: latex
%%% TeX-master: "../sep"
%%% End:


\section{Операционна семантика}
\index{операционна семантика}

\subsection{Предаване на параметрите по име}\index{операционна семантика!по име}

% Правилата за извод с предаване на параметрите по име, които означаваме като $\mu \Downarrow^P c$,
% са същите като тези с предаване на параметрите по стойност като 
% единствената разлика е, че вместо правилата $(4)_\Nat$ и $(4)_\bot$ имаме правилото $(4)$.


Дефинираме релация $\Downarrow_P$ по следния начин:
\begin{description}
\item
  % За всяко $a \in \Nat$,
  \begin{figure}[h!]
    \begin{prooftree}
      \AxiomC{}
      \RightLabel{\scriptsize{(1)}}
      \UnaryInfC{$\vv{a}\Downarrow_P a$}
    \end{prooftree}
  \end{figure}
\item
  \begin{figure}[h!]
    \begin{prooftree}
      \AxiomC{$\mu_1\Downarrow_P a_1$}
      \AxiomC{$\mu_2\Downarrow_P a_2$}
      \AxiomC{$a = \texttt{plus}(a_1, a_2)$}
      \RightLabel{\scriptsize{$(2_+)$}}
      \TrinaryInfC{$\mu_1 + \mu_2 \Downarrow_P a$}
    \end{prooftree}
  \end{figure}
\item
  \begin{figure}[h!]
    \begin{prooftree}
      \AxiomC{$\mu_1\Downarrow_P a_1$}
      \AxiomC{$\mu_2\Downarrow_P a_2$}
      \AxiomC{$a = \texttt{eq}(a_1, a_2)$}
      \RightLabel{\scriptsize{$(2_{\vv{==}})$}}
      \TrinaryInfC{$\mu_1\ \vv{==}\ \mu_2 \Downarrow_P a$}
    \end{prooftree}
  \end{figure}
\item
  \begin{figure}[h!]
    \begin{prooftree}
      \AxiomC{$\mu_0\Downarrow_P a_0$}
      \AxiomC{$\mu_1 \Downarrow_P a_1$}
      \AxiomC{$a_0 > 0$}
      \RightLabel{\scriptsize{$(3_\true)$}}
      \TrinaryInfC{$\ifelse{\mu_0}{\mu_1}{\mu_2} \Downarrow_P a_1$}
    \end{prooftree}
  \end{figure}  
\item
  \begin{figure}[h!]
    \begin{prooftree}
      \AxiomC{$\mu_0\Downarrow_P 0$}
      \AxiomC{$\mu_2 \Downarrow_P a_2$}
      \RightLabel{\scriptsize{$(3_\false)$}}
      \BinaryInfC{$\ifelse{\mu_0}{\mu_1}{\mu_2} \Downarrow_P a_2$}
    \end{prooftree}
  \end{figure}
\item
  \begin{figure}[h!]
    \begin{prooftree}
      \AxiomC{$\tau_i[\vv{x}_1/\mu_1,\dots,\vv{x}_{m_i}/\mu_{m_i}] \Downarrow_P a$}
      \RightLabel{\scriptsize{(4)}}
      \UnaryInfC{$\vv{f}_i(\mu_1,\dots,\mu_{m_i}) \Downarrow_P a$}
    \end{prooftree}
  \end{figure}
\end{description}

\begin{lemma}
  Докажете, че за всеки затворен терм $\mu$,
  ако $\mu \Downarrow_P a$ и $\mu \Downarrow_P a'$, то $a = a'$.
\end{lemma}


За фиксирана декларация $P$, нека за всеки {\em функционален} терм $\mu$ да дефинираме
\[\eval{\mu}\dff
  \begin{cases}
    b, & \text{ ако }\mu \Downarrow_P b\\
    \bot, & \text{ ако }\mu \text{ няма извод до константа}.
\end{cases}\]

\begin{framed}
  Операционната семантика по име на рекурсивната програма $\vv{P}[\varsx,\varsf]$ представлява
  изображението 
  \[\O\val{\vv{P}} \in \Cont{\Nat^{m_1}_\bot}{\Nat_\bot},\] където
  \[\O\val{\vv{P}}(a_1,\dots,a_{m_1}) \dff
    \begin{cases}
      \eval{\vv{f}_1(\vv{a}_1,\dots,\vv{a}_{m_1})}, & \text{ако }\bot \not\in\{a_1,\dots,a_{m_1}\}\\
      \bot, & \text{ако }\bot \in \{a_1,\dots,a_{m_1}\}
    \end{cases}\]
  за произволни $a_1,\dots,a_{m_1} \in \Nat_\bot$.
\end{framed}

\begin{remark}
  Всъщност ние все още няма как да знаем, че за всяка програма $\vv{P}$,
  $\O\val{\vv{P}}$ е непрекъснато изображение.
  Този факт може да се докаже директно, но вместо това, ние ще видим, че
  $\O\val{\vv{P}} = \D\val{\vv{P}}$ и оттам ще получим непрекъснатостта на $\O\val{\vv{P}}$,
  защото от дефиницията на $\D\val{\vv{P}}$ е ясно, че то е непрекъснато изображение.
\end{remark}

В някои доказателства ще се наложи да правим индукция по дължината на извода $\to_P$.
Затова дефинираме релацията $\mu \to^\ell_P \vv{a}$, която казва, че функционалният терм $\mu$
се свежда до константата $\vv{a}$ след $\ell$ на брой прилагания на правилата на операционната семантика.
Формално дефиницията е следната:

\marginpar{Това се нарича cost dynamics в \cite[стр. 58]{practical-foundations}.}

\begin{description}
\item
  % За всяко $a \in \Nat$,
  \begin{figure}[h!]
    \begin{prooftree}
      \AxiomC{}
      \RightLabel{\scriptsize{(const)}}
      \UnaryInfC{$\vv{a} \Downarrow^0_P \vv{a}$}
    \end{prooftree}
  \end{figure}
\item
  \begin{figure}[h!]
    \begin{prooftree}
      \AxiomC{$\mu_1\Downarrow^{\ell_1}_P \vv{a}_1$}
      \AxiomC{$\mu_2\Downarrow^{\ell_2}_P \vv{a}_2$}
      \AxiomC{$a = a_1 + a_2$}
      \RightLabel{\scriptsize{(plus)}}
      \TrinaryInfC{$\mu_1 + \mu_2 \Downarrow^{\ell_1+\ell_2+1}_P \vv{a}$}
    \end{prooftree}
  \end{figure}
\item
  \begin{figure}[h!]
    \begin{prooftree}
      \AxiomC{$\mu_1\Downarrow^{\ell_1}_P \vv{a}_1$}
      \AxiomC{$\mu_2\Downarrow^{\ell_2}_P \vv{a}_2$}
      \AxiomC{$a = \texttt{eq}(a_1, a_2)$}
      \RightLabel{\scriptsize{$(eq)$}}
      \TrinaryInfC{$\mu_1\ \vv{==}\ \mu_2 \Downarrow^{\ell_1+\ell_2+1}_P \vv{a}$}
    \end{prooftree}
  \end{figure}
\item
  \begin{figure}[h!]
    \begin{prooftree}
      \AxiomC{$\mu_1\Downarrow^{\ell_1}_P \vv{a}_1$}
      \AxiomC{$\mu_2 \Downarrow^{\ell_2}_P \vv{a}_2$}
      \AxiomC{$\vv{a}_1 \not\equiv \vv{0}$}
      \RightLabel{\scriptsize{(if$_\true$)}}
      \TrinaryInfC{$\ifelse{\mu_1}{\mu_2}{\mu_3} \Downarrow^{\ell_1+\ell_2+1}_P \vv{a}_2$}
    \end{prooftree}
  \end{figure}  
\item
  \begin{figure}[h!]
    \begin{prooftree}
      \AxiomC{$\mu_1\Downarrow^{\ell_1}_P \vv{0}$}
      \AxiomC{$\mu_3 \Downarrow^{\ell_3}_P \vv{a}_3$}
      \RightLabel{\scriptsize{(if$_\false$)}}
      \BinaryInfC{$\ifelse{\mu_1}{\mu_2}{\mu_3} \Downarrow^{\ell_1+\ell_3+1}_P \vv{a}_3$}
    \end{prooftree}
  \end{figure}
\item
  \begin{figure}[h!]
    \begin{prooftree}
      \AxiomC{$\tau_i[\vv{x}_1/\mu_1,\dots,\vv{x}_{m_i}/\mu_{m_i}] \Downarrow^{\ell}_P \vv{a}$}
      \RightLabel{\scriptsize{(cbn)}}
      \UnaryInfC{$\vv{f}_i(\mu_1,\dots,\mu_{m_i}) \Downarrow^{\ell+1}_P \vv{a}$}
    \end{prooftree}
  \end{figure}
\end{description}

\newpage

Също така, понякога ще пишем $\mu \Downarrow^{<\ell}_P \vv{a}$, когато искаме да кажем, че
$\mu$ се свежда до $\vv{a}$ след прилагане на по-малко от $\ell$ на брой правила от операционната семантика.

\begin{example}
  Нека за програмата \vv{P}:
  \begin{haskellcode}
    f(x,y) = if x == y then 0 else 1 + f(x,y+1)
  \end{haskellcode}
  да разгледаме няколко извода с правилата на операционната семантика по име.
\end{example}


%%% Local Variables:
%%% mode: latex
%%% TeX-master: "../sep"
%%% End:


\newpage
\def\defaultHypSeparation{\hskip .6cm}

\begin{landscape}

\begin{figure}[h!]
  \begin{framed}
    \begin{prooftree}
      \AxiomC{}
      \LeftLabel{\scriptsize{(1)}}
      \UnaryInfC{$\vv{3} \to^P_N 3$}
      \AxiomC{}
      \RightLabel{\scriptsize{(1)}}
      \UnaryInfC{$\vv{2} \to^P_N 2$}
      \LeftLabel{\scriptsize{$(2_{\vv{==}})$}}
      \BinaryInfC{$\vv{3 == 2} \to^P_N 0$}
      \AxiomC{}
      \LeftLabel{\scriptsize{(1)}}
      \UnaryInfC{$\vv{1} \to^P_N 1$}
      \AxiomC{}
      \LeftLabel{\scriptsize{(1)}}
      \UnaryInfC{$\vv{3} \to^P_N 3$}
      \AxiomC{}
      \RightLabel{\scriptsize{(1)}}
      \UnaryInfC{$\vv{2+1} \to^P_N 3$}
      \LeftLabel{\scriptsize{($2_{\vv{==}}$)}}
      \BinaryInfC{$\vv{3 == 2 + 1} \to^P_N 1$}
      \AxiomC{}
      \RightLabel{\scriptsize{(1)}}
      \UnaryInfC{$\vv{0} \to^P_N 0$}
      \RightLabel{\scriptsize{$(2_\true)$}}
      \BinaryInfC{$\ifelse{\vv{3 == 2+1}}{\vv{0}}{\vv{1+f(3,2+1+1)}} \to^P_N 0$}
      \RightLabel{\scriptsize{(4)}}
      \UnaryInfC{$\vv{f(3,2+1)} \to^P_N 0$}
      \RightLabel{\scriptsize{$(2_{\vv{+}})$}}
      \BinaryInfC{$\vv{1 + f(3,2+1)} \to^P_N 1$}
      \RightLabel{\scriptsize{$(2_{\false})$}}
      \BinaryInfC{$\ifelse{\vv{3 == 2}}{\vv{0}}{\vv{1 + f(3,2+1)}} \to^P_N 1$ }
      \RightLabel{\scriptsize{(4)}}
      \UnaryInfC{$\vv{f(3,2)} \to^P_N 1$}
    \end{prooftree}
  \end{framed}
  \caption{Крайно дърво на извод започващо от функционалния терм $\vv{f(3,2)}$}
\end{figure}

\begin{figure}[h!]
  \begin{framed}
    \begin{prooftree}
      \AxiomC{}
      \LeftLabel{\scriptsize{(1)}}
      \UnaryInfC{$\vv{2} \to^P_N 2$}
      \AxiomC{}
      \RightLabel{\scriptsize{(1)}}
      \UnaryInfC{$\bot \to^P_N \bot$}
      \RightLabel{\scriptsize{$(2_{\vv{==}})$}}
      \BinaryInfC{$\vv{3 == }\bot \to^P_N \bot$}
      \RightLabel{\scriptsize{$(2_{\bot})$}} 
      \UnaryInfC{$\ifelse{\vv{3 == }\bot}{\vv{0}}{\vv{1 + f(3,}\bot\vv{+1)}} \to^P_N \bot$ }
      \RightLabel{\scriptsize{(4)}}
      \UnaryInfC{$\vv{f(3,}\bot\vv{)} \to^P_N \bot$}
    \end{prooftree}
\end{framed}
\caption{Крайно дърво на извод започващо от функционалния терм $\vv{f(3,}\bot\vv{)}$}
\end{figure}


\def\defaultHypSeparation{\hskip .4cm}

\begin{figure}[h!]
  \begin{framed}
    \begin{prooftree}
      \AxiomC{}
      \LeftLabel{\scriptsize{(1)}}
      \UnaryInfC{$\vv{2} \to^P_N 2$}
      \AxiomC{}
      \RightLabel{\scriptsize{(1)}}
      \UnaryInfC{$\vv{3} \to^P_N 3$}
      \RightLabel{\scriptsize{$(2_{\vv{==}})$}}
      \BinaryInfC{$\vv{2 == 3} \to^P_N 0$}
      \AxiomC{}
      \RightLabel{\scriptsize{(1)}}
      \UnaryInfC{$\vv{1} \to^P_N 1$}
      \AxiomC{}
      \LeftLabel{\scriptsize{(1)}}
      \UnaryInfC{$\vv{2} \to^P_N 2$}
      \AxiomC{}
      \LeftLabel{\scriptsize{(1)}}
      \UnaryInfC{$\vv{3} \to^P_N 3$}
      \AxiomC{}
      \RightLabel{\scriptsize{(1)}}
      \UnaryInfC{$\vv{1} \to^P_N 1$}
      \RightLabel{\scriptsize{$(2_{\vv{+}})$}}
      \BinaryInfC{$\vv{3+1} \to^P_N 4$}
      \RightLabel{\scriptsize{$(2_{\vv{==}})$}}
      \BinaryInfC{$\vv{2 == 3+1} \to^P_N 0$}
      \AxiomC{$\vdots$}
      \UnaryInfC{$\vv{1 + f(2,3+1+1)} \to^P_N \square$}
      \RightLabel{\scriptsize{$(2_{\false})$}}
      \BinaryInfC{$\ifelse{\vv{2 == 3+1}}{\vv{0}}{\vv{1 + f(2,3+1+1)}} \to^P_N \square$}
      \RightLabel{\scriptsize{(4)}}
      \UnaryInfC{$\vv{f(2,3+1)} \to^P_N \square$}
      \RightLabel{\scriptsize{$(2_{\vv{+}})$}}
      \BinaryInfC{$\vv{1+f(2,3+1)} \to^P_N \square$}
      \RightLabel{\scriptsize{$(2_{\false})$}} 
      \BinaryInfC{$\ifelse{\vv{2 == 3}}{\vv{0}}{\vv{1 + f(2,3+1)}} \to^P_N \square$ }
      \RightLabel{\scriptsize{(4)}}
      \UnaryInfC{$\vv{f(2,3)} \to^P_N \square$}
    \end{prooftree}
\end{framed}
\caption{Част от безкрайното дърво на извод започващо от функционалния терм $\vv{f(2,3)}$}
\end{figure}


\end{landscape}

%%% Local Variables:
%%% mode: latex
%%% TeX-master: "../sep-notes"
%%% End:


\subsection{Теорема за еквивалентност}
\marginpar{Винаги с $\mu$ ще означаваме функционални термове; с $\tau$ - произволни термове}

\begin{prop}
  \label{pr:rec:op-name-inclusion1}
  \marginpar{В \cite[стр. 192]{ditchev-soskov}, \cite[стр. 157]{winskel} доказателството е друго}
  \marginpar{Сравнете с \Prop{rec:op-value-inclusion1}}
  Да разгледаме една програма \vv{P}.
  Нека $\mu[\vv{f}_1,\dots,\vv{f}_n]$ е {\em функционален} терм. Тогава
  \[\evaln{\mu} \sqsubseteq \val{\mu}(\bar{\gamma}),\]
  \marginpar{За това твърдение не е необходимо $\bar{\gamma}$ да бъде най-малката неподвижна точка на $\Gamma$, а просто решение.
    Само за другата посока ще е важно да е най-малкото решение.}
  където $\bar{\gamma} = (\gamma_1,\dots,\gamma_n)$ е 
  някоя неподвижна точка на оператора
  \[\Gamma \dff \Gamma_{\tau_1}\times \cdots \times \Gamma_{\tau_n},\]
  който съответства на програмата \vv{P}.
\end{prop}
\begin{proof}
  Ако от терма $\mu$ {\em няма извод} до елемент на $\Nat_\bot$, то
  по дефиниция $\evaln{\mu} = \bot$ и в този случай е очевидно, че
  \[\evaln{\mu} \sqsubseteq \val{\mu}(\ov{\gamma}).\]
  
  Интересният случай е когато от терма $\mu$ {\em има извод} до елемент на $\Nat_\bot$.
  Тогава ще докажем, че за произволен функционален терм $\mu$ и елемент $a \in \Nat_\bot$, 
  \begin{equation}
    \label{eq:16}
    \mu\to^{\vv{P}}_N a\ \implies \val{\mu}(\ov{\gamma}) = a.
  \end{equation}
  Доказателството на (\ref{eq:16}) ще проведем с индукция по дължината на извода $\mu\to^P_N a$.

  \marginpar{Константата $\vv{a}$ има стойност елементът $a$, която принадлежи на $\Nat_\bot$}
  Нека изводът има дължина 1. Понеже $\mu$ е функционален терм, според правилата на операционната семантика, единствената възможност е $\mu \equiv \vv{a}$.
  Тогава е очевидно, че $\val{\mu}(\bar{\gamma}) = a$. Ясно е, че в този случай,
  \[\vv{a} \to^{\vv{P}}_N a\ \implies\ \val{\vv{a}}(\ov{\gamma}) = a.\]
  
  Нека (\ref{eq:16}) е изпълнено, когато $\mu \to^P_N a$ има извод с дължина $< \ell$.
  Ще докажем твърдението когато изводът има дължина $\ell$.
  Трябва да разгледаме различни случаи в зависимост от вида на функционалния терм $\mu$.
  \begin{itemize}
  \item
    Да разгледаме случая, когато $\mu \equiv \mu_1 + \mu_2$.
    Нека $\mu_1 + \mu_2 \to^P_N a$ като дължината на извода е $\ell$.
    Според правилата за извод в операционната семантика по име, имаме следната ситуация:
    \begin{prooftree}
      \AxiomC{$\vdots$}
      \LeftLabel{\scriptsize{(Извод с дълж. $\ell_1$)}}
      \UnaryInfC{$\mu_1 \to^P_N a_1$}
      \AxiomC{$\vdots$}
      \RightLabel{\scriptsize{(Извод с дълж. $\ell_2$)}}
      \UnaryInfC{$\mu_2 \to^P_N a_2$}
      \RightLabel{\scriptsize{\text{правило }($2_{+}$)}}
      \BinaryInfC{$\mu_1 + \mu_2 \to^P_N a$}
    \end{prooftree}

    където $a = \texttt{plus}(a_1,a_2)$, и дължината на извода $\ell = \ell_1 + \ell_2 + 1$.
    Ясно е, че изводите на $\mu_1\to^P_N a_1$ и $\mu_2 \to^P_N a_2$ са с дължини $< \ell$.
    Следователно можем да приложим {\bf И.П.} за $\mu_1$ и $\mu_2$, откъдето получаваме, че
    \begin{align*}
      & \mu_1 \to^P_N a_1\ \implies \val{\mu_1}(\ov{\gamma}) =a _1\\
      & \mu_2 \to^P_N a_2\ \implies \val{\mu_2}(\ov{\gamma}) =a _2.
    \end{align*}
    Тогава получаваме, че ако $\mu_1 + \mu_2 \to^P_N a$, то
    \begin{align*}
      \val{\mu_1 + \mu_2}(\ov{\gamma}) & \dff \texttt{plus}(\val{\mu_1}(\ov{\gamma}), \val{\mu_2}(\ov{\gamma}))\\
                                       & = \texttt{plus}(a_1,a_2)\\
                                       & = a.
    \end{align*}
  \item
    Случаят, когато $\mu \equiv \mu_1\ \vv{==}\ \mu_2$ е лесен.
  \item
    Случаят, когато $\mu \equiv \ifelse{\mu_0}{\mu_1}{\mu_2}$, е лесен.
  \item
    Нека имаме функционалния терм $\mu \equiv \vv{f}_i(\mu_1,\dots,\mu_{m_i})$ и $\mu \to^P_N a$.
    Според правилата за извод в операционната семантика по име, имаме следната ситуация:
    \begin{prooftree}
      \AxiomC{$\vdots$}
      \RightLabel{\scriptsize{Извод с дълж. $(\ell-1)$}}
      \UnaryInfC{$\tau_i[\vv{x}_1/\mu_1,\dots,\vv{x}_{m_i}/\mu_{m_i}] \to^P_N a$}
      \RightLabel{\scriptsize{\text{правило }(4)}}
      \UnaryInfC{$\vv{f}_i(\mu_1,\dots,\mu_{m_i}) \to^P_N a$}
    \end{prooftree}
    
    Понеже изводът $\tau_i[\varsx/\ov{\mu}] \to^P_N a$ има дължина $\ell-1 < \ell$, 
    от {\bf И.П.} имаме, че 
    \[\tau_i[\varsx/\ov{\mu}] \to^P_N a\ \implies\ \val{\tau_i[\varsx/\ov{\mu}]}(\ov{\gamma}) = a.\]
    \marginpar{Озн. $\ov{\gamma} = (\gamma_1,\dots,\gamma_n)$. Единствено в този случай се използва, че $\ov{\gamma}$ е решение на системата от оператори, като не е задължително да е най-малкото решение.}
    Лесно се съобразява, че:
    \begin{align*}
      \val{\tau_i[\varsx/\ov{\mu}]}(\ov{\gamma}) & = \val{\tau_i}(\ov{\gamma})(\val{\mu_1}(\ov{\gamma}),\dots,\val{\mu_{m_i}}(\ov{\gamma})) & \comment{\text{\hyperref[lem:rec:substitution]{Лема за замяната}}}\\
                                                 & \dff \Gamma_{\tau_i}(\ov{\gamma})(\val{\mu_1}(\ov{\gamma}),\dots, \val{\mu_{m_i}}(\ov{\gamma})) \\
                                                 & = \gamma_i(\val{\mu_1}(\ov{\gamma}),\dots, \val{\mu_{m_i}}(\ov{\gamma})) & \comment{\gamma_i = \Gamma_{\tau_i}(\ov{\gamma})}\\
                                                 & \dff \val{\vv{f}_i(\mu_1,\dots,\mu_{m_i})}(\ov{\gamma}). & \comment{\text{стойност на терм}}\\
    \end{align*}
    Обединявайки всичко, което знаем, получаваме:
    \begin{align*}
      \vv{f}_i(\mu_1,\dots,\mu_{m_i}) \to^P_N a & \implies \tau_i[\varsx/\ov{\mu}] \to^P_N a & \comment{\text{правило }(4)}\\
                                                & \implies \val{\tau_i[\varsx/\ov{\mu}]}(\ov{\gamma}) = a & \comment{\text{{\bf И.П.}}}\\
                                                & \implies  \val{\vv{f}_i(\mu_1,\dots,\mu_{m_i})}(\ov{\gamma}) = a.
    \end{align*}
  \end{itemize}
  С това доказателството на (\ref{eq:16}) е завършено.
\end{proof}

\begin{framed}
  \begin{cor}
    \label{cr:on-in-dn}
    За всяка рекурсивна програма $\vv{P}$ на езика {\bf REC} имаме, че 
    \[\O_N\val{\vv{P}} \sqsubseteq \D_N\val{\vv{P}}.\]
  \end{cor}  
\end{framed}
\begin{proof}
  Нека $\ov{\gamma}$ е най-малкото решение на системата от непрекъснати оператори, която съответства на програмата $\vv{P}$.
  Тогава
  \begin{align*}
    \O_N\val{\vv{P}}(a_1,\dots,a_{m_1}) & \dff \evaln{\vv{f}_1(\vv{a}_1,\dots,\vv{a}_{m_1})} \\
                                        & \sqsubseteq \val{\vv{f}_1(\vv{a}_1,\dots,\vv{a}_{m_1})}(\ov{\gamma}) & \comment{\text{\Prop{rec:op-name-inclusion1}}} \\
                                        & \dff \gamma_1(\val{\vv{a}_1}(\ov{\gamma}),\dots,\val{\vv{a}_{m_1}}(\ov{\gamma})) & \comment{\text{стойност на терм}}\\
                                        & = \gamma_1(a_1,\dots,a_{m_1}) & \comment \val{\vv{a}_i}(\ov{\gamma}) = a_i\\
                                        & \dff \D_N\val{\vv{P}}(a_1,\dots,a_{m_1}).
  \end{align*}
\end{proof}

Така получихме едната посока на теоремата за еквивалентност.
Сега преминаваме към доказателството на другата посока.

Нека сега $\ov{\gamma}$ е най-малкото решение на системата от оператори, която съответства на програма \vv{P}
за денотационната семантика по име.
Да напомним, че това означава, че $\ov{\gamma} = \bigsqcup_r \ov{\gamma}_r$, 
където $\gamma^i_{r+1} = \Gamma_{\tau_i}(\ov{\gamma}_r)$ и $\ov{\gamma}_r = ( \gamma^1_r, \dots, \gamma^k_r)$.

\begin{prop}
  \label{pr:op-name-inclusion2}
  \marginpar{Съобразете защо не можем да докажем по-простото твърдение, че за всеки функционален терм $\mu$ и всяко $r$,
  $\val{\mu}(\ov{\gamma}_r) \sqsubseteq \evaln{\mu}$.}
  Да разгледаме една програма \vv{P} в езика $\REC$, като
  $\ov{\gamma} = \bigsqcup_r\ov{\gamma}_r$ е най-малкото решение на системата от оператори за \vv{P}.
  Тогава за произволен терм $\tau[\vv{x}_1,\dots,\vv{x}_n,\vv{f}_1,\dots,\vv{f}_k]$ и
  произволни функционални термове
  \[\mu_1[\vv{f}_1,\dots,\vv{f}_k],\dots,\mu_n[\vv{f}_1\dots,\vv{f}_k],\]
  и всяко $r$, е изпълнено, че:
  \[\val{\tau}(\ov{\gamma}_r)(\evaln{\mu_1},\dots,\evaln{\mu_n}) \sqsubseteq \evaln{\tau[\varsx/\ov{\mu}]}.\]
\end{prop}
\begin{proof}
  За произволно естествено число $r$, нека твърдението $\texttt{Include}(r)$ да гласи следното:

  ,,за произволен терм $\tau[\vv{x}_1,\dots,\vv{x}_n,\vv{f}_1,\dots,\vv{f}_k]$ и
  произволни функционални термове $\mu_1[\vv{f}_1,\dots,\vv{f}_k],\dots,\mu_n[\vv{f}_1\dots,\vv{f}_k]$,
  е изпълнено, че:
  \[\val{\tau}(\ov{\gamma}_r)(\evaln{\mu_1},\dots,\evaln{\mu_n}) \sqsubseteq \evaln{\tau[\varsx/\ov{\mu}]}.\text{''}\]
  
  Трябва да докажем, че $\texttt{Include}(r)$ е изпълнено за всяко $r$.
  Това ще направим с индукция по $r$.

  \begin{itemize}
  \item 
    Първо ще докажем $\texttt{Include}(0)$.
    Това ще направим с индукция по построението на терма $\tau$.
    Да разгледаме произволни функционални термове $\mu_i$ и за улеснение да положим $a_i \dff \evaln{\mu_i}$.
    \begin{itemize}
    \item
      Нека $\tau \equiv \vv{c}$. Тогава:
      \begin{align*}
        \val{\vv{c}}(\ov{\gamma}_0)(\ov{a}) & \dff c\\
                                            & \dff \evaln{\vv{c}} & \comment{\text{правило (1)}}\\
                                            & = \evaln{\vv{c}[\varsx/\ov{\mu}]}.
      \end{align*}
    \item
      Нека $\tau \equiv \vv{x}_i$. Тогава:
      \begin{align*}
        \val{\vv{x}_i}(\ov{\gamma}_0)(\ov{a}) & \dff a_i\\
                                              & = \evaln{\mu_i} & \comment{a_i \dff \evaln{\mu_i}}\\
                                              & = \evaln{\vv{x}_i[\varsx/\ov{\mu}]}.                    
      \end{align*}
    \item
      Нека $\tau \equiv \tau_1 + \tau_2$.
      В този случай, $\tau$ е съставен от по-простите термове $\tau_1$ и $\tau_2$.
      От {\bf И.П.} за $\tau_1$ и $\tau_2$ следва, че за $i = 1,2$ е изпълнено следното:
      \begin{equation}
        \label{eq:15}
        \val{\tau_i}(\ov{\gamma}_0)(\ov{a}) \sqsubseteq \evaln{\tau_i[\varsx/\ov{\mu}]}.
      \end{equation}
      Да напомним, че изображението $\texttt{plus}$ е непрекъснато, откъдето следва, че също така е монотонно. 
      Тогава:
      \begin{align*}
        \val{\tau_1 + \tau_2}(\ov{\gamma}_0)(\ov{a}) & \dff \texttt{plus}(\val{\tau_1}(\ov{\gamma}_0)(\ov{a}), \val{\tau_2}(\ov{\gamma}_0)(\ov{a}))\\
                                                    & \sqsubseteq \texttt{plus}(\evaln{\tau_1[\varsx/\ov{\mu}]}, \evaln{\tau_2[\varsx/\ov{\mu}]}) & \comment{\text{от (\ref{eq:15})}}\\
                                                    & = \evaln{\tau[\varsx/\ov{\mu}]} & \comment{\text{правило }(2_{+})}
      \end{align*}
    \item
      \marginpar{\writedown Тези два случая са за домашно!}
      Нека $\tau \equiv \tau_1\ \vv{==}\ \tau_2$.
    \item
      Нека $\tau \equiv \ifelse{\tau_0}{\tau_1}{\tau_2}$.
    \item
      Нека $\tau \equiv \vv{f}_i(\rho_1,\dots,\rho_{m_i})$. Тогава
      \begin{align*}
        \val{\tau}(\ov{\gamma}_0)(\ov{a}) & \dff \gamma^i_0(\val{\rho_1}(\ov{\gamma}_0)(\ov{a}), \dots,\val{\rho_{m_i}}(\ov{\gamma}_0)(\ov{a}))\\
                                          & = \bot & \comment{\gamma^i_0(\ov{x}) \dff \bot}\\
                                          & \sqsubseteq \evaln{\tau[\varsx/\ov{\mu}]}.
      \end{align*}
      Така доказахме, че $\texttt{Include}(0)$ е изпълнено.
    \end{itemize}
  \item
    Нека $r > 0$. Да приемем, че $\texttt{Include}(r-1)$ е изпълнено. Ще докажем $\texttt{Include}(r)$.
    \marginpar{\writedown Разгледайте сами останалите случаи за $\tau$ и се убедете, че те наистина се доказват по същия начин както при $r = 0$} 
    Единственият случай, който заслужава внимание е 
    \[\tau \equiv \vv{f}_i(\rho_1,\dots,\rho_{m_i}).\]
    Доказателствата на всички останали случаи за $\tau$ протичат по абсолютно същия начин както при $r = 0$.

    Понеже термът $\tau$ е построен с помощта на термовете $\rho_j$, за $j = 1, \dots, m_i$,
    можем да приложим {\bf И.П.} за тях и да получим, че 
    \begin{align*}
      \val{\rho_{j}}(\ov{\gamma}_r)(\underbrace{\evaln{\mu_1}}_{a_1},\dots,\underbrace{\evaln{\mu_n}}_{a_n}) & = \underbrace{\val{\rho_{j}}(\ov{\gamma}_r)(a_1,\dots, a_n)}_{b_j} \\
                                                                                                             & \sqsubseteq \underbrace{\evaln{\rho_j[\varsx/\ov{\mu}]}}_{c_j}. & \comment{\text{от {\bf И.П.}}}
    \end{align*}

    \marginpar{Обърнете внимание, че $\rho'_j$ са функционални термове}
    Нека за наше улеснение да положим $\rho'_j \dff \rho[\varsx/\ov{\mu}]$.
    Това означава, че до момента имаме следното:
    \[\val{\tau_i}(\ov{\varphi})(\underbrace{\val{\rho_1}(\ov{\gamma}_r)(\ov{a})}_{b_1}, \dots, \underbrace{\val{\rho_{m_i}}(\ov{\gamma}_r)(\ov{a})}_{b_{m_i}}) \sqsubseteq  \val{\tau_i}(\ov{\varphi})(\underbrace{\evaln{\rho'_1}}_{c_1},\dots, \underbrace{\evaln{\rho'_{m_i}}}_{c_{m_i}}),\]
    за произволни непрекъснати изображения $\ov{\varphi}$.
    
    Като обединим всичко от по-горе, получаваме следното:
    \begin{align*}
      \val{\tau}(\ov{\gamma}_r)(\ov{a}) & = \val{\vv{f}_i(\rho_1,\dots,\rho_{m_i})}(\ov{\gamma}_r)(\ov{a})\\
                                        & \dff \gamma^i_r(\underbrace{\val{\rho_1}(\ov{\gamma}_r)(\ov{a})}_{b_1}, \dots, \underbrace{\val{\rho_{m_i}}(\ov{\gamma}_r)(\ov{a})}_{b_{m_i}}) \\
                                        & = \gamma^i_r(b_1, \dots, b_{m_i})\\
                                        & \sqsubseteq \gamma^i_r(c_1, \dots, c_{m_i}) & \comment{\gamma^i_r\text{ е непр. и следователно мон.}}\\
                                        & = \Gamma_{\tau_i}(\ov{\gamma}_{r-1})(c_1,\dots,c_{m_i}) & \comment{\gamma^i_r \dff \Gamma_{\tau_i}(\ov{\gamma}_{r-1})}\\
                                        & \dff \val{\tau_i}(\ov{\gamma}_{r-1})(c_1,\dots, c_{m_i})\\
                                        & = \val{\tau_i}(\ov{\gamma}_{r-1})(\underbrace{\evaln{\rho'_1}}_{c_1},\dots,\underbrace{\evaln{\rho'_{m_i}}}_{c_{m_i}})  \\
                                        & \sqsubseteq \evaln{\tau_i[\vv{x}_1/\rho'_1,\dots,\vv{x}_{m_i}/\rho'_{m_i}]} & \comment{\text{от }\texttt{Include}(r-1)}\\
                                        & = \evaln{\vv{f}_i(\rho'_1,\dots,\rho'_{m_i})} & \comment{\text{от правило (4)}}\\
                                        & = \evaln{\vv{f}_i(\rho_1[\varsx/\ov{\mu}],\dots,\rho_{m_i}[\varsx/\ov{\mu}])} & \comment{\rho'_j \dff \rho[\varsx/\ov{\mu}]}\\
                                        & = \evaln{\vv{f}_i(\rho_1,\dots,\rho_{m_i})[\varsx/\ov{\mu}]} & \comment{\text{правила за замяна}}\\
                                        & = \evaln{\tau[\varsx/\ov{\mu}]}.
    \end{align*}
    Заключаваме, че
    \[\val{\tau}(\ov{\gamma}_r)(\underbrace{\evaln{\mu_1}}_{a_1},\dots,\underbrace{\evaln{\mu_n}}_{a_n}) \sqsubseteq  \evaln{\tau[\varsx/\ov{\mu}]}.\]
    Така доказахме $\texttt{Include}(r)$.
  \end{itemize}
  Най-накрая заключаваме, че $(\forall r)\texttt{Include}(r)$.
\end{proof}

\begin{cor}
  \label{cr:rec:equivalence-cbn:inclusion2}
  Нека $\mu$ е функционален терм.
  Тогава $\evaln{\mu}$ е горна граница на веригата $(\val{\mu}(\ov{\gamma}_r))^{\infty}_{r=0}$.
\end{cor}

\begin{lemma}
  За всяка рекурсивна програма $\vv{P}$,
  произволен {\em функционален} терм $\mu$,
  \[\val{\mu}(\ov{\gamma}) \sqsubseteq \evaln{\mu},\]
  където $\ov{\gamma} = \lfp(\Gamma)$, а $\Gamma = \Gamma_{\tau_0} \times \Gamma_{\tau_2} \times \cdots \times \Gamma_{\tau_k}$ е операторът, който съответства на програмата $\vv{P}$.
\end{lemma}
\begin{hint}

  \begin{align*}
    \val{\mu}(\ov{\gamma}) & = \val{\mu}(\bigsqcup_r\ov{\gamma}_r) & \comment{\ov{\gamma} = \bigsqcup_r \ov{\gamma}_r}\\
                           & = \bigsqcup_r \val{\mu}(\ov{\gamma}_r) & \comment{\text{от \Lem{rec:functional:term:continuous}}}\\
                           & \sqsubseteq \evaln{\mu}. & \comment{\text{от \Cor{rec:equivalence-cbn:inclusion2}}}
  \end{align*}
\end{hint}

\begin{framed}
  \begin{thm}
    За всяка рекурсивна програма $\vv{P}$ на езика {\bf REC} е изпълнено:
    \[\O_N\val{\vv{P}} = \D_N\val{\vv{P}}.\]
  \end{thm}  
\end{framed}
\begin{proof}
  \marginpar{Озн. $\ov{\gamma} = (\gamma_1,\dots,\gamma_n)$ е най-малкото решение на системата от оператори съответстващи на програмата \vv{P} }
  Ние вече знаем от \Cor{on-in-dn}, че 
  \[\O_N\val{\vv{P}} \sqsubseteq \D_N\val{\vv{P}}.\]
  Остава да докажем обратната посока, а именно 
  \[\D_N\val{\vv{P}} \sqsubseteq \O_N\val{\vv{P}}.\]
  За произволни елементи $\ov{a} \in \Nat^n_\bot$, имаме следните връзки:
  \begin{align*}
    \D_N\val{\vv{P}}(\ov{a}) & \dff \gamma_1(\ov{a})\\
                             & = \Gamma_{\tau_1}(\ov{\gamma})(\ov{a}) & \comment{\gamma_1 \dff \Gamma_{\tau_1}(\ov{\gamma})}\\
                             & \dff \val{\tau_1}(\ov{\gamma})(\ov{a}) \\
                             & = \val{\tau_1[\varsx/\ov{\vv{a}}]}(\ov{\gamma}) & \comment{\text{\hyperref[lem:rec:substitution]{Лема за замяната}}}\\
                             & \sqsubseteq \evaln{\tau_1[\varsx/\ov{\vv{a}}]} & \comment{\text{\Prop{op-name-inclusion2}}}\\
                             & = \evaln{\vv{f}_1(\vv{a}_1,\dots,\vv{a}_{m_1})} & \comment{\text{правило (4) на опер. сем.}}\\
                             & \dff \O_N\val{\vv{P}}(\ov{a}).
  \end{align*}
\end{proof}


%%% Local Variables:
%%% mode: latex
%%% TeX-master: "../sep"
%%% End:


\subsection{Предаване на параметрите по стойстност}
\index{операционна семантика!по стойност}


В операционната семантика показва как свеждаме един {\em функционален} терм до естествено число или $\bot$.
Да разгледаме една програмата \vv{P}.

За всеки {\em функционален терм} $\mu$ дефинираме {\bf извод $\mu \to^P_V a$ с предаване на параметрите по стойност}
посредством индукция по построението на функционалния терм $\mu$.
\marginpar{Тук разликата, в сравнение с \cite{ditchev-soskov, winskel} е, че $\bot$ е константа в езика}
\marginpar{Да напомним, че функционален терм е терм без свободни обектови променливи}
\marginpar{В \cite{ditchev-soskov} се използва означението $\vv{P} \vdash_V \mu \to n$}
\begin{description}
\item
  % За произволно $a \in \Nat_\bot$,
  \begin{figure}[h!]
    \begin{prooftree}
      \AxiomC{}
      \RightLabel{\scriptsize{(1)}}
      \UnaryInfC{$\vv{a}\to^P_V a$}
    \end{prooftree}
  \end{figure}
\item
  \begin{figure}[h!]
    \begin{prooftree}
      \AxiomC{$\mu_1\to^P_V a_1$}
      \AxiomC{$\mu_2\to^P_V a_2$}
      \AxiomC{$a = \texttt{plus}(a_1, a_2)$}
      \RightLabel{\scriptsize{$(2_+)$}}
      \TrinaryInfC{$\mu_1 + \mu_2 \to^P_V a$}
    \end{prooftree}
  \end{figure}
\item
  \begin{figure}[h!]
    \begin{prooftree}
      \AxiomC{$\mu_1\to^P_N a_1$}
      \AxiomC{$\mu_2\to^P_N a_2$}
      \AxiomC{$a = \texttt{eq}(a_1, a_2)$}
      \RightLabel{\scriptsize{$(2_{\vv{==}})$}}
      \TrinaryInfC{$\mu_1\ \vv{==}\ \mu_2 \to^P_N a$}
    \end{prooftree}
  \end{figure}
\item
  \begin{figure}[h!]
    \begin{prooftree}
      \AxiomC{$\mu_0\to^P_V a_0$}
      \AxiomC{$\mu_1 \to^P_V a_1$}
      \AxiomC{$a_0 \in \Nat^+$}
      \RightLabel{\scriptsize{$(3_\true)$}}
      \TrinaryInfC{$\ifelse{\mu_0}{\mu_1}{\mu_2} \to^P_V a_1$}
    \end{prooftree}
  \end{figure}  
\item
  \begin{figure}[h!]
    \begin{prooftree}
      \AxiomC{$\mu_0\to^P_V 0$}
      \AxiomC{$\mu_2 \to^P_V a_2$}
      \RightLabel{\scriptsize{$(3_\false)$}}
      \BinaryInfC{$\ifelse{\mu_0}{\mu_1}{\mu_2} \to^P_V a_2$}
    \end{prooftree}
  \end{figure}
\item
  \begin{figure}[h!]
    \begin{prooftree}
      \AxiomC{$\mu_0\to^P_V \bot$}
      \RightLabel{\scriptsize{$(3_\bot)$}}
      \UnaryInfC{$\ifelse{\mu_0}{\mu_1}{\mu_2} \to^P_V \bot$}
    \end{prooftree}
  \end{figure}
\item
  \begin{figure}[h!]
    \begin{prooftree}
      \AxiomC{$\mu_1\to^P_V a_1$}
      \AxiomC{$\cdots$}
      \AxiomC{$\mu_{m_i}\to^P_V a_{m_i}$}
      \AxiomC{$\tau_i[\vv{x}_1/\vv{a}_1,\dots,\vv{x}_{m_i}/\vv{a}_{m_i}] \to^P_V a$}
      \AxiomC{$\bot \not\in \{a_1,\dots, a_{m_i}\}$}
      \RightLabel{\scriptsize{$(4_\Nat)$}}
      \QuinaryInfC{$\vv{f}_i(\mu_1,\dots,\mu_{m_i}) \to^P_V a$}
    \end{prooftree}
  \end{figure}
\item
  \begin{figure}[h!]
    \begin{prooftree}
      \AxiomC{$\mu_1\to^P_V a_1$}
      \AxiomC{$\cdots$}
      \AxiomC{$\mu_{m_i}\to^P_V a_{m_i}$}
      \AxiomC{$\bot\in\{a_1,\dots, a_{m_i}\}$}
      \RightLabel{\scriptsize{$(4_\bot)$}}
      \QuaternaryInfC{$\vv{f}_i(\mu_1,\dots,\mu_{m_i}) \to^P_V \bot$}
    \end{prooftree}
  \end{figure}
\end{description}

\Stefan{Всъщност това правило $3_\bot$ за какво ми е ? Ако го махна, всичко ще продължи да си е ОК.}

\marginpar{Докато доказателства за денотационна семантика протичаха с индукция по построението на термовете,
  доказателствата за операционна семантика обикновено протичат с индукция по дължината на извода.}

За фиксираната програма $\vv{P}$, с всеки {\em функционален} терм $\mu[\varsf]$ асоциираме 
\[\evalv{\mu}\dff
\begin{cases}
  b, & \text{ ако }\mu \to^P_V b\\
  \bot, & \text{ ако }\mu \text{ няма извод до константа}.
\end{cases}\]

\begin{framed}
  % \index{$\O_V$}
  Операционната семантика по стойност на рекурсивната програма $\vv{P}[\varsx,\varsf]$ представлява изображението
  $\O_V\val{\vv{P}} \in \Strict{\Nat^{m_0}_\bot}{\Nat_\bot}$
  дефинирано като
  \[\O_V\val{\vv{P}}(a_1,\dots,a_{m_1}) \dff \evalv{\vv{f}_1(\vv{a}_1,\dots,\vv{a}_{m_1})},\]
  за произволни $a_1,\dots,a_{m_1} \in \Nat_\bot$.
\end{framed}

\begin{remark}
  Можем директно да докажем, че $\O_V\val{\vv{P}}$ е точно изображение.
  Но това е излишно. Това свойство ще следва от теоремата за еквивалентност, която ще докажем след малко, 
  защото вече знаем, че денотационната семантика по стойност представлява точно изображение.
\end{remark}

\begin{example}
  Да разгледаме отново програмата \vv{P}, където:
  \begin{haskellcode}
f(x) = g(f(x)) where
  g(x) = 0
  \end{haskellcode}
  За нея вече видяхме, че $\D_V\val{\vv{P}} \sqsubseteq \D_N\val{\vv{P}}$,
  но $\D_V\val{\vv{P}} \neq \D_N\val{\vv{P}}$.
  
  Да разгледаме операционната семантика на тази програма.
  Лесно се съобразява, че $\vv{f(1)} \to^P_N 0$.
  
  \begin{prooftree}
    \AxiomC{}
    \RightLabel{\scriptsize{правило (1)}}
    \UnaryInfC{$\vv{0}[\vv{x}/\vv{g(f(1))}] \to^P_N 0$}
    \RightLabel{\scriptsize{правило (4)}}
    \UnaryInfC{$\vv{g(f(x))}[\vv{x}/\vv{1}] \to^P_N 0$}
    \RightLabel{\scriptsize{правило (4)}}
    \UnaryInfC{$\vv{f(1)} \to^P_N 0$}
  \end{prooftree}

  
  Да видим защо $\vv{f(1)}$ няма извод до елемент в операционната семантика по стойност.

  \begin{prooftree}
    \AxiomC{}
    \LeftLabel{\scriptsize{(1)}}
    \UnaryInfC{$\vv{1} \to^P_V 1$}
    \AxiomC{$\vdots$}
    \LeftLabel{\scriptsize{правило $(4_\Nat)$}}
    \UnaryInfC{$\vv{f(1)} \to^P_V \square$}
    \AxiomC{}
    \LeftLabel{\scriptsize{(1)}}
    \UnaryInfC{$\vv{0}[\vv{x}/\square] \to^P_V 0$}
    \RightLabel{\scriptsize{$(4_\bot)$ или $(4_\Nat)$}}
    \BinaryInfC{$\vv{g(f(x))}[\vv{x}/\vv{1}] \to^P_V \square$}
    \RightLabel{\scriptsize{правило $(4_\Nat)$}}
    \BinaryInfC{$\vv{f(1)} \to^P_V \square$}
  \end{prooftree}

\end{example}


%%% Local Variables:
%%% mode: latex
%%% TeX-master: "../sep-notes"
%%% End:


\subsection{Теорема за еквивалентност}

\begin{prop}
  \label{pr:rec:op-value-inclusion1}
  \marginpar{Сравнете с \Prop{rec:op-name-inclusion1}}
  Нека $\mu$ е {\em функционален} терм и 
  \[\bar{\delta} \in \DomOpCBV\]
  е решение на системата от уравнения $\Delta_{\tau_i}$ съответстващи на програмата \vv{P}.
  \marginpar{Тук не е необходимо $\bar{\delta}$ да бъде най-малкото решение}
  Тогава 
  \marginpar{\cite[стр. 182]{ditchev-soskov}}
  \[\evalv\mu \sqsubseteq \val{\mu}(\ov{\delta}).\]
\end{prop}
\begin{proof}
  \marginpar{Тук до голяма степен повтаряме същите разсъждения както в доказателството на \Prop{rec:op-name-inclusion1}}
  Ако от терма $\mu$ {\em няма извод} до елемент на $\Nat_\bot$, то
  по дефиниция $\evaln{\mu} = \bot$ и в този случай е очевидно, че
  \[\evalv{\mu} \sqsubseteq \val{\mu}(\ov{\delta}).\]

  \marginpar{Ако $\mu \to^P_V \bot$, то е ясно, че $\evalv{\mu} \sqsubseteq \val{\mu}(\ov{\delta})$}
  Интересният случай е когато от терма $\mu$ {\em има извод} до елемент на $\Nat$.
  Тогава ще докажем, че за произволен функционален терм $\mu$ и елемент $a \in \Nat$, 
  \begin{equation}
    \label{eq:18}
    \mu\to^P_V a\ \implies \val{\mu}(\ov{\delta}) = a.
  \end{equation}
  Доказателството на (\ref{eq:18}) ще проведем с индукция по дължината $\ell$ на извода $\mu\to^P_V a$.

  Първо, нека $\ell = 1$. Тогава единствения случай, който трябва да разгледаме е $\mu \equiv \vv{a}$.
  Този случай е прекалено лесен.

  Нека сега изводът $\mu \to^P_V a$ има дължина $\ell > 1$.
  Понеже правилата за извод строго следват дефиницията на термовете, 
  трябва да разгледаме следните случаи.
  \begin{itemize}
  \item
    Нека $\mu \equiv \mu_1 + \mu_2$, като $\evalv{\mu} = a$. Тогава:
    \begin{prooftree}
      \AxiomC{$\vdots$}
      \LeftLabel{\scriptsize{(извод с дълж. $\ell_1$)}}
      \UnaryInfC{$\mu_1 \to^P_V a_1$}
      \AxiomC{$\vdots$}
      \RightLabel{\scriptsize{(извод с дълж. $\ell_2$)}}
      \UnaryInfC{$\mu_2 \to^P_V a_2$}
      \RightLabel{\scriptsize{\text{правило }$(2_+)$}}
      \BinaryInfC{$\mu \to^P_V a$}
    \end{prooftree}
    Тук имаме, че $\ell = \ell_1 + \ell_2 + 1$ и $a = \texttt{plus}(a_1,a_2)$.
    Следователно можем да приложим {\bf И.П.} за $\mu_1$ и $\mu_2$, откъдето получаваме, че
    \begin{align*}
      & \mu_1 \to^P_V a_1\ \implies \val{\mu_1}(\ov{\delta}) = a _1\\
      & \mu_2 \to^P_V a_2\ \implies \val{\mu_2}(\ov{\delta}) = a _2.
    \end{align*}
    % Получаваме, че:
    % \begin{align*}
    %   & \evalv{\mu_1} = a_1 \sqsubseteq \val{\mu}(\ov{\delta})\\
    %   & \evalv{\mu_2} = a_2 \sqsubseteq \val{\mu}(\ov{\delta}).
    % \end{align*}
    % Тогава, понеже изображението $\texttt{plus}$ е непрекъснато, а следователно и монотонно, то
    % можем да заключим, че 
    Тогава получаваме, че ако $\mu_1 + \mu_2 \to^P_V a$, то
    \begin{align*}
      \val{\mu_1 + \mu_2}(\ov{\delta}) & \dff \texttt{plus}(\val{\mu_1}(\ov{\gamma}), \val{\mu_2}(\ov{\delta}))\\
                                       & = \texttt{plus}(a_1,a_2)\\
                                       & = a.
    \end{align*}
  \item
    Нека $\mu \equiv \mu_1\ \vv{==}\ \mu_2$. Този случай не крие изненади.
    \marginpar{\writedown Домашно!}
  \item
    Нека $\mu \equiv \ifelse{\tau_0}{\tau_1}{\tau_2}$.    
  \item
    Нека $\mu \equiv \vv{f}_i(\mu_1,\dots,\mu_{m_i})$ и да приемем, че $\mu \to^P_V a$.
    Според правилата на операционната семантика, единствената възможна ситуация е следната.
    Съществуват елементи $a_1,\dots,a_{m_i} \in \Nat$, за които:
    % \begin{itemize}
    % \item 
    %   Ако $\bot \not\in \{a_1,\dots,a_{m_i}\}$, където:
    \begin{prooftree}
      \AxiomC{$\vdots$}
      \LeftLabel{\scriptsize(дълж. $\ell_1$)}
      \UnaryInfC{$\mu_1\to^P_V a_1$}
      \AxiomC{$\cdots$}
      \AxiomC{$\vdots$}
      \LeftLabel{\scriptsize($\ell_{m_i}$)}
      \UnaryInfC{$\mu_{m_i} \to^P_V a_{m_i}$}
      \AxiomC{$\vdots$}
      \RightLabel{\scriptsize(дълж. $\ell_0$)}
      \UnaryInfC{$\tau_i[\varsx/\bar{\vv{a}}] \to^P_V a$}
      \LeftLabel{\scriptsize{(дълж. $\ell$)}}
      \RightLabel{\scriptsize{правило ($4_\Nat$)}}
      \QuaternaryInfC{$\vv{f}_i(\mu_1,\dots,\mu_{m_i}) \to^P_V a$}
    \end{prooftree}

    Понеже за $j = 1,\dots, m_i$ изводът на $\mu_j\to^P_V a_j$ има дължина $\ell_j < \ell$,
    то от {\bf И.П.} за (\ref{eq:18}) имаме, че 
    \begin{equation}
      \label{eq:20}
      \val{\mu_j}(\bar{\delta}) = a_j.
    \end{equation}
    \marginpar{$\ell_0 + \ell_1+\cdots \ell_{m_i} + 1 = \ell$}
    Аналогично, понеже изводът на $\tau_i[\varsx/\bar{\vv{a}}] \to^{\vv{P}}_V a$ има дължина $\ell_0 < \ell$,
    и очевидно термът $\tau_i[\varsx/\bar{\vv{a}}]$ е функционален, то от {\bf И.П.} за (\ref{eq:18}) имаме, че 
    \begin{equation}
      \label{eq:3}
      \val{\tau_i[\varsx/\bar{\vv{a}}]}(\bar{\delta}) = a.
    \end{equation}
    
    Получаваме следното:
    \begin{align*}
      \val{\mu}(\ov{\delta}) & = \val{\vv{f}_i(\mu_1,\dots,\mu_{m_i})}(\ov{\delta}) \\
                             & = \delta_i(\underbrace{\val{\mu_1}(\bar{\delta})}_{a_1},\dots,\underbrace{\val{\mu_{m_i}}(\bar{\delta})}_{a_{m_i}}) & \comment{\text{стойност на терм}}\\
                             & = \delta_i(a_1,\dots,a_{m_i}) & \comment{\text{от (\ref{eq:20})}}\\
                             & = \Delta_{\tau_i}(\bar{\delta})(a_1,\dots,a_{m_i}) & \comment{\delta_i = \Delta_{\tau_i}(\ov{\delta})}\\
                             & = \Sigma_{\star}(\Gamma_{\tau_i}(\ov{\delta}))(\bar{a}) & \comment{\Delta_{\tau_i} \dff \Sigma_{\star} \circ \Gamma_{\tau_i}}\\
                             & = \Gamma_{\tau_i}(\bar{\delta})(\ov{a}) & \comment{\text{защото }\bot\not\in\{a_1,\dots,a_n\}}\\
                             & \dff \val{\tau_i}(\ov{\delta})(\ov{a}) & \comment{\ov{\delta}\text{ са непрекъснати}}\\
                             & = \val{\tau_i[\varsx/\bar{\vv{a}}]}(\bar{\delta}) & \comment{\text{\hyperref[lem:rec:substitution]{Лема за замяната}}}\\
                             & = a & \comment{\text{от (\ref{eq:3})}}
    \end{align*}
  % \item
  %   Ако $\bot \in \{a_1,\dots,a_{m_i}\}$, където:
  %   \begin{prooftree}
  %     \AxiomC{$\vdots$}
  %     \LeftLabel{\scriptsize(дълж. $l_1$)}
  %     \UnaryInfC{$\mu_1\to^P_V a_1$}
  %     \AxiomC{$\cdots$}
  %     \AxiomC{$\vdots$}
  %     \LeftLabel{\scriptsize($l_{m_i}$)}
  %     \UnaryInfC{$\mu_{m_i} \to^P_V a_{m_i}$}
  %     \RightLabel{\scriptsize{правило ($4_\bot$)}}
  %     \TrinaryInfC{$\vv{f}_i(\mu_1,\dots,\mu_{m_i}) \to^P_V \bot$}
  %   \end{prooftree}
  %   \marginpar{$l_0 + l_1+\cdots l_{m_i} = l$ и $l_j \geq 1$}
  %   В този случай е ясно, че $a = \bot$. Тогава понеже за $j = 1,\dots, m_i$ изводът на $\mu_j\to^P_V a_j$ има дължина $l_j < l$,
  %   то от {\bf И.П.} за (\ref{eq:18}) имаме, че 
  %   \begin{equation}
  %     \label{eq:19}
  %     \val{\mu_j}(\bar{\delta}) = a_j.
  %   \end{equation}
  %   Накрая получаваме следното:
  %   \begin{align*}
  %     \val{\mu}(\ov{\delta}) & = \val{\vv{f}_i(\mu_1,\dots,\mu_{m_i})}(\ov{\delta}) \\
  %                            & \dff \delta_i(\underbrace{\val{\mu_1}(\bar{\delta})}_{a_1},\dots,\underbrace{\val{\mu_{m_i}}(\bar{\delta})}_{a_{m_i}}) & \comment{\text{ стойност на терм}}\\
  %                            & = \delta_i(a_1,\dots,a_{m_i}) & \comment{\text{от (\ref{eq:18})}}\\
  %                            & = \bot. & \comment{\delta_i\text{ е точно изображение}}
  %   \end{align*}
  % \end{itemize}
  \end{itemize}
\end{proof}

Обърнете внимание, че дотук използвахме само, че $\bar{\delta}$ е решение на системата от оператори $\Delta_{\tau_i}$ зададена от програмата \vv{P}. 
Не сме използвали, че $\ov{\delta}$ е най-малкото решение.

\begin{framed}
  \begin{cor}
    \label{cr:ov-in-dv}
    За всяка рекурсивна програма $\vv{P}$ на езика $\REC$  имаме, че
    \[\O_V\val{\vv{P}} \sqsubseteq \D_V\val{\vv{P}}.\]
  \end{cor}
\end{framed}
\begin{proof}
  \marginpar{Озн. с $\ov{\delta}$ най-малкото решение на системата от оператори за програмата \vv{P}} Тогава
  \begin{align*}
    \O_V\val{\vv{P}}(\ov{a}) & \dff \evalv{\vv{f}_1(\vv{a}_1,\dots,\vv{a}_{m_1})} \\
                             & \sqsubseteq \val{\vv{f}_1(\vv{a}_1,\dots,\vv{a}_{m_1})}(\ov{\delta}) & \comment{\text{от \Prop{rec:op-value-inclusion1}}}\\
                             & = \delta_1(\val{\vv{a}_1}(\ov{\delta}), \dots, \val{\vv{a}_{m_1}}(\ov{\delta})) \\
                             & = \delta_1(a_1,\dots,a_{m_1})\\
                             & \dff \D_V\val{\vv{P}}(\ov{a}).
  \end{align*}

  
  % \begin{itemize}
  % \item 
  %   Нека $\bot \not\in \{a_1,\dots,a_{m_0}\}$. Тогава:
  %   \begin{align*}
  %     \O_V\val{\vv{P}}(\ov{a}) & \dff \evalv{\vv{f}_0(\vv{a}_1,\dots,\vv{a}_{m_0})} \\
  %                              & = \evalv{\tau_0[\varsx/\ov{\vv{a}}]} & \comment{\text{Правило }(4_\Nat)}\\
  %                              & \sqsubseteq \val{\tau_0[\varsx/\ov{\vv{a}}]}(\ov{\delta})& (\text{от \Prop{rec:op-value-inclusion1}})\\
  %                              & = \val{\tau_0}(\ov{\delta})(\ov{a}) & \comment{\text{\hyperref[lem:rec:substitution]{Лема за замяната}}}\\
  %                              & \dff \Gamma_{\tau_0}(\ov{\delta})(\ov{a})\\
  %                              & = \Sigma_{\star}(\Gamma_{\tau_0}(\ov{\delta})(\ov{a}) & \comment{\bot\not\in\{a_1,\dots,a_{m_0}\}}\\
  %                              & \dff \Delta_{\tau_0}(\ov{\delta})(\ov{a}) & \comment{\Delta_{\tau_0} = \Sigma_{\star} \circ \Gamma_{\tau_0}}\\
  %                              & = \delta_0(\ov{a}) & \comment{\delta_0 = \Delta_{\tau_0}(\ov{\delta})}\\
  %                              & \dff \D_V\val{\vv{P}}(\ov{a}).
  %   \end{align*}
  % \item
  %   Нека $\bot \in \{a_1,\dots,a_{m_0}\}$. Тогава:
  %   \begin{align*}
  %     \O_V\val{\vv{P}}(\ov{a}) & \dff \evalv{\vv{f}_0(\vv{a}_1,\dots,\vv{a}_{m_0})} \\
  %                              & = \bot & \comment{\text{Правило }(4_\bot)}\\
  %                              & \sqsubseteq \D_V\val{\vv{P}}.
  %   \end{align*}
  % \end{itemize}
\end{proof}


Обръщаме нашето внимание към доказателството на обратната посока, т.е. $\D_V\val{h} \sqsubseteq \O_V\val{h}$. 
Нека сега $\ov{\delta}$ е най-малкото решение на системата от оператори, която съответства на програма \vv{P}
за денотационната семантика по стойност.
Да напомним, че това означава, че $\ov{\delta} = \bigsqcup_r \ov{\delta}_r$, 
където $\delta^i_{r+1} = \Delta_{\tau_i}(\ov{\delta}_r)$

\begin{prop}
  \label{pr:rec:op-value-inclusion2}
  Тогава за всеки {\em функционален} терм $\mu[\vv{f}_1,\dots,\vv{f}_k]$ и всяко $r$,
  \[\val{\mu}(\ov{\delta}_r) \sqsubseteq \evalv{\mu}.\]
\end{prop}
\begin{hint}
  \marginpar{Сравнете с \Prop{op-name-inclusion2}. Основната разлика е, че тук можем да минем само с функционални термове. Това не можем да направим при семантиката по име поради разликата в правилата за извод. Всъщност можем да го направим, ако имаме лема за симулацията, но тя сега е сложена като задача}
  
  Доказателството на това твърдение следва същата схема като доказателството на \Prop{op-name-inclusion2}.
  За произволно естествено число $r$, нека твърдението $\texttt{Include}(r)$ да гласи следното:

  ,,за произволен функционален терм $\mu[\vv{f}_1,\dots,\vv{f}_l]$
  е изпълнено, че:
  \[\val{\mu}(\ov{\delta}_r) \sqsubseteq \evalv{\mu}.\text{''}\]
  
  Трябва да докажем, че $\texttt{Include}(r)$ е изпълнено за всяко $r$.
  Това ще направим с индукция по $r$.


  \begin{itemize}
  \item 
    Първо ще докажем $\texttt{Include}(0)$.
    Това ще направим с индукция по построението на терма $\tau$.
    Доказателството протича по същия начин ( с индукция по построението на термовете ) както доказателството в този случай на \Prop{op-name-inclusion2}.
    \begin{itemize}
    \item
      Нека $\mu \equiv \vv{a}$.
    \item
      Нека $\mu \equiv \mu_1 + \mu_2$.
      От {\bf И.П.} имаме, че за $i = 1,2$ е изпълнено
      \[\val{\mu_i}(\ov{\delta}_0) \sqsubseteq \evalv{\mu_i}.\]
      Тогава
      \begin{align*}
        \val{\mu_1 + \mu_2}(\ov{\delta}_0) & = \texttt{plus}(\val{\mu_1}(\ov{\delta}_0), \val{\mu_2}(\ov{\delta}_0))\\
                                           & \sqsubseteq \texttt{plus}(\evalv{\mu_1}, \evalv{\mu_2}) & \comment{\text{ от И.П. и мон. на }\texttt{plus}}\\
                                           & = \evalv{\mu_1 + \mu_2} & \comment{\text{ от правило }(2_+)}
      \end{align*}
    \item
      Нека $\mu \equiv \mu_1 \vv{==} \mu_2$.
    \item
      Нека $\mu \equiv \ifelse{\mu_1}{\mu_2}{\mu_3}$.
    \item
      Нека $\mu \equiv \vv{f}_i(\mu_1,\dots,\mu_{m_i})$.
    \end{itemize}
  \item 
    Нека $r > 0$. Да приемем, че $\texttt{Include}(r-1)$ е изпълнено. Ще докажем $\texttt{Include}(r)$
    с индукция по построението на термовете.
    Единственият случай, който заслужава внимание е 
    \[\mu \equiv \vv{f}_i(\mu_1,\dots,\mu_{m_i}).\]
    Доказателствата на всички останали случаи за $\mu$ протичат по абсолютно същия начин както при $r = 0$.
    
    Понеже термът $\mu$ е построен с помощта на термовете $\mu_j$, за $j = 1, \dots, m_i$,
    можем да приложим {\bf И.П.} за тях и да получим, че 
    \[b_j \dff \val{\mu_{j}}(\ov{\delta}_r) \sqsubseteq \evalv{\mu_j}.\]
    Трябва да разгледаме два случая.
    \begin{itemize}
    \item 
      Ако $\bot \not\in \{b_1,\dots,b_{m_i}\}$. Тогава, понеже работим в плоска наредба, \[\evalv{\mu_j} = b_j.\]
      От правилата на операционната семантика, това означава, че в този случай имаме следния извод:
      \begin{figure}[h!]
        \begin{prooftree}
          \AxiomC{$\vdots$}
          \UnaryInfC{$\mu_1\to^P_V b_1$}
          \AxiomC{$\cdots$}
          \AxiomC{$\vdots$}
          \UnaryInfC{$\mu_{m_i}\to^P_V b_{m_i}$}
          \AxiomC{$\vdots$}
          \UnaryInfC{$\tau_i[\vv{x}_1/\vv{b}_1,\dots,\vv{x}_{m_i}/\vv{b}_{m_i}] \to^P_V b$}
          \RightLabel{$(4_\Nat)$}
          \QuaternaryInfC{$\vv{f}_i(\mu_1,\dots,\mu_{m_i}) \to^P_V b$}
        \end{prooftree}
      \end{figure}
      
      Оттук следва, че:
      \begin{equation}
        \label{eq:14}
        \evalv{\vv{f}_i(\mu_1,\dots,\mu_{m_i})} = \evalv{\tau_i[\vv{x}_1/\vv{b}_1,\dots,\vv{x}_{m_i}/\vv{b}_{m_i}]}.
      \end{equation}
      Получаваме, че:
      \begin{align*}
        \val{\mu}(\ov{\delta}_r) & = \val{\vv{f}_i(\mu_1,\dots,\mu_{m_i})}(\ov{\delta}_r)\\
                                 & \dff \delta^i_r(\underbrace{\val{\mu_1}(\ov{\delta}_r)}_{b_1}, \dots, \underbrace{\val{\mu_{m_i}}(\ov{\delta}_r)}_{b_{m_i}}) & \comment{\ov{\delta}_r = (\delta^1_r,\dots,\delta^k_r)}\\
                                 & = \delta^i_r(b_1,\dots,b_{m_i}) \\
                                 & = \Delta_{\tau_i}(\ov{\delta}_{r-1})(b_1,\dots,b_{m_i}) & \comment{\delta^i_r = \Delta_{\tau_i}(\ov{\delta}_{r-1})}\\
                                 & = \Sigma_{\star}(\Gamma_{\tau_i}(\ov{\delta}_{r-1}))(b_1,\dots,b_{m_i}) & \comment{\Delta_{\tau_i} \dff \Sigma_{\star} \circ \Gamma_{\tau_i}}\\
                                 & = \Gamma_{\tau_i}(\ov{\delta}_{r-1})(b_1,\dots,b_{m_i}) & \comment{\text{всяко }b_j \neq \bot}\\
                                 & \dff \val{\tau_i}(\ov{\delta}_{r-1})(b_1,\dots,b_{m_i}) & \comment{\ov{\delta}_{r-1}\text{ са непрекъснати}}\\
                                 & = \val{\tau_i[\vv{x}_1/\vv{b}_1,\dots,\vv{x}_{m_i}/\vv{b}_{m_i}]}(\ov{\delta}_{r-1}) & \comment{\text{\hyperref[lem:rec:substitution]{Лема за замяната}}}\\
                                 & \sqsubseteq \evalv{\tau_i[\vv{x}_1/\vv{b}_1,\dots,\vv{x}_{m_i}/\vv{b}_{m_i}]} & \comment{\text{от }\texttt{Include}(r-1)}\\
                                 & = \evalv{\vv{f}_i(\mu_1,\dots,\mu_{m_i})} & \comment{\text{от (\ref{eq:14})}}\\
                                 & = \evalv{\mu}.
      \end{align*}
    \item
      Нека $\bot \in \{b_1,\dots,b_{m_i}\}$. Тогава е лесно, защото:
      \begin{align*}
        \val{\mu}(\ov{\delta}_r)  & = \val{\vv{f}_i(\mu_1,\dots,\mu_{m_i})}(\ov{\delta}_r)\\
                                  & \dff \delta^i_r(\underbrace{\val{\mu_1}(\ov{\delta}_r)}_{b_1}, \dots, \underbrace{\val{\mu_{m_i}}(\ov{\delta}_r)}_{b_{m_i}})  & \comment{\ov{\delta}_r = (\delta^1_r,\dots,\delta^k_r)}\\
                                  & = \delta^i_r(b_1,\dots,b_{m_i}) & \comment{b_j \dff \val{\mu_j}(\ov{\delta}_r)}\\
                                  & = \bot & \comment{\delta^i_r\text{ е точно изображение}}\\
                                  & \sqsubseteq \evalv{\vv{f}_i(\mu_1,\dots,\mu_{m_i})}\\
                                  & = \evalv{\mu}.
      \end{align*}      
    \end{itemize}
  \end{itemize}
\end{hint}

\begin{cor}
  \label{cr:rec:equivalence-cbv-inclusion2}
  Нека $\mu$ е функционален терм.
  Тогава $\evalv{\mu}$ е горна граница на веригата $(\val{\mu}(\ov{\delta}_r))^{\infty}_{r=0}$.
\end{cor}

\begin{lemma}
  За всяка рекурсивна програма $\vv{P}$,
  произволен {\em функционален} терм $\mu$,
  \[\val{\mu}(\ov{\delta}) \sqsubseteq \evalv{\mu},\]
  където $\delta = \lfp(\Delta)$, а $\Delta$ е операторът, който съответства на системата от 
  уравнения за $\vv{P}$.
\end{lemma}
\begin{hint}
  Знаем, че $\ov{\delta} = \bigsqcup_r \ov{\delta}_r$. Тогава:
  \begin{align*}
    \val{\mu}(\ov{\delta}) & = \val{\mu}(\bigsqcup_r\ov{\delta}_r)\\
                           & = \bigsqcup_r \val{\mu}(\ov{\delta}_r) & \comment{\text{от \Cor{rec:equivalence-cbv-inclusion2}}}\\
                           & \sqsubseteq \evalv{\mu}. & \comment{\text{от \Prop{rec:op-value-inclusion2}}}
  \end{align*}
\end{hint}

\begin{framed}
  \begin{thm}
    \label{th:dv-equivalent-ov}
    За всяка рекурсивна програма $\vv{P}$ на езика $\REC$ е изпълнено:
    \[\D_V\val{\vv{P}} = \O_V\val{\vv{P}}.\]
  \end{thm}  
\end{framed}
\begin{proof}
  От \Cor{ov-in-dv} знаем, че 
  \[\O_V\val{\vv{P}} \sqsubseteq \D_V\val{\vv{P}}.\]
  Остава да докажем другата посока, а именно, че 
  \[\D_V\val{\vv{P}} \sqsubseteq \O_V\val{\vv{P}}.\]
  Но това е лесно:

  \begin{align*}
    \D_V\val{\vv{P}}(\ov{a}) & \dff \delta_1(\ov{a})\\
                             & = \delta_1(\val{\vv{a}_1}(\ov{\delta}),\dots,\val{\vv{a}_{m_1}}(\ov{\delta})) & \comment{ a_j = \val{\vv{a}_j}(\ov{\delta})}\\
                             & = \val{\vv{f}_1(\vv{a}_1,\dots,\vv{a}_{m_1})}(\ov{\delta}) & \comment{\text{ деф. на стойност на терм}}\\
                             % & = \Delta_{\tau_1}(\ov{\delta})(\ov{a}) & \comment{\delta_1 = \Delta_{\tau_1}(\ov{\delta})}\\
                             % & = \Sigma_{m_1}(\Gamma_{\tau_1}(\ov{\delta}))(\ov{a}) & \comment{\Delta_{\tau_1} \dff \Sigma_{m_1}\circ\Gamma_{\tau_1}}\\
                             % & = \Gamma_{\tau_1}(\ov{\delta})(\ov{a}) &  \comment{\text{защото }a_j \neq \bot}\\
                             % & \dff \val{\tau_1}(\bar{a},\bar{\delta}) \\
                             % & = \val{\tau_1[\varsx/\ov{\vv{a}}]}(\ov{\delta}) & \comment{\text{от \hyperref[lem:rec:substitution]{Лема за замяната}}}\\
                             & \sqsubseteq \evalv{\vv{f}_1(\vv{a}_1,\dots\vv{a}_{m_1})} & \comment{\text{от \Prop{rec:op-value-inclusion2}}}\\
                             & \dff \O_V\val{\vv{P}}(\ov{a}).
  \end{align*}

  
  
  % \begin{itemize}
  % \item 
  %   Ако $\bot \not\in\{a_1,\dots,a_{m_0}\}$, то 
  %   \begin{align*}
  %     \D_V\val{\vv{P}}(\ov{a}) & \dff \delta_0(\ov{a})\\
  %                               & = \Delta_{\tau_1}(\ov{\delta})(\ov{a}) & \comment{\delta_1 = \Delta_{\tau_1}(\ov{\delta})}\\
  %                               & = \Sigma_{m_1}(\Gamma_{\tau_1}(\ov{\delta}))(\ov{a}) & \comment{\Delta_{\tau_1} \dff \Sigma_{m_1}\circ\Gamma_{\tau_1}}\\
  %                               & = \Gamma_{\tau_1}(\ov{\delta})(\ov{a}) &  \comment{\text{защото }a_j \neq \bot}\\
  %                               & \dff \val{\tau_1}(\bar{a},\bar{\delta}) \\
  %                               & = \val{\tau_1[\varsx/\ov{\vv{a}}]}(\ov{\delta}) & \comment{\text{от \hyperref[lem:rec:substitution]{Лема за замяната}}}\\
  %                               & \sqsubseteq \evalv{\tau_1[\varsx/\ov{\vv{a}}]} & \comment{\text{от \Prop{rec:op-value-inclusion2}}}\\
  %                               & = \evalv{\vv{f}_1(\vv{a}_1,\dots,\vv{a}_{m_1})} & \comment{\text{правило }(4_\Nat)}\\
  %                               & \dff \O_V\val{\vv{P}}(\ov{a}).
  %   \end{align*}
  % \item
  %   Ако $\bot \in \{a_1,\dots,a_{m_1}\}$, то в този случай, 
  %   \begin{align*}
  %     \D_V\val{\vv{P}}(\ov{a}) & \dff \delta_1(\ov{a})\\
  %                              & = \bot & \comment{\delta_1 \text{ е точно изображение}}\\
  %                              & \sqsubseteq \O_V\val{\vv{P}}(\ov{a}).
  %   \end{align*}
  % \end{itemize}
\end{proof}


%%% Local Variables:
%%% mode: latex
%%% TeX-master: "../sep"
%%% End:

% \newpage
% \section{Задачи}

\begin{problem}
  % \label{pr:op-name-monotone}
  \Stefan{Да се премести при задачите}
  Нека с всеки терм $\tau$ да асоциираме изображението $f_\tau:\Nat^n_\bot \to \Nat_\bot$,
  където 
  \[f_\tau(\ov{a}) \dff \evaln{\tau[\varsx/\ov{\vv{a}}]}.\]
  Докажете, че:
  \begin{itemize}
  \item 
    $f_\tau$ е функция;
  \item
    $f_\tau \in \Mon{\Nat^n_\bot}{\Nat_\bot}$;
  \item
    $f_\tau \in \Cont{\Nat^n_\bot}{\Nat_\bot}$.
  \end{itemize}

\end{problem}
% \begin{hint}
%   \marginpar{\writedown Домашно!}
%   От \Cor{monotone-is-continuous} достатъчно е да проверим, че $\evaln\tau \in \Mon{\Nat^n_\bot}{\Nat_\bot}$.
%   Трябва да проверим дали за всяко $\bar{a},\bar{b} \in \Nat^n_\bot$,
%   \[\bar{a} \sqsubseteq \bar{b} \implies \evaln\tau(\bar{a}) \sqsubseteq \evaln\tau(\bar{b}).\]
%   Индукция по дължината на извода.
% \end{hint}

\begin{problem}
  \marginpar{Лема за симулацията \cite[стр. 177]{ditchev-soskov}}
  \Stefan{Да се премести при задачи}
  Да разгледаме една рекурсивна програма $\vv{P}[\vv{x}_1,\dots,\vv{x}_n,\varsf]$.
  Нека $\vv{y}_1,\dots,\vv{y}_k$ са обектови променливи, различни от тези в $\vv{P}$.
  Нека $\rho[\vv{y}_1,\dots,\vv{y}_k, \varsf]$ е произволен терм.
  Докажете, че
  \marginpar{Когато слагаме $\vdots$ означаваме, че не знаем колко дълъг е извода. Една линия означава, че имаме директен извод. 
    Възможно е някои $c_j = \bot$}
  \label{pr:simulation}
  \begin{figure}[h!]
    \begin{prooftree}
      \AxiomC{$\mu_1\to^P_N c_1$}
      \AxiomC{$\cdots$}
      \AxiomC{$\mu_{k}\to^P_N c_{k}$}
      \AxiomC{$\rho[\vv{y}_1/\vv{c}_1,\dots,\vv{y}_{k}/\vv{c}_{k}] \to^P_N b$}
      \QuaternaryInfC{$\vdots$}
      \UnaryInfC{$\rho[\vv{y}_1/\mu_1,\dots,\vv{y}_{m_i}/\mu_{m_i}] \to^P_N b$}
    \end{prooftree}
  \end{figure}
  
  Оттук заключете, че 
  \[\O_V\val{\vv{P}} \sqsubseteq \O_N\val{\vv{P}}.\]
\end{problem}
% \begin{proof}
%   Индукция по дължината на извода на \[\rho[\vv{y}_1/\vv{c}_1,\dots,\vv{y}_{k}/\vv{c}_{k}] \to^P_N b.\]
%   Да приемем, че твърдението е вярно за дължини на извода $< l$.
%   Ще докажем, че то е вярно за дължина на извода $l$.
%   За целта ще разгледаме вида на терма $\rho$.
%   \begin{itemize}
%   \item
%     Нека $\rho \equiv \vv{c}$. Единствената възможност тук е $c = b$.
%   \item
%     Нека $\rho \equiv \vv{y}_i$. Този случай е ясен, защото 
%     $\rho[\bar{\vv{y}}/\bar{\mu}] \equiv \mu_i$ и е очевидно, че
%     \begin{prooftree}
%       \AxiomC{$\mu_i \to^P_N c_i$}
%       \UnaryInfC{$\rho[\bar{\vv{y}}/\bar{\mu}] \to^P_N c_i$}
%     \end{prooftree}
%     Тук също имаме, че $c_i = b$.
%   \item
%     Нека $\rho \equiv \rho_1\ \op\ \rho_2$. Тогава ако $\rho[\vv{y}_1/\vv{c}_1,\dots,\vv{y}_{k}/\vv{c}_{k}] \to^P_N b$ е извод с дължина $l$, то той е получен по правилото:
%     \begin{prooftree}
%       \AxiomC{$\vdots$}
%       \LeftLabel{\scriptsize{(извод с дълж. $l_1$)}}
%       \UnaryInfC{$\rho_1[\bar{\vv{y}}/\bar{\vv{c}}]\to^P_N b_1$}
%       \AxiomC{$\vdots$}
%       \RightLabel{\scriptsize{(извод с дълж. $l_2$)}}
%       \UnaryInfC{$\rho_2[\bar{\vv{y}}/\bar{\vv{c}}] \to^P_N b_2$}
%       \RightLabel{$(2_\op)$}
%       \BinaryInfC{$\rho[\bar{\vv{y}}/\bar{\vv{c}}] \to^P_N b$}
%     \end{prooftree}
%     Тук имаме, че $l = l_1 + l_2 + 1$ и и $b = b_1 op^\star b_2$. Следователно можем да приложим индукционното предположение.
%     Получаваме, че:

%     \begin{prooftree}
%       \AxiomC{$\mu_1\to^P_N c_1$}
%       \AxiomC{$\cdots$}
%       \AxiomC{$\mu_{k}\to^P_N c_{k}$}
%       \AxiomC{$\vdots$}
%       \RightLabel{\scriptsize{(дълж. $< l$)}}
%       \UnaryInfC{$\rho_1[\bar{\vv{y}}/\bar{\vv{c}}]\to^P_N b_1$}
%       % \doubleLine
%       \QuaternaryInfC{$\vdots$}
%       \RightLabel{\scriptsize{(\bf{И.П.})}}
%       \UnaryInfC{$\rho_1[\bar{\vv{y}}/\bar{\mu}] \to^P_N b_1$}
%     \end{prooftree}
%     Аналогично получаваме, че:
%     \begin{prooftree}
%       \AxiomC{$\mu_1\to^P_N c_1$}
%       \AxiomC{$\cdots$}
%       \AxiomC{$\mu_{k}\to^P_N c_{k}$}
%       \AxiomC{$\vdots$}
%       \RightLabel{\scriptsize{(дълж. $< l$)}}
%       \UnaryInfC{$\rho_2[\bar{\vv{y}}/\bar{\vv{c}}]\to^P_N b_2$}
%       % \doubleLine
%       \QuaternaryInfC{$\vdots$}
%       \RightLabel{\scriptsize{(\bf{И.П.})}}
%       \UnaryInfC{$\rho_2[\bar{\vv{y}}/\bar{\mu}] \to^P_N b_2$}
%     \end{prooftree}
    
%     Накрая прилагаме правило $(2_\op)$ и завършваме:

%     \begin{prooftree}
%       \AxiomC{$\rho_1[\bar{\vv{y}}/\bar{\mu}]\to^P_N b_1$}
%       \AxiomC{$\rho_2[\bar{\vv{y}}/\bar{\mu}]\to^P_N b_2$}
%       \RightLabel{$(2_\op)$}
%       \BinaryInfC{$\underbrace{\rho_1[\bar{\vv{y}}/\bar{\mu}]\ \op^\star\ \rho_2[\bar{\vv{y}}/\bar{\mu}]}_{\rho[\bar{\vv{y}}/\bar{\mu}]} \to^P_N b$}
%     \end{prooftree}
    
%   \item
%     Нека $\rho \equiv \ifelse{\tau_0}{\tau_1}{\tau_2}$. Тогава ако $\rho[\vv{y}_1/\vv{c}_1,\dots,\vv{y}_{k}/\vv{c}_{k}] \to^P_N b$ е извод с дължина $l$, то той е получен по правилото:
%     \begin{prooftree}
%       \AxiomC{$\vdots$}
%       \LeftLabel{\scriptsize{(извод с дълж. $l_0$)}}
%       \UnaryInfC{$\tau_0[\bar{\vv{y}}/\bar{\vv{c}}]\to^P_N 0$}
%       \AxiomC{$\vdots$}
%       \RightLabel{\scriptsize{(извод с дълж. $l_2$)}}
%       \UnaryInfC{$\tau_2[\bar{\vv{y}}/\bar{\vv{c}}] \to^P_N b$}
%       \RightLabel{$(3_\false)$}
%       \BinaryInfC{$\rho[\bar{\vv{y}}/\bar{\vv{c}}] \to^P_N b$}
%     \end{prooftree}
%     Тук имаме, че $l = l_0 + l_2 + 1$ и следователно можем да приложим индукционното предположение. 
%     Така получаваме, че:
    
%     \begin{prooftree}
%       \AxiomC{$\mu_1\to^P_N c_1$}
%       \AxiomC{$\cdots$}
%       \AxiomC{$\mu_{k}\to^P_N c_{k}$}
%       \AxiomC{$\vdots$}
%       \RightLabel{\scriptsize{(дълж. $< l$)}}
%       \UnaryInfC{$\tau_0[\bar{\vv{y}}/\bar{\vv{c}}]\to^P_N 0$}
%       % \doubleLine
%       \QuaternaryInfC{$\vdots$}
%       \RightLabel{\scriptsize{(\bf{И.П.})}}
%       \UnaryInfC{$\tau_0[\bar{\vv{y}}/\bar{\mu}] \to^P_N 0$}
%     \end{prooftree}

%     Аналогично получаваме, че:

%     \begin{prooftree}
%       \AxiomC{$\mu_1\to^P_N c_1$}
%       \AxiomC{$\cdots$}
%       \AxiomC{$\mu_{k}\to^P_N c_{k}$}
%       \AxiomC{$\vdots$}
%       \RightLabel{\scriptsize{(дълж. $< l$)}}
%       \UnaryInfC{$\tau_2[\bar{\vv{y}}/\bar{\vv{c}}]\to^P_N b$}
%       % \doubleLine
%       \QuaternaryInfC{$\vdots$}
%       \RightLabel{\scriptsize{(\bf{И.П.})}}
%       \UnaryInfC{$\tau_2[\bar{\vv{y}}/\bar{\mu}] \to^P_N b$}
%     \end{prooftree}

%     Накрая прилагаме правило $(3_\false)$ и завършваме:

%     \begin{prooftree}
%       \AxiomC{$\tau_0[\bar{\vv{y}}/\bar{\mu}]\to^P_N 0$}
%       \AxiomC{$\tau_2[\bar{\vv{y}}/\bar{\mu}]\to^P_N b$}
%       \RightLabel{$(3_\false)$}
%       \BinaryInfC{$\underbrace{\ifelse{\tau_0[\bar{\vv{y}}/\bar{\mu}]}{\tau_1[\bar{\vv{y}}/\bar{\mu}]}{\tau_2[\bar{\vv{y}}/\bar{\mu}]}}_{\rho[\bar{\vv{y}}/\bar{\mu}]} \to^P_N b$}
%       % \BinaryInfC{$\rho[\bar{\vv{y}}/\bar{\mu}] \to^P_N b$}
%     \end{prooftree}
    
%     Случаят, когато $\rho[\vv{y}_1/\vv{c}_1,\dots,\vv{y}_{k}/\vv{c}_{k}] \to^P_N b$ е получен по правилото $(3_\true)$ е аналогичен.
%   \item 
%     Нека $\rho \equiv \vv{f}_i(\rho_1,\dots,\rho_{m_i})$.
%     Да отбележим най-напред, че 
%     \begin{align*}
%       \rho[\vv{y}_1/\vv{c}_1,\dots,\vv{y}_{k}/\vv{c}_{k}] & \equiv \vv{f}_i(\rho_1,\dots,\rho_{m_i})[\vv{y}_1/\vv{c}_1,\dots,\vv{y}_{k}/\vv{c}_{k}]\\
%       & \equiv \vv{f}_i(\rho_1[\bar{\vv{y}}/\bar{\vv{c}}],\dots,\rho_{m_i}[\bar{\vv{y}}/\bar{\vv{c}}]).
%     \end{align*}

%     Щом $\rho[\vv{y}_1/\vv{c}_1,\dots,\vv{y}_{k}/\vv{c}_{k}] \to^P_N b$ е извод с дължина $l$, то
%     от правилата на операционна семантика с предаване на параметрите по има имаме, че     
%     \begin{prooftree}
%       \AxiomC{$\vdots$}
%       \RightLabel{\scriptsize{Извод с дълж. $(l-1)$}}
%       \UnaryInfC{$\tau_i[\vv{x}_1/\rho_1[\bar{\vv{y}}/\bar{\vv{c}}],\dots,\vv{x}_{m_i}/\rho_{m_i}[\bar{\vv{y}}/\bar{\vv{c}}]] \to^P_N b$}
%       \RightLabel{\scriptsize($4'$)}
%       \UnaryInfC{$\underbrace{\vv{f}_i(\rho_1[\bar{\vv{y}}/\bar{\vv{c}}],\dots,\rho_{m_i}[\bar{\vv{y}}/\bar{\vv{c}}])}_{\rho[\bar{\vv{y}}/\bar{\vv{c}}]} \to^P_N b$}
%     \end{prooftree}

%     Да отбележим, че имаме тъждествата:
%     \begin{align*}
%       \tau_i[\vv{x}_1/\rho_1,\dots,\vv{x}_{m_i}/\rho_{m_i}][\bar{\vv{y}}/\bar{\vv{c}}] & \equiv \tau_i[\vv{x}_1/\rho_1[\bar{\vv{y}}/\bar{\vv{c}}],\dots,\vv{x}_{m_i}/\rho_{m_i}[\bar{\vv{y}}/\bar{\vv{c}}]]\\
%       \tau_i[\vv{x}_1/\rho_1,\dots,\vv{x}_{m_i}/\rho_{m_i}][\bar{\vv{y}}/\bar{\mu}] & \equiv \tau_i[\vv{x}_1/\rho_1[\bar{\vv{y}}/\bar{\mu}],\dots,\vv{x}_{m_i}/\rho_{m_i}[\bar{\vv{y}}/\bar{\mu}]].
%     \end{align*}

%     Получаваме, че
%     \begin{prooftree}
%       \AxiomC{$\mu_1\to^P_N c_1$}
%       \AxiomC{$\cdots$}
%       \AxiomC{$\mu_{k}\to^P_N c_{k}$}
%       \AxiomC{$\vdots$}
%       \RightLabel{\scriptsize{Извод с дълж. $(l-1)$}}
%       \UnaryInfC{$\tau_i[\vv{x}_1/\rho_1,\dots,\vv{x}_{m_i}/\rho_{m_i}][\bar{\vv{y}}/\bar{\vv{c}}] \to^P_N b$}
%       \QuaternaryInfC{$\vdots$}
%       \RightLabel{\scriptsize{({\bf И.П.})}}
%       % \doubleLine
%       \UnaryInfC{$\tau_i[\vv{x}_1/\rho_1[\bar{\vv{y}}/\bar{\mu}],\dots,\vv{x}_{m_i}/\rho_{m_i}[\bar{\vv{y}}/\bar{\mu}]] \to^P_N b$}
%       \RightLabel{\scriptsize{$(4')$}}
%       \UnaryInfC{$\underbrace{\vv{f}_i(\rho_1[\bar{\vv{y}}/\bar{\mu}],\dots,\rho_{m_i}[\bar{\vv{y}}/\bar{\mu}])}_{\rho[\bar{\vv{y}}/\bar{\mu}]} \to^P_N b$}
%     \end{prooftree}
%   \end{itemize}
% \end{proof}

% \begin{prop}
%   \label{pr:op-name-function}
%   За произволна декларация $P$ и {\em функционален} терм $\tau$,
%   \[\tau \to^P_N n_1\ \&\ \tau \to^P_N n_2\ \implies\ n_1 = n_2.\]
% \end{prop}
% \begin{hint}
%   \marginpar{\writedown Домашно!}
%   Доказателство с индукция по извода.
% \end{hint}


% \begin{prop}
%   \label{pr:op-name-inclusion2}
%   Нека разгледаме една декларация \vv{P} в езика $\REC$ и произволен {\em функционален} терм $\mu[\vv{f}_1\dots,\vv{f}_k]$.
%   Тогава 
%   \[\val{\mu}(\evaln{\tau_1},\dots,\evaln{\tau_k}) \sqsubseteq \evaln{\mu}.\]
% \end{prop}
% \begin{proof}
%   Нека за улеснение да положим $\psi_i \dff \evaln{\tau_i}$.
%   Ще докажем с индукция по построението на функционалния терм $\mu$, че за всяко $b \in \Nat$,
%   \marginpar{Ако $b = \bot$, то е тривиално}
%   \[\val{\mu}(\psi_1,\dots,\psi_k) = b \implies \evalv{\mu} = b.\]
%   \begin{itemize}
%   \item
%     Нека $\mu \equiv \vv{b}$.
%     Тогава $\val{\vv{b}}(\bar{\psi}) = b$.
%     От правилата на операционната семантика имаме, че:
%     \begin{prooftree}
%       \AxiomC{$$}
%       \RightLabel{$(1)$}
%       \UnaryInfC{$\vv{b} \to^P_N b$}
%     \end{prooftree}
%     Според нашите означения това означава, че $\evaln{\mu} = b$.
%   \item
%     Нека $\mu \equiv \mu_1\ \op\ \mu_2$. Тогава
    
%     \[\val{\mu}(\bar{\psi}) = \underbrace{\val{\mu_1}(\bar{\psi})}_{b_1\neq \bot}\ op^\star\ \underbrace{\val{\mu_2}(\bar{\psi})}_{b_2\neq\bot} = b\neq \bot.\]
%     Понеже $op^\star$ е точно изображение и $b \in \Nat$, то $b_1,b_2 \in \Nat$.
%     Тогава от {\bf И.П.}, $\evaln{\mu_1} = b_1$ и $\evaln{\mu_2} = b_2$, и
%     \begin{prooftree}
%       \AxiomC{$\mu_1\to^P_N b_1$}
%       \AxiomC{$\mu_2\to^P_N b_2$}
%       \AxiomC{$b = b_1\ op^\star\ b_2$}
%       \RightLabel{$(2_\op)$}
%       \TrinaryInfC{$\mu_1\ \op\ \mu_2 \to^P_N b$}
%     \end{prooftree}
%     Заключаваме, че в този случай, \[\evalv{\mu} = b.\]
%   \item
%     Случаят, когато $\mu \equiv \ifelse{\mu_0}{\mu_1}{\mu_2}$, не крие изненади 
%     и е трудно да не го пропуснем.
%   \item
%     Нека $\mu \equiv \vv{f}_i(\mu_1,\dots,\mu_{m_i})$ и да приемем, че $\val{\mu}(\bar{\psi}) = b \neq \bot$.
%     Ще докажем, че $\evaln{\mu} = b$. Имаме, че 
%     \begin{align*}
%       \val{\mu}(\bar{\psi}) & = \psi_i(\underbrace{\val{\mu_1}(\bar{\psi})}_{c_1},\dots,\underbrace{\val{\mu_{m_i}}(\bar{\psi})}_{c_{m_i}}) & (\text{стойност на терм })\\
%       & = \psi(c_1,\dots,c_{m_i}) & (\text{може някои }c_j = \bot)\\
%       & = \evaln{\tau_i}(c_1,\dots,c_{m_i}) & (\text{от деф. на }\psi_i)\\
%       & = b \neq \bot,
%     \end{align*}
%     откъдето следва, че 
%     \[\tau_i[\varsx/\bar{\vv{c}}] \to^P_V b.\]

%     Сега от {\bf И.П.} имаме, че за всяко $j = 1,\dots,m_i$
%     \[c_j \dff \val{\mu_j}(\bar{\psi}) \sqsubseteq \evaln{\mu_j} \dff d_j.\]
%     Ясно е, че $\bar{c} \sqsubseteq \bar{d}$.
%     От \Prop{op-name-monotone} имаме, че $\evaln{\tau_i}$ е монотонно изображение, откъдето следва, че
%     \[\bot \neq b = \evaln{\tau_i}(\bar{c}) \sqsubseteq \evaln{\tau_i}(\bar{d}) = b.\]

%     \begin{prooftree}
%       \AxiomC{$\mu_1\to^P_N d_1$}
%       \AxiomC{$\cdots$}
%       \AxiomC{$\mu_{m_i}\to^P_N d_{m_i}$}
%       \AxiomC{$\tau_i[\varsx/\bar{\vv{d}}] \to^P_N b$}
%       \QuaternaryInfC{$\vdots$}
%       \RightLabel{\scriptsize(\Prop{simulation})}
%       % \doubleLine
%       \UnaryInfC{$\tau_i[\varsx/\bar{\mu}] \to^P_N b$}
%       \RightLabel{\scriptsize{$(4')$}}
%       \UnaryInfC{$\vv{f}_i(\mu_1,\dots,\mu_{m_i}) \to^P_N b$}
%     \end{prooftree}
    
%     Заключаваме, че $\evaln{\mu} = b$.
%     \end{itemize}
% \end{proof}

% \begin{cor}
%   \label{cr:prefixed-point-name}
%   Нека е дадена една декларация $\vv{P}$ в езика $\REC$
%   и $\bar{\gamma} = \lfp(\Gamma)$, където $\Gamma = \Gamma_{\tau_1}\times\cdots\times \Gamma_{\tau_n}$
%   е операторът, който съответства на декларацията $\vv{P}$.
%   Тогава за всяко $i = 1,\dots,n$,
%   \[\gamma_i \sqsubseteq \evaln{\tau_i}.\]
% \end{cor}
% \begin{proof}
%   Нека отново за улеснение да означим $\psi_i \dff \evaln{\tau_i}$.
%   Да видим първо, че $\Gamma_{\tau_i}(\bar{\psi}) \sqsubseteq \psi_i$.
%   \begin{align*}
%     \Gamma_{\tau_i}(\bar{\psi})(\bar{a}) & = \val{\tau_i}(\bar{a},\bar{\psi}) & (\text{от деф. на }\Gamma_{\tau_i})\\
%     & = \val{\tau_i[\bar{x}/\bar{\vv{a}}]}(\bar{\psi}) & (\text{от \hyperref[lem:rec:substitution]{Лема за замяната}})\\
%     & \sqsubseteq \evaln{\tau_i[\bar{x}/\bar{\vv{a}}]} & (\text{от \Prop{op-name-inclusion2}})\\
%     & = \psi_i(\bar{a}) & (\text{от деф. на }\psi_i).
%   \end{align*}
%   Така получихме, че $\Gamma(\bar{\psi}) \sqsubseteq \bar{\psi}$, 
%   т.е. $\bar\psi \in \texttt{Pref}(\Gamma)$.
%   Понеже $\bar{\gamma} = \lfp(\Gamma)$, то от \Prop{prefix-point}
%   следва, че $\bar{\gamma} \sqsubseteq \bar{\psi}$.
% \end{proof}

\begin{problem}
  % \label{pr:op-name-inclusion2}
  Нека разгледаме една декларация \vv{P} в езика $\REC$ и произволен {\em функционален} терм $\mu[\vv{f}_1\dots,\vv{f}_k]$.
  Докажете, че
  \[\val{\mu}(\evaln{\tau_1},\dots,\evaln{\tau_k}) \sqsubseteq \evaln{\mu}.\]
\end{problem}
% \begin{proof}
%   Нека за улеснение да положим $\psi_i \dff \evaln{\tau_i}$.
%   Ще докажем с индукция по построението на функционалния терм $\mu$, че за всяко $b \in \Nat$,
%   \marginpar{Ако $b = \bot$, то е тривиално}
%   \[\val{\mu}(\psi_1,\dots,\psi_k) = b \implies \evalv{\mu} = b.\]
%   \begin{itemize}
%   \item
%     Нека $\mu \equiv \vv{b}$.
%     Тогава $\val{\vv{b}}(\bar{\psi}) = b$.
%     От правилата на операционната семантика имаме, че:
%     \begin{prooftree}
%       \AxiomC{$$}
%       \RightLabel{$(1)$}
%       \UnaryInfC{$\vv{b} \to^P_N b$}
%     \end{prooftree}
%     Според нашите означения това означава, че $\evaln{\mu} = b$.
%   \item
%     Нека $\mu \equiv \mu_1\ \op\ \mu_2$. Тогава
    
%     \[\val{\mu}(\bar{\psi}) = \underbrace{\val{\mu_1}(\bar{\psi})}_{b_1\neq \bot}\ op^\star\ \underbrace{\val{\mu_2}(\bar{\psi})}_{b_2\neq\bot} = b\neq \bot.\]
%     Понеже $op^\star$ е точно изображение и $b \in \Nat$, то $b_1,b_2 \in \Nat$.
%     Тогава от {\bf И.П.}, $\evaln{\mu_1} = b_1$ и $\evaln{\mu_2} = b_2$, и
%     \begin{prooftree}
%       \AxiomC{$\mu_1\to^P_N b_1$}
%       \AxiomC{$\mu_2\to^P_N b_2$}
%       \AxiomC{$b = b_1\ op^\star\ b_2$}
%       \RightLabel{$(2_\op)$}
%       \TrinaryInfC{$\mu_1\ \op\ \mu_2 \to^P_N b$}
%     \end{prooftree}
%     Заключаваме, че в този случай, \[\evalv{\mu} = b.\]
%   \item
%     Случаят, когато $\mu \equiv \ifelse{\mu_0}{\mu_1}{\mu_2}$, не крие изненади 
%     и е трудно да не го пропуснем.
%   \item
%     Нека $\mu \equiv \vv{f}_i(\mu_1,\dots,\mu_{m_i})$ и да приемем, че $\val{\mu}(\bar{\psi}) = b \neq \bot$.
%     Ще докажем, че $\evaln{\mu} = b$. Имаме, че 
%     \begin{align*}
%       \val{\mu}(\bar{\psi}) & = \psi_i(\underbrace{\val{\mu_1}(\bar{\psi})}_{c_1},\dots,\underbrace{\val{\mu_{m_i}}(\bar{\psi})}_{c_{m_i}}) & (\text{стойност на терм })\\
%       & = \psi(c_1,\dots,c_{m_i}) & (\text{може някои }c_j = \bot)\\
%       & = \evaln{\tau_i}(c_1,\dots,c_{m_i}) & (\text{от деф. на }\psi_i)\\
%       & = b \neq \bot,
%     \end{align*}
%     откъдето следва, че 
%     \[\tau_i[\varsx/\bar{\vv{c}}] \to^P_V b.\]

%     Сега от {\bf И.П.} имаме, че за всяко $j = 1,\dots,m_i$
%     \[c_j \dff \val{\mu_j}(\bar{\psi}) \sqsubseteq \evaln{\mu_j} \dff d_j.\]
%     Ясно е, че $\bar{c} \sqsubseteq \bar{d}$.
%     От \Prop{op-name-monotone} имаме, че $\evaln{\tau_i}$ е монотонно изображение, откъдето следва, че
%     \[\bot \neq b = \evaln{\tau_i}(\bar{c}) \sqsubseteq \evaln{\tau_i}(\bar{d}) = b.\]

%     \begin{prooftree}
%       \AxiomC{$\mu_1\to^P_N d_1$}
%       \AxiomC{$\cdots$}
%       \AxiomC{$\mu_{m_i}\to^P_N d_{m_i}$}
%       \AxiomC{$\tau_i[\varsx/\bar{\vv{d}}] \to^P_N b$}
%       \QuaternaryInfC{$\vdots$}
%       \RightLabel{\scriptsize(\Prop{simulation})}
%       % \doubleLine
%       \UnaryInfC{$\tau_i[\varsx/\bar{\mu}] \to^P_N b$}
%       \RightLabel{\scriptsize{$(4')$}}
%       \UnaryInfC{$\vv{f}_i(\mu_1,\dots,\mu_{m_i}) \to^P_N b$}
%     \end{prooftree}
    
%     Заключаваме, че $\evaln{\mu} = b$.
%     \end{itemize}
% \end{proof}

% \begin{cor}
%   \label{cr:prefixed-point-name}
%   Нека е дадена една декларация $\vv{P}$ в езика $\REC$
%   и $\bar{\gamma} = \lfp(\Gamma)$, където $\Gamma = \Gamma_{\tau_1}\times\cdots\times \Gamma_{\tau_n}$
%   е операторът, който съответства на декларацията $\vv{P}$.
%   Тогава за всяко $i = 1,\dots,n$,
%   \[\gamma_i \sqsubseteq \evaln{\tau_i}.\]
% \end{cor}
% \begin{proof}
%   Нека отново за улеснение да означим $\psi_i \dff \evaln{\tau_i}$.
%   Да видим първо, че $\Gamma_{\tau_i}(\bar{\psi}) \sqsubseteq \psi_i$.
%   \begin{align*}
%     \Gamma_{\tau_i}(\bar{\psi})(\bar{a}) & = \val{\tau_i}(\bar{a},\bar{\psi}) & (\text{от деф. на }\Gamma_{\tau_i})\\
%     & = \val{\tau_i[\bar{x}/\bar{\vv{a}}]}(\bar{\psi}) & (\text{от \hyperref[lem:rec:substitution]{Лема за замяната}})\\
%     & \sqsubseteq \evaln{\tau_i[\bar{x}/\bar{\vv{a}}]} & (\text{от \Prop{op-name-inclusion2}})\\
%     & = \psi_i(\bar{a}) & (\text{от деф. на }\psi_i).
%   \end{align*}
%   Така получихме, че $\Gamma(\bar{\psi}) \sqsubseteq \bar{\psi}$, 
%   т.е. $\bar\psi \in \texttt{Pref}(\Gamma)$.
%   Понеже $\bar{\gamma} = \lfp(\Gamma)$, то от \Prop{prefix-point}
%   следва, че $\bar{\gamma} \sqsubseteq \bar{\psi}$.
% \end{proof}



\begin{problem}
  За произволна декларация $P$ и {\em функционален} терм $\tau$,
  \[\tau \to^P_V n_1\ \&\ \tau \to^P_V n_2 \implies n_1 = n_2.\]
\end{problem}
\begin{hint}
  \marginpar{\writedown Домашно!}
  Индукция по дължината на извода.
\end{hint}

\begin{problem}
  \label{pr:equiv-value1}
  Нека разгледаме една деклрация \vv{P} в езика $\REC$ и произволен {\em функционален} терм $\mu[\vv{f}_0,\dots,\vv{f}_k]$.
  Докажете, че 
  \marginpar{\cite[стр. 151]{winskel}, \cite[стр. 183]{ditchev-soskov}}
  \marginpar{Тук има леки разлики}
  \[\val{\mu}(\evalv{\vv{f}_0(\vv{x}_1,\dots,\vv{x}_{m_0})},\dots,\evalv{\vv{f}_k(\vv{x}_1,\dots,\vv{x}_{m_k})}, \ov{\delta}) \sqsubseteq \evalv{\mu}.\]
\end{problem}
% \begin{proof}
%   Нека за улеснение да положим 
%   \[\psi_i \dff \evalv{\vv{f}_i(\vv{x}_1,\dots,\vv{x}_{m_i})}.\]
%   Ще докажем с индукция по построението на функционалния терм $\mu$, че за всяко $b \in \Nat$,
%   \marginpar{Ако $b = \bot$, то е тривиално}
%   \[\val{\mu}(\psi_1,\dots,\psi_k) = b \implies \evalv{\mu} = b.\]
%   \begin{itemize}
%   \item
%     Нека $\mu \equiv \vv{b}$.
%     Тогава $\val{\vv{b}}(\bar{\psi}) = b$.
%     От правилата на операционната семантика имаме, че:
%     \begin{prooftree}
%       \AxiomC{$$}
%       \RightLabel{$(1)$}
%       \UnaryInfC{$\vv{b} \to^P_V b$}
%     \end{prooftree}
%     Според нашите означения това означава, че $\evalv{\mu} = b$.
%   \item
%     Нека $\mu \equiv \mu_1\ \op\ \mu_2$. Тогава
    
%     \[\val{\mu_1\ \op\ \mu_2}(\bar{\psi}) = \underbrace{\val{\mu_1}(\bar{\psi})}_{b_1\neq \bot}\ op^\star\ \underbrace{\val{\mu_2}(\bar{\psi})}_{b_2\neq\bot} = b\neq \bot.\]
%     Понеже $op^\star$ е точно изображение и $b \in \Nat$, то $b_1,b_2 \in \Nat$.
%     Тогава от {\bf И.П.}, $\evalv{\mu_1} = b_1$ и $\evalv{\mu_2} = b_2$, и
%     \begin{prooftree}
%       \AxiomC{$\mu_1\to^P_V b_1$}
%       \AxiomC{$\mu_2\to^P_V b_2$}
%       \AxiomC{$b = b_1\ op^\star\ b_2$}
%       \RightLabel{$(2_\op)$}
%       \TrinaryInfC{$\mu_1\ \op\ \mu_2 \to^P_V b$}
%     \end{prooftree}
%     Заключаваме, че в този случай, \[\evalv{\mu} = b.\]
%   \item
%     Случаят, когато $\mu \equiv \ifelse{\pi}{\mu_1}{\mu_2}$, не крие изненади 
%     и е трудно да не го пропуснем.
%   \item
%     Нека $\mu \equiv \vv{f}_i(\mu_1,\dots,\mu_{m_i})$ и да приемем, че $\val{\mu}(\bar{\psi}) = b \neq \bot$.
%     % защото ако $\val{\mu}(\bar{\psi}) = \bot$, то е очевидно, че 
%     % $\val{\mu}(\bar{\psi}) \sqsubseteq \evalv{\mu}$.
%     Ще докажем, че $\evalv{\mu} = b$. Имаме, че 
%     \begin{align*}
%       \val{\mu}(\ov{\psi}) & = \val{\vv{f}_i(\mu_1,\dots,\mu_{m_i})}(\bar{\psi})\\
%                            & = \psi_i(\underbrace{\val{\mu_1}(\bar{\psi})}_{c_1},\dots,\underbrace{\val{\mu_{m_i}}(\bar{\psi})}_{c_{m_i}}) & (\text{стойност на терм})\\
%                            & = \psi_i(c_1,\dots,c_{m_i}) \\
%                            & = \evalv{\vv{f}_i(\vv{x}_1,\dots,\vv{x}_{m_i})}(c_1,\dots,c_{m_i}) & (\text{от деф. на }\psi_i)\\
%                            & = \evalv{\vv{f}_i(\vv{c}_1,\dots,\vv{c}_{m_i})} \\
%                            & = b \neq \bot.
%     \end{align*}
%     Получихме, че $\vv{f}_i(\vv{c}_1,\dots,\vv{c}_{m_i}) \to^P_V b$.

%     Единственото правило, по което можем да стигнем до този извод е правилото $(4_\Nat)$, т.е.
%     \marginpar{Константите $\vv{c}_j$ са едни много хубави термове}
%     \begin{prooftree}
%       \AxiomC{$\vv{c}_1\to^P_V c_1$}
%       \AxiomC{$\cdots$}
%       \AxiomC{$\vv{c}_{m_i}\to^P_V c_{m_i}$}
%       \AxiomC{$\tau_i[\varsx/\bar{\vv{c}}] \to^P_V b$}
%       \RightLabel{\scriptsize($4_\Nat$)}
%       \QuaternaryInfC{$\vv{f}_i(\vv{c}_1,\dots,\vv{c}_{m_i}) \to^P_V b$}
%     \end{prooftree}
%     Това е така, защото $b \neq \bot$.
%     Оттук следва, че $\bot \not\in \{c_1,\dots,c_{m_i}\}$.

%     Сега от {\bf И.П.} имаме, че
%     \[\bot \neq c_i = \val{\mu_i}(\bar{\psi}_i) \sqsubseteq \evalv{\mu_i} = c_i,\]
%     и тогава от правилата за извод в операционната семантика, получаваме:
%     \begin{prooftree}
%       \AxiomC{$\mu_1\to^P_V c_1$}
%       \AxiomC{$\cdots$}
%       \AxiomC{$\mu_{m_i}\to^P_V c_{m_i}$}
%       \AxiomC{$\tau_i[\varsx/\bar{\vv{c}}] \to^P_V b$}
%       \RightLabel{\scriptsize($4_\Nat$)}
%       \QuaternaryInfC{$\vv{f}_i(\mu_1,\dots,\mu_{m_i}) \to^P_V b$}
%     \end{prooftree}
    
%     Заключаваме, че $\evalv{\mu} = b$.
%     \end{itemize}
% \end{proof}

\begin{problem}
  % \label{cr:prefixed-point-value}
  \marginpar{\cite[стр. 151]{winskel}}
  Нека е дадена една декларация $\vv{P}$ в езика $\REC$
  и $\bar{\delta} = \lfp(\Delta)$, където $\Delta = \Delta_{\tau_1}\times\cdots\times \Delta_{\tau_n}$.
  Тогава за всяко $i = 1,\dots,n$,
  \[\delta_i \sqsubseteq \evalv{\vv{f}_i(\vv{x}_1,\dots,\vv{x}_{m_i})}\]
\end{problem}
% \begin{proof}
%   Нека отново за улеснение да положим
%   \[\psi_i \dff \evalv{\vv{f}_i(\vv{x}_1,\dots,\vv{x}_{m_i})}.\]
%   Ще докажем, че $\Delta_{\tau_i}(\bar{\psi}) \sqsubseteq \psi_i$.
%   Нека приемем, че $\bar{a} \in \Nat^{m_i}$.
%   \begin{align*}
%     \Delta_{\tau_i}(\bar{\psi})(\bar{a}) & = \Sigma_\star(\Gamma_{\tau_i})(\bar{\psi})(\bar{a}) & (\text{деф. на }\Delta_{\tau_i}) \\
%                                          & = \Gamma_{\tau_i}(\bar{\psi})(\bar{a}) & (\bar{a} \in \Nat^{m_i})\\
%                                          & = \val{\tau_i}(\bar{a},\bar{\psi}) & (\text{от деф.})\\
%                                          & = \val{\tau_i[\bar{x}/\bar{\vv{a}}]}(\bar{\psi}) & (\text{от \hyperref[lem:subst2]{Лема за замяната}})\\
%                                          & \sqsubseteq \evalv{\tau_i[\bar{x}/\bar{\vv{a}}]} & (\text{от \Prop{equiv-value1}})\\
%                                          & = \evalv{\tau_i}(\bar{a}) & (\text{от деф. на }\evalv{\tau_i})\\
%                                          & = \psi_i(\bar{a}) & (\text{защото }\bot\not\in\{a_1,\dots,a_{m_i}\}).
%   \end{align*}
%   Така получихме, че $\Delta(\bar{\psi}) \sqsubseteq \bar{\psi}$,
%   т.е. $\bar\psi \in \texttt{Pref}(\Delta)$. Тогава, понеже $\bar{\delta} = \lfp(\Delta)$, то от
%   \Prop{prefix-point} следва, че $\bar{\delta} \sqsubseteq \bar{\psi}$.
% \end{proof}



\begin{problem}
  Да разгледаме следната програма на хаскел:

  \begin{haskellcode}
    {-# LANGUAGE BangPatterns #-}

    f :: (Int, Int) -> Int
    f(!x, !y) = if x `rem` 4 == 0 then 0 
                  else f(x + 2, f(x, y + 2)) + 2

    g :: (Int, Int) -> Int
    g(x, !y) = if x `rem` 4 == 0 then 0 
                 else g(x + 2, g(x, y + 2)) + 2

    h :: (Int, Int) -> Int
    h(!x, y) = if x `rem` 4 == 0 then 0 
                 else h(x + 2, h(x, y + 2)) + 2
  \end{haskellcode}
  Обяснете каква е разликата между \vv{f}, \vv{g} и \vv{h}.
\end{problem}


\begin{problem}
  Да разгледаме следната програма на езика \REC:
  \begin{haskellcode}
  h(x) = f(g(x)) where 
    f(x) = 42 
    g(x) = if x == 0 then 0 else g(f(x))
  \end{haskellcode}
  Намерете $\D_V\val{\vv{h}}$ и $\D_N\val{\vv{h}}$.
\end{problem}


% \Stefan{Тази задача е за правило на Скот?}
% \begin{problem}
%   \marginpar{\cite[стр. 159]{nikolova-soskova}}
%   Да разгледаме следната програма:
%   \begin{haskellcode}
%   g(x) = f(x, 0, x) where 
%     f(x, y, z) = if x == 0 then y 
%                  else f(x - 1, y + z, z)

%   g'(x) = f'(x, 0) where
%     f'(x,y) = if x == y then y 
%                 else f'(x - 1, 2 * x + y - 1)
%   \end{haskellcode}
%   Докажете, че $\D_V\val{\vv{g}} = \D_V\val{\vv{g'}}$.
% \end{problem}


\begin{problem}
  Да разгледаме следната програма на езика \REC:
  \begin{haskellcode}
    h(x) = f(x, x) where 
      f(x, y) = if x `rem` 3 == 0 then x `div` 3 
                 else f(x - 1, f(2*x - 2, y))
  \end{haskellcode}
  Намерете $\D_V\val{\vv{h}}$ и $\D_N\val{\vv{h}}$
  за да покажете, че $\D_V\val{\vv{h}} \neq \D_N\val{\vv{h}}$.
\end{problem}
\begin{solution}
  Нека първо да разгледаме термалния оператор $\Gamma:\C_2 \to \C_2$, който съответства на терма 
  задаващ дефиницията на функционалния символ \vv{f}:
  \begin{align*}
    \Gamma(\varphi)(a,b) & =
    \begin{cases}
      \val{\vv{x/3}}(a,b,\varphi), & \text{ ако } \val{x \vv{ div } 3}(a,b,\varphi) = 0\\
      \val{\vv{f(x-1,f(2x - 2, y))}}(a,b,\varphi), & \text{ ако }\val{x \vv{ div } 3}(a,b,\varphi) \neq 0\\
      \bot, & \text{ ако } \val{x \vv{ div } 3}(a,b,\varphi) = \bot\\
    \end{cases}
    \\
    & = \begin{cases}
      a/3, & \text{ ако } x\equiv 0\ (\bmod\ 3)\ \&\ x\in\Nat\\
      \varphi(a-1, \varphi(2a-2, b)), & \text{ ако } x \not\equiv 0\ (\bmod\ 3)\ \&\ x\in\Nat\\
      \bot, & \text{ ако } a = \bot.
    \end{cases}
  \end{align*}
  
  За да намерим денотационната семантика по стойност, 
  дефинираме оператора $\Delta:\S_2 \to \S_2$ 
  като $\Delta \dff \Sigma_\star \circ \Gamma$.
  Това означава, че :
  \begin{align*}
    \Delta(\varphi)(a,b) =
    \begin{cases}
      a/3, & \text{ако } a\equiv 0\ (\bmod\ 3)\ \&\ a\in\Nat\\
      \varphi(a-1, \varphi(2a-2, b)), & \text{ако } a \not\equiv 0\ (\bmod\ 3)\ \&\ a\in\Nat\\
      \bot, & \text{ако } a = \bot \text{ или } b = \bot.
    \end{cases}
  \end{align*}
  
  \marginpar{Обърнете внимание, че $\Gamma(\Omega^{(2)})(3,\bot) = 1$,
  докато $\Delta(\Omega^{(2)})(3,\bot) = \bot$}
  
  Семантиката по стойност на нашата програма \vv{h} на практика 
  представлява следната програма \vv{h'} на хаскел:
  \begin{haskellcode}
    h'(x) = f(x, x) where 
      f(!x, !y) = if x `rem` 3 == 0 then x `div` 3 
                    else f(x - 1, f(2*x - 2, y))
  \end{haskellcode}

  Семантиката по стойност на програмата се дефинира с помощта на $\lfp(\Delta)$.
  От \hyperref[th:knaster-tarski]{Теоремата на Клини} знаем, че 
  $\lfp(\Delta) = \bigsqcup_i \delta_i$, където веригата $(\delta_i)_{i\in\Nat}$ е дефинирана като
  $\delta_0 = \Omega^{(2)}$ и $\delta_{i+1} = \Delta(\delta_i)$.
  Да пресметнем първите няколко члена на тази редица.
  \begin{align*}
    \delta_0(x,y) & = \bot \text{, за всяко }x,y\in\Nat_\bot\\
    \\
    \delta_1(x,y) & = \Delta(\delta_0)(x,y)\\
    & =
    \begin{cases}
      x/3,  & x \equiv 0\ (\bmod\ 3)\ \&\ x\in\Nat\\
      \bot, & x \not\equiv 0\ (\bmod\ 3)\ \&\ x\in\Nat\\
      \bot, & x = \bot \text{ или } y = \bot
    \end{cases}\\
    \\
    \delta_2(x,y) & = \Delta(\delta_1)(x,y)\\
    & =
    \begin{cases}
      k, & x = 3k\text{ за някое }k \in \Nat\\
      \delta_1(3k,\delta_1(6k,y)), &  x = 3k+1\text{ за някое }k \in \Nat \\
      \delta_1(3k+1,\delta_1(6k+2,y)),&   x = 3k+2\text{ за някое }k \in \Nat \\
      \bot, & x = \bot \text{ или } y = \bot
    \end{cases}\\
    & = 
    \begin{cases}
      k, & \ x = 3k\text{ или }x = 3k+1\text{ за някое }k \in \Nat\\
      \bot, & x = 3k+2\text{ за някое }k \in \Nat\\
      \bot, & x = \bot \text{ или } y = \bot
    \end{cases}
  \end{align*}
  \begin{align*}
    \delta_3(x,y) & = \Delta(\delta_2)(x,y)\\
    & =
    \begin{cases}
      k, & \ x = 3k\text{ за някое }k \in \Nat\\
      \delta_2(3k,\delta_2(6k,y)), &\ x = 3k+1\text{ за някое }k \in \Nat\\
      \delta_2(3k+1,\delta_2(6k+2,y)), &\ x = 3k+2\text{ за някое }k \in \Nat\\
      \bot, & x = \bot \text{ или } y = \bot
    \end{cases}\\
    & = 
    \begin{cases}
      k, & \ x = 3k\text{ за някое }k \in \Nat\\
      \delta_2(3k,2k), &\ x = 3k+1\text{ за някое }k \in \Nat\\
      \delta_2(3k+1,\bot), &\ x = 3k+2\text{ за някое }k \in \Nat\\
      \bot, & x = \bot \text{ или } y = \bot
    \end{cases}\\
    & =
    \begin{cases}
      k, & \ x = 3k\text{ или }x = 3k+1\text{ за някое }k \in \Nat\\
      \bot, &\ x = 3k+2\text{ за някое }k \in \Nat\\
      \bot, & x = \bot \text{ или } y = \bot
    \end{cases}
    \end{align*}
    Получаваме, че $\delta_2 = \delta_3$ и следователно $\delta_2 = \bigsqcup_n \delta_n$ и 
  \[\D_V\val{\vv{h}}(x) = \delta_2(x,x).\]
  Тогава за всяко $x \in \Nat_\bot$, 
  \begin{align*}
    \D_V\val{\vv{h}}(x) \simeq 
    \begin{cases}
      \lfloor{x/3}\rfloor, & \text{ ако } x\equiv 0\ (\bmod\ 3)\text{ или } x\equiv 1\ (\bmod\ 3)\\
      \bot, & \text{ ако } x\equiv 2\ (\bmod\ 3) \text{ или }x = \bot.
    \end{cases}
  \end{align*}

  Преминаваме към намирането на денотационната семантика по име на програмата \vv{h},
  която се определя с помощта на $\lfp(\Gamma)$.
  Започваме да търсим елементите на веригата $(\gamma_i)^{\infty}_{i=0}$,
  където $\gamma_0 = \Omega^{(2)}$ и $\gamma_{i+1} = \Gamma(\gamma_i)$.
  Да пресметнем първите няколко члена на тази верига.
  \begin{align*}
    \gamma_0(x,y) & = \bot \text{ за всяко }x,y\in\Nat_\bot\\
    \\
    \gamma_1(x,y) & = \Gamma(\gamma_0)(x,y) = 
    \begin{cases}
      x/3, & \quad x \equiv 0\ (\bmod\ 3)\\
      \bot, & \quad \text{иначе}
    \end{cases}\\
    \\
    \gamma_2(x,y) & = \Gamma(\gamma_1)(x,y)\\
    & = 
    \begin{cases}
      k, & x = 3k\text{ за някое }k \in \Nat\\
      \gamma_1(3k,\gamma_1(6k,y)), & x = 3k+1\text{ за някое }k \in \Nat\\
      \gamma_1(3k+1,\gamma_1(6k+2,y)), & x = 3k+2\text{ за някое }k \in \Nat\\
      \bot, & x = \bot
    \end{cases}\\
    & = 
    \begin{cases}
      k, & \quad x = 3k\text{ или }x=3k+1\text{ за някое }k \in \Nat\\
      \bot, & \quad x=\bot\text{ или } x = 3k+2\text{ за някое }k \in \Nat
    \end{cases}
  \end{align*}
  \begin{align*}
    \gamma_3(x,y) & = \Gamma(\gamma_2)(x,y) \\
    & = 
    \begin{cases}
      k, & x = 3k\text{ за някое }k \in \Nat\\
      \gamma_2(3k, \gamma_2(6k,y)), & x = 3k+1\text{ за някое }k  \in \Nat\\
      \gamma_2(3k+1, \gamma_2(6k+2,y)), & x = 3k+2\text{ за някое }k \in \Nat\\
      \bot, & x = \bot
    \end{cases}\\
    & = 
    \begin{cases}
      k, & x = 3k\text{ за някое }k \in \Nat\\
      \gamma_2(3k, 2k), & x = 3k+1\text{ за някое }k \in \Nat\\
      \gamma_2(3k+1, \bot), & x = 3k+2\text{ за някое }k \in \Nat\\
      \bot, & x = \bot
    \end{cases}\\
    & = 
    \begin{cases}
      k, & x = 3k\text{ за някое }k \in \Nat\\
      k, & x = 3k+1\text{ за някое }k \in \Nat\\
      k, & x = 3k+2\text{ за някое }k \in \Nat\\
      \bot, & x = \bot
    \end{cases}\\
    & = 
    \begin{cases}
      \lfloor{x/3}\rfloor, & \quad x\in\Nat\\
      \bot, & \quad x = \bot\\
    \end{cases}
  \end{align*}
  
  Знаем, че $\gamma_3 \sqsubseteq \gamma_4$.
  Следователно, $\gamma_4(x,y) = \lfloor{x/3}\rfloor$ за всяко $x\in\Nat,y\in\Nat_\bot$.
  Освен това,  $\gamma_4(\bot,y) = \Gamma(\gamma_3)(\bot,y) = \bot$ и следователно $\gamma_3 = \gamma_4$.
  Тогава $\gamma_3 = \bigsqcup_i \gamma_i$.
  Получаваме, че:
  \begin{align*}
    \D_N\val{\vv{h}}(x) & = 
    \begin{cases}
      \gamma_3(x,x), & \text{ако }\gamma_3(x,x) \neq \bot\\
      \bot, & \text{ако }\gamma_3(x,x) = \bot
    \end{cases}\\
    & =
    \begin{cases}
      \lfloor{x/3}\rfloor, & \text{ако } x \in \Nat\\
      \bot, & \text{ако } x = \bot.
    \end{cases}
  \end{align*}

  Получихме, че двете семантики се различават.
\end{solution}

\begin{problem}
  \marginpar{\cite[стр. 158]{winskel}}
  % . Доказателството на Лема 8.3.3 на \cite[стр. 192]{ditchev-soskov} е сходно, но има недостатъка, че използва лема за симулацията на \cite[стр. 177]{ditchev-soskov}}
  Нека разгледаме една деклрация \vv{P} в езика $\REC$, произволен терм $\tau$ и {\em функционални} термове $\mu_1,\dots,\mu_{k}$.
  Нека $\bar{\gamma} = \lfp(\Gamma)$, където $\Gamma$ е операторът, който съответства на декларацията \vv{P}.
  Тогава
  \marginpar{Понеже $\mu_i$ е функционален терм, то $\evaln{\mu_i} \in \Nat_\bot$}
  \[\val{\tau}(\evaln{\mu_1},\dots,\evaln{\mu_k},\bar{\gamma}) \sqsubseteq \evaln{\tau[\vv{x}_1/\mu_1\dots,\vv{x}_k/\mu_k]}.\]
  
  Използвайте това твърдение за да докажете, че за всяка програма $\vv{h}$ на езика {\bf REC} е изпълнено,
  че \[\D_N\val{\vv{h}} \sqsubseteq \O_N\val{\vv{h}}.\]
\end{problem}
% \begin{hint}
%   Да напомним, че $\bar{\gamma} = \bigsqcup_r \bar{\gamma}_r$,
%   където $\bar{\gamma}_r = (\gamma^1_r,\dots,\gamma^n_r)$ и 
%   \begin{align*}
%     & \gamma^i_0 = \Omega^{(m_i)}\\
%     & \gamma^i_{r+1} = \Gamma_{\tau_i}(\bar{\gamma}_r).
%   \end{align*}

%   Ще докажем с индукция по $r$, че за всяко $r$, произволен терм $\tau$ и произволни функционални термове $\mu_1,\dots,\mu_k$,
%   \[\val{\tau}(\evaln{\mu_1},\dots,\evaln{\mu_k},\bar{\gamma}_r) \sqsubseteq \evaln{\tau[\vv{x}_1/\mu_1\dots,\vv{x}_k/\mu_k]}.\]
%   Нека $r = 0$ не крие изненади.

%   Да приемем, че твърдението е вярно за $r$. Ще докажем, че то е вярно и за $r+1$.
%   Сега правим вътрешна индукция по построението на терма $\tau$. Ще разгледаме само най-интересния случай.
%   \begin{itemize}
%   \item 
%     Нека $\tau = f_i(\tau_1,\dots,\tau_{m_i})$.
%     Да означим за $j = 1,\dots,m_i$,
%     \begin{align*}
%       b_j & \dff \val{\tau_j}(\evaln{\mu_1},\dots,\evaln{\mu_k},\bar{\gamma}_{r+1}),\\
%       c_j & \dff \evaln{\tau_j[x_1/\mu_1,\dots,x_k/\mu_k]}.
%     \end{align*}
%     Тогава
%     \begin{align*}
%       \val{\tau}(\evaln{\mu_1},\dots,\evaln{\mu_k},\bar{\gamma}_{r+1}) & = \gamma^i_{r+1}(b_1,\dots,b_{m_i}) & (\text{стойност на терм})\\
%       & \sqsubseteq \gamma^i_{r+1}(c_1,\dots,c_{m_i}) & (\text{мон. на } \gamma^i_{r+1})\\
%       & = \Gamma_{\tau_i}(\bar{\gamma}_r)(c_1,\dots,c_{m_i}) & (\gamma^i_{r+1} = \Gamma_{\tau_i}(\bar{\gamma}))\\
%       & = \val{\tau_i}(c_1,\dots,c_{m_i},\bar{\gamma}_r) & (\text{деф. на }\Gamma_{\tau_i})\\
%       & \sqsubseteq \evaln{\tau_i[\varsx/\bar{\mu}]} & (\text{{\bf И.П.} за }r).
%     \end{align*}
%     Сега вече можем да приложим правилото $(4')$ и да получим

%      \begin{prooftree}
%        \AxiomC{$\tau_i[\varsx/\bar{\mu}] \to^P_V a$}
%        \RightLabel{\scriptsize($4'$)}
%        \UnaryInfC{$\vv{f}_i(\mu_1,\dots,\mu_{m_i}) \to^P_N a$}
%      \end{prooftree}
%    \end{itemize}
%    Завършваме доказателството като отбележим, че
%    \begin{align*}
%      \val{\tau}(\evaln{\mu_1},\dots,\evaln{\mu_k},\bar{\gamma}) & = \val{\tau}(\evaln{\mu_1},\dots,\evaln{\mu_k},\bigsqcup_r\bar{\gamma}_r)\\
%      & = \bigsqcup_r \val{\tau}(\evaln{\mu_1},\dots,\evaln{\mu_k},\bar{\gamma}_r)\\
%      & \sqsubseteq \evaln{\tau[\vv{x}_1/\mu_1\dots,\vv{x}_k/\mu_k]}.
%    \end{align*}
% \end{hint}



%%% Local Variables:
%%% mode: latex
%%% TeX-master: "../sep-notes"
%%% End:


%%% Local Variables:
%%% mode: latex
%%% TeX-master: "../sep"
%%% End:

% \chapter{Доказване на свойства на програми}

След като вече имаме точна дефиниция на семантиката на една рекурсивна програма, можем да видим как можем да доказваме
формално свойства на рекурсивни програми.

\section{Свойства}
\begin{itemize}
\item 
  Да фиксираме една област на Скот $\A$. Подмножествата $P \subseteq \A$ ще наричаме {\bf свойства}.
\item
  \index{непрекъснато свойство}
  \marginpar{В \cite[стр. 166]{winskel} се наричат {\em inclusive subsets}. В \cite{bird-haskell} се наричат {\em chain-complete assertions}}
  Казваме, че $P$ е {\bf непрекъснато (или допустимо, индуктивно) свойство} над областта на Скот $\A$, ако за всяка верига $(a_i)^{\infty}_{i=0}$ от елементи на $\A$, е изпълнено:
  \begin{prooftree}
    \AxiomC{$P(a_0)$}
    \AxiomC{$P(a_1)$}
    \AxiomC{$P(a_2)$}
    \AxiomC{$P(a_3)$}
    \AxiomC{$\ldots$}
    \QuinaryInfC{$P(\bigsqcup_i a_i)$}
  \end{prooftree}
  % за която $P(a_i)$, то е изпълнено и $P(\bigsqcup_i a_i)$.
  \marginpar{Понеже $\A$ е област на Скот, ние знаем, че $\bigsqcup_i a_i$ съществува}
\end{itemize}

Нека да видим, че има свойства, който не са непрекъснати.

\begin{example}
  \label{ex:complement-not-inclusive}
  Нека да разгледаме областта на Скот от точните изображения $\Strict{\Nat_\bot}{\Nat_\bot}$.
  Да разгледаме свойството $P \subseteq \Strict{\Nat_\bot}{\Nat_\bot}$, което е дефинирано по следния начин:
  \[P(f) \dfff (\exists x \in \Nat)[f(x) = \bot].\]
  Да разгледаме изображенията $f_i$, дефинирани по следния начин:
  \begin{align*}
    f_i(x) & =
    \begin{cases}
      42, & x \in \Nat\ \&\ x \leq i\\
      \bot, & x \in \Nat\ \&\ x > i\\
      \bot, & x = \bot.
    \end{cases}
  \end{align*}
  Лесно се съобразява, че $(f_i)^{\infty}_{i=0}$ е верига в $\Strict{\Nat_\bot}{\Nat_\bot}$ и че 
  за всяко $i$, $P(f_i)$. Да разгледаме точната горна граница $g$ на тази верига, за която знаем, че
  за всяко $y \in \Nat$,
  \[g(x) = y \iff (\exists i)[f_i(x) = y].\]
  Според конструкцията на $g$, $g(x) = 42$ за всяко $x \in \Nat$.
  Оттук директно получаваме, че $\neg P(g)$.
  Така видяхме, че $P$ {\em не е непрекъснато свойство}.
\end{example}

% \begin{example}
%   Нека да разгледаме областта на Скот $\S_1$, т.е. точните функции на един аргумент.
%   Да разгледаме свойството $P \subseteq \S_1$, което е дефинирано по следния начин:
%   \[P(f) \dfff (\exists x \in \Nat)[f(x) = \bot].\]
%   Да разгледаме функциите $f_i$, дефинирани по следния начин:
%   \begin{align*}
%     f_i(x) & =
%     \begin{cases}
%       42, & x \leq i\\
%       \bot, & x > i\\
%       \bot, & x = \bot.
%     \end{cases}
%   \end{align*}
%   Ясно е, че $P(f_i)$ е изпълнено за всяко $i$.
%   Да разгледаме функцията $g = \bigsqcup_i f_i$, която е дефинирана като:
%   \[g(x) = \bigsqcup_i\{f_i(x)\}.\]
%   Лесно се съобразява, че за всяко $x \in \Nat$, $g(x) \neq \bot$. Следователно, $\neg P(g)$.
%   Така видяхме, че $P$ не е непрекъснато свойство.
% \end{example}

\begin{problem}
  \label{prob:inclusive-property}
  Докажете, че свойството $Q$ над $\Strict{\Nat_\bot}{\Nat_\bot}$, където
  \[Q(f) \iff (\forall x \in \Nat)[f(x) \neq \bot],\]
  е непрекъснато.
\end{problem}

\subsection{Основни свойства}

\begin{prop}
  \label{pr:continuous-property}
  Нека $\A$ и $\B$ са области на Скот и $f_1,f_2 \in \Cont{\A}{\B}$.
  Тогава следните свойства над $\A$ са непрекъснати:
  \begin{itemize}
  \item 
    $P(a) \dfff f_1(a) \sqsubseteq f_2(a)$;
  \item
    $P(a) \dfff f_1(a) = f_2(a)$;
  \end{itemize}
\end{prop}

\begin{prop}
  \label{pr:fixed-element-property}
  Нека $\A$ е област на Скот.
  Да фиксираме произволен елемент $a_0 \in \A$.
  Тогава следните свойства над $\A$ са непрекъснати:
  \begin{itemize}
  \item 
    $P(a) \dfff a \sqsubseteq a_0$;
  \item
    $P(a) \dfff a = a_0$;
  \end{itemize}
\end{prop}

\subsection{Сечение}
\begin{prop}
  Нека $P_1$ и $P_2$ непрекъснати свойства над областта на Скот $\A$.
  Тогава $P_1 \cap P_2$ също е непрекъснато свойство.
\end{prop}

\subsection{Обединение}

\begin{prop}
  Нека $P_1$ и $P_2$ непрекъснати свойства над областта на Скот $\A$.
  Тогава $P_1 \cup P_2$ също е непрекъснато свойство.
\end{prop}

\subsection{Допълнение}

\begin{prop}
  Съществува непрекъснато свойство $P$ над областта на Скот $\A$,
  за което $\A \setminus P$ {\bf не} е непрекъснато свойство.
\end{prop}
\begin{proof}
  Да вземем $\A = \S_1$.
  Свойството $Q$ от \Problem{inclusive-property} е непрекъснато, 
  докато $\S_1 \setminus Q = P$, което е точно свойството от \Ex{complement-not-inclusive}, а 
  за него него знаем, че не е непрекъснато.
\end{proof}

\subsection{Образ}

\begin{problem}
  Нека $P$ е непрекъснато свойство в областта на Скот $\B$.
  Нека $f \in \Cont{\A}{\B}$.
  Да разгледаме свойството 
  \[f[P] \dff \{f(a) \mid f(a) \in P\}.\]
  Не винаги $f[P]$ е непрекъснато свойство в $\A$.
\end{problem}
\ifhints
\begin{hint}
  Нека $(b_n)^\infty_{n=0}$ е верига в $\B$,
  като $\bigsqcup_n b_n$ не е елемент на веригата.
  Нека дефинираме изображението $f$ по следния начин:
  $f(\bot) \dff \bot$ и $f(n) \dff b_n$.
  Лесно се съобразява, че $f \in \Cont{\Nat_\bot}{\B}$.
  Нека $P = \Nat$, което очевидно е непрекъснато свойство, защото елементите на $\Nat$ 
  не са сравними относно плоската наредба.
  Тогава $f[P] = \{b_n \mid n \in \Nat\}$.
  Лесно се проверява, че $f[P]$ не е непрекъснато свойство.
\end{hint}
\fi

\subsection{Първообраз}

\begin{problem}
  Нека $P$ е непрекъснато свойство в областта на Скот $\B$.
  Нека $f \in \Cont{\A}{\B}$.
  Да разгледаме свойството 
  \[f^{-1}[P] \dff \{a \in \A \mid f(a) \in P\}.\]
  Докажете, че $f^{-1}[P]$ е непрекъснато свойство в $\A$.
\end{problem}

\subsection{Композиция}

\begin{problem}
  Нека $P$ е непрекъснато свойство в областта на Скот $\A \times \B$,
  а $Q$ е непрекъснато свойство в областта на Скот $\B \times \C$.
  Композицията 
  \[Q \circ P = \{\pair{a,c} \in \A\times \C \mid (\exists b\in\B)[\pair{a,b} \in P\ \&\ \pair{b,c} \in Q]\}.\]
  Докажете, че {\bf не винаги} $Q \circ P$ е непрекъснато свойство.
\end{problem}
\ifhints
\begin{hint}
  Нека $(a_n)^\infty_{n=0}$ е верига в $\A$, а $(c_n)^{\infty}_{n=0}$ е верига в $\C$,
  като и двете вериги са такива, че $\bigsqcup_n a_n$ не е елемент на $(a_n)^{\infty}_{n=0}$
  и $\bigsqcup_n c_n$ не е елемент на $(c_n)^\infty_{n=0}$.
  Нека $\B = \Nat_\bot$.  
  Тогава дефинираме свойствата по следния начин:
  \begin{align*}
    & P \dff \{\pair{a_n,n} \mid n \in \Nat\};\\
    & Q \dff \{\pair{n,c_n} \mid n \in \Nat\}.
  \end{align*}
  Лесно се проверява, че тези свойства са непрекъснати.
  Тогава,
  \[Q \circ P = \{\pair{a_n,c_n} \mid n \in \Nat\},\]
  което очевидно не е непрекъснато свойство.
\end{hint}
\fi

\subsection{Проекции}

\begin{problem}
  Нека $P$ е непрекъснато свойство в областта на Скот $\A \times \B$.
  Нека за произволно $a$ да дефинираме свойството 
  \[P_a \dff \{b \in \B \mid \pair{a,b} \in P\}.\]
  Тогава $P_a$ е непрекъснато свойство.
  Наричаме $P_a$ проекция на $P$ по първата компонента.
\end{problem}

Ние знаем, че едно изображение $f:\A\times \B \to \C$ е непрекъснато точно тогава, когато
$f$ е непрекъснато по всеки от аргументите си.
Ако $P$ е непрекъснато свойство в $\A\times\B$, то е ясно, че $P$ е непрекъснато по всяка от проекциите си.

\begin{problem}
  Да разгледаме свойството $P$ в $\A \times \B$, за което 
  имаме, че $P$ е непрекъснато свойство по всяка от проекциите.
  Вярно ли е, че тогава $P$ е непрекъснатото?
\end{problem}
\ifhints
\begin{hint}
  Нека $\A$ и $\B$ са такива области на Скот, в които има вериги съответно $(a_n)^\infty_{n=0}$ и $(b_n)^\infty_{n=0}$,
  за които $\bigsqcup_n a_n$ и $\bigsqcup_n a_n$ не са елементи на съответните вериги.
  Нека $P = \{\pair{a_n,b_n} \mid n \in \Nat\}$.
  Тогава за всяко $n$, $P_{a_n}$ и $P_{b_n}$ са непрекъснати, защото
  $P_{a_n} = \{b_n\}$ и $P_{b_n} = \{a_n\}$,
  но е очевидно, че $P$ не е непрекъснато свойство.
\end{hint}
\fi

\subsection{Универсално затваряне}

\begin{problem}
  Нека $P$ е непрекъснато свойство в областта на Скот $\A \times \B$.
  Нека за произволно $a$ да дефинираме свойството 
  \[Q \dff \{b \in \B \mid (\forall a \in \A)[\pair{a,b} \in P]\}.\]
  Вярно ли е, че $Q$ е непрекъснато свойство?
  Обосновете отговора си!
\end{problem}
\ifhints
\begin{hint}
  Вярно е.
\end{hint}
\fi
% \ifhints
% \begin{hint}
%   Нека $\A$ и $\B$ са такива области на Скот, в които има вериги съответно $(a_n)^\infty_{n=0}$ и $(b_n)^\infty_{n=0}$,
%   за които $\bigsqcup_n a_n$ и $\bigsqcup_n a_n$ не са елементи на съответните вериги.
%   Нека 
%   \[P \dff \{\pair{a_n,b_k} \mid n,k \in \Nat\} \cup \{\pair{\bigsqcup_n a_n, b_k} \mid k \in \Nat\} \cup \{\pair{\bigsqcup_n a_n, \bigsqcup_k b_k}\}.\]
%   Тогава $Q = \{a_n \mid n \in \Nat\}$.
% \end{hint}
% \fi

\subsection{Екзистенциално затваряне}

\begin{problem}
  Нека $P$ е непрекъснато свойство в областта на Скот $\A \times \B$.
  Нека за произволно $a$ да дефинираме свойството 
  \[Q \dff \{b \in \B \mid (\exists a \in \A)[\pair{a,b} \in P]\}.\]
  Вярно ли е, че $Q$ е непрекъснато свойство?
\end{problem}
\ifhints
\begin{hint}
  Разгледайте $\A = \B = \Int \cup\{-\infty,+\infty\}$.
  Нека 
  \[P = \{\pair{-n,n} \mid n \in \Nat\}.\]
  Тогава $P$ е непрекъснато свойство.
  Но тогава $Q = \{n \mid n \in \Nat\}$ не е непрекъснато свойство.
\end{hint}
\fi

\subsection{Непрекъснати изображения}

Знаем, че ако $\A$ и $\B$ са области на Скот, то съвкупността от всички непрекъснати изображения $\Cont{\A}{\B}$ образува
област на Скот. Сега ще разгледаме аналог на тази теорема за непрекъснати свойства.

\begin{prop}
  Нека $P$ и $Q$ са свойства съответно в $\A$ и $\B$.
  Да разгледаме свойството $R$ в $\Cont{\A}{\B}$ дефинирано като:
  \[R \dff \{f \in \Cont{\A}{\B} \mid (\forall a \in \A)[P(a) \implies Q(f(a))]\}.\]
  Докажете, че ако $Q$ е непрекъснато свойство, то $R$ е непрекъснато свойство.
\end{prop}

\subsection{Частична коректност}

\begin{itemize}
\item
  Да разгледаме едно свойство $I$ в областта на Скот $\A$, който наричаме условие за входа, и
  свойство $O$ в областта на Скот $\A \times \B$, който наричаме условие за изхода.
\item
  \index{частична коректност}
  {\bf Свойство от тип частична коректност} относно $I$ и $O$ представлява 
  свойство $P \subseteq \Mapping{\A}{\B}$ със следната дефиниция
  \[P(f) \dfff (\forall a \in \A)[I(a)\ \&\ f(a) \neq \bot \implies O(a,f(a))].\]
\end{itemize}

\begin{prop}
  Нека $I \subseteq \A$, а $O$ е непрекъснато свойство в $\A \times \B$.
  \[P(f) \dfff (\forall a \in \A)[I(a)\ \&\ f(a) \neq \bot \implies O(a,f(a))].\]
  Тогава свойството $P$ е непрекъснато в $\Mapping{\A}{\B}$.
\end{prop}

\begin{example}
  Нека $\A = \B = \Nat_\bot$ и $I(x) \dfff x > 1$, а $O(x,y) \dfff x,y\in\Nat\ \&\ x = y^2$.
  Ясно е, че $O$ е непрекъснато свойство в $\Nat^2_\bot$. Да разгледаме свойството
  \[P(f) \dfff (\forall x \in \Nat_\bot)[x > 1\ \&\ f(x) \neq \bot \implies O(x,f(x))].\]
  Знаем, че $P$ е от тип частична коректно и следователно е непрекъснато в областта на Скот 
  $\Mapping{\Nat^2_\bot}{\Nat_\bot}$.
  Понеже $\Sigma^\star:\Cont{\Partial{\Nat^2}{\Nat}}{\Strict{\Nat^2_\bot}{\Nat_\bot}}$, то
  $Q \dfff (\Sigma^\star)^{-1}[P]$ е непрекъснато свойство в областта на Скот $\Partial{\Nat^2}{\Nat}$.
  Ясно е, че
  \[Q(f) \equiv (\forall x \in \Nat)[x > 1\ \&\ !f(x) \implies O(x,f(x))].\]
\end{example}


\subsection{Тотална коректност}

\index{тотална коректност}
{\bf Свойство от тип тотална коректност} относно $I$ и $O$ представлява 
свойство $P \subseteq \Mapping{\A}{\B}$ със следната дефиниция
\[P(f) \dfff (\forall a \in \A)[I(a) \implies (f(a) \neq \bot\ \&\  O(a,f(a)))].\]

\begin{prop}
  Нека $I$ е свойство в $\A$, а $O$ е непрекъснато свойство в $\A \times \B$.
  Тогава свойството
  \[P(f) \dfff (\forall a \in \A)[I(a) \implies (f(a) \neq \bot\ \&\ O(a,f(a)))]\]
  е непрекъснато в $\Mapping{\A}{\B}$.
\end{prop}


%%% Local Variables:
%%% mode: latex
%%% TeX-master: "../sep-notes"
%%% End:


% \newpage
% \setcounter{problem}{0}

% \begin{problem}
%   Нека е даден следния оператор $\Gamma:\F_1\to\F_1$:
%   \begin{align*}
%     \Gamma(f)(x) \simeq 
%     \begin{cases}
%       0, & f\text{ е крайна функция}\\
%       1, & \text{ иначе }.
%     \end{cases}
%   \end{align*}
%   Проверете дали:
%   \begin{enumerate}[a)]
%   \item 
%     $\Gamma$ е монотонен оператор;
%   \item
%     $\Gamma$ е компактен оператор.
%   \end{enumerate}
% \end{problem}
% \begin{proof}
%   \begin{enumerate}[a)]
%   \item 
%     Трябва да проверим дали:
%     \[(\forall f,g\in\F_1)[f \subseteq g\ \Rightarrow \Gamma(f) \subseteq \Gamma(g)].\]
%     Да изберем $f \subseteq g$ да бъдат такива функции, че $f$ е крайна функция, но $g$ не е крайна функция.
%     Тогава $(\forall x \in \Nat)[\Gamma(f)(x) \simeq 0\ \&\ \Gamma(g)(x) \simeq 1]$.
%     Очевидно е, че за така избраните $f$ и $g$, $\Gamma(f) \not\subseteq \Gamma(g)$.
%   \item
%     Знаем, че всеки компактен оператор е монотонен.
%     Щом $\Gamma$ не е монотонен, то със сигурност $\Gamma$ не е компактен.
%   \end{enumerate}
% \end{proof}

% \begin{problem}
%   Нека е даден следния оператор $\Gamma:\F_1\to\F_1$:
%   \begin{align*}
%     \Gamma(f)(x) \simeq 
%     \begin{cases}
%       \neg !, & f\text{ е крайна функция}\\
%       1, & \text{ иначе }.
%     \end{cases}
%   \end{align*}
%   Проверете дали:
%   \begin{enumerate}[a)]
%   \item 
%     $\Gamma$ е монотонен оператор;
%   \item
%     $\Gamma$ е компактен оператор.
%   \end{enumerate}
% \end{problem}
% \begin{proof}
%   \begin{enumerate}[a)]
%   \item 
%     Трябва да проверим дали:
%     \[(\forall f,g\in\F_1)[f \subseteq g\ \Rightarrow \Gamma(f) \subseteq \Gamma(g)].\]
%     Нека $f \subseteq g$ са произволни функции от $\F_1$.
%     Ще разгледаме два случая.
%     \begin{itemize}
%     \item 
%       $f$ е крайна функция. Тогава $\Gamma(f) = \emptyset^{(1)}$ и е очевидно, че 
%       \[\Gamma(f) = \emptyset^{(1)} \subseteq \Gamma(g).\]
%     \item
%       $f$ не е крайна функция. Щом $f \subseteq g$, то $g$ също не е крайна функция.
%       Тогава 
%       \[(\forall x \in \Nat)[\Gamma(f)(x) \simeq 1 \simeq \Gamma(g)(x)],\]
%       от което следва, че 
%       \[\Gamma(f) \subseteq \Gamma(g).\]
%     \end{itemize}
%     Разгледахме всички възможни случаи за $f$ и във всеки от тях получихме, че $\Gamma(f) \subseteq \Gamma(g)$.
%     Следователно, $\Gamma$ е монотонен оператор.
%   \item
%     Сега ще проверим дали е изпълнено, че:
%     \begin{equation}
%       \label{eq:compact}
%       (\forall f \in \F_1)(\forall x,y\in\Nat)[\Gamma(f)(x)\simeq y\ \Rightarrow\ (\exists \theta \subseteq f)[\Gamma(\theta)(x) \simeq y]].
%     \end{equation}
%     Тук с $\theta$ означаваме крайна функция.
%     Нека $f$ е не е крайна функция.% , например, $f(x) = x^2$ за всяко $x \in \Nat$.
%     Тогава е ясно, че за всяко $x \in \Nat$, $\Gamma(f)(x) \simeq 1$.
%     От друга страна, понеже $\theta$ е крайна, $\neg ! \Gamma(\theta)(x)$ за всяко $x \in \Nat$.
    
%     Така видяхме, че ако $f$ не е крайна, то за произволно $x$, $\Gamma(f)(x) \simeq 1$,
%     но не съществува крайна $\theta \subseteq f$, за която $\Gamma(\theta)(x) \simeq 1$.
%     От това следва, че Формула (\ref{eq:compact}) не е изпълнена.
%     Следователно, $\Gamma$ не е компактен оператор.
%   \end{enumerate}
% \end{proof}


% \begin{problem}
%   Нека е даден следния оператор $\Gamma:\F_2\to\F_2$:
%   \begin{align*}
%     \Gamma(f)(x,y) \simeq &
%     \begin{cases}
%       y, & x = 0\\
%       f(x, f(x-1,y)), & x > 0.
%     \end{cases}
%   \end{align*}
%   \begin{enumerate}[a)]
%   \item 
%     Докажете, че $\Gamma$ е компактен оператор.
%   \item
%     Коя е най-малката неподвижна точка на $\Gamma$?
%   \item
%     Има ли $\Gamma$ други неподвижни точки ?
%   \end{enumerate}
% \end{problem}


% \begin{problem}
%   Нека е даден следния оператор $\Gamma:\F_1\to\F_1$:
%   \begin{align*}
%     \Gamma(f)(x) \simeq &
%     \begin{cases}
%       0, & x = 0\\
%       f(f(x-1)+1)), & x > 0.
%     \end{cases}
%   \end{align*}
%   \begin{enumerate}[a)]
%   \item 
%     Докажете, че $\Gamma$ е компактен оператор.
%   \item
%     Коя е най-малката неподвижна точка на $\Gamma$?
%   \item
%     Има ли $\Gamma$ други неподвижни точки ?
%   \end{enumerate}
% \end{problem}

% \section{Структурна индукция}

% % \Stefan{Да се махне оттук.}
% \begin{problem}
%   Да разгледаме програмите $\texttt{fib}$ и $\texttt{fib'}$:
  
%   \begin{haskellcode}
% fib(n) = f(n) where
%   f(n) = if n == 0 then 0
%            else if n == 1 then 1
%              else f(n-1) + f(n-2)

% fib'(n) = g(0,1,n) where
%   g(a,b,n) = if n == 0 then a
%                else g(b, a+b, n-1)
%   \end{haskellcode}
  
%   Докажете, че $\D_V\val{\texttt{fib}} = \D_V\val{\texttt{fib'}}$.
% \end{problem}
% \begin{hint}
%   Да разгледаме операторите:
%   \begin{align*}
%     \Gamma(f)(x) =
%     \begin{cases}
%       0, & x = 0\\
%       1, & x = 1\\
%       f(x-1) + f(x-2), & x \geq 2\\
%       \bot, & x = \bot.
%     \end{cases}
%   \end{align*}

%   \begin{align*}
%     \Delta(g)(x,y,z) =
%     \begin{cases}
%       x, & z = 0\\
%       g(y,x+y,z-1), & z \geq 1\\
%       \bot, & z = \bot.
%     \end{cases}
%   \end{align*}
%   \Stefan{Малко е объркващо, че от една страна работим с $\Nat_\bot$, където имаме плоска наредба, а 
%   от друга правим индукция по наредбата на естествените числа}

%   Очевидно е, че тези оператори са непрекъснати.
%   Нека $\gamma$ е най-малката неподвижна точка на $\Gamma$ и
%   $\delta$ е най-малката неподвижна точка на $\Delta$.

%   Докажете, че с индукция по $n \in \Nat$, че 
%   \[(\forall i \in \Nat)[\delta(\gamma(i), \gamma(i+1), n) = \gamma(n+i)].\]
% \end{hint}

\section{Правило на Скот}
\marginpar{\cite[стр. 166]{winskel}}
\index{правило на Скот}
\begin{itemize}
\item 
  Нека $\A$ е област на Скот и нека $f \in \Cont{\A}{\A}$.
\item
  Всяко $P \subseteq A$ ще наричаме свойство.
\item
  \marginpar{С $\texttt{lfp}(f)$ означаваме най-малката неподвижна точка на $f$ (от англ. least fixed point)}
  Тогава {\bf правилото на Скот} гласи следното:
  \begin{prooftree}
  \AxiomC{$P(\bot)$}
  \AxiomC{$(\forall a \in \A)[P(a) \implies P(f(a))]$}
  \BinaryInfC{$P(\texttt{lfp}(f))$}
\end{prooftree}
\end{itemize}

\begin{proof}
  
\end{proof}


\begin{problem}
  \marginpar{Сравнете с \Prop{prefix-point}}
  Нека $f \in \Cont{\A}{\A}$.
  Да означим множеството от преднеподвижни точки на $f$ като
  \[\texttt{Pref}(f) \dff \{a \in \A \mid f(a) \sqsubseteq a\}.\]
  Тогава 
  \[(\forall a \in \A)[a \in \texttt{Pref}(f) \implies \lfp(f) \sqsubseteq a].\]
\end{problem}
\begin{proof}
  Да фиксираме елемент $a \in \texttt{Pref}(f)$.
  Да разгледаме непрекъснатото свойство
  \marginpar{Сами се убедете, че $P$ е непрекъснато свойство!}
  \[P(b) \dfff b \sqsubseteq a.\]
  Ясно е, че $P(\bot)$.
  Нека $b\in \A$, за който $P(b)$. Ще докажем, че $P(f(b))$.
  \begin{align*}
    b \sqsubseteq a & \implies f(b) \sqsubseteq f(a) & \comment{f \text{ е мон.}}\\
    & \implies f(b) \sqsubseteq f(a) \sqsubseteq a & \comment{a \in \texttt{Pref}(f)}\\
    & \implies f(b) \sqsubseteq a & \comment{\sqsubseteq \text{ е транз. рел.}}.
  \end{align*}
  От правилото на Скот, заключаваме, че $P(\lfp(f))$, т.е.
  $\lfp(f) \sqsubseteq a$.
\end{proof}
  
\begin{example}
  Нека $f,g \in \Cont{\A}{\A}$ като имаме свойствата:
  \begin{itemize}
  \item
    $f(\bot) \sqsubseteq g(\bot)$;
  \item
    $f \circ g \sqsubseteq g \circ f$.
  \end{itemize}
  Докажете, че $\lfp(f) \sqsubseteq \lfp(g)$.
\end{example}
\begin{proof}
  Разгледайте непрекъснатото свойството 
  \[P(a) \dfff f(a) \sqsubseteq g(a).\]
  От условието имаме, че $P(\bot)$.
  Нека $P(a)$. Ще докажем, че $P(g(a))$.
  \begin{align*}
    P(a) & \iff f(a) \sqsubseteq g(a)\\
         & \implies g(f(a)) \sqsubseteq g(g(a)) & \comment{g \text{ е мон.}}\\
         & \implies f(g(a)) \sqsubseteq g(g(a)) & \comment{ f\circ g \sqsubseteq g\circ f}\\
         & \iff P(g(a)).
  \end{align*}
  Тогава по правилото на Скот ще заключим, че $P(\lfp(g))$, откъдето
  \[f(\lfp(g)) \sqsubseteq g(\lfp(g)) = \lfp(g).\]
  Това означава, че $\lfp(g)$ е преднеподвижна точка на $f$, т.е.
  \[\lfp(g) \in \texttt{Pref}(f).\]
  Понеже $\lfp(f)$ е най-малката преднеподвижна точка на $f$,
  то заключаваме, че $\lfp(f) \sqsubseteq \lfp(g)$.
\end{proof}

\begin{problem}
  Нека $h \in \Cont{\A}{\B}$, $f \in \Cont{\A}{\A}$, $g \in \Cont{\B}{\B}$,
  като имаме свойствата:
  \begin{itemize}
  \item 
    $h$ е точна, т.е. $h(\bot^\A) = \bot^\B$;
  \item
    $g\circ h = h \circ f$.
  \end{itemize}
  Докажете, че $\lfp(g) = h(\lfp(f))$.
\end{problem}
\ifhints
\begin{hint}
  \begin{itemize}
  \item 
    Разгледайте непрекъснатото свойство 
    \[P_1(a) \dfff h(a) \sqsubseteq g(h(a)).\]
    Докажете с правилото на Скот, че $P_1(\lfp(f))$.
    Тогава
    \begin{align*}
      h(\lfp(f)) \sqsubseteq g(h(\lfp(f)) & \iff h(f(\lfp(f))) \sqsubseteq g(h(\lfp(f)\\
                                          & \iff h(f(\lfp(f))) \sqsubseteq h(f(\lfp(f)))\\
                                          & \iff g(h(\lfp(f))) \sqsubseteq h(\lfp(f)).
    \end{align*}
    Това означава, че $h(\lfp(f))$ е преднеподвижна точка на $g$, т.е.
    \[h(\lfp(f)) \in \texttt{Pref}(g).\]
    Заключаваме, че $\lfp(g) \sqsubseteq h(\lfp(f))$.
  \item
    Другата посока е по-лесна. Разгледайте непрекъснатото свойство
    \[P_2(a) \dfff h(a) \sqsubseteq \lfp(g).\]
    Докажете, че $P_2(\lfp(f))$.
  \end{itemize}
\end{hint}
\fi

\begin{problem}
  Нека $f,g \in \Cont{\A}{\A}$ като имаме свойствата:
  \begin{itemize}
  \item
    $f(\bot) = g(\bot)$;
  \item
    $f \circ g = g \circ f$.
  \end{itemize}
  Докажете, че $\lfp(f \circ g) = \lfp(g \circ f)$.
\end{problem}
\ifhints
\begin{hint}
  Разгледайте непрекъснатото свойство
  \marginpar{Лесно се вижда, че $P$ е непрекъснато свойство, защото $f$ и $g$ са непрекъснати изображения.}
  \[P(a) \dfff g(f(a)) \sqsubseteq a.\]
  Ясно е, че $P(\bot)$.
  Нека $P(a)$. Ще докажем, че $P(f(g(a))$.
  \begin{align*}
    g(f(a)) \sqsubseteq a & \implies g(f(g(f(a)))) \sqsubseteq g(f(a))\\
    & \implies g(f(f(g(a)))) \sqsubseteq f(g(a))\\
    & \implies P(f(g(a))).
  \end{align*}
  От правилото на Скот заключваме, че $P(\lfp(f\circ g))$.
  Това означава, че 
  \[g(f(\lfp(f\circ g))) \sqsubseteq \lfp(f\circ g),\] т.е.
  $\lfp(f\circ g) \in \texttt{Pref}(g \circ f)$.
  Следователно,
  \[\lfp(g \circ f) \sqsubseteq \lfp(f\circ g).\]

  За другата посока разсъждаваме аналогично.
\end{hint}
\fi


\begin{problem}
  Нека $p \in \Cont{\A}{\Nat_\bot}$ и $h \in \Cont{\A}{\A}$, като $h$ е точна, т.е. $h(\bot) = \bot$.
  Да разгледаме 
  \[\Gamma \in \Cont{\Cont{\A\times\A}{\A}}{\Cont{\A\times\A}{\A}},\]
  където
  \begin{align*}
    \Gamma(f)(x,y) =
    \begin{cases}
      y, & p(x) = 0\\
      h(f(h(x),y)), & p(x) \in \Nat^+\\
      \bot, & p(x) = \bot.
    \end{cases}
  \end{align*}
  Докажете, че ако $f_\Gamma \dff \lfp(\Gamma)$, то
  \[(\forall a,b\in\A)[h(f_\Gamma(a,b)) = f_\Gamma(a,h(b))].\]
\end{problem}
\ifhints
\begin{hint}
  Разгледайте непрекъснатото свойство
  \[P(g) \dfff (\forall a,b\in\A)[h(g(a,b)) = g(a,h(b))].\]
\end{hint}
\fi

\begin{problem}
  Нека $p \in \Cont{\A}{\Nat_\bot}$ и $h \in \Cont{\A}{\A}$, като $p$ е точна, т.е. $p(\bot) = \bot$.
  Да разгледаме 
  \[\Gamma \in \Cont{\Cont{\A}{\A}}{\Cont{\A}{\A}},\] 
  където:
  \begin{align*}
    \Gamma(f)(x) =
    \begin{cases}
      x, & p(x) = 0\\
      f(f(h(x))), & p(x) \in \Nat^+\\
      \bot, & p(x) = \bot.
    \end{cases}
  \end{align*}
  Докажете, че ако $f_\Gamma \dff \lfp(\Gamma)$, то
  \[(\forall a\in\A)[f_\Gamma(f_\Gamma(a)) = f_\Gamma(a)].\]
\end{problem}
\ifhints
\begin{hint}
  Разгледайте непрекъснатото свойство
  \[P(g) \dfff (\forall a \in \A)[f_\Gamma(g(a)) = g(a)].\]
\end{hint}
\fi

\begin{problem}
  Нека $p \in \Cont{\A}{\Nat_\bot}$ и $h,k \in \Cont{\A}{\A}$, като $h$ е точна, т.е. $h(\bot) = \bot$.
  Да разгледаме $\Gamma_{1,2} \in \Cont{\Cont{\A\times\A}{\A}}{\Cont{\A\times\A}{\A}}$, където:
  \begin{align*}
    & \Gamma_1(f)(x,y) =
    \begin{cases}
      y, & p(x) = 0\\
      h(f(k(x),y)), & p(x) \in \Nat^+\\
      \bot, & p(x) = \bot;\\
    \end{cases}\\
   & \Gamma_2(f)(x,y) =
    \begin{cases}
      y, & p(x) = 0\\
      f(k(x),h(y)), & p(x) \in \Nat^+\\
      \bot, & p(x) = \bot;
    \end{cases}
  \end{align*}
  Докажете, че ако $f_1 \dff \lfp(\Gamma_1)$ и $f_2 = \lfp(\Gamma_2)$, то
  $f_1 = f_2$.
\end{problem}
\ifhints
\begin{hint}
  Разгледайте непрекъснатото изображение $\Delta$, където
  \[\Delta(f,g) = \pair{\Gamma_1(f),\Gamma_2(g)}.\]
  Разгледайте свойството:
  \[P(f,g) \dfff f = g\ \&\ (\forall a,b \in \A)[h(f(a,b))) = f(a,h(b))].\]
  Първо трябва да се съобрази, че това свойство е непрекъснато, което не е трудно.
  Ясно е, че $P(\Omega,\Omega)$.
  Докажете, че $P(f,g) \implies P(\Delta(f,g))$.
\end{hint}
\fi

%%% Local Variables:
%%% mode: latex
%%% TeX-master: "../sep-notes"
%%% End:



% \begin{problem}
%   Операторът $\Gamma:\mathcal{F}_2\rightarrow \mathcal{F}_2$ действа по правилото:
%   \begin{equation*}
%     \Gamma(f)(x,y)\simeq 
%     \begin{cases} 
%       1, & \text{ ако } x + y \text{ е просто},\\
%       f(x+y,y)+1, & \text{ иначе.}
%     \end{cases}
%   \end{equation*}
%   Да се докаже, че:
%   \begin{enumerate}[a)]
%   \item
%     операторът $\Gamma$ е компактен.
%   \item 
%     ако $f_{\Gamma}$ е най-малката неподвижна точка на $\Gamma$, то:
%     \begin{equation*}
%       (\forall x,y \in \Nat)[!f_{\Gamma}(x,y) \Rightarrow x + y.f_\Gamma(x,y) \text{ е просто}].
%     \end{equation*}
%   \end{enumerate}
% \end{problem}
% \begin{solution}
%   \begin{enumerate}[a)]
%   \item
%     Да се убедим, че $\Gamma$ е компактен.
%     \begin{itemize}
%     \item 
%       Първо да видим, че $\Gamma$ е монотонен.
%       Нека $f \subseteq g$. Ще докажем, че $\Gamma(f) \subseteq \Gamma(g)$, т.е.
%       \[(\forall x,y,z\in\Nat)[\Gamma(f)(x,y) \simeq z\ \rightarrow\ \Gamma(g)(x,y) \simeq z].\]
%       И така, нека $\Gamma(f)(x,y) \simeq z$. Гледайки дефиницията на $\Gamma$, трябва да разгледаме два случая:
%       \begin{itemize}
%       \item 
%         ако $x+y$ е просто число, то по дефиницията на $\Gamma$,
%         \[\Gamma(f)(x,y) \simeq 1 \simeq \Gamma(g)(x,y).\]
%       \item
%         ако $x+y$ не е просто число, то 
%         \[\Gamma(f)(x,y) \simeq f(x+y,y)+1 \simeq z.\]
%         Това означава, че съществува число $u$, такова че $f(x+y,y) \simeq u$ и $z = u+1$.
%         Понеже $f \subseteq g$, то $g(x+y,y) \simeq u$.
%         Тогава 
%         \begin{align*}
%           \Gamma(g)(x,y) & \simeq g(x+y,y) + 1 & (\text{от деф. на }\Gamma)\\
%           & \simeq u+1 & (\text{защото }f \subseteq g)\\
%           & \simeq z.
%         \end{align*}
%       \end{itemize}
%       За всички възможни двойки $x$, $y$ доказахме, че ако $!\Gamma(f)(x,y)\simeq z$, то
%       $\Gamma(g)(x,y) \simeq z$.
%       Следователно, $\Gamma$ е монотонен.
%     \item
%       За втората част, нека $\Gamma(f)(x,y) \simeq z$, за някои $f \in \F_2$ и $x,y,z \in \Nat$.
%       Ще докажем, че съществува крайна функция $\theta \subseteq f$, за която $\Gamma(\theta)(x,y) \simeq z$.
%       За целта ще разгледаме два случая за $x$ и $y$.
%       \begin{itemize}
%       \item 
%         $x+y$ е просто число. Тогава $\Gamma(f)(x,y) \simeq 1$. 
%         Да вземем крайната функция $\theta = \emptyset^{(2)}$.
%         Очевидно е, че $\Gamma(\emptyset^{(2)})(x,y) \simeq 1$.
%       \item
%         $x+y$ не е просто число. Тогава 
%         \[\Gamma(f)(x,y) \simeq f(x+y,y)+1 \simeq z.\]
%         Да положим $u \simeq f(x+y,y)$. Тогава $z = u+1$.
%         В този случай, избираме крайната функция $\theta \subseteq f$ да бъде такава, че
%         \begin{align*}
%           \theta(a,b) \simeq 
%           \begin{cases}
%             u, & a = x+y\ \&\ b = y\\
%             \neg!, & \text{ иначе}.
%           \end{cases}
%         \end{align*}
%         Тогава 
%         \begin{align*}
%           \Gamma(\theta)(x,y) & \simeq \theta(x+y,y) + 1 & (\text{от деф. на }\theta)\\
%           & \simeq u + 1 & (\text{от деф. на }\theta)\\
%           & \simeq z.
%         \end{align*}
%       \end{itemize}
%       Така видяхме, че във всички случаи за $x$ и $y$  съществува крайна $\theta \subseteq f$, 
%       за която $\Gamma(\theta)(x,y) \simeq z$.
%     \end{itemize}
%     От всичко по-горе следва, че операторът $\Gamma$ е компактен.
%   \item
%     Да дефинираме свойството $P$ като
%     \[P(f)\ \iff\ (\forall x,y \in \Nat)[!f(x,y) \Rightarrow x + y.f(x,y) \text{ е просто}].\]
%     Това е свойство от тип частична коректност, защото ако положим
%     \begin{align*}
%       I(x,y) & \equiv\ x,y\in\Nat,\\
%       O(x,y,r) & \equiv\ x + yr\text{ е просто число},
%     \end{align*}
%     то можем да представим $P$ като
%     \[P(f)\ \equiv\ (\forall x,y)[I(x,y)\ \&\ !f(x,y)\ \Rightarrow\ O(x,y,f(x,y))].\]
%     Щом $P$ е от тип частична коректност, то $P$ е непрекъснато свойство.
%     Понеже $\Gamma$ е компактен оператор, а $P$ е непрекъснато, можем да приложим индукционното
%     правило на Скот. Така ще докажем, че $P(f_\Gamma)$.
%     \begin{itemize}
%     \item 
%       \marginpar{няма $x,y$, за които $!\emptyset^{(2)}(x,y)$}
%       $P(\emptyset^{(2)})$ е изпълнено, защото лява страна на импликацията в дефиницията на $P$
%       винаги е неистина и следователно импликацията винаги е истина.
%     \item
%       Нека приемем, че за някое $f \in \F_2$ е изпълнено $P(f)$.
%       Ще докажем, че $P(\Gamma(f))$, т.е.
%       \[(\forall x,y \in \Nat)[!\Gamma(f)(x,y) \Rightarrow x + y.\Gamma(f)(x,y) \text{ е просто}]\]
%       Нека $!\Gamma(f)(x,y)$. Отново ще разгледаме два случая за $x$ и $y$.
%       \begin{itemize}
%       \item 
%         $x+y$ е просто число. Тогава от дефиницията на $\Gamma$ имаме, че $\Gamma(f)(x,y) \simeq 1$
%         и е ясно, че \[x + y.\Gamma(f)(x,y) \simeq x+y.1 = x+y\] е просто число.
%       \item
%         $x+y$ не е просто число. Тогава $\Gamma(f)(x,y) \simeq f(x+y,y)+1$.
%         Да положим $u \simeq f(x+y,y)$. Получаваме, че:
%         \begin{align*}
%           x + y.\Gamma(f)(x,y) & \simeq x+y.(f(x+y,y)+1) & (\text{от деф. на }\Gamma)\\
%           & \simeq x+y.(u+1) \\
%           & \simeq (x+y) + y.u
%         \end{align*}
       
%         Сега използваме, че $P(f)$ е изпълнено. Тогава:
%         \[!f(x+y,y) \simeq u\ \Rightarrow\ (x+y) + y.u+ \text{ е просто число}. \]
%         Заключаваме, че 
%         \[x + y.\Gamma(f)(x,y) \simeq (x+y) + y.u\]
%         е просто число.
%       \end{itemize}      
%     \end{itemize}
%     Разгледахме всички случаи за $x$ и $y$, и следователно, $P(\Gamma(f))$ е изпълнено.    
%     От индукционното правило на Скот получаваме, че $P(f_\Gamma)$, което ни дава точно това, което 
%     трябваше да докажем.
%   \end{enumerate}
% \end{solution}

% \begin{problem}
%   Да рагледаме оператора $\Gamma:\F_3 \to \F_3$, където:
%   \begin{align*}
%     \Gamma(f)(x,y,z) \simeq 
%     \begin{cases}
%       y, & x = 0\\
%       f(x-1, y+2z, y), & x > 0.
%     \end{cases}
%   \end{align*}
%   Докажете, че 
%   \[(\forall x,u\in\Nat)[u\geq 1\ \&\ !f_\Gamma(x,2^u,2^{u-1})\ \Rightarrow\ f_\Gamma(x,2^u,2^{u-1}) \simeq 2^{x+u}].\]
% \end{problem}
% \begin{hint}
%   Лесно се съобразява, че $\Gamma$ е компактен (непрекъснат) оператор.
%   Целта ни е да дефинираме непрекъснато свойство $P$, за което да докажем с индукционното
%   правило на Скот, че $P(f_\Gamma)$.
%   Разгледайте правилото:
%   \begin{align*}
%     P(f)\ \equiv\ (\forall x,y,z\in\Nat)[& (\exists u \geq 1)[y = 2^u\ \&\ z = 2^{u-1}]\ \&\ !f(x,y,z) \Rightarrow\ \\
%     & (\exists u \geq 1)[y = 2^u\ \&\ z = 2^{u-1}\ \&\ f(x,y,z) \simeq 2^{x+u}]].
%   \end{align*}
%   $P$ е свойство от тип частична коректност, защото използвайки предикатите:
%   \begin{align*}
%     I(x,y,z) & \equiv x,y,z\in\Nat\ \&\ (\exists u \geq 1)[y = 2^u\ \&\ z = 2^{u-1}],\\
%     O(x,y,z,r) & \equiv (\exists u \geq 1)[y = 2^u\ \&\ z = 2^{u-1}\ \&\ r = 2^{x+u}],
%   \end{align*}
%   можем да представим свойството $P$ по следния начин:
%   \[P(f) \equiv  (\forall x,y,z)[I(x,y,z)\ \&\ !f(x,y,z) \Rightarrow\ O(x,y,z,f(x,y,z))].\]
  
%   \marginpar{Очевидно е, че $P(\emptyset^{(3)})$}
%   Да приемем, че имаме $P(f)$. Ще докажем $P(\Gamma(f))$.
%   Нека $x,y,z\in\Nat$ са такива, че $!\Gamma(f)(x,y,z)$ и да фиксираме $u \geq 1$, за което $y = 2^u$, $z = 2^{u-1}$.
%   Ще докажем, че $\Gamma(f)(x,y,z) \simeq 2^{x+u}$.
%   Според дефиницията на $\Gamma$, трябва да разгледаме два случая.
%   \begin{itemize}
%   \item 
%     $x = 0$. Тогава $\Gamma(x,y,z) \simeq y = 2^{u+0}$.
%   \item
%     $x > 0$. Тогава $\Gamma(x,y,z) \simeq f(x-1,y+2z,y)$.
%     Понеже $y = 2^u$ и $z = 2^{u-1}$, $y+2z = 2^u+2.2^{u-1} = 2^{u+1}$.
%     Имаме, че $I(x-1,y+2z,y)$ е истина и $!f(x-1,y+2z,y)$.
%     От $P(f)$ следва, че $O(x-1,y+2z,y,f(x-1,y+2z,y)$, т.е. 
%     \[y+2z = 2^{u+1}\ \&\ y = 2^{(u+1)-1}\ \&\ f(x-1,y+2z,y) \simeq 2^{(x-1)+(u+1)}.\]
%     Като обединим всичко, което знаем до момента, получаваме:
%     \begin{align*}
%       \Gamma(f)(x,y,z) & \simeq f(x-1,y+2z,y) & (\text{от деф. на }\Gamma)\\
%       & \simeq f(x-1,2^{u+1},2^{u}) & (y+2z = 2^{u+1},y=2^{(u+1)-1})\\
%       & \simeq 2^{(x-1)+(u+1)} & (\text{от }P(f))\\
%       & = 2^{x+u}
%     \end{align*}
%     Заключаваме, че $O(x,y,z,\Gamma(f)(x,y,z))$, откъдето следва, че $P(\Gamma(f))$.
%   \end{itemize}
%   От правилото на Скот следва, че $P(f_\Gamma)$.
% \end{hint}




% \begin{problem}
%   Операторът $\Gamma:\F_1 \to \F_1$ е зададен с равенството:
%   \begin{align*}
%     \Gamma(f)(x) \simeq
%     \begin{cases}
%       \sqrt{x}, & \text{ ако } x \text{ е точен квадрат}\\
%       f(f(x(x+5))), & \text{ иначе}.
%     \end{cases}
%   \end{align*}
%   Докажете, че:
%   \begin{enumerate}[a)]
%   \item 
%     операторът $\Gamma$ е компактен;
%   \item
%     $(\forall x\in\Nat)[!f_\Gamma(x)\ \&\ x\text{ не е точен квадрат}\ \Rightarrow\ f_\Gamma(x) > \sqrt{x}]$.
%   \end{enumerate}
% \end{problem}

% \begin{problem}
%   Нека $g:\Nat \to \Nat$ е тотална функция, за която $(\forall x\in\Nat)[g(x) \leq x]$.
%   Операторът $\Gamma:\F_2 \to \F_2$ е зададен с равенството:
%   \begin{align*}
%     \Gamma(f)(x,y) = 
%     \begin{cases}
%       g(y), & \text{ ако } x = 0\\
%       f(f(x-1,g(y-1)), y-1)+ 1, & \text{ иначе}.
%     \end{cases}
%   \end{align*}
%   Докажете, че:
%   \begin{enumerate}[a)]
%   \item 
%     операторът $\Gamma$ е компактен;
%   \item
%     $(\forall x,y\in\Nat)[!f_\Gamma(x,y)\ \Rightarrow\ f_\Gamma(x,y) \leq \max(x,y)]$.
%   \end{enumerate}
% \end{problem}

% \begin{problem}
%   Нека $g:\Nat \to \Nat$ е тотална функция, за която $(\forall x\in\Nat)[g(x) \leq x]$.
%   Операторът $\Gamma:\F_2 \to \F_2$ е зададен с равенството:
%   \begin{align*}
%     \Gamma(f)(x,y) = 
%     \begin{cases}
%       g(y), & \text{ ако } x = 0\\
%       f(x-1, f(x-1, g(y-1))) + 1, & \text{ иначе}.
%     \end{cases}
%   \end{align*}
%   Докажете, че:
%   \begin{enumerate}[a)]
%   \item 
%     операторът $\Gamma$ е компактен;
%   \item
%     $(\forall x,y\in\Nat)[!f_\Gamma(x,y)\ \Rightarrow\ f_\Gamma(x,y) > \max(x,y)]$.
%   \end{enumerate}
% \end{problem}

\begin{problem}
  Операторът $\Gamma:\F_2 \to \F_2$ е зададен с равенството:
  \begin{align*}
    \Gamma(f)(x,y) = 
    \begin{cases}
      y, & \text{ ако } y\vert x\\
      f(f(x,y+1)), x), & \text{ иначе}.
    \end{cases}
  \end{align*}
  Докажете, че:
  \begin{enumerate}[a)]
  \item 
    операторът $\Gamma$ е компактен;
  \item
    $(\forall x,y\in\Nat)[!f_\Gamma(x,y)\ \&\ y\not| x\ \Rightarrow\ f_\Gamma(x,y)\ \vert\ x]$.
  \end{enumerate}  
\end{problem}

\begin{problem}
  Операторът $\Gamma:\F_1 \to \F_1$ е зададен с равенството:
  \begin{align*}
    \Gamma(f)(x) = 
    \begin{cases}
      1, & \text{ ако } x \leq 1\\
      x/2, & \text{ ако } 2\vert x\ \&\ x > 1\\
      f(f(3\lfloor{x/2}\rfloor+2)), & \text{ иначе}.
    \end{cases}
  \end{align*}
  Докажете, че:
  \begin{enumerate}[a)]
  \item 
    операторът $\Gamma$ е компактен;
  \item
    $(\forall x\in\Nat)[!f_\Gamma(x)\ \&\ x > 1\ \Rightarrow\ f_\Gamma(x)\ \leq x/2]$.
  \end{enumerate}  
\end{problem}

% \begin{problem}
%   Операторът $\Gamma:\F_2 \to \F_2$ е зададен с равенството:
%   \begin{align*}
%     \Gamma(f)(x,y) \simeq
%     \begin{cases}
%       y, & x = 0\\
%       f(x-1,2) + y, & x \equiv 1\ (\bmod\ 2)\\
%       f(\frac{x}{2},f(\frac{x}{2},y)), & \text{ иначе}.
%     \end{cases}
%   \end{align*}
%   Докажете, че:
%   \begin{enumerate}[a)]
%   \item 
%     операторът $\Gamma$ е компактен;
%   \item
%     $(\forall x\in\Nat)[!f_\Gamma(x,0)\ \Rightarrow\ f_\Gamma(x,0) \simeq 2x]$.
%   \end{enumerate}  
% \end{problem}
% \begin{hint}
%   Разгледайте свойството
%   \[P(f) \equiv (\forall x,y\in\Nat)[!f(x,y)\ \Rightarrow\ f(x,y) \simeq 2x+y].\]
% \end{hint}

% \begin{problem}
%   Да разгледаме оператора $\Gamma:\mathcal{F}_1\to \mathcal{F}_1$, където:
%   \begin{align*}
%     \Gamma(f)(x) \simeq 
%     \begin{cases}
%       x/3, & \text{ ако }x \equiv 0\ (\bmod\ 3)\\
%       f(3f(x+1)\dotminus 1), & \text{ ако }x \equiv 1\ (\bmod\ 3)\\
%       f(3f(2x-1)+1), & \text{ ако }x \equiv 2\ (\bmod\ 3),
%     \end{cases}
%   \end{align*}
%   където $x\dotminus y = 0$, ако $x < y$, и $x \dotminus y = x-y$, ако $x \geq y$.
%   \begin{enumerate}[a)]
%   \item 
%     Докажете, че операторът $\Gamma$ е компактен.
%   \item
%     Докажете, че
%     $(\forall x\in\Nat)[!f_\Gamma(x) \implies f_\Gamma(x) \leq x/3]$,
%     където с $f_\Gamma$ означаваме най-малката неподвижна точка на оператора $\Gamma$.
%   \end{enumerate}
% \end{problem}
% \begin{hint}
%   \begin{enumerate}[1)]
%   \item 
%     Най-лесно се решава задачата като намерете явния вид на $f_\Gamma$ с теоремата на Кнастер-Тарски.
%     Оказва се, че 
%     \begin{align*}
%       f_\Gamma(x) \simeq 
%       \begin{cases}
%         x/3, & x \equiv 0\ (\bmod\ 3)\\
%         \neg!, & \text{ иначе}.
%       \end{cases}
%     \end{align*}
    
%     Знаем, че $f_0 = \emptyset^{(1)}$, $f_{i+1} = \Gamma(f_i)$ и $f_\Gamma = \bigcup_i f_i$.
%     \begin{align*}
%       f_1(x) \simeq \Gamma(f_0)(x) \simeq\ & \Gamma(\emptyset^{(1)})(x) \simeq 
%       \begin{cases}
%         x/3, & x \equiv 0\ (\bmod\ 3)\\
%         \neg!, & \text{иначе}\\
%       \end{cases}\\
%       f_2(x) \simeq \Gamma(f_1)(x) \simeq &
%       \begin{cases}
%         x/3, & x \equiv 0\ (\bmod\ 3)\\
%         f_1(3f_1(x+1)\dotminus 1), & x \equiv 1\ (\bmod\ 3)\\
%         f_1(3f_1(2x-1) + 1), & x \equiv 2\ (\bmod\ 3)\\
%       \end{cases}\\
%       \simeq &
%       \begin{cases}
%         x/3, & x \equiv 0\ (\bmod\ 3)\\
%         \neg!, & x \equiv 1\ (\bmod\ 3)\\
%         f_1(3\frac{2x-1}{3} + 1), & x \equiv 2\ (\bmod\ 3)\\
%       \end{cases}\\
%       \simeq &
%       \begin{cases}
%         x/3, & x \equiv 0\ (\bmod\ 3)\\
%         \neg!, & x \equiv 1\ (\bmod\ 3)\\
%         f_1(2x), & x \equiv 2\ (\bmod\ 3)\\
%       \end{cases}\\
%       \simeq &
%       \begin{cases}
%         x/3, & x \equiv 0\ (\bmod\ 3)\\
%         \neg!, & \text{иначе}.
%       \end{cases}
%     \end{align*}
%     Виждаме, че $f_1 = f_2$. 
%     Заключаваме, че за всяко $k \geq 1$, $f_k = f_1$.
%     Тогава $f_\Gamma = f_1$.
%   \item
%     Ако искате да използвате правилото на Скот, възможно е да разгледате непрекъснатото свойство
%     \begin{align*}
%       P(f) \equiv\ & (\forall x\in\Nat)[x \equiv 0\ (\bmod\ 3)\ \&\ !f(x) \implies f(x) \simeq x/3]\ \&\ \\
%       & (\forall x\in\Nat)[x \equiv 1\ (\bmod\ 3)\ \&\ !f(x) \implies f(x) \leq x/12]\ \&\ \\
%       & (\forall x\in\Nat)[x \equiv 2\ (\bmod\ 3)\ \&\ !f(x) \implies f(x) \leq x/6].
%     \end{align*}    
%   \end{enumerate}
% \end{hint}

% \begin{problem}
%   Нека предварително имаме дадена частичната функция $h \in \F_2$.
%   Операторът $\Gamma:\F_2 \to \F_2$ е зададен с равенството:
%   \begin{align*}
%     \Gamma(f)(x,y) \simeq
%     \begin{cases}
%       0, & h(x,y) \simeq 0\\
%       f(x, y+1) + 1, & h(x,y) > 0\\
%       \neg !, & \neg ! h(x,y)
%     \end{cases}
%   \end{align*}  
%   Докажете, че 
%   \[(\forall x,y,z\in\Nat)[!f_\Gamma(x,y) \simeq z \implies h(x, y+z) \simeq 0\ \&\ (\forall t < z)[h(x,y+t) > 0]]\]
% \end{problem}

% \newpage

\section{Задачи}

\begin{problem}
  Да фиксираме $a_0 \in \A$ и да разгледаме $P \subseteq \Cont{\A}{\A}$, където
  \[P(f) \dfff \lfp(f) = a_0.\]
  Проверете дали $P$ е непрекъснато свойство.
\end{problem}


\begin{problem}
  Даден е следния оператор $\Gamma:\F_1 \to \F_1$:
  \begin{align*}
    \Gamma(f)(x,y) \simeq
    \begin{cases}
      3.f(\sqrt{x},y) + 2, & \text{ ако $x$ е точен квадрат}\\
      y, & \text{ ако $x$ не е точен квадрат}\\
    \end{cases}
  \end{align*}
  Да се докаже, че:
  \begin{enumerate}[a)]
  \item
    операторът $\Gamma$ е компактен.
  \item 
    ако $f_{\Gamma} = \lfp(\Gamma)$, то:
    \begin{equation*}
      (\forall x,y \in \Nat)[3.f_\Gamma(x,y) + 2 \simeq f_\Gamma(x,3y+2)].
    \end{equation*}
  \end{enumerate}
\end{problem}
\ifhints
\begin{hint}
  Подточка а) е стандартна.
  \begin{itemize}
  \item 
    Нека първо да разгледаме операторите $\Gamma_1$ и $\Gamma_2$, където
    \begin{align*}
      & \Delta_1(f) \dff 3f(x,y)+2\\
      & \Delta_2(f) \dff f(x,3y+2).
    \end{align*}
    Съобразете, че те са непрекъснати.
  \item
    От \Prop{continuous-property} следва, че свойството 
    \[P(f) \dfff \Delta_1(f) = \Delta_2(f)\]
    е непрекъснато.
  \item
    Използвайте правилото на Скот върху $P$ 
    за да докажете, че $P(f_\Gamma)$.
    Това означава, че $3f_\Gamma(x,y) + 2 \simeq f_\Gamma(x,3y+2)$ за всяко $x,y \in \Nat$.
  \end{itemize}  
\end{hint}
\fi

\begin{problem}
  Даден е следния оператор $\Gamma:\F_1 \to \F_1$:
  \begin{align*}
    \Gamma(f)(x,y) \simeq
    \begin{cases}
      (f(\sqrt{x},y))^2, & \text{ ако $x$ е точен квадрат}\\
      y, & \text{ иначе }
    \end{cases}
  \end{align*}
  Да се докаже, че:
  \begin{enumerate}[a)]
  \item
    операторът $\Gamma$ е компактен.
  \item 
    ако $f_{\Gamma}$ е най-малката неподвижна точка на $\Gamma$, то:
    \begin{equation*}
      (\forall x,y \in \Nat)[(f_\Gamma(x,y))^2 \simeq f_\Gamma(x,y^2)].
    \end{equation*}
  \end{enumerate}
\end{problem}

\begin{problem}
  Нека $p_0,p_1,p_2\dots\ $ е редицата от всички прости числа в нарастващ ред.
  Операторът $\Gamma: \F_3\to\F_3$ действа по правилото:
  \begin{align*}
    \Gamma(f)(x,y,z) \simeq
    \begin{cases}
      x^xy, & \text{ ако }p_z = x\ \&\ x,y,z\in\Nat\\
      f(x+x,y,z+2), & \text{ ако }p_z = x\ \&\ x,y,z\in\Nat.
    \end{cases}
  \end{align*}
  Да се докаже, че:
  \begin{enumerate}[a)]
  \item
    операторът $\Gamma$ е компактен.
  \item 
    ако $f_{\Gamma} = \lfp(\Gamma)$, то:
    \begin{equation*}
      (\forall x,y,z \in \Nat)[!f_{\Gamma}(x,y,z) \implies (\exists\text{ просто число }p)[p \geq x\ \&\ p^py\ |\ f_\Gamma(x,y,z)].
    \end{equation*}
  \end{enumerate}
\end{problem}

\begin{problem}
  Операторът $\Gamma:\F_3 \to \F_3$ е зададен с равенството:
  \begin{align*}
    \Gamma(f)(x,y,z) =
    \begin{cases}
      z, & y = 0\ \&\ x,z\in\Nat\\
      f(x,y-1, xy+z), & y > 0\ \&\ x,z \in \Nat.
    \end{cases}
  \end{align*}
  \begin{enumerate}[a)]
  \item 
    Докажете, че $\Gamma$ е компактен;
  \item
    Докажете, че ако $f_\Gamma = \lfp(\Gamma)$, то
    \[(\forall x,y\in\Nat)[f_\Gamma(x,y,0) = \frac{xy(y+1)}{2}].\]
  \end{enumerate}
\end{problem}
\begin{hint}
  Да разгледаме свойството над ествествените числа
  \[P(y) \dfff (\forall x,z\in\Nat)[f_\Gamma(x,y,z) = \frac{xy(y+1)}{2} + z].\]
  Докажете с математическа индукция по $y \in \Nat$, че $(\forall y\in\Nat)[P(y)]$.
  % \begin{itemize}
  % \item 
  %   Нека $y = 0$. Тогава за произволни $x$ и $z$,
  %   \begin{align*}
  %     f_\Gamma(x,0,z) & = \Gamma(f_\Gamma)(x,0,z) \\
  %     & = z.
  %   \end{align*}
  % \item
  %   Нека $y > 0$. Тогава 
  %   \begin{align*}
  %     f_\Gamma(x,y,z) & = \Gamma(f_\Gamma)(x,y,z)\\
  %                     & = f_\Gamma(x,y-1,xy+z) & (\text{от деф. на }\Gamma)\\
  %                     & = \frac{x(y-1)(y-1+1)}{2} + xy + z & (\text{от И.П. за }y-1)\\
  %                     & = \frac{xy(y-1)+2xy}{2} + z \\
  %                     & = \frac{xy(y+1)}{2} + z.
  %   \end{align*}
  % \end{itemize}
\end{hint}

% \begin{hint}
%   \marginpar{Ясно е, че $\Gamma$ е компактен}
%   Да разгледаме свойството
%   \[P(f)\ \dfff\ (\forall x,y,z\in\Nat_\bot)[f(x,y,z) \neq \bot\ \to\ f(x,y,z) = \frac{xy(y+1)}{2}+z].\]
%   Лесно се вижда, че това е свойство от тип частична коректност, защото ако положим
%   \begin{align*}
%     I(x,y,z) & \dfff\ \texttt{True},\\
%     O(x,y,z,r) & \dfff\ r = \frac{xy(y+1)}{2},
%   \end{align*}
%   то можем да представим $P$ във вида:
%   \[P(f)\ \equiv\ (\forall x,y,z\in\Nat_\bot)[I(x,y,z)\ \&\ f(x,y,z) \neq \bot\ \to\ O(x,y,z,f(x,y,z))].\]
  
%   Да приложим правилото на Скот.
%   \marginpar{Очевидно е, че $P(\Omega^{(3)})$}
%   Нека е изпълнено $P(f)$. Ще докажем, че $P(\Gamma(f))$.
%   И така, нека да разгледаме елементи $x,y,z \in \Nat_\bot$, за които $\Gamma(f)(x,y,z) \neq \bot$. 
%   Ясно е от дефиницията на оператора $\Gamma$, че $\bot \not\in \{x,y,z\}$.
%   Ще разгледаме два случая.
%   \begin{itemize}
%   \item 
%     Нека $y = 0$. Тогава $\Gamma(f)(x,y,z) = z = \frac{xy(y+1)}{2} + z$.
%   \item
%     Нека $y > 0$. Тогава 
%     \begin{align*}
%       \Gamma(f)(x,y,z) & = f(x,y-1,xy+z) & (\text{от деф. на }\Gamma)\\
%       & = \frac{x(y-1)(y-1+1)}{2} + xy + z & (\text{от }P(f))\\
%       & = \frac{xy(y-1)+2xy}{2} + z \\
%       & = \frac{xy(y+1)}{2} + z.
%     \end{align*}
%   \end{itemize}
%   Заключаваме, че $P(\Gamma(f))$, откъдето следва, по правилото на Скот, че $P(\lfp(\Gamma))$.
  
%   Накрая, за произволни $x,y\in\Nat$ и $z = 0$, получаваме, че 
%   \[f_\Gamma(x,y,0) = \frac{xy(y+1)}{2}.\]
% \end{hint}

\begin{remark}
Да разгледаме програмата на езика \REC:

\begin{haskellcode}
h(x,y) = f(x,y,0) where
  f(x,y,z) = if y == 0 then z
               else f(x, y-1, x*y + z)
\end{haskellcode}

Съобразете, че ние горе на практика доказахме, че
\[(\forall x,y \in \Nat)[\D_V\val{\vv{h}}(x,y) = \frac{xy(y+1)}{2}].\]
\end{remark}


% \begin{problem}
%   Нека е дадена програмата на езика хаскел:

%   \begin{minted}[frame=lines,framesep=2mm,baselinestretch=1.2]{haskell}
%     rev :: [a] -> [a]
%     rev x = f(x, []) where 
%       f([], y) = y
%       f(x:xs, y) = f(xs, x:y)
%   \end{minted}

%   \noindent 
%   Докажете, че:
%   \begin{enumerate}[a)]
%   \item 
%     $rev:\Sigma^\star \to \Sigma^\star$ е тотална.
%   \item
%     $(\forall x \in \Sigma^\star)[rev(rev(x)) = x]$.
%   \item
%     $(\forall x \in \Sigma^\star)[rev(x) = x^R]$.
%   \end{enumerate}
% \end{problem}
% \begin{hint}
%   % Ще използваме следното правило:
%   % \begin{prooftree}
%   %   \AxiomC{$P(\varepsilon)$}
%   %   \AxiomC{$(\forall x \in \Sigma^\star)[x\neq\varepsilon\ \&\ P(cdr(x)) \to P(x)]$}
%   %   \RightLabel{\scriptsize(1)}
%   %   \BinaryInfC{$(\forall x\in\Sigma^\star)[P(x)]$}
%   % \end{prooftree}
%   % % \item
%   %   Докажете валидността на правилото $(1)$. % е еквивалентно на структурна индукция върху фундираната наредба
%     % $(\Sigma^\star,\prec)$, където $x \prec y \iff (\exists z\in\Sigma^\star)[z\cdot x = y]$, т.е.
%     % $x$ е суфикс на $y$.
%   Да разгледаме фундираната наредба $L = (\Sigma^\star, \prec)$, където
%   $x \prec y \iff \abs{x} < \abs{y}$.
%   \begin{enumerate}[a)]
%   \item 
%     Да разгледаме свойството 
%     \[P(x) \equiv (\forall y\in \Sigma^\star)[f(x,y)\text{ е дефинирана}].\]
%     Докажете със структурна индукция по $L$, че $(\forall x\in\Sigma^\star)[P(x)]$.
%   \item
%     Да разгледаме свойството 
%     \[P(x) \equiv (\forall y\in \Sigma^\star)[rev(f(x,y)) = f(y,x)].\]
%     Докажете със структурна индукция по $L$, че $(\forall x\in\Sigma^\star)[P(x)]$.
%     Тогава в частния случай $y = \varepsilon$, 
%     \[rev(rev(x)) = rev(f(x,\varepsilon)) = f(\varepsilon,x) = x.\]
%   \item
%     Разгледайте свойството
%     \[P(x) \equiv (\forall y\in\Sigma^\star)[f(x,y) = x^R \cdot y].\]
%   \end{enumerate}
% \end{hint}

% \section{Частична коректност}

\begin{problem}
  Дадена е следната програма на езика \REC:
  
  \begin{haskellcode}
h(x,y) = f(x,y,1) where
  f(x,y,z) = if x <= 1 then z 
               else f(x-1, y, z*g(x,y))
  g(x,y) = if y == 0 then 1 
             else x * g(x, y - 1)
  \end{haskellcode}
  
  Докажете, че $(\forall x,y\in\Nat)[\D_V\val{\vv{h}}(x,y) \neq \bot \implies \D_V\val{\vv{h}}(x,y) = (x!)^y]$.
\end{problem}
% \begin{solution}
%   \Stefan{В главата за REC означавах операторите с $\Delta$}
%   \marginpar{Понеже разглеждаме денотационна семантика по стойност, то работим с {\bf точни} изображения}
%   Да разгледаме непрекъснатите термални оператори 
%   \begin{align*}
%     & \Delta_1:\S_3\times\S_2 \to \S_3\\
%     & \Delta_2:\S_3\times\S_2 \to \S_2,
%   \end{align*}
%   съответстващи на термовете дефиниращи \vv{f} и \vv{g}:
%   \begin{align*}
%     \Delta_1(f,g)(x,y,z) = &
%     \begin{cases}
%       z & \text{, ако } x \leq 1\\
%       f(x-1,y,z\cdot g(x,y)) &  \text{, ако } x > 1\\
%       \bot & \text{, ако }\bot \in \{x,y,z\}
%     \end{cases}
%     \\
%     \Delta_2(f,g)(x,y) = &
%     \begin{cases}
%       1 & \text{, ако } y = 0\\
%       x \cdot g(x,y-1) & \text{, ако }y > 0\\
%       \bot & \text{, ако } \bot \in \{x,y\}.
%     \end{cases}
%   \end{align*}
%   Тогава операторът $\Delta:\S_3\times\S_2 \to \S_3\times\S_2$, дефиниран като:
%   \marginpar{Знаем, че $\delta_1, \delta_2$ е най-малкото решение на системата от уравнения:
%     \begin{align*}
%       & \Delta_1(f,g) = f\\
%       & \Delta_2(f,g) = g
%     \end{align*}}
%   \[\Delta(f,g) = (\Delta_1(f,g),\Delta_2(f,g)),\] също е непрекъснат.
%   Ако означим $(\delta_1,\delta_2) = \lfp(\Delta)$, то по дефиниция
%   % \marginpar{Следвайки дефинициите, $\D_V\val{\vv{h}}(x,y) \simeq \delta_1(x,y,1)$}
%   \[\D_V\val{\vv{h}}(x,y) = \delta_1(x,y,1).\]
%   Сега дефинираме следните свойства:
%   \begin{align*}
%     & P_1(f,g) \dfff (\forall x,y,z\in\Nat)[f(x,y,z) \neq \bot \implies f(x,y,z) = z.(x!)^y],\\
%     & P_2(f,g) \dfff (\forall x,y \in \Nat)[g(x,y) \neq \bot \implies g(x,y) = x^y].
%   \end{align*}
%   Понеже те са от тип частична коректност, те са и непрекъснати.
%   Ще докажем с индукционно правило на Скот, приложено върху областта на Скот $\S_3 \times \S_2$ за непрекъснатото изображение $\Gamma$,
%   че $P(\delta_1,\delta_2)$, където 
%   \[P(f,g) \dfff P_1(f,g)\ \&\ P_2(f,g).\]
%   $P$ също е непрекъснато свойство, защото е конюнкция на две непрекъснати свойства.

%   Очевидно е, че $P(\Omega^{(3)},\Omega^{(2)})$.
%   Сега да приемем, че $P(f,g)$ е изпълнено. Ще докажем $P(\Delta(f,g))$.
%   \begin{enumerate}[1)]
%   \item 
%     Ще докажем, че е изпълнено $P_1(\Delta(f,g))$, т.е.
%     \[(\forall x,y,z\in\Nat)[\Delta_1(f,g)(x,y,z) \neq \bot \implies \Delta_1(f,g)(x,y,z) = z.(x!)^y].\]
%     Нека $\Delta_1(f,g)(x,y,z) \neq \bot$, за произволни $x,y,z \in \Nat$.
%     \begin{itemize}
%     \item 
%       ако $x \leq 1$, то $\Delta_1(f,g)(x,y,z) = z = z.(x!)^y$, което следва от дефиницията на $\Delta_1$.
%     \item
%       ако $x > 1$, тогава имаме, че
%       \[\Delta_1(f,g)(x, y, z) = f(x-1, y, z.g(x,y)) \neq \bot,\]
%       откъдето следва, че $g(x,y) \neq \bot$, защото $f$ е точна функция.
%       Тогава от $P_2(f,g)$ получаваме, че $g(x,y) = x^y$.
%       Обединявайки всичко, получаваме:
%       \begin{align*}
%         \Delta_1(f,g)(x,y,z)&  = f(x-1,y,z.g(x,y)) & (\text{от деф.})\\
%                             & = f(x-1,y,z.x^y) & (\text{от }P_2(f,g))\\
%                             & = (z.x^y).((x-1)!)^y & (\text{от }P_1(f,g))\\
%                             & = z.(x!)^y.
%       \end{align*}
%     \end{itemize}
%   \item
%     Сега ще докажем, че е изпълнено $P_2(\Delta(f,g))$, т.е.
%     \[(\forall x,y\in\Nat)[\Delta_2(f,g)(x,y) \neq \bot \implies \Delta_2(f,g)(x,y) = x^y].\]
%     Нека $\Delta_2(f,g)(x,y) \neq \bot$, за произволни $x,y \in \Nat$.
%     \begin{itemize}
%     \item 
%       ако $y = 0$, то $\Delta_2(f,g)(x,0) = 1 = x^0$, което следва от дефиницията на $\Delta_2$.
%     \item
%       ако $y > 0$, то имаме $\Delta_2(f,g)(x,y) = x \cdot g(x, y-1) \neq \bot$.
%       Оттук следва, че $g(x,y-1) \neq \bot$ и от $P_2(f,g)$ получаваме, че 
%       $g(x,y-1) = x^{y-1}$.
%       Обединявайки всичко, получаваме:
%       \begin{align*}
%         \Delta_2(f,g)(x,y) & = x.g(x,y-1) & (\text{от деф.})\\
%         & = x.x^{y-1} & (\text{от }P_2(f,g))\\
%         & = x^y.
%       \end{align*}
%     \end{itemize}
%   \end{enumerate}
%   Заключаваме, че $P(\Delta(f,g))$.
%   Следователно, $P(\delta_1,\delta_2)$ и в частност, 
%   \begin{align*}
%     \D_V\val{\vv{h}}(x) & = \delta_1(x,y,1) & (\text{от деф.})\\
%     & = 1.(x!)^y & (\text{от }P_1(\delta_1,\delta_2))\\
%     & = (x!)^y.
%   \end{align*}
% \end{solution}

\begin{problem}
  Дадена е следната програма на езика $\REC$:
  
  \begin{haskellcode}
h(x) = g(x,0) where
  f(x,y) = if y == 0 then 2^x
             else if x == y then 3^y
               else 3*f(x - 1, y - 1) + 2*f(x- 1, y)
  g(x,y) = if x < y then 0 
             else g(x, y + 1) + f(x, y)
  \end{haskellcode}
  
  Докажете, че $(\forall x\in \Nat)[\D_V\val{\vv{h}}(x) \neq \bot \implies \D_V\val{\vv{h}}(x) = 5^x]$.
\end{problem}
\begin{hint}
  Използвайте следните свойства:
  \marginpar{$\binom{x}{y}$ - Нютонов бином}
  \begin{itemize}
  \item 
    $5^x = \sum^x_{i=0}3^i2^{x-i}\binom{x}{i}$;
  \item
    $\binom{x}{i} = \binom{x-1}{i-1} + \binom{x-1}{i}$;
  \item
    $3^i2^{x-i}\binom{x}{i} = 3.3^{i-1}2^{x-i}\binom{x-1}{i-1} + 2.3^{i}2^{x-1-i}\binom{x-1}{i}$.
  \end{itemize}
  Тогава приложете правилото на Скот за:
  \begin{align*}
    & P_1(f,g) \dfff (\forall x,y)[f(x,y) \neq \bot\ \&\ x\geq y \implies f(x,y) = 3^y2^{x-y}\binom{x}{y}],\\
    & P_2(f,g) \dfff (\forall x,y)[g(x,y) \neq \bot \implies g(x,y) = \sum^{x}_{i=y}3^{i}2^{x-i}\binom{x}{i}].
  \end{align*}
\end{hint}

Сега ще дадем пълно решение на тази задача.

\begin{solution}
  Да разгледаме следните две непрекъснати изображения, които съответстват на термовете дефиниращи 
  \vv{f} и \vv{g}:
  
  \begin{align*}
    \Delta_1(f,g)(x,y) \dff &
    \begin{cases}
      \bot, & \text{ако } \bot \in \{x,y\}\\
      2^x, & \text{ако } y = 0\\
      3^y, & \text{ако } x = y\\
      2\cdot  f(x-1,y-1)+3\cdot f(x-1,y), & \text{иначе}
    \end{cases}
    \\
    \Delta_2(f,g)(x,y) \dff &
    \begin{cases}
      \bot, & \text{ако } \bot \in \{x,y\}\\
      0, & \text{ако } y > x\\
      g(x,y+1) + f(x,y), & \text{иначе}.
    \end{cases}
  \end{align*}
  Тогава $\Delta:\S_2\times\S_2 \to \S_2\times\S_2$, дефинирано като:
  \[\Delta(f,g) \dff (\Delta_1(f,g),\Delta_2(f,g)),\] също е непрекъснато изображение.
  Ако означим с $(\delta_1,\delta_2) = \lfp(\Delta)$, то по дефиниция
  \[\D_V\val{\vv{h}}(x) = \delta_2(x,0).\]
  Сега дефинираме следните свойства:
  \begin{align*}
    P_1(f,g) & \iff (\forall x,y\in\Nat)[f(x,y) \neq \bot\ \&\ y\leq x \implies f(x,y) = 2^{y}3^{x-y}\binom{x}{y}],\\
    P_2(f,g) & \iff (\forall x,y\in\Nat)[g(x,y) \neq \bot \implies g(x,y) = \sum^{x}_{z=y}2^z3^{x-z}\binom{x}{z}].
  \end{align*}
  Понеже те са от тип частична коректност, те са и непрекъснати.
  Ще докажем с индукционно правило на Скот, приложено върху областта на Скот $\S_2 \times \S_2$ за непрекъснатото изображение $\Delta$,
  че $P(\delta_1,\delta_2)$, където 
  \[P(f,g) \iff P_1(f,g)\ \&\ P_2(f,g).\]
  $P$ също е непрекъснато свойство, защото е конюнкция на две непрекъснати свойства.

  Очевидно е, че $P(\Omega^{(2)},\Omega^{(2)})$.
  Сега да приемем, че $P(f,g)$ е изпълнено. Ще докажем $P(\Delta(f,g))$.
  \begin{enumerate}[1)]
  \item 
    Ще докажем, че $P_1(\Delta(f,g))$, т.е.
    \[(\forall x,y\in\Nat)[\Delta_1(f,g)(x,y) \neq \bot\ \&\ y\leq x \implies \Delta_1(f,g)(x,y) = 2^{y}3^{x-y}\binom{x}{y}].\]
    Нека $\Delta_1(f,g)(x,y) = u \neq \bot$ и $y \leq x$.
    Според дефиницията на $\Delta_1$, трябва да разгледаме три случая:
    \begin{itemize}
    \item 
      \marginpar{$\binom{x}{0} = 1$}
      ако $y = 0$, тогава 
      \[\Delta_1(f,g)(x,y) = 3^x = 2^{0}3^{x-0}\binom{x}{0}.\]
    \item
      \marginpar{$\binom{x}{x} = 1$}
      ако $x = y$, тогава
      \[\Delta_1(f,g)(x,y) = 2^y = 2^{y}3^{x-y}\binom{x}{y}.\]
    \item
      иначе, $0 < y < x$ и 
      \[\Delta_1(f,g)(x,y) = 2.\underbrace{f(x-1,y-1)}_{\neq \bot} + 3.\underbrace{f(x-1,y)}_{\neq \bot} = u \neq \bot.\]
      % Щом $u \neq \bot$, то $f(x-1,y-1) \neq \bot$ и $f(x-1,y) \neq \bot$.
      Понеже е изпълнено $P_1(f,g)$, то имаме, че:
      \begin{align*}
        & f(x-1,y-1) = 2^{y-1}3^{x-1-(y-1)}\binom{x-1}{y-1},\\
        & f(x-1,y) = 2^{y}3^{x-1-y}\binom{x-1}{y}.
      \end{align*}
      Обединявай всичко това, получаваме:
      \begin{align*}
        \Delta_1(f,g)(x,y) & = 2.f(x-1,y-1) + 3.f(x-1,y) & (\text{от деф.})\\
        & = 2.2^{y-1}3^{x-y}\binom{x-1}{y-1} + 3.2^{y}3^{x-1-y}\binom{x-1}{y} & (\text{от }P_1(f,g))\\
        & = 2^y3^{x-y}\binom{x-1}{y-1} + 2^y3^{x-y}\binom{x-1}{y}\\
        & = 2^y3^{x-y}[\binom{x-1}{y-1} + \binom{x-1}{y}]\\
        & = 2^y3^{x-y}\binom{x}{y}.
      \end{align*}
    \end{itemize}
    \item
      Ще докажем, че $P_2(\Delta(f,g))$, т.е.
      \[(\forall x,y\in\Nat)[\Delta_2(f,g)(x,y) \neq \bot \implies \Delta_2(f,g)(x,y) = \sum^{x}_{z=y}2^{z}3^{x-z}\binom{x}{z}].\]
      Нека $\Delta_2(f,g)(x,y) = u \neq \bot$. Според дефиницията на $\Delta_2$, трябва да разгледаме два случая:
      \begin{itemize}
      \item 
        ако $y > x$, то
        \[\Delta_2(f,g)(x,y) = 0 = \sum^{x}_{z=y}2^{z}3^{x-z}\binom{x}{z},\]
        защото сумата от елементите на празното множество е $0$.
      \item
        ако $y \leq x$, то
        \[\Delta_2(f,g)(x,y) = \underbrace{g(x,y+1)}_{\neq \bot} + \underbrace{f(x,y)}_{\neq \bot} = u \neq \bot.\]
        Понеже е изпълнено $P_1(f,g)$ и $P_2(f,g)$, то имаме, че:
        \begin{align*}
          \Delta_2(f,g)(x,y) & = g(x,y+1) + f(x,y) & (\text{от деф. на }\Delta_2)\\
          & = g(x,y+1) + 2^{y}3^{x-y}\binom{x}{y} & (\text{от }P_1(f,g))\\
          & = \sum^{x}_{z=y+1}2^{z}3^{x-z}\binom{x}{z} +  2^{y}3^{x-y}\binom{x}{y} & (\text{от }P_2(f,g))\\
          & = \sum^x_{z=y}2^{z}3^{x-z}\binom{x}{z}.
        \end{align*}
      \end{itemize}
    \end{enumerate}
    Накрая заключаваме, че $P(\Delta(f,g))$.
    Следователно, $P(\delta_1,\delta_2)$ и в частност, 
    \begin{align*}
      D_V\val{\vv{h}}(x) & = \delta_2(x,0) & (\text{от деф.})\\
      & = \sum^x_{z=0}2^z3^{x-z}\binom{x}{z} & (\text{от }P_2(\delta_1,\delta_2))\\
      & = (2+3)^x\\
      & = 5^x.
    \end{align*}
\end{solution}

\begin{problem}
  Дадена е следната програма на езика $\REC$:
  
  \begin{haskellcode}
h(x) = g(x, x) where
  f(x, y) = if y == 0 || x == y then 1
              else f(x - 1, y - 1) + f(x - 1, y)
  g(x, y) = if y == 0 then 1
              else g(x, y - 1) + f(x, y)^2
  \end{haskellcode}
  
  Докажете, че $(\forall x\in \Nat)[\D_V\val{\vv{h}}(x) \neq \bot \implies \D_V\val{\vv{h}}(x) = \binom{2x}{x}]$.
\end{problem}
\begin{hint}
  Използвайте, че:
  \begin{itemize}
  \item
    $\binom{x+y}{z} = \sum^z_{i=0} \binom{x}{i}\binom{y}{z-i}$;
  \item
    $\binom{x}{y} = \binom{x}{x-y}$;
  \item
    $\binom{2x}{x} = \sum^x_{i=0} \binom{x}{i}\binom{x}{x-i} = \sum^{x}_{i=0}\binom{x}{i}^2$.
  \end{itemize}
\end{hint}

\begin{problem}
  Да разгледаме програмата на езика $\REC$:
  
  \begin{haskellcode}
h(x) = g(x,x) where
  f(x,y) = if y == 0 || x == y then 1
             else f(x - 1,y - 1) + f(x - 1,y)
  g(x,y) = if y == 0 then 1
             else g(x,y-1) + f(x+y,y)
  \end{haskellcode}
  
  Докажете, че $(\forall x\in \Nat)[\D_V\val{\vv{h}}(x) \neq \bot \implies \D_V\val{\vv{h}}(x) = \binom{2x+1}{x}]$.
\end{problem}

\begin{problem}
  Дадена е следната програма на езика $\REC$:
  
  \begin{haskellcode}
h(x) = f(0, x, x + 1) where
  f(x, y, z) = if x > y then z 
                 else f(x + 1, y, g(0, x, z))
  g(x, y, z) = if x == y then z
                 else g(x + 1, y, 2 + z)
  \end{haskellcode}
  
  Да се докаже, че:
  $(\forall x\in\Nat)[\D_V\val{\vv{h}}(x) \neq \bot \implies \D_V\val{\mathbf{h}}(x) = (x+1)^2]$.
\end{problem}

\begin{problem}
  Дадена е следната програма на езика $\REC$:
  
  \begin{haskellcode}
h(y) = f(0, y) where
  f(x, y) = if x == y then x
              else  f(x + 1, y) + g(0, x)
  g(x, y) = if x == y then 1
              else g(x + 1, y) + 2
  \end{haskellcode}
  
  Да се докаже, че:
  $(\forall x\in\Nat)[\D_V\val{\vv{h}}(x) \neq \bot \implies \D_V\val{\vv{h}}(x) = x^2+x]$.
\end{problem}

\begin{problem}
  Да разгледаме програмата на езика $\REC$:
  
  \begin{haskellcode}
h(x) = f(x) where 
  f(x) = if x <= 1 then 4 
           else g((f(x-1))^2,(f(x-2))^4)
  g(x,y) = if y == 0 then 0 
             else g(x,y-1) + x
  \end{haskellcode}
  
  Докажете, че $(\forall a\in\Nat)[\D_V\val{\vv{h}}(a) \neq \bot \implies \log_2(\D_V\val{\vv{h}}(a)) \equiv 2 \bmod 10]$.
\end{problem}

\begin{problem}
  Да разгледаме следната програма на езика $\REC$:
  
  \begin{haskellcode}
rm(x,y) = f(x,y) where
  f(x, y) = if x == 0 then 0
              else if f(x - 1, y) + 1 == y then 0
                else f(x-1, y) + 1
  \end{haskellcode}
  
  Докажете частична коректност на $\D_V\val{\vv{rm}}$ относно $I$ и $O$, където
  \begin{align*}
    & I(x,y) \dfff x,y\in\Nat\\
    & O(x,y,z) \dfff z\text{ е остатъкът при делението на $x$ с $y$}.
  \end{align*}
\end{problem}


\begin{problem}
  Да разгледаме следната програма на езика $\REC$:
  
  \begin{haskellcode}
qt(x, y) = g(x, y, 0) where
  f(x, y) = if x == 0 then 0
              else if f(x - 1, y) + 1 == y then 0
                else f(x-1, y) + 1
  g(x, y, z) = if x == 0 then z
                 else if f(x-1, y) + 1 == y then g(x-1, y, z+1)
                   else g(x - 1, y, z)
  \end{haskellcode}
  
  Докажете частична коректност на $\D_V\val{\vv{qt}}$ относно $I$ и $O$, където
  \begin{align*}
    & I(x,y) \dfff x,y\in\Nat\\
    & O(x,y,z) \dfff z\text{ е цялата част при делението на $x$ с $y$}.
  \end{align*}
\end{problem}

\begin{problem}
  Лесно се вижда, че всяко крайно множество 
  \[D = \{y_0 < y_1 < \cdots < y_k\}\] можем
  еднозначно да кодираме като числото $x = \sum^k_{i=0} 2^{y_i}$. Нека с $D_x$ да означим крайното множество с код $x$.
  \marginpar{Имаме, че $x = \sum_{y \in D_x}2^y$}
  Да разгледаме следната програма на езика $\REC$:

  \begin{haskellcode}
mem(x, y) = f(g(x, 2^y, 0), 2) where
  f(x, y) = if x == 0 then 0
              else if f(x - 1, y) + 1 == y then 0
                else f(x-1, y) + 1
  g(x, y, z) = if x == 0 then z
                 else if f(x-1, y) + 1 == y then g(x-1, y, z+1)
                   else g(x - 1, y, z)
  \end{haskellcode}
  
  Докажете частична коректност на $\D_V\val{\vv{mem}}$ относно свойствата $I$ и $O$, където
  \begin{align*}
    & I(x,y) \dfff x,y\in\Nat\\
    & O(x,y,z) \dfff (z = 1 \iff y \in D_x).
  \end{align*}
\end{problem}

%%% Local Variables:
%%% mode: latex
%%% TeX-master: "../sep"
%%% End:




%%% Local Variables:
%%% mode: latex
%%% TeX-master: "../sep"
%%% End:

% \chapter{Лениви спъсъци}
\marginpar{До известна степен следваме \cite[Глава 6]{bird-haskell}}

\Stefan{Дали не е по-добре вместо $[x,y,z]$ е съкратен запис за $x:y:z:[]$ ?}

\section{Основни понятия}
\begin{itemize}
\item
  Нека с $\nil$ да означаваме празния спъсък, т.е. единствения краен списък с дължина $0$.
\item
  {\bf Краен спъсък} с елементи естествените числа $a_0,\dots,a_{k-1}$
  ще записваме като $\pair{a_0,\dots,a_{k-1},\nil}$, т.е. това е елемент придандлежащ на множеството  
  \[\underbrace{\Nat \times \Nat \cdots \times \Nat}_{k} \times \{\nil\}.\]
  Дефинираме множеството от всички крайни списъци като
  \[\FinL \dff \bigcup_{k\geq 0} \Nat^k \times \{\nil\}.\]
\end{itemize}

Нека $P(l)$ е свойство на крайните списъци.
\begin{prooftree}
  \AxiomC{$P(\nil)$}
  \AxiomC{$(\forall a\in\Nat)(\forall l\in\FinL)[P(l) \implies P(\pair{a,l})]$}
  \BinaryInfC{$(\forall l\in \FinL)[P(l)]$}
\end{prooftree}

% \subsection{Частични списъци}

\marginpar{Това на практика са недовършени списъци}
{\bf Частичен списък} с елементи естествените числа $a_0,\dots,a_{k-1}$
ще записваме като $\pair{a_0,a_1,a_2,\dots,a_{k-1},\bot}$, т.е. това е елемент придандлежащ на множеството  
\[\underbrace{\Nat \times \Nat \cdots \times \Nat}_{k} \times \{\bot\}.\]
Дефинираме множеството от всички частични списъци като
\[\PartL \dff \bigcup_{k\geq 0} \Nat^k \times \{\bot\}.\]
Тогава $\bot$ може да се интерпретира като единствения частичен списък с дължина $0$.
Да видим какво ще стане ако изпълним следния ред:

\begin{haskellcode}
ghci> filter (<4) [1..]
[1,2,3                      -- == $(1:2:3:\bot)$
\end{haskellcode}
  
Програмата работи безкрайно дълго време след като вече е отпечатала първите три числа,
защото хаскел не знае, че след $3$ в безкрайния списък няма други числа по-малки от $4$.
Това означава, че резултатът от изпълнението на тази програма е частичния списък $\pair{1,2,3,\bot}$,

Нека $P(l)$ е свойство на частичните списъци.
\begin{prooftree}
  \AxiomC{$P(\bot)$}
  \AxiomC{$(\forall a\in\Nat)(\forall l\in\PartL)[P(l) \implies P( \pair{a,l} ) ]$}
  \BinaryInfC{$(\forall l\in \PartL)[P(l)]$}
\end{prooftree}


% \subsection*{Безкрайни списъци}

{\bf Безкраен списък} с елементи естествените числа $a_0,a_1,\dots$
ще записваме като $\pair{a_0,a_1,\dots}$, т.е. това е елемент придандлежащ на множеството  
\[\InfL \dff \Nat^\Nat = \{f:\Nat \to \Nat \mid f \text{ е тотална}\}.\]
Както вече знаем, на хаскел е много лесно да работим с безкрайни списъци.
\begin{haskellcode}
ghci> take 10 [1,3..]                 -- $a_n = 1 + (3-1)*n$
[1,3,5,7,9,11,13,15,17,19]
ghci> take 10 [x*x | x <- [0..]]      -- $a_n = n^2$
[0,1,4,9,16,25,36,49,64,81]
\end{haskellcode}

\marginpar{chain-complete property \cite[стр. 218]{bird-haskell}}
Нека $P(l)$ е {\em непрекъснато свойство} на частичните и безкрайни списъци, ако 
за всяка верига от частични списъци $(l_n)^\infty_{n=0}$ имаме, че $(\forall n)[P(l_n)]$,
то тогава имаме, че $P(\bigsqcup_n l_n)$.

\begin{prooftree}
  \AxiomC{$P(\bot)$}
  \AxiomC{$(\forall a\in\Nat)(\forall l\in\PartL)[P(l) \implies P(\pair{a,l})]$}
  \BinaryInfC{$(\forall l\in \PartL)[P(l)]$}
\end{prooftree}



% \begin{itemize}
% \item 
%   Нека с $\nil$ да означаваме празния спъсък, т.е. единствения краен списък с дължина $0$.
% \item
%   {\bf Краен спъсък} с елементи естествените числа $a_0,\dots,a_{k-1}$
%   ще записваме като $\pair{a_0,\dots,a_{k-1},\nil}$, т.е. това е елемент придандлежащ на множеството  
%   \[\underbrace{\Nat \times \Nat \cdots \times \Nat}_{k} \times \{\nil\}.\]
%   Дефинираме множеството от всички крайни списъци като
%   \[\FinL \dff \bigcup_{k\geq 0} \Nat^k \times \{\nil\}.\]
% \item
%   \marginpar{Това на практика са недовършени списъци}
%   {\bf Частичен списък} с елементи естествените числа $a_0,\dots,a_{k-1}$
%   ще записваме като $\pair{a_0,\dots,a_{k-1},\bot}$, т.е. това е елемент придандлежащ на множеството  
%   \[\underbrace{\Nat \times \Nat \cdots \times \Nat}_{k} \times \{\bot\}.\]
%   Дефинираме множеството от всички частични списъци като
%   \[\PartL \dff \bigcup_{k\geq 0} \Nat^k \times \{\bot\}.\]
%   Тогава $\bot$ може да се интерпретира като единствения частичен списък с дължина $0$.
%   Да видим какво ще стане ако изпълним следния ред:
  
%   \begin{haskellcode}
% ghci> filter (<4) [1..]
% [1,2,3
%   \end{haskellcode}
  
% Програмата работи безкрайно дълго време след като вече е отпечатала първите три числа,
% защото хаскел не знае, че след $3$ в безкрайния списък няма други числа по-малки от $4$.
% Това означава, че резултатът от изпълнението на тази програма е частичния списък $\pair{1,2,3,\bot}$,
% \item
%   {\bf Безкраен списък} с елементи естествените числа $a_0,a_1,\dots$
%   ще записваме като $\pair{a_0,a_1,\dots}$, т.е. това е елемент придандлежащ на множеството  
%   \[\InfL \dff \Nat^\Nat = \{f:\Nat \to \Nat \mid f \text{ е тотална}\}.\]
%   Както вече знаем, на хаскел е много лесно да работим с безкрайни списъци.
%   \begin{haskellcode}
% ghci> take 10 [1,3..]
% [1,3,5,7,9,11,13,15,17,19]
% ghci> take 10 [x*x | x <- [1..]]
% [1,4,9,16,25,36,49,64,81,100]
%   \end{haskellcode}
% \end{itemize}



\section{Област на Скот от списъци}


Дефинираме областта на Скот $L = (L,\sqsubseteq, \bot)$, където
\[L = \FinL \cup \PartL \cup \InfL,\]
а частичната наредба $\sqsubseteq$ е дефинирана следвайки правилата:
\begin{itemize}
\item
  $(a_0 : a_1 : \dots: a_{n-1} : \bot) \sqsubseteq (a_0 : a_1 : \dots : a_{n-1} : b_0 : \dots : b_{k-1} : \bot)$;
\item
  $(a_0 : a_1 : \dots : a_{n-1} : \bot) \sqsubseteq (a_0 : \dots : a_{n-1} : b_0 : \dots : b_{k-1} : \nil)$;
\item
  $(a_0 : \dots : a_{n-1} : \bot) \sqsubseteq (a_0 : \dots : a_{n-1} : b_0 : b_1 : \cdots)$.
\end{itemize}

% \begin{haskellcode}
% data LazyList = Nil | Cons Int LazyList deriving (Show)
% \end{haskellcode}

Според правилата, когато $n = 0$, получаваме, че $\bot$ е най-малкият елемент на $L$.
Обърнете внимание, че
\[(\forall \alpha_1,\alpha_2 \in \FinL \cup \InfL)[\alpha_1 \sqsubseteq \alpha_2 \iff \alpha_1 = \alpha_2].\]
Това означава, че например,
\begin{itemize}
% \item 
%   $\pair{0,1,2,\bot} \sqsubseteq \pair{0,1,2,\nil}$;
\item
  $(0:1:\nil) \not\sqsubseteq (0:1:2:\nil)$;
\item
  $\bot \sqsubseteq (0:\bot) \sqsubseteq (0:1:\bot) \sqsubseteq \cdots \sqsubseteq [0,1,\dots]$.
\end{itemize}

\begin{framed}
  \begin{figure}[H]
    \label{fig:lazy-list}
    \centering
    \begin{tikzpicture}[shorten >=1pt,->]
      \tikzstyle{vertex}=[circle,minimum size=17pt,inner sep=0pt]
      
      \node[vertex] (bot) at (3,0) {$\bot$};
      \node[vertex] (0) at (2,1) {$\nil$};
      \node[vertex] (1) at (4,1) {$(0:\bot)$};

      \node[vertex] (11) at (3,2.5) {$(0:\nil)$};
      \node[vertex] (12) at (5,2.5) {$(0:1:\bot)$};

      \node[vertex] (122) at (3.5,4.3) {$(0:1:\nil)$};
      \node[vertex] (123) at (7,4) {$(0:1:2:\bot)$};
      
      \node[vertex] (1233) at (4.6,6) {$(0:1:2:\nil)$};
      \node[vertex] (dots) at (9,5.5) {};

      \draw (bot) -- node[below left]{$\scriptstyle{\sqsupseteq}$} (0);
      \draw (bot) -- node[below right]{$\scriptstyle{\sqsubseteq}$} (1);
      \draw (1) -- node[below left]{$\scriptstyle{\sqsupseteq}$} (11);
      \draw (1) -- node[below right]{$\scriptstyle{\sqsubseteq}$} (12);
      \draw (12) -- node[below left]{$\scriptstyle{\sqsupseteq}$} (122);
      \draw (12) -- node[below right]{$\scriptstyle{\sqsubseteq}$} (123);
      \draw (123) -- node[below left]{$\scriptstyle{\sqsupseteq}$} (1233);
      \draw[dashed] (123) -- node[below right]{$\scriptstyle{\sqsubseteq}$} (dots);
    \end{tikzpicture}    
    \caption{Графично представяне на $\sqsubseteq$ върху част от $L$}
  \end{figure}
\end{framed}

\subsection*{Канонични апроксимации}

\marginpar{Тук следваме \cite[стр. 218]{bird-haskell}}
\index{канонична апроксимация}
За всеки елемент $l \in L$, дефинираме неговата {\bf $n$-та канонична апроксимация} $l\upharpoonright{n}$, където:
\begin{align*}
  & l \upharpoonright{0} \dff \bot\\
  & \nil \upharpoonright{(n+1)} \dff \nil\\
  % & l \upharpoonright{(n+1)} \dff \texttt{cons}(\texttt{car}(l), l\upharpoonright{n}).
  & (a:l) \upharpoonright{(n+1)} \dff a:(l\upharpoonright{n}).
\end{align*}
Имаме свойството, че за всеки списък $l \in L$, 
\[n \leq n' \implies l\upharpoonright n \sqsubseteq l\upharpoonright n'.\]
Това означава, че $(l\upharpoonright n)^{\infty}_{n=0}$ е верига.
Лесно се съобразява, че 
\[l = \bigsqcup_n (l \upharpoonright n).\]
Друго важно свойство, което имаме е, че 
\[l_1 \sqsubseteq l_2 \implies l_1 \upharpoonright n \sqsubseteq l_2 \upharpoonright n.\]
\begin{itemize}
\item 
  Да обърнем внимание на апроксимациите на крайни списъци.
  Нека $l = \pair{a_0,a_1,\dots,a_{n-1},\nil}$. Тогава според дефиницията:
  \begin{align*}
    & l \upharpoonright {0} = \bot,\\
    & l \upharpoonright k = \pair{a_0,\dots,a_{k-1},\bot} \text{, за }k = 1,\dots, n,\\
    & l \upharpoonright k = l \text{, за }k\geq n+1.
  \end{align*}

\item
  Да разгледаме няколко примера.
  \begin{itemize}
  \item 
    Нека $l = (0:1:\nil) \in \FinL$. Тогава 
    \[\underbrace{\bot}_{l\upharpoonright 0} \sqsubseteq \underbrace{(0:\bot)}_{l\upharpoonright 1} \sqsubseteq \underbrace{(0:1:\bot)}_{l\upharpoonright 2} \sqsubseteq \underbrace{(0:1:\nil)}_{l\upharpoonright 3} = \underbrace{(0:1:\nil)}_{l\upharpoonright 4} = \cdots\]
  \item
    Нека $l = (0:1:\bot) \in \PartL$. Тогава
    \[\underbrace{\bot}_{l\upharpoonright 0} \sqsubseteq \underbrace{(0:\bot)}_{l\upharpoonright 1} \sqsubseteq \underbrace{(0:1:\bot)}_{l\upharpoonright 2} = \underbrace{(0:1:\bot)}_{l\upharpoonright 3} = \underbrace{(0:1:\bot)}_{l\upharpoonright 4} = \cdots\]
  \item
    Нека $l = (0:1:2:\cdots) \in \InfL$. Тогава
    \[\underbrace{\bot}_{l\upharpoonright 0} \sqsubseteq \underbrace{(0:\bot)}_{l\upharpoonright 1} \sqsubseteq \underbrace{(0:1:\bot)}_{l\upharpoonright 2} \sqsubseteq \underbrace{(0:1:2:\bot)}_{l\upharpoonright 3} \sqsubseteq \underbrace{(0:1:2:3:\bot)}_{l\upharpoonright 4} \sqsubseteq \cdots\]
  \end{itemize}
\end{itemize}


Можем да дефинираме апроксимацията на списъци по следния начин на хаскел:
\begin{haskellcode}
approx _      0         = undefined         -- $l \upharpoonright 0 = \bot$
approx []     n | n > 0 = []                -- $\nil \upharpoonright (n+1) = \nil$
approx (x:xs) n | n > 0 = x:approx xs (n-1) -- $(a:l) \upharpoonright (n+1) = (a: (l\upharpoonright n))$
\end{haskellcode}

\begin{problem}
  \label{prob:approx}
  Да дефинираме изображението 
  \[\texttt{approx}_n(l) \dff l \upharpoonright n.\]
  Докажете, че $\texttt{approx}_n \in \Cont{L}{L}$.
\end{problem}

\begin{haskellcode}
ghci> approx [1..10] 0    --  == $\bot$
*** Exception: Prelude.undefined
ghci> approx [1..10] 10   --  == (1:2:3:4:5:6:7:8:9:10:$\bot$)
[1,2,3,4,5,6,7,8,9,10*** Exception: Prelude.undefined  
ghci> approx [1..10] 11   --  == (1:2:3:4:5:6:7:8:9:10:$\nil$)
[1,2,3,4,5,6,7,8,9,10]
ghci> approx [1..10] 101  --  == (1:2:3:4:5:6:7:8:9:10:$\nil$)
[1,2,3,4,5,6,7,8,9,10]
\end{haskellcode}

\Stefan{Да се обасни каква е разликата между take и approx}

\begin{prop}
  Областта на Скот $L$ е алгебрична като крайните елементи $K(L) = \FinL$ и $\PartL$.
\end{prop}
% \begin{hint}
%   Банално.
% \end{hint}

\section{Алгебричност на непрекъснатите изображения}

Целта ни тук ще бъде да докажем, че $\Cont{L}{L}$ е алгебрична област на Скот.
За да направим това трябва да намерим {\em крайните елементи} на $\Cont{L}{L}$.
Понеже крайните елементи на $L$ са $\FinL$ и $\PartL$, то е логично да предположим,
че крайните елементи на $\Cont{L}{L}$ са дефинирани чрез крайно много елементи на $K(L)$.

\begin{problem}
  Докажете, че за всяко $f \in \Cont{L}{L}$ е изпълнено, че:
  \[f(l) = \bigsqcup_n \{f(l \upharpoonright n)\upharpoonright n\}.\]
\end{problem}
\begin{hint}
  Използвайте \Th{double-chain}.
\end{hint}

\begin{problem}
  За произволно изображение $f \in \Cont{L}{L}$ и число $n$, да ознчим с $f \upharpoonright n$ изображението, където
  \[(f\upharpoonright n)(l) \dff f(l\upharpoonright n)\upharpoonright n.\]
  Докажете, че $f\upharpoonright n \in \Cont{L}{L}$.
\end{problem}
\begin{hint}
  Използвайте \Problem{approx}.
\end{hint}

\Stefan{Това не е ясно!}
Можем да считаме изображенията $f\upharpoonright n$ като апроксимации на $f$.
Обаче те не са крайни апроксимации, защото за да дефинираме $f \upharpoonright n$
се нуждаем от безкрайна информация. Това е така, защото има безкрайно много списъци с дължина $\leq n$.
Оттук следва, че изображенията $f \upharpoonright n$ не са крайните елементи на $\Cont{L}{L}$.

\begin{problem}
  Нека $a \in K(\A)$ и $b \in K(\B)$.
  Да дефинираме изображението
  \[\theta(x) \dff
  \begin{cases}
    b, & \text{ ако } a \sqsubseteq x\\
    \bot, & \text{ иначе}.
  \end{cases}\]
  Докажете, че:
  \begin{enumerate}[1)]
  \item 
    $\theta \in \Cont{\A}{\B}$.
  \item
    ако $f \in \Cont{\A}{\B}$, то $\theta \sqsubseteq f \iff b \sqsubseteq f(a)$.
  \end{enumerate}
\end{problem}
\begin{hint}
  
\end{hint}

Сега трябва да видим кога и по какъв начин можем да правим крайни обединения на такива изображения.
Това не е толкова просто, защото искаме полученото изображение също да бъде непрекъснато.

Две редици от елементи на $K(L)$
\begin{align*}
  \bar{a} & = (a_0,\dots,a_{n-1}),\\
  \bar{b} & = (b_0,\dots,b_{n-1})
\end{align*}
се наричат {\bf съвместими}, ако е изпълнено свойството:
\begin{equation}
  \label{eq:mon}
  (\forall i,j < n)[a_i \sqsubseteq a_j \implies b_i \sqsubseteq b_j].
\end{equation}

\begin{example}
  Редиците от крайни елементи
  \begin{align*}
    \bar{a} & = (\ (0:\bot),\ (1:\bot),\ (0:1:\nil)\ ),\\
    \bar{b} & = (\ (1:2:\nil),\ \bot,\ (1:2:3:\nil)\ ),
  \end{align*}
  не са съвместими, защото
  \[(0:\bot) \sqsubseteq (0:1:\nil)\text{, но } (1:2:\nil) \not\sqsubseteq (1:2:3:\nil).\]
\end{example}

Възможно е някои от елементите на $\bar{a}$ да са несравними помежду си, но обърнете внимание, че за произволен елемент $x$, 
всички елементи на множеството 
\[A \dff \{a_i \mid i < n\ \&\ a_i \sqsubseteq x\}\]
са сравними помежду си.
Това означава, че 
\[A = \{a_{i_0} \sqsubseteq a_{i_1} \sqsubseteq \cdots \sqsubseteq a_{i_k}\},\] за някое $k$,
и следователно, $A$ притежава точна горна граница.
Естествено, множеството $A$ може и да е празно. Тогава точната горна граница на $A$ ще бъде $\bot$,
защото по дефиниция $\bigsqcup\emptyset = \bot$.

Да разгледаме изображението 
\begin{align*}
  \theta_{\bar{a},\bar{b}}(x) & \dff \bigsqcup_{i<n}\{b_i \mid a_i \sqsubseteq x\}\\
  & = \begin{cases}
    b_0, & \text{ ако } a_0 = \bigsqcup_{i<n}\{a_i \mid a_i \sqsubseteq x\}\\
    \vdots & \\
    b_{n-1}, & \text{ ако } a_{n-1} = \bigsqcup_{i<n}\{a_i \mid a_i \sqsubseteq x\}\\
    \bot, & \text{ иначе}.
  \end{cases}
\end{align*}

\Stefan{На картинка нещата не изглеждат толкова сложни.}

\begin{prop}
  Ако $\bar{a}$ и $\bar{b}$ са съвместими редици от крайни елементи на $L$, то имаме свойствата:
  \begin{enumerate}[1)]
  \item 
    $\theta_{\bar{a},\bar{b}} \in \Cont{L}{L}$;
  \item
    За всяко $f \in \Cont{L}{L}$, 
    \[\theta_{\bar{a},\bar{b}} \sqsubseteq f\ \iff\ (\forall i < n)[b_i \sqsubseteq f(a_i)];\]
  \item
    $\theta_{\bar{a},\bar{b}}$ е компактен елемент в $\Cont{L}{L}$.
  \end{enumerate}
\end{prop}
\begin{proof}
  \begin{enumerate}[1)]
  \item 
    Лесно. Използва се, че $\bar{a}$ са крайни елементи.
  \item
    Използва се само монотонността на $f$.
  \item
    Тук съществено се използва, че $\bar{b}$ са крайни елементи.

    Нека $(f_r)^\infty_{r=0}$ е верига в $\Cont{L}{L}$ и 
    $\theta_{\bar{a},\bar{b}} \sqsubseteq \bigsqcup_r f_r$.
    От $2)$, това означава, че за $i < n$,
    \[b_i \sqsubseteq (\bigsqcup_r f_r)(a_i) = \bigsqcup_r \{f_r(a_i)\}.\]
    Понеже $b_i$ е краен елемент, то съществува индекс $r_i$, за който
    \[b_i \sqsubseteq f_{r_i}(a_i).\]
    Накрая взимаме $r = \max\{r_1,\dots,r_n\}$.
    Получаваме, че за $i < n$,
    \[b_i \sqsubseteq f_{r}(a_i).\]
    Тогава отново от 2) следва, че $\theta_{\bar{a},\bar{b}} \sqsubseteq f_r$.
  \end{enumerate}
\end{proof}

\begin{framed}
\begin{prop}
  Областта на Скот $\Cont{L}{L}$ е алгебрична.
\end{prop}
\end{framed}
\begin{proof}
  Да разгледаме произволен елемент $f \in \Cont{L}{L}$.
  Ще покажем, че съществува верига от крайни елементи $(\theta_n)^\infty_{n=0}$, за която $f = \bigsqcup_n \theta_n$.
  Нека подредим елементите на $K(L)$ в една редица $a_0,a_1,\dots$,
  което можем да направим, защото те са изброимо много.
  Да означим 
  \begin{align*}
    \bar{a}_n & = (a_0,a_1,\dots,a_{n})\\
    \bar{b}_n & = (f(a_0)\upharpoonright n, \dots, f(a_{n})\upharpoonright n).
  \end{align*}
  Нека да положим
  \[\theta_n \dff \theta_{\bar{a}_n,\bar{b}_n}.\]

  Лесно се проверява, че $(\theta_n)^\infty_{n=0}$ е верига, защото $f$ е монотонно изображение.
  
  Трябва да проверим, че $f = \bigsqcup_n \theta_n$. 
  От дефиницията на $\theta_n$ е ясно, че $\theta_n \sqsubseteq f$ за всяко $n$.
  Следователно, $\bigsqcup_n \theta_n \sqsubseteq f$.
  За другата посока, нека разгледаме произволен елемент $l \in L$.
  Ще докажем, че $f(l) \sqsubseteq (\bigsqcup_n \theta_n)(l)$.
  Да разгледаме верига от елементи на $K(L)$, $(a_{i_n})^\infty_{n=0}$, за която $i_0 < i_1 < \cdots$
  и $\bigsqcup_n a_{i_n} = l$. Знаем, че такава съществува, защото $L$ е алгебрична област на Скот.
  \Stefan{Трябва някъде да сложа твърдение, че ако $\bigsqcup_n a_n = b$, то за произволна подредица
  $\bigsqcup_n a_{i_n} = b$.}
  Тогава 
  \begin{align*}
    f(l) & = \bigsqcup_n \{f(a_{i_n}) \upharpoonright n\}\\
         & \sqsubseteq \bigsqcup_n \{f(a_{i_n}) \upharpoonright i_n \} & (i_n \geq n)\\
         & = \bigsqcup_n \{\theta_{i_n}(a_{i_n})\} & (\text{от деф. на }\theta_{i_n})\\
         & = \bigsqcup_n\{ \bigsqcup_m \{\theta_{i_n}(a_{i_m})\}\} & (\text{\Th{double-chain}})\\
         & = \bigsqcup_n \{\theta_{i_n}(\bigsqcup_m a_{i_m})\} & (\theta_{i_n} \in \Cont{L}{L})\\
         & = \bigsqcup_n \{\theta_{i_n}(l)\} & (l = \bigsqcup_m a_{i_m})\\
         & = (\bigsqcup_n \theta_{i_n})(l) \\
         & \sqsubseteq (\bigsqcup_n \theta_n)(l).
  \end{align*}
\end{proof}


\section{Задачи}

\Stefan{Дали да дефинираме изображение cons, което да докажа, че е непрекъснато?}

\begin{problem}
  Да се даде пример за $f \in \Mon{L}{L}$, но $f \not\in \Cont{L}{L}$.
\end{problem}

\begin{problem}
  Да се даде пример за $f \in \Strict{L}{L}$, но $f \not\in \Mon{L}{L}$.
\end{problem}

\begin{problem}
  Да разгледаме изображението $f \in \Cont{L}{L}$, където
  \[f(l) \dff (0:l).\]
  Намерете $\lfp(f)$.
\end{problem}
\begin{solution}
  Прилагаме \Th{knaster-tarski}.
  \begin{itemize}
  \item 
    $l_0 = \bot$;
  \item
    $l_1 = f(\bot) = (0:\bot)$;
  \item
    $l_2 = f(l) = (0:0:\bot)$;
  \end{itemize}
  Така получаваме, че
  \[l_n = \pair{\underbrace{0,0,\dots,0}_{n},\bot} \in \PartL.\]
  Тогава 
  \[\lfp(f) = \bigsqcup_n l_n = \pair{0,0,\dots}.\]



\begin{haskellcode}
ghci> let f(l) = (0:l)
ghci> let approx = undefined : [f(x) | x <- approx]
ghci> approx !! 10
[0,0,0,0,0,0,0,0,0,0*** Exception: Prelude.undefined  
\end{haskellcode}

\end{solution}

\begin{problem}
  Да разгледаме изображението 
  \[\Gamma \in \Cont{\Cont{\Nat_\bot\times L}{L}}{\Cont{\Nat_\bot\times L}{L}},\] където
  \[\Gamma(f)(n,l) =
  \begin{cases}
    \bot, & n = \bot\\
    \nil, & n = 0\\
    \nil, & n > 0\ \&\ l = \nil\\
    \bot, & n > 0\ \&\ l = \bot\\
    a : f(l'), & n > 0\ \&\ l = (a:l')\\
  \end{cases}\]
  Намерете $\lfp(\Gamma)$.
\end{problem}


\begin{haskellcode}
ghci> let l = (0:l)
ghci> take 10 l
[0,0,0,0,0,0,0,0,0,0]
ghci> take 10 [1,2,3]
[1,2,3]
ghci> take 0 undefined
[]
ghci> take 2 [1,2,undefined]
[1,2]
ghci> take 3 [1,2,undefined]
[1,2,*** Exception: Prelude.undefined
ghci> take undefined []
*** Exception: Prelude.undefined
\end{haskellcode}


\Stefan{Да се разгледа функцията drop n l}

% \begin{problem}
%   Да разгледаме оператора $\Gamma:[\Nat\to L] \to [\Nat \to L]$, където
%   \[\Gamma(f)(a,b) = \pair{a, f(b,a+b)}.\]
%   Намерете безкрайния списък $\lfp(\Gamma)(1,1)$.
% \end{problem}

\begin{problem}
  Да разгледаме оператора $\Gamma:[L\times L \to L] \to [L\times L \to L]$, където
  \[\Gamma(f)(l_1,l_2) = 
  \begin{cases}
    \bot, & \text{ ако }l_1 = \bot \\
    l_2, & \text{ ако }l_1 = \nil\\
    a:f(l'_1,l_2) & \text{ ако } l_1 = a:l'_1.
  \end{cases}\]
  Да означим $\texttt{conc} \dff \lfp(\Gamma)$.
  Докажете, че
  \[(\forall l_1 \in \FinL)(\forall l_2,l_3 \in L)[\texttt{conc}(l_1, \texttt{conc}(l_2,l_3)) = \texttt{conc}(\texttt{conc}(l_1, l_2), l_3))]\]
\end{problem}

%%% Local Variables:
%%% mode: latex
%%% TeX-master: "../sep-notes"
%%% End:


%%% Local Variables:
%%% mode: latex
%%% TeX-master: "../sep"
%%% End:


% \include{call-by-value}
% \include{call-by-name}
% \include{verification}
% include{dataflow}
% \include{abstract-interpretation}
%\include{extra}
% \include{struct-ind}

% \include{tail}
%\backmatter

\bibliographystyle{plain}
\bibliography{sep}

\printindex

\end{document}


%%% Local Variables:
%%% mode: latex
%%% TeX-master: t
%%% End:
